\chapter{First Order Differential Equations} \label{fo:chapter}
\renewcommand{\thesection}{\thechapter.\the\value{section}}

In this chapter, we begin by discussing first order differential equations. As they have the lowest possible order, only containing one derivative of the unknown function, they tend to be the simplest equations to try to analyze and solve. This doesn't mean that we'll be able to solve all of them, but we can make a decent effort at a fair number of them. These equations are also very common in modeling problems, as most principles from science and engineering give us a way to express the rate of change of a given quantity. This setup gives rise to a first order differential equation involving that quantity, which, if we can solve it, will tell us how the quantity evolves over time. Even if we can't solve the equation analytically, a numerical solution may allow us to predict the behavior of a system over time and design it to best fit our needs. 

%%%%%%

\input sec-integral-sols.tex

%%%%%%
\sectionnewpage
\input sec-slope-fields.tex

%%%%%%
\sectionnewpage
\input sec-separable.tex

%%%%%%
\sectionnewpage
\input sec-first-order-lin.tex

%%%%%%
\sectionnewpage
\input sec-first-existence.tex

%%%%%% 
\sectionnewpage
\input sec-numerical-euler.tex

%%%%%% 
\sectionnewpage
\input sec-autonomous-eqn.tex

%%%%%%
\sectionnewpage
\input sec-bifurcation-diag.tex

%%%%%% 
\sectionnewpage
\input sec-exact.tex

%%%%%%%
\sectionnewpage
\input sec-modeling-first.tex

%%%%%%
\sectionnewpage
\input sec-modeling-parameters.tex

%%%%%% 
\sectionnewpage
\input sec-substitution.tex

%%%%%% 
% \sectionnewpage
% \input sec-first-pde.tex

%%%%%%
% \sectionnewpage
% \input sec-chapFO-review.tex
