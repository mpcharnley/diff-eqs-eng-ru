
\section{The Laplace transform}
\label{laplace:section}

\LAtt{6.1}

\LO{
\item Define the Laplace transform,
\item Compute the Laplace transform of a variety of functions, and
\item Compute the inverse Laplace transform of functions when it exists. 
}

% \sectionnotes{Verbatim from Lebl}

% \sectionnotes{1.5--2 lectures\EPref{, \S10.1 in \cite{EP}}\BDref{,\S6.1 and parts of \S6.2 in \cite{BD}}}

\subsection{The transform}

In this chapter we will discuss the Laplace transform%
\footnote{Just like the Laplace equation and the Laplacian, the Laplace
transform is also named after 
\href{https://en.wikipedia.org/wiki/Laplace}{Pierre-Simon, marquis de Laplace}
(1749--1827).}.
The Laplace transform
is a very efficient method to solve certain ODE or PDE problems.
The transform takes a differential equation and turns it into
an algebraic equation.  If the algebraic equation can be solved, applying the
inverse transform gives us our desired solution.
The Laplace transform also has applications in
the analysis of 
electrical circuits, NMR spectroscopy, signal processing, and elsewhere.
Finally,
understanding the Laplace
transform will also help with understanding the related Fourier transform,
which, however, requires more
understanding of complex numbers.  We will not cover the Fourier transform.

The Laplace transform also gives a lot of insight into the nature of the
equations we are dealing with.  It can be seen as converting between the time
and the frequency domain.  For example, take the standard equation
\begin{equation*}
m x''(t) + c x'(t) + k x(t) = f(t) .
\end{equation*}
We can think of $t$ as time and $f(t)$ as incoming signal.  The Laplace
transform will convert the equation from a differential equation in time to
an algebraic (no derivatives) equation, where the new independent variable
$s$ is the frequency.

We can think of the \emph{\myindex{Laplace transform}} as a black box.  It
eats functions and spits out functions in a new variable.  We write
$\mathcal{L} \bigl\{ f(t) \bigr\} = F(s)$ for the Laplace transform of $f(t)$.
It is common to write lower case letters for
functions in the time domain and upper case letters for functions in the
frequency domain.  We use the same letter to denote that one function
is the Laplace transform of the other.  For example, $F(s)$ is the Laplace
transform of $f(t)$.  Let us define the transform.
\begin{equation*}
\mathcal{L} \bigl\{ f(t) \bigr\} =
F(s) \overset{\text{def}}{=} \int_0^\infty e^{-st} f(t) \, dt .
\end{equation*}
We note that we are only considering $t \geq 0$ in the transform.  Of course,
if we think of $t$ as time there is no problem, we are generally interested in
finding out what will happen in the future (Laplace transform is one place
where it is safe to ignore the past).  Let us compute some simple
transforms.

\begin{example}
Suppose $f(t) = 1$, then
\begin{equation*}
\mathcal{L} \{1\} = \int_0^\infty e^{-st} \, dt
=
\left[ \frac{e^{-st}}{-s} \right]_{t=0}^\infty
=
\lim_{h\to\infty}
\left[ \frac{e^{-st}}{-s} \right]_{t=0}^h
=
\lim_{h\to\infty}
\left( \frac{e^{-sh}}{-s} - \frac{1}{-s} \right)
= \frac{1}{s} .
\end{equation*}
The limit (the improper integral) only exists if $s > 0$.  So 
$\mathcal{L} \{1\}$ is only defined for $s > 0$.
\end{example}

\begin{example}
Suppose $f(t) = e^{-at}$, then
\begin{equation*}
\mathcal{L} \bigl\{e^{-at}\bigr\}
= \int_0^\infty e^{-st} e^{-at} \, dt
= \int_0^\infty e^{-(s+a)t} \, dt
=
\left[ \frac{e^{-(s+a)t}}{-(s+a)} \right]_{t=0}^\infty
= \frac{1}{s+a} .
\end{equation*}
The limit only exists if $s+a > 0$.  So 
$\mathcal{L} \bigl\{e^{-at}\bigr\}$ is only defined for $s+a > 0$.
\end{example}

\begin{example}
Suppose $f(t) = t$, then using integration by parts
\begin{equation*}
\begin{split}
\mathcal{L} \{t\}
& = \int_0^\infty e^{-st} t \, dt \\
& =
\left[ \frac{-te^{-st}}{s} \right]_{t=0}^\infty
+
\frac{1}{s}
\int_0^\infty e^{-st} \,dt \\
& =
0
+
\frac{1}{s}
\left[ \frac{e^{-st}}{-s} \right]_{t=0}^\infty \\
& =
\frac{1}{s^2} .
\end{split}
\end{equation*}
Again, the limit only exists if $s > 0$.
\end{example}

\begin{example}
A common function is the \emph{\myindex{unit step function}}, which is
sometimes called the \emph{\myindex{Heaviside function}}%
\footnote{The function is named after the English mathematician, engineer, and
physicist
\href{https://en.wikipedia.org/wiki/Heaviside}{Oliver Heaviside}
(1850--1925).  Only
by coincidence is the function \myquote{heavy} on \myquote{one side.}}.
This function is generally given as
\begin{equation*}
u(t) = 
\begin{cases}
0 & \text{if } \; t < 0 , \\
1 & \text{if } \; t \geq 0 .
\end{cases}
\end{equation*}
%Note that some authors prefer to define $u(0) = \frac{1}{2}$.
Let us find the Laplace transform of $u(t-a)$, where $a \geq 0$
is some constant.
That is, the function that is 0 for $t < a$ and 1 for $t \geq a$.
\begin{equation*}
\mathcal{L} \bigl\{ u(t-a) \bigr\}
=
\int_0^{\infty} e^{-st} u(t-a) \, dt
=
\int_a^{\infty} e^{-st} \, dt
=
\left[ \frac{e^{-st}}{-s} \right]_{t=a}^\infty \\
=
\frac{e^{-as}}{s} ,
\end{equation*}
where of course $s > 0$ (and $a \geq 0$ as we said before).
\end{example}

By applying similar procedures we can compute the transforms of many
elementary functions.  Many basic transforms are listed in
\tablevref{lt:table}.

\begin{table}[h!t]
\mybeginframe
\capstart
\begin{center}
\begin{tabular}{@{}lllll@{}}
\toprule
$f(t)$ & $\mathcal{L} \bigl\{ f(t) \bigr\}$ & $\qquad\quad$ &
$f(t)$ & $\mathcal{L} \bigl\{ f(t) \bigr\}$ \\
\midrule
$C$ & $\frac{C}{s}$
& &
$\sin (\omega t)$ & $\frac{\omega}{s^2+\omega^2}$
\\[4pt]
$t$ & $\frac{1}{s^2}$
& &
$\cos (\omega t)$ & $\frac{s}{s^2+\omega^2}$
\\[4pt]
$t^2$ & $\frac{2}{s^3}$
& &
$\sinh (\omega t)$ & $\frac{\omega}{s^2-\omega^2}$
\\[4pt]
$t^3$ & $\frac{6}{s^4}$
& &
$\cosh (\omega t)$ & $\frac{s}{s^2-\omega^2}$
\\[4pt]
$t^n$ & $\frac{n!}{s^{n+1}}$
& &
$u(t-a)$ & $\frac{e^{-as}}{s}$
\\[4pt]
$e^{-at}$ & $\frac{1}{s+a}$
& & &
\\[4pt]
\bottomrule
\end{tabular}
\end{center}
\caption{Some Laplace transforms ($C$, $\omega$, and $a$ are
constants).\label{lt:table}}
\myendframe
\end{table}

\begin{exercise}
Verify \tablevref{lt:table}.
\end{exercise}

Since the transform is defined by an integral.  We can use the linearity
properties of the integral.  For example, suppose $C$ is a constant, then
\begin{equation*}
\mathcal{L} \bigl\{ C f(t) \bigr\} =
\int_0^\infty e^{-st} C f(t) \,dt =
C \int_0^\infty e^{-st} f(t) \,dt =
C \mathcal{L} \bigl\{ f(t) \bigr\} .
\end{equation*}
So we can \myquote{pull out} a constant out of the transform.  Similarly  we have
linearity.
Since linearity is very important we state it as a theorem.

\begin{theorem1}{Linearity of the Laplace transform}
\index{linearity of the Laplace transform}
Suppose that $A$, $B$, and $C$ are constants, then
\begin{equation*}
%\mybxbg{~~
\mathcal{L} \bigl\{ A f(t) + B g(t) \bigr\} =
A \mathcal{L} \bigl\{ f(t) \bigr\} +
B \mathcal{L} \bigl\{ g(t) \bigr\} ,
%~~}
\end{equation*}
and in particular
\begin{equation*}
\mathcal{L} \bigl\{ C f(t) \bigr\} =
C \mathcal{L} \bigl\{ f(t) \bigr\} .
\end{equation*}
\end{theorem1}

\begin{exercise}
Verify the theorem.  That is, show that
$\mathcal{L} \bigl\{ A f(t) + B g(t) \bigr\} =
A \mathcal{L} \bigl\{ f(t) \bigr\} +
B \mathcal{L} \bigl\{ g(t) \bigr\}$.
\end{exercise}

These rules together with \tablevref{lt:table} make it easy to find
the Laplace transform of a whole lot of functions already.
But be careful.
It is a common mistake to think that the Laplace transform of a product
is the product of the transforms.  In general 
\begin{equation*}
\mathcal{L} \bigl\{ f(t) g(t) \bigr\} \not=
\mathcal{L} \bigl\{ f(t) \bigr\}
\mathcal{L} \bigl\{ g(t) \bigr\} .
\end{equation*}

It must also be noted that not all functions have a Laplace transform.  For
example, the function $\frac{1}{t}$ does not have a Laplace transform as the
integral diverges for all $s$.  Similarly,
$\tan t$ or $e^{t^2}$ do not have Laplace transforms.

\subsection{Existence and uniqueness}

When does the Laplace transform exist?  A function $f(t)$ is of
\emph{\myindex{exponential order}} as $t$ goes to infinity if
\begin{equation*}
\lvert f(t) \rvert \leq M e^{ct} ,
\end{equation*}
for some constants $M$ and $c$, for
sufficiently large $t$ (say for all $t > t_0$ for some $t_0$).  The simplest
way to check this condition is to try and compute
\begin{equation*}
\lim_{t\to \infty} \frac{f(t)}{e^{ct}} .
\end{equation*}
If the limit exists and is finite (usually zero), then $f(t)$ is of
exponential order.

\begin{exercise}
Use L'Hopital's rule from calculus to show that a polynomial is of
exponential order.  Hint: Note that a sum of two exponential order functions
is also of exponential order.  Then show that $t^n$ is of exponential order
for any $n$.
\end{exercise}

For an exponential order function we have existence and uniqueness of the
Laplace transform.

\begin{theorem1}{Existence}
Let $f(t)$ be continuous and of exponential order for a certain
constant $c$.  Then $F(s) = \mathcal{L} \bigl\{ f(t) \bigr\}$ is defined for
all $s > c$.
\end{theorem1}

The existence is not difficult to see.  Let $f(t)$ be of exponential order,
that is $\lvert f(t) \rvert \leq M e^{ct}$ for all $t > 0$ (for simplicity $t_0 = 0$).
Let $s > c$, or in other words $(c-s) < 0$.
By the comparison theorem from calculus, the improper integral defining
$\mathcal{L} \bigl\{ f(t) \bigr\}$ exists if the following integral exists
\begin{equation*}
\int_0^\infty e^{-st} ( M e^{ct} ) \,dt
=
M \int_0^\infty e^{(c-s)t} \,dt = M \left[ \frac{e^{(c-s)t}}{c-s}
\right]_{t=0}^\infty = \frac{M}{c-s} .
\end{equation*}

The transform also exists for some other functions
that are not of exponential
order, but that will not be relevant to us.
Before dealing with uniqueness, let
us note that for exponential order functions we obtain that their
Laplace transform decays at infinity:
\begin{equation*}
\lim_{s\to\infty} F(s) = 0 .
\end{equation*}

\begin{theorem1}[lt:uniqthm]{Uniqueness}
Let $f(t)$ and $g(t)$ be continuous and of exponential order.
Suppose that there exists a constant $C$,
such that $F(s) = G(s)$ for all $s > C$.
Then $f(t) = g(t)$ for all $t \geq 0$.
\end{theorem1}

Both theorems hold for piecewise continuous functions as well.
Recall that piecewise continuous means that the function
is continuous except perhaps at a discrete set of points, where it has jump
discontinuities like the Heaviside function.  Uniqueness, however, does
not \myquote{see} values at the discontinuities.  So we can only conclude that
$f(t) = g(t)$ outside of discontinuities.  For example, the unit step
function is sometimes defined using $u(0) = \nicefrac{1}{2}$.  This new
step function, however, has the exact same Laplace transform
as the one we defined earlier where $u(0) = 1$.

\subsection{The inverse transform}

As we said, the Laplace transform will allow us to convert a differential
equation into an algebraic equation.  Once we solve the
algebraic equation in the frequency domain we will want to get back to the
time domain, as that is what we are interested in.
Given a function
$F(s)$, we wish to find a function
$f(t)$ such that $\mathcal{L} \bigl\{ f(t) \bigr\} = F(s)$.
\thmref{lt:uniqthm} says that the solution $f(t)$ is unique.
So we can without fear make the following definition.

Suppose $F(s) = \mathcal{L} \bigl\{ f(t) \bigr\}$ for some function $f(t)$.
Define the
\emph{\myindex{inverse Laplace transform}} as
\begin{equation*}
{\mathcal{L}}^{-1} \bigl\{ F(s) \bigr\} \overset{\text{def}}{=} f(t) .
\end{equation*}
There is an integral formula for the inverse, but it is not as simple
as the transform itself---it requires complex numbers and path integrals.
For us it will
suffice to
compute the inverse using \tablevref{lt:table}.

\begin{example}
Take
$F(s) = \frac{1}{s+1}$.  Find the inverse Laplace transform.

We look at the table to find
\begin{equation*}
{\mathcal{L}}^{-1} \left\{ \frac{1}{s+1} \right\} = 
e^{-t} .
\end{equation*}
\end{example}

As the Laplace transform is linear, the inverse Laplace
transform is also linear.  That is,
\begin{equation*}
{\mathcal{L}}^{-1} \bigl\{ A F(s) + B G(s) \bigr\} =
A {\mathcal{L}}^{-1} \bigl\{ F(s) \bigr\} +
B {\mathcal{L}}^{-1} \bigl\{ G(s) \bigr\} .
\end{equation*}
Of course, we also have
${\mathcal{L}}^{-1} \bigl\{ A F(s) \bigr\} =
A {\mathcal{L}}^{-1} \bigl\{ F(s) \bigr\}$.
Let us demonstrate how linearity can be used.

\begin{example}
Take
$F(s) = \frac{s^2+s+1}{s^3+s}$.  Find the inverse Laplace transform.
\end{example}

\begin{exampleSol}
First we use the \emph{\myindex{method of partial fractions}} to write $F$ in a form where
we can use \tablevref{lt:table}.  We factor the denominator as
$s(s^2+1)$ and write
\begin{equation*}
\frac{s^2+s+1}{s^3+s}
=
\frac{A}{s} + 
\frac{Bs+C}{s^2+1} .
\end{equation*}
Putting the right-hand side over a common
denominator and equating the numerators we get
$A(s^2+1) + s(Bs+C) = s^2+s+1$.  Expanding and equating coefficients
we obtain $A+B = 1$, $C=1$, $A=1$,
and thus $B=0$.  In
other words,
\begin{equation*}
F(s) =
\frac{s^2+s+1}{s^3+s}
=
\frac{1}{s} +
\frac{1}{s^2+1} .
\end{equation*}
By linearity of the inverse Laplace transform we get 
\begin{equation*}
{\mathcal{L}}^{-1} \left\{ 
\frac{s^2+s+1}{s^3+s} \right\}
=
{\mathcal{L}}^{-1} \left\{ 
\frac{1}{s} \right\} 
+
{\mathcal{L}}^{-1} \left\{ 
\frac{1}{s^2+1} \right\}
=
1 + 
\sin t .
\end{equation*}
\end{exampleSol}

Another useful property is the 
so-called \emph{\myindex{shifting property}} or
the \emph{\myindex{first shifting property}}
\begin{equation*}
\mybxbg{~~
\mathcal{L} \bigl\{ e^{-at} f(t) \bigr\} = F(s+a) ,
~~}
\end{equation*}
where $F(s)$ is the Laplace transform of $f(t)$.

\begin{exercise}
Derive the first shifting property
from the definition of the Laplace transform.
\end{exercise}

The shifting property can be used, for example, when the denominator is a
more complicated quadratic that may come up in the method of partial
fractions.  We complete the square and write such quadratics as ${(s+a)}^2+b$
and then use the shifting property.

\begin{example}
Find
${\mathcal{L}}^{-1} \left\{ \frac{1}{s^2+4s+8} \right\}$.
\end{example}

\begin{exampleSol}
First we complete the square to make the denominator ${(s+2)}^2+4$.  
Next we find
\begin{equation*}
{\mathcal{L}}^{-1} \left\{ \frac{1}{s^2+4} \right\}
=
\frac{1}{2} \sin (2t) .
\end{equation*}
Putting it all together with the shifting property, we find
\begin{equation*}
{\mathcal{L}}^{-1} \left\{ \frac{1}{s^2+4s+8} \right\} = 
{\mathcal{L}}^{-1} \left\{ \frac{1}{{(s+2)}^2+4} \right\}
=
\frac{1}{2}\,e^{-2t} \sin (2t) .
\end{equation*}
\end{exampleSol}

In general, we want to be able to apply the Laplace transform to
rational functions, that is functions of the form
\begin{equation*}
\frac{F(s)}{G(s)}
\end{equation*}
where $F(s)$ and $G(s)$ are polynomials.  Since normally, for the functions
that we are considering, the Laplace transform goes
to zero as $s \to \infty$, it is not hard to see that the degree of $F(s)$
must be smaller than that of $G(s)$.  Such rational functions
are called \emph{proper rational functions\index{proper rational function}}
and we can always apply the method of partial fractions.  Of
course this means we need to be able to factor the denominator into
linear and quadratic terms, which involves finding the roots of the
denominator.

\subsection{Exercises}

\begin{exercise}
Find the Laplace transform of $3+t^5+\sin (\pi t)$.
\end{exercise}

\begin{exercise}
Find the Laplace transform of $a+bt+ct^2$ for some constants $a$, $b$, and
$c$.
\end{exercise}

\begin{exercise}\ansMark
Find the Laplace transform of $4{(t+1)}^2$.
\end{exercise}
\exsol{%
$\frac{8}{s^3} + \frac{8}{s^2} + \frac{4}{s}$
}

\begin{exercise}
Find the Laplace transform of $A \cos (\omega t) + B \sin (\omega t)$.
\end{exercise}

\begin{exercise}
Find the Laplace transform of $\cos^2 (\omega t)$.
\end{exercise}

\begin{exercise}
Find the inverse Laplace transform of $\frac{4}{s^2-9}$.
\end{exercise}

\begin{exercise}
Find the inverse Laplace transform of $\frac{2s}{s^2-1}$.
\end{exercise}

\begin{exercise}\ansMark%
Find the inverse Laplace transform of $\frac{8}{s^3(s+2)}$.
\end{exercise}
\exsol{%
$2t^2-2t+1-e^{-2t}$
}

\begin{exercise}
Find the inverse Laplace transform of $\frac{1}{{(s-1)}^2(s+1)}$.
\end{exercise}

\begin{exercise}
Find the Laplace transform of $f(t) =
\begin{cases}
t & \text{if } \; t \geq 1, \\
0 & \text{if } \; t < 1.
\end{cases}$
\end{exercise}

\begin{exercise}
Find the inverse Laplace transform of $\frac{s}{(s^2+s+2)(s+4)}$.
\end{exercise}

\begin{exercise}
Find the Laplace transform of $\sin\bigl(\omega (t-a)\bigr)$.
\end{exercise}

\begin{exercise}\ansMark
Find the Laplace transform of $te^{-t}$ (Hint: integrate by parts).
\end{exercise}
\exsol{%
$\frac{1}{{(s+1)}^2}$
}

\begin{exercise}
Find the Laplace transform of $t\sin(\omega t)$.  Hint: Several integrations
by parts.
\end{exercise}


\begin{exercise}\ansMark
Find the Laplace transform of $\sin(t)e^{-t}$ (Hint: integrate by parts).
\end{exercise}
\exsol{%
$\frac{1}{s^2+2s+2}$
}


\setcounter{exercise}{100}

