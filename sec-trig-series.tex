\section{The trigonometric series} \label{ts:section}

\LAtt{4.2}

\LO{
\item Find and compute periodic extensions of functions and
\item Find the Fourier series for a given function.
}

% \sectionnotes{Verbatim from Lebl}

% \sectionnotes{2 lectures\EPref{, \S9.1 in \cite{EP}}\BDref{,
% \S10.2 in \cite{BD}}}

\subsection{Periodic functions and motivation}

As motivation for studying Fourier series, suppose we have the problem
\begin{equation} \label{ts:deq}
x'' + \omega_0^2 x = f(t) ,
\end{equation}
for some periodic function $f(t)$.
We already solved
\begin{equation} \label{ts:deqcos}
x'' + \omega_0^2 x = F_0 \cos ( \omega t) .
\end{equation}
One way to solve \eqref{ts:deq} is to
decompose $f(t)$ as a sum of cosines (and sines) and then
solve many problems of the form \eqref{ts:deqcos}.  We then use
the principle of superposition, to sum up all the solutions we got
to get a solution to \eqref{ts:deq}.

Before we proceed, let us talk a little bit more in detail about
periodic functions.
A function is said to be \emph{\myindex{periodic}} with period $P$ if
$f(t) = f(t+P)$ for all $t$.  For brevity we say $f(t)$ is $P$-periodic.
Note that a $P$-periodic function is also $2P$-periodic, $3P$-periodic
and so on.
For example, $\cos (t)$ and $\sin (t)$ are
$2\pi$-periodic.  So are $\cos (kt)$ and $\sin (kt)$ for all integers $k$.  The
constant functions are an extreme example.  They are periodic for any period
(exercise).

Normally we start with a function $f(t)$ defined on some interval $[-L,L]$,
and we want to
\emph{extend $f(t)$ periodically}\index{extend periodically}\index{periodic extension}
to make it
a $2L$-periodic function.  We do this extension
by defining a new function $F(t)$
such that for $t$ in $[-L,L]$, $F(t) = f(t)$.  For $t$ in $[L,3L]$,
we define $F(t) = f(t-2L)$, for $t$ in $[-3L,-L]$, $F(t) = f(t+2L)$, and
so on.
To make that work we needed $f(-L) = f(L)$.
We could have also started with $f$
defined only on the half-open interval $(-L,L]$ and then define $f(-L) = f(L)$.

\begin{example}
Define $f(t) = 1-t^2$ on $[-1,1]$.  Now extend $f(t)$ periodically to
a 2-periodic function.  See \figurevref{ts:perextofinvertedparabolafig}.
\begin{myfig}
\capstart
\diffyincludegraphics{width=3in}{width=4.5in}{ts-perextofinvertedparabola}
\caption{Periodic extension of the function
$1-t^2$.\label{ts:perextofinvertedparabolafig}}
\end{myfig}
\end{example}

You should be careful to distinguish between $f(t)$ and its extension.  A common
mistake is to assume that a formula for $f(t)$ holds for its extension.  It
can be confusing when the formula for $f(t)$ is periodic, but with perhaps
a different period.

\begin{exercise}
Define $f(t) = \cos t$ on $[\nicefrac{-\pi}{2},\nicefrac{\pi}{2}]$.  Take the $\pi$-periodic
extension and sketch its graph.  How does it compare to the graph of
$\cos t$?
\end{exercise}

\subsection{Inner product and eigenvector decomposition}

Suppose we have a \emph{\myindex{symmetric matrix}},
that is $A^T = A$.  As we remarked before, 
eigenvectors of $A$ are then orthogonal.  Here the word
\emph{orthogonal}\index{orthogonal!vectors} means
that if $\vec{v}$ and $\vec{w}$ are two 
eigenvectors of $A$ for distinct eigenvalues,
then $\langle \vec{v} , \vec{w} \rangle = 0$.
In this case the inner product $\langle \vec{v} , \vec{w} \rangle$
is the \emph{\myindex{dot product}},
which can be computed as $\vec{v}^T\vec{w}$.

To decompose a vector $\vec{v}$ in terms of mutually orthogonal
vectors $\vec{w}_1$ and $\vec{w}_2$ we write
\begin{equation*}
\vec{v} = a_1 \vec{w}_1  + a_2 \vec{w}_2 .
\end{equation*}
Let us find the formula for $a_1$ and $a_2$.  First let us compute
\begin{equation*}
\langle \vec{v} , \vec{w_1} \rangle
=
\langle a_1 \vec{w}_1  + a_2 \vec{w}_2 , \vec{w_1} \rangle
=
a_1 \langle \vec{w}_1 , \vec{w_1} \rangle
+
a_2 \underbrace{\langle \vec{w}_2 , \vec{w_1} \rangle}_{=0}
=
a_1 \langle \vec{w}_1 , \vec{w_1} \rangle .
\end{equation*}
Therefore,
\begin{equation*}
a_1 = 
\frac{\langle \vec{v} , \vec{w_1} \rangle}{
\langle \vec{w}_1 , \vec{w_1} \rangle} .
\end{equation*}
Similarly
\begin{equation*}
a_2 = 
\frac{\langle \vec{v} , \vec{w_2} \rangle}{
\langle \vec{w}_2 , \vec{w_2} \rangle} .
\end{equation*}
You probably remember this formula from vector calculus.

\begin{example}
Write
$\vec{v} = \left[ \begin{smallmatrix} 2 \\ 3 \end{smallmatrix} \right]$
as a linear combination of 
$\vec{w_1} = \left[ \begin{smallmatrix} 1 \\ -1 \end{smallmatrix} \right]$
and
$\vec{w_2} = \left[ \begin{smallmatrix} 1 \\ 1 \end{smallmatrix} \right]$.

First note that $\vec{w}_1$ and $\vec{w}_2$ are orthogonal
as $\langle \vec{w}_1 , \vec{w}_2 \rangle = 1(1) + (-1)1 = 0$.
Then
\begin{align*}
& a_1 = 
\frac{\langle \vec{v} , \vec{w_1} \rangle}{
\langle \vec{w}_1 , \vec{w_1} \rangle}
=
\frac{2(1) + 3(-1)}{1(1) + (-1)(-1)} = \frac{-1}{2} ,
\\
& a_2 = 
\frac{\langle \vec{v} , \vec{w_2} \rangle}{
\langle \vec{w}_2 , \vec{w_2} \rangle}
=
\frac{2 + 3}{1 + 1} = \frac{5}{2} .
\end{align*}
Hence
\begin{equation*}
\begin{bmatrix} 2 \\ 3 \end{bmatrix}
=
\frac{-1}{2}
\begin{bmatrix} 1 \\ -1 \end{bmatrix}
+
\frac{5}{2}
\begin{bmatrix} 1 \\ 1 \end{bmatrix} .
\end{equation*}
\end{example}

\subsection{The trigonometric series}

Instead of decomposing a vector in terms of eigenvectors of a matrix,
we decompose a function in terms of eigenfunctions of a certain
eigenvalue problem.  The eigenvalue problem we use for
the Fourier series is 
\begin{equation*}
x'' + \lambda x = 0, \quad x(-\pi) = x(\pi), \quad x'(-\pi) = x'(\pi) .
\end{equation*}
We computed that eigenfunctions are 1, $\cos (k t)$,
$\sin (k t)$.  That is, we want to find a representation of a
$2\pi$-periodic function $f(t)$ as
\begin{equation*}
\mybxbg{~~
f(t) = \frac{a_0}{2} +
\sum_{n=1}^\infty a_n \cos (n t) + b_n \sin (n t) .
~~}
\end{equation*}
This series is called the \emph{\myindex{Fourier series}}%
\footnote{Named after the French mathematician
\href{https://en.wikipedia.org/wiki/Joseph_Fourier}{Jean Baptiste Joseph Fourier}
(1768--1830).} or the
\emph{\myindex{trigonometric series}} for $f(t)$.
We write the coefficient of the eigenfunction 1 as $\frac{a_0}{2}$
for convenience.
We could also think of $1 = \cos (0t)$, so that
we only need to look at $\cos (kt)$ and $\sin (kt)$.

As for matrices we want to find a \emph{\myindex{projection}}
of $f(t)$ onto the subspaces given by the eigenfunctions.  So we want to
define an \emph{\myindex{inner product of functions}}.  For example, to
find $a_n$ we want to compute $\langle \, f(t) \, , \, \cos (nt) \, \rangle$.
We define the inner product as
\begin{equation*}
\langle \, f(t)\, , \, g(t) \, \rangle \overset{\text{def}}{=}
\int_{-\pi}^\pi f(t) \, g(t) \, dt .
\end{equation*}
With this definition of the inner product,
we saw in the previous section that the eigenfunctions $\cos (kt)$
(including the constant eigenfunction), and
$\sin (kt)$ are \emph{orthogonal\index{orthogonal!functions}} in the sense
that
\begin{align*}
\langle \, \cos (mt)\, , \, \cos (nt) \, \rangle = 0 & \qquad \text{for } m \not= n , \\
\langle \, \sin (mt)\, , \, \sin (nt) \, \rangle = 0 & \qquad \text{for } m \not= n , \\
\langle \, \sin (mt)\, , \, \cos (nt) \, \rangle = 0 & \qquad \text{for all } m \text{ and } n .
\end{align*}
For $n=1,2,3,\ldots$
we have
\begin{align*}
\langle \, \cos (nt) \, , \, \cos (nt) \, \rangle &=
\int_{-\pi}^\pi \cos(nt)\cos(nt) \, dt
=
\pi,
\\
\langle \, \sin (nt) \, , \, \sin (nt) \, \rangle &=
\int_{-\pi}^\pi \sin(nt)\sin(nt) \, dt
=
\pi,
\end{align*}
by elementary calculus.  For the constant we get
\begin{equation*}
\langle \, 1 \, , \, 1 \, \rangle
=
\int_{-\pi}^\pi 1 \cdot 1 \, dt
 = 2\pi.
\end{equation*}
The coefficients are given by
\begin{equation*}
\mybxbg{~~
\begin{aligned}
& a_n =
\frac{\langle \, f(t) \, , \, \cos (nt) \, \rangle}{\langle \, \cos (nt) \, , \,
\cos (nt) \, \rangle}
= 
\frac{1}{\pi} \int_{-\pi}^\pi f(t) \cos (nt) \, dt , \\
& b_n =
\frac{\langle \, f(t) \, , \, \sin (nt) \, \rangle}{\langle \, \sin (nt) \, , \,
\sin (nt) \, \rangle}
= 
\frac{1}{\pi} \int_{-\pi}^\pi f(t) \sin (nt) \, dt .
\end{aligned}
~~}
\end{equation*}
Compare these expressions with the finite-dimensional example.
For $a_0$ we get a similar formula
\begin{equation*}
\mybxbg{~~
a_0 = 2
\frac{\langle \, f(t) \, , \, 1 \, \rangle}{\langle \, 1 \, , \,
1 \, \rangle}
=
\frac{1}{\pi} \int_{-\pi}^\pi f(t) \, dt .
~~}
\end{equation*}

Let us check the formulas using the orthogonality properties.  Suppose for
a moment that
\begin{equation*}
f(t) = \frac{a_0}{2} + \sum_{n=1}^\infty a_n \cos (n t) + b_n
\sin (n t) .
\end{equation*}
Then for $m \geq 1$ we have
\begin{equation*}
\begin{split}
\langle \, f(t)\,,\,\cos (mt) \, \rangle
& =
\Bigl\langle \, \frac{a_0}{2} + \sum_{n=1}^\infty a_n \cos (n t) + b_n
\sin (n t) \,,\, \cos (mt) \, \Bigr\rangle \\
& =
\frac{a_0}{2}
\langle \, 1 \, , \, \cos (mt) \, \rangle
+ \sum_{n=1}^\infty
a_n \langle \, \cos (nt) \, , \, \cos (mt) \, \rangle +
b_n \langle \, \sin (n t) \, , \, \cos (mt) \, \rangle \\
& =
a_m \langle \, \cos (mt) \, , \, \cos (mt) \, \rangle .
\end{split}
\end{equation*}
And hence
$a_m =
\frac{\langle \, f(t) \, , \, \cos (mt) \, \rangle}{\langle \, \cos (mt) \, , \,
\cos (mt) \, \rangle}$.

\begin{exercise}
Carry out the calculation for $a_0$ and $b_m$.
\end{exercise}

\begin{example}
Take the function
\begin{equation*}
f(t) = t
\end{equation*}
for $t$ in $(-\pi,\pi]$.  Extend $f(t)$ periodically and write it 
as a Fourier series.  This function is called the \emph{\myindex{sawtooth}}.

\begin{myfig}
\capstart
\diffyincludegraphics{width=3in}{width=4.5in}{ts-sawtooth}
\caption{The graph of the sawtooth function.\label{ts:sawtoothfig}}
\end{myfig}
\end{example}

\begin{exampleSol}
The plot of the extended periodic function is given in
\figurevref{ts:sawtoothfig}.
Let us compute the coefficients.  We start with $a_0$,
\begin{equation*}
a_0 = \frac{1}{\pi} \int_{-\pi}^\pi t \,dt = 0 .
\end{equation*}
We will often use the result from calculus that says that the integral of an odd
function over a symmetric interval is zero.  Recall that an
\emph{\myindex{odd function}} is a
function $\varphi(t)$ such that $\varphi(-t) = -\varphi(t)$.  For example
the functions $t$, $\sin t$, or (importantly for us)
$t \cos (nt)$ are all odd functions.  Thus
\begin{equation*}
a_n = \frac{1}{\pi} \int_{-\pi}^\pi t \cos (nt) \,dt = 0 .
\end{equation*}
Let us move to $b_n$.  Another useful fact from calculus
is that the integral of an even function over
a symmetric interval is
twice the integral of the same function over half the interval.  
Recall an \emph{\myindex{even function}}
is a
function $\varphi(t)$ such that $\varphi(-t) = \varphi(t)$.  For example
$t \sin (nt)$ is even.
\begin{equation*}
\begin{split}
b_n & = \frac{1}{\pi} \int_{-\pi}^\pi t \sin (nt) \,dt \\
& = \frac{2}{\pi} \int_{0}^\pi t \sin (nt) \,dt \\
& = \frac{2}{\pi} \left(
\left[ \frac{-t \cos (nt)}{n} \right]_{t=0}^{\pi}
+
\frac{1}{n}
\int_{0}^\pi \cos (nt) \,dt
\right)
\\
& = \frac{2}{\pi} \left(
\frac{-\pi \cos (n\pi)}{n}
+
0
\right) \\
& =  \frac{-2 \cos (n\pi)}{n}
=  \frac{2 \,{(-1)}^{n+1}}{n} .
\end{split}
\end{equation*}
We have used the fact that 
\begin{equation*}
\cos (n\pi) = {(-1)}^n =
\begin{cases}
1 & \text{if } n \text{ even} , \\
-1 & \text{if } n \text{ odd} .
\end{cases}
\end{equation*}
The series, therefore, is
\begin{equation*}
\sum_{n=1}^\infty
\frac{2 \,{(-1)}^{n+1}}{n} \,
\sin (n t) .
\end{equation*}

Let us write out the first 3 harmonics of the series for $f(t)$.
\begin{equation*}
2 \, \sin (t)
- \sin (2t)
+\frac{2}{3} \sin (3t)
+ \cdots
\end{equation*}
The plot of these first three terms of the series, along with a plot
of the first 20 terms is given in
\figurevref{ts:sawtoothfsfig}.

\begin{myfig}
\capstart
%original files ts-sawtooth-fs3 ts-sawtooth-fs20
\diffyincludegraphics{width=6.24in}{width=9in}{ts-sawtooth-fs3-fs20}
\caption{First 3 (left graph) and 20 (right graph) harmonics of the sawtooth
function.\label{ts:sawtoothfsfig}}
\end{myfig}
\end{exampleSol}

\begin{example}
Take the function
\begin{equation*}
f(t) =
\begin{cases}
0 & \text{if } \;{-\pi} < t \leq 0 , \\
\pi & \text{if } \;\phantom{-}0 < t \leq \pi .
\end{cases}
\end{equation*}
\nopagebreak[4]%
Extend $f(t)$ periodically and write it 
as a Fourier series.  This function or its variants appear often
in applications and the function is called the
\emph{\myindex{square wave}}.

\begin{myfig}
\capstart
\diffyincludegraphics{width=3in}{width=4.5in}{ts-squarewave}
\caption{The graph of the square wave function.\label{ts:squarewavefig}}
\end{myfig}
\end{example}

\begin{exampleSol}
The plot of the extended periodic function is given in
\figurevref{ts:squarewavefig}.
Now we compute the coefficients.  Let us start with $a_0$
\begin{equation*}
a_0 = \frac{1}{\pi} \int_{-\pi}^\pi f(t) \,dt
= \frac{1}{\pi} \int_{0}^\pi \pi \,dt = \pi .
\end{equation*}
Next,
\begin{equation*}
a_n = \frac{1}{\pi} \int_{-\pi}^\pi f(t) \cos (nt) \,dt 
= \frac{1}{\pi} \int_{0}^\pi \pi \cos (nt) \,dt = 0 .
\end{equation*}
And finally
\begin{equation*}
\begin{split}
b_n & = \frac{1}{\pi} \int_{-\pi}^\pi f(t) \sin (nt) \,dt \\
& = \frac{1}{\pi} \int_{0}^\pi \pi \sin (nt) \,dt \\
& = \left[ \frac{- \cos (nt)}{n} \right]_{t=0}^\pi \\
& = \frac{1 - \cos (\pi n)}{n}
= \frac{1 - {(-1)}^n}{n}
=
\begin{cases}
\frac{2}{n} & \text{if } n \text{ is odd} , \\
0 & \text{if } n \text{ is even} .
\end{cases}
\end{split}
\end{equation*}
The Fourier series is
\begin{equation*}
\frac{\pi}{2} +  \sum_{\substack{n=1\\n \text{ odd}}}^\infty
\frac{2}{n} 
\sin (n t)
=
\frac{\pi}{2} + \sum_{k=1}^\infty
\frac{2}{2k-1} 
\sin \bigl( (2k-1)\, t \bigr) .
\end{equation*}

Let us write out the first 3 harmonics of the series for $f(t)$.
\begin{equation*}
\frac{\pi}{2}
+
2 \, \sin (t)
+
\frac{2}{3}  \sin (3t)
+ \cdots
\end{equation*}
The plot of these first three and also of the first 20 terms of the series
is given in
\figurevref{ts:squarewavefsfig}.

\begin{myfig}
\capstart
%original files ts-squarewave-fs3 ts-squarewave-fs20
\diffyincludegraphics{width=6.24in}{width=9in}{ts-squarewave-fs3-fs20}
\caption{First 3 (left graph) and 20 (right graph) harmonics of the
square wave function.\label{ts:squarewavefsfig}}
\end{myfig}
\end{exampleSol}

We have so far skirted the issue of convergence.  For example,
if $f(t)$ is the square wave function,
the equation
\begin{equation*}
f(t) = 
\frac{\pi}{2} + \sum_{k=1}^\infty
\frac{2}{2k-1} 
\sin \bigl( (2k-1)\, t \bigr) .
\end{equation*}
is only an equality for such $t$ where $f(t)$ is continuous.  That is,
we do not get an equality for $t=-\pi,0,\pi$ and all the other discontinuities
of $f(t)$.  It is not hard to see that when $t$ is an integer multiple of
$\pi$ (which includes all the discontinuities), then
\begin{equation*}
\frac{\pi}{2} + \sum_{k=1}^\infty
\frac{2}{2k-1} 
\sin \bigl( (2k-1)\, t \bigr) = \frac{\pi}{2} .
\end{equation*}
We redefine $f(t)$ on $[-\pi,\pi]$ as
\begin{equation*}
f(t) =
\begin{cases}
0 & \text{if } \; {-\pi} < t < 0 , \\
\pi & \text{if } \; \phantom{-}0 < t < \pi , \\
\nicefrac{\pi}{2} & \text{if } \; \phantom{-}t = -\pi, 
t = 0,\text{ or }
t = \pi,
\end{cases}
\end{equation*}
and extend periodically.
The series equals this extended $f(t)$ everywhere, including the
discontinuities.
We will generally not worry about changing the function values
at several (finitely many) points.

We will say more about convergence in the next section.  Let us however
mention briefly an effect of the discontinuity.  Let us zoom in near the
discontinuity in the square wave.  Further, let us plot the first 100
harmonics, see
\figurevref{ts:squarewavegibbsfig}.  While the
series is a very good approximation away from the discontinuities, the error
(the overshoot)
near the discontinuity at $t=\pi$ does not seem to be getting any smaller.
This behavior is known as the \emph{\myindex{Gibbs phenomenon}}.
The region where the error is large does get smaller, however, the more
terms in the series we take.

\begin{myfig}
\capstart
\diffyincludegraphics{width=3in}{width=4.5in}{ts-squarewave-gibbs}
\caption{Gibbs phenomenon in action.\label{ts:squarewavegibbsfig}}
\end{myfig}

We can think of a periodic function as a \myquote{signal} being a superposition of
many signals of pure frequency.  For example, we could think of the square
wave as a tone of certain base frequency.
This base frequency is called the
\emph{\myindex{fundamental frequency}}.
The square wave
will be
a superposition of
many different pure tones of frequencies that are multiples of the
fundamental frequency.
In music,
the higher frequencies are called the \emph{\myindex{overtones}}.
All the frequencies that appear are called the
\emph{\myindex{spectrum}} of the signal.
On the other hand a simple sine wave is only the pure tone (no overtones).  The
simplest way to make sound using a computer is the square wave, and the sound
is very different from a pure tone.  If you ever played video games
from the 1980s or so, then you heard what square waves sound like.

\subsection{Exercises}

\begin{exercise}\ansMark%
Suppose $f(t)$ is defined on $[-\pi,\pi]$ as $f(t) = \sin(t)$.  Extend
periodically and compute the Fourier series.
\end{exercise}
\exsol{%
$\sin(t)$
}

\begin{exercise}
Suppose $f(t)$ is defined on $[-\pi,\pi]$ as $\sin (5t) + \cos (3t)$.  Extend
periodically and compute the Fourier series of $f(t)$.
\end{exercise}

\begin{exercise}
Suppose $f(t)$ is defined on $[-\pi,\pi]$ as $\lvert t \rvert$.
Extend periodically and compute the Fourier series of $f(t)$.
\end{exercise}

\begin{exercise}
Suppose $f(t)$ is defined on $[-\pi,\pi]$ as $\lvert t \rvert^3$.
Extend periodically and compute the Fourier series of $f(t)$.
\end{exercise}

\begin{exercise}\ansMark%
Suppose $f(t)$ is defined on $(-\pi,\pi]$ as $f(t) = \sin(\pi t)$.  Extend
periodically and compute the Fourier series.
\end{exercise}
\exsol{%
$\sum\limits_{n=1}^\infty
\frac{(\pi-n) \sin( \pi n+{\pi}^{2})
+(\pi+n)\sin(\pi n-{\pi}^{2}) }{\pi {n}^{2}-{\pi}^{3}}
\sin(nt)$
%$a_0 = \frac{1}{\pi} \int_{-\pi}^\pi \sin(\pi t) \, dt$
%\\
%$a_n =
%\frac{1}{\pi} \int_{-\pi}^\pi \sin(\pi t) \cos (nt) \, dt$
%\\
%$b_n =
%\frac{1}{\pi} \int_{-\pi}^\pi f(t) \sin (nt) \, dt$
}

\begin{exercise}
Suppose $f(t)$ is defined on $(-\pi,\pi]$ as
\begin{equation*}
f(t) =
\begin{cases}
-1 & \text{if } \; {-\pi} < t \leq 0 , \\
1 & \text{if } \; \phantom{-}0 < t \leq \pi .
\end{cases}
\end{equation*}
Extend periodically and compute the Fourier series of $f(t)$.
\end{exercise}

\begin{exercise}
Suppose $f(t)$ is defined on $(-\pi,\pi]$ as $t^3$.
Extend periodically and compute the Fourier series of $f(t)$.
\end{exercise}

\begin{exercise}\ansMark%
Suppose $f(t)$ is defined on $(-\pi,\pi]$ as $f(t) = \sin^2(t)$.
Extend
periodically and compute the Fourier series.
\end{exercise}
\exsol{%
$\frac{1}{2}-\frac{1}{2}\cos(2t)$
}

\begin{exercise}
Suppose $f(t)$ is defined on $[-\pi,\pi]$ as $t^2$.
Extend periodically and compute the Fourier series of $f(t)$.
\end{exercise}

\begin{exercise}\ansMark%
Suppose $f(t)$ is defined on $(-\pi,\pi]$ as $f(t) = t^4$.
Extend periodically and compute the Fourier series.
\end{exercise}
\exsol{%
$\frac{\pi^4}{5} + \sum\limits_{n=1}^\infty
\frac{{(-1)}^{n} (8{\pi}^{2}{n}^{2}-48) }{{n}^{4}}
\cos(nt)$
}

There is another form of the Fourier series using complex exponentials
$e^{nt}$ for $n=\ldots,-2,-1,0,1,2,\ldots$ instead of
$\cos(nt)$ and $\sin(nt)$ for positive $n$.  This form may be easier
to work with sometimes.  It is certainly more compact to write,
and there is only one formula for the coefficients.  On the downside,
the coefficients are complex numbers.

\begin{exercise}
\begin{samepage}
Let 
\begin{equation*}
f(t) = \frac{a_0}{2} + \sum_{n=1}^\infty a_n \cos (n t)
+ b_n \sin (n t) .
\end{equation*}
Use Euler's formula $e^{i\theta} = \cos (\theta) + i \sin (\theta)$ to
show that there exist complex numbers $c_m$ such that
\begin{equation*}
f(t) = 
\sum_{m=-\infty}^\infty c_m e^{imt} .
\end{equation*}
Note that the sum now ranges over all the integers including negative ones.
Do not worry about convergence in this calculation.
Hint: It may be better to start from the complex exponential form and write
the series as
\begin{equation*}
c_0 + \sum_{m=1}^\infty \Bigl( c_m e^{imt} + c_{-m} e^{-imt}  \Bigr).
\end{equation*}
\end{samepage}
\end{exercise}

\setcounter{exercise}{100}

