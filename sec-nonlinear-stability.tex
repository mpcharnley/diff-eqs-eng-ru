\section{Stability and classification of isolated critical points}
\label{nlinstability:section}

\LAtt{8.2}

\LO{
\item Determine if a critical point of a non-linear system is isolated,
\item Use the Jacobian matrix to classify the critical point of a non-linear system, and
\item Determine the stability of a critical point from the classification.
}

\subsection{Isolated critical points and almost linear systems}
The next step in this process is to try to figure out a way to analyze what is happening to a non-linear system of differential equations near equilibrium solutions \emph{without} using a slope field/phase portrait. We would like to be able to determine this from the equations alone, not any of the pictures that come from them. Thankfully, our ability to analyze linear systems helps us accomplish this goal. 

\begin{definition}
A critical point is
\emph{isolated}\index{isolated critical point}
if it is the only critical point in some small
\myquote{neighborhood} of the point. 
\end{definition}
That is, if we zoom in far enough it is the
only critical point we see.  In the example above, the critical point was
isolated.  If on the other hand there would be a whole curve of critical
points, then it would not be isolated. For example, the system
\begin{equation*}
x' = y(x-1) \qquad y' = (x-2)(x-1)
\end{equation*}
has the entire line $x=1$ as critical points. Therefore, these are not isolated.

\begin{definition}
A system is called \emph{\myindex{almost linear}} at a critical point
$(x_0,y_0)$, if the critical point is isolated and the Jacobian matrix at the point
is invertible, or equivalently if the linearized system has an isolated
critical point.
\end{definition}
This is also equivalent to zero not being an eigenvalue of the Jacobian matrix at the critical point.
In such a case, the nonlinear terms are very small
and the system behaves like its linearization, at least if we are close
to the critical point.

For example, the system in
Examples~\ref{example:nlin-1b-example} and \ref{example:nlin-1b-examplelin}
has two isolated critical points $(0,0)$ and $(0,1)$, and
is almost linear at both critical points as 
the Jacobian matrices at both points,
$\left[ \begin{smallmatrix} 0 & 1 \\ -1 & 0 \end{smallmatrix} \right]$ and
$\left[ \begin{smallmatrix} 0 & 1 \\ 1 & 0 \end{smallmatrix} \right]$,
are invertible.

On the other hand, the system $x' = x^2$, $y' = y^2$ has an isolated
critical point at $(0,0)$, however the Jacobian matrix
\begin{equation*}
\begin{bmatrix} 2x & 0 \\ 0 & 2y \end{bmatrix}
\end{equation*}
is zero when $(x,y) = (0,0)$.  So the system is not almost
linear. 
Even a worse example is the system $x' = x$, $y' = x^2$, which does not have 
isolated critical points; $x'$ and $y'$ are both zero
whenever $x=0$, that is, the entire $y$-axis.

Fortunately, most often critical
points are isolated, and the system is almost linear at the critical
points.  So if we learn what happens there, we will have figured out the majority
of situations that arise in applications.



\subsection{Stability and classification of isolated critical points}

Once we have an isolated critical point, the system is almost linear at
that critical point, and we computed the
associated linearized system, we can classify what happens to the 
solutions.  The classifications for linear
two-variable systems from \sectionref{sec:twodimaut} are generally the same as what we use here, with one minor
caveat.
Let us list the behaviors depending on the eigenvalues of
the Jacobian matrix at the critical point in \tablevref{pln:behtab2}.
This table is very similar to \tablevref{pln:behtab}, with
the exception of missing \myquote{center} points.
The repeated eigenvalue cases are also missing. They behave similarly to the real eigenvalue descriptions
in the table below, but similar to centers, the behavior can change slightly.
It can behave like either a spiral or a node, but will be either a source or sink based on the sign of the repeated eigenvalue. 
%There is also a new column
%that we will discuss.
We will discuss centers later, as they are more complicated.

\begin{table}[h!t]
\mybeginframe
\capstart
\begin{center}
\begin{tabular}{@{}lll@{}}
\toprule
Eigenvalues of the Jacobian matrix & Behavior & Stability \\
\midrule
real and both positive & source / unstable node & unstable \\
real and both negative & sink / stable node & asymptotically stable \\
real and opposite signs & saddle & unstable \\
complex with positive real part & spiral source & unstable \\
complex with negative real part & spiral sink & asymptotically stable \\
\bottomrule
\end{tabular}
\end{center}
\caption{Behavior of an almost linear system near an isolated critical
point.  \label{pln:behtab2}}
\myendframe
\end{table}

In the third column,
we mark points as \emph{asymptotically stable} or \emph{unstable}.  
\begin{definition}
Let $(x_0, y_0)$ be a critical point for a non-linear system of two differential equations.
\begin{enumerate}[1.]
\item We say that the critical point is a \emph{\myindex{stable critical point}} if, given any small distance $\epsilon$ to
$(x_0,y_0)$, and any initial condition within a perhaps smaller radius
around $(x_0,y_0)$, the trajectory
of the system never goes further away from $(x_0,y_0)$ than $\epsilon$.
\item The critical point is an \emph{\myindex{unstable critical point}} if it is not stable; that is, there are trajectories that start within a distance $\epsilon$ of $(x_0, y_0)$ and end up farther than $\epsilon$ from that point.
\item The critical point is called \emph{\myindex{asymptotically stable}} if
given any initial condition sufficiently close to $(x_0,y_0)$ and any
solution $\bigl( x(t), y(t) \bigr)$ satisfying that condition, then
\begin{equation*}
\lim_{t \to \infty} \bigl( x(t), y(t) \bigr) = (x_0,y_0) .
\end{equation*}
\end{definition}

Informally, a point is stable if we start close to a critical point and
follow a trajectory we either go towards, or at least not away
from, this critical point. If the point is asymptotically stable, then 
any trajectory for a sufficiently close initial condition
goes towards the critical point $(x_0,y_0)$, and unstable means that, in general, trajectories move away from the critical point. 

\begin{example} \label{example:nlin-xplusy}
Find and analyze the critical points of 
$x'=-y-x^2$,
$y'=-x+y^2$.
\end{example}

\begin{exampleSol}
See \figurevref{fig:nlin-ex813-new} for the phase diagram.
Let us find the critical points.  These are the points where
$-y-x^2 = 0$ and $-x+y^2=0$.  The first equation means $y = -x^2$, and
so $y^2 = x^4$.  Plugging into the second equation we obtain 
$-x+x^4 = 0$.  Factoring we obtain $x(1-x^3)=0$.  Since we are looking only
for real solutions we get either $x=0$ or $x=1$.  Solving for the
corresponding $y$ using $y = -x^2$, we get two critical points, one being $(0,0)$
and the other being $(1,-1)$.  Clearly the critical points are isolated.

\begin{myfig}
\capstart
\diffyincludegraphics{width=3in}{width=4.5in}{nlin-ex813-new}
\caption{The phase portrait with few sample trajectories of 
$x'=-y-x^2$, $y'=-x+y^2$.  \label{fig:nlin-ex813-new}}
\end{myfig}


Let us compute the Jacobian matrix:
\begin{equation*}
\begin{bmatrix}
-2x & -1 \\
-1 & 2y
\end{bmatrix} .
\end{equation*}
At the point $(0,0)$ we get the matrix
$\left[ \begin{smallmatrix} 0 & -1 \\ -1 & 0 \end{smallmatrix} \right]$ and
so the two eigenvalues are $1$ and $-1$.  As the matrix is invertible, the system is almost linear
at $(0,0)$.  As the eigenvalues are real
and of opposite signs, we get a saddle point, which is an unstable
equilibrium point. Looking at the phase portrait, we can see trajectories that would start near $(0,0)$ and end up farther away from $(0,0)$. These trajectories may end up at $(1,-1)$, but that is away from $(0,0)$. 

At the point $(1,-1)$ we get the matrix
$\left[ \begin{smallmatrix} -2 & -1 \\ -1 & -2 \end{smallmatrix} \right]$ and
computing the eigenvalues we get $-1$, $-3$.
The matrix is invertible, and so the system is almost linear at $(1,-1)$.
As we have real eigenvalues and both negative, the critical
point is a sink, and therefore an asymptotically stable equilibrium point.
That is, if we start with any point $(x(0),y(0))$ close to $(1,-1)$ as
an initial condition and plot a trajectory, it approaches $(1,-1)$.
In other words,
\begin{equation*}
\lim_{t \to \infty} \bigl( x(t), y(t) \bigr) = (1,-1) .
\end{equation*}
As you can 
see from the diagram, this behavior is true even for some
initial points quite far from $(1,-1)$, but it is definitely not true for all
initial points.
\end{exampleSol}

\begin{example} \label{example:nlin-withexp}
Find and analyze the critical points of 
$x'=y+y^2e^x$,
$y'=x$.
\end{example}

\begin{exampleSol}
First let us find the critical points.  These are the points where
$y+y^2e^x = 0$ and $x=0$.  Simplifying we get $0=y+y^2 = y(y+1)$.  So the
critical points are $(0,0)$ and $(0,-1)$, and hence are isolated.  Let us
compute the Jacobian matrix:
\begin{equation*}
\begin{bmatrix}
y^2e^x & 1+2ye^x \\
1 & 0
\end{bmatrix}.
\end{equation*}

At the point $(0,0)$ we get the matrix
$\left[ \begin{smallmatrix} 0 & 1 \\ 1 & 0 \end{smallmatrix} \right]$ and
so the two eigenvalues are $1$ and $-1$.  As the matrix is invertible, the system is almost linear
at $(0,0)$.  And, as the eigenvalues are real
and of opposite signs, we get a saddle point, which is an unstable
equilibrium point.

At the point $(0,-1)$ we get the matrix
$\left[ \begin{smallmatrix} 1 & -1 \\ 1 & 0 \end{smallmatrix} \right]$ whose
eigenvalues are $\frac{1}{2} \pm i \frac{\sqrt{3}}{2}$.
The matrix is invertible, and so the system is almost linear at $(0,-1)$.
As we have complex eigenvalues with positive real part, the critical
point is a spiral source, and therefore an unstable equilibrium point.

\begin{myfig}
\capstart
\diffyincludegraphics{width=3in}{width=4.5in}{nlin-ex813}
\caption{The phase portrait with few sample trajectories of 
$x'=y+y^2e^x$, $y'=x$.  \label{fig:nlin-ex813}}
\end{myfig}

See \figurevref{fig:nlin-ex813} for the phase diagram.  Notice the two
critical points, and the behavior of the arrows in the vector field around
these points.
\end{exampleSol}

\subsection{The trouble with centers}

Recall, a linear system with a center means that trajectories
travel in closed elliptical orbits
in some direction around the critical point.  Such
a critical point we call a \emph{\myindex{center}} or
a \emph{\myindex{stable center}}.  It is not an asymptotically 
stable critical point, as the trajectories never approach the critical
point, but at least if you start sufficiently close to the critical point,
you stay close to the critical point.  The simplest example of such
behavior is the linear system with a center.  Another
example is the critical point $(0,0)$ in
\examplevref{example:nlin-1b-example}.

The trouble with a center in a nonlinear system is that whether the
trajectory goes towards or away from the critical point is governed by the
sign of the real part of the eigenvalues of the Jacobian matrix, and the Jacobian
matrix
in a nonlinear system changes from point to point.  Since this real
part is zero at the critical point itself, it can have either sign nearby,
meaning the trajectory could be pulled towards or away from the critical
point.

\begin{example}
Find and analyze the critical point(s) of 
$x'=y, y' = -x+y^3$.  
\end{example}

\begin{exampleSol}
The only critical point
is the origin $(0,0)$.  The Jacobian matrix is 
\begin{equation*}
\begin{bmatrix}
0 & 1 \\
-1 & 3 y^2 \\
\end{bmatrix} .
\end{equation*}
At 
$(0,0)$ the Jacobian matrix is
$\left[ \begin{smallmatrix}
0 & 1 \\
-1 & 0 \\
\end{smallmatrix} \right]$, which has eigenvalues $\pm i$.  So the
linearization has a center.

Using the quadratic equation, the eigenvalues of the
Jacobian matrix at any point $(x,y)$ are
\begin{equation*}
\lambda = 
\frac{3}{2}y^2 \pm
i
\frac{\sqrt{4-9y^4}}{2} .
\end{equation*}
At any point where $y \not= 0$ (so at most points near the origin), the eigenvalues have a positive real part ($y^2$ can
never be negative).  This positive real part 
pulls the trajectory away from the origin.  A sample trajectory for an
initial condition near the origin is given in
\figurevref{fig:nlin-unstable-center}.
\begin{myfig}
\capstart
\diffyincludegraphics{width=3in}{width=4.5in}{nlin-unstable-centerfig}
\caption{An unstable critical point (spiral source) at the origin
for $x'=y, y' = -x+y^3$, even if the linearization has a center.  \label{fig:nlin-unstable-center}}
\end{myfig}
\end{exampleSol}

The same process could be carried out with the system $x'=y, y' = -x-y^3$. This one will also have a center as the linearization at the origin, but the non-linear system will have a spiral sink at the origin. The moral of the example is that further analysis is needed when the
linearization has a center.  The analysis will in general be more
complicated than in the example above, and is more likely to involve
case-by-case consideration.  Such a complication should not be
surprising to you.  By now in your mathematical career, you have
seen many places where a simple test is inconclusive, recall for example
the second derivative test for maxima or minima, and requires more careful,
and perhaps ad hoc analysis of the situation.


\subsection{Exercises}

\begin{exercise}
For the systems below, find and classify the critical points, also indicate
if the equilibria are stable, asymptotically stable, or unstable.
\begin{tasks}(2)
\task $x'=-x+3x^2, y'=-y$
\task $x'=x^2+y^2-1$, $y'=x$
\task $x'=ye^x$, $y'=y-x+y^2$
\end{tasks}
\end{exercise}

\begin{exercise}\ansMark%
For the systems below, find and classify the critical points.
\begin{tasks}(3)
\task $x'=-x+x^2$, $y'=y$
\task $x'=y-y^2-x$, $y'=-x$
\task $x'=xy$, $y'=x+y-1$
\end{tasks}
\end{exercise}
\exsol{%
a) $(0,0)$: saddle (unstable), $(1,0)$: source (unstable), \qquad
b) $(0,0)$: spiral sink (asymptotically stable), $(0,1)$: saddle (unstable), \qquad
c) $(1,0)$: saddle (unstable), $(0,1)$: saddle (unstable)
}

\begin{exercise}
Find and classify all critical points of the system
\[ \frac{dx}{dt} = (x+1)(x-y+3) \qquad \frac{dy}{dt} = (x-2)(x-y) .\]
\end{exercise}

\begin{exercise}
Find and classify all critical points of the system
\[ \frac{dx}{dt} = x^2 - y^2 \qquad \frac{dy}{dt} = (x+4)(y-2) .\]
\end{exercise}


\begin{exercise}
Find and classify the critical point(s) of $x' = -x^2$, $y' = -y^2$.
\end{exercise}

\begin{samepage}
\begin{exercise}
Suppose $x'=-xy$, $y'=x^2-1-y$.
\begin{tasks}
\task
Show there are two spiral sinks at
$(-1,0)$ and $(1,0)$.
\task
For any initial point of the form $(0,y_0)$, find what is the trajectory.
\task
Can a trajectory starting at $(x_0,y_0)$ where $x_0 > 0$ spiral into 
the critical point at $(-1,0)$?  Why or why not?
\end{tasks}
\end{exercise}
\end{samepage}

\begin{exercise} \label{exercise:increasing}
In the example $x'=y$, $y'=y^3-x$ show that for any trajectory, the distance
from the origin is an increasing function.
Conclude
that the origin behaves like is a spiral source.
Hint: Consider $f(t) =
{\bigl(x(t)\bigr)}^2 + 
{\bigl(y(t)\bigr)}^2$ and show it has positive derivative.
\end{exercise}

\begin{exercise}\ansMark%
Derive an analogous classification of critical points for equations in one dimension,
such as $x'= f(x)$ based on the derivative.  A point $x_0$ is critical when $f(x_0) = 0$ and
almost linear if in addition $f'(x_0) \not= 0$.  Figure out if the critical point is stable or unstable
depending on the sign of $f'(x_0)$.  Explain.  Hint: see \sectionref{auteq:section}.
\end{exercise}
\exsol{%
A critical point $x_0$ is stable if $f'(x_0) < 0$ and unstable when $f'(x_0)
> 0$.
}


\setcounter{exercise}{100}
