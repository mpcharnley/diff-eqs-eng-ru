\section{Stability and classification of isolated critical points}
\label{nlinstability:section}

\LAtt{8.2}

\subsection{Isolated critical points and almost linear systems}

A critical point is
\emph{isolated}\index{isolated critical point}
if it is the only critical point in some small
\myquote{neighborhood} of the point.  That is, if we zoom in far enough it is the
only critical point we see.  In the example above, the critical point was
isolated.  If on the other hand there would be a whole curve of critical
points, then it would not be isolated.

A system is called \emph{\myindex{almost linear}} at a critical point
$(x_0,y_0)$, if the critical point is isolated and the Jacobian matrix at the point
is invertible, or equivalently if the linearized system has an isolated
critical point. This is also equivalent to zero not being an eigenvalue of the Jacobian matrix at the critical point.
In such a case, the nonlinear terms are very small
and the system behaves like its linearization, at least if we are close
to the critical point.

For example, the system in
Examples~\ref{example:nlin-1b-example} and \ref{example:nlin-1b-examplelin}
has two isolated critical points $(0,0)$ and $(0,1)$, and
is almost linear at both critical points as 
the Jacobian matrices at both points,
$\left[ \begin{smallmatrix} 0 & 1 \\ -1 & 0 \end{smallmatrix} \right]$ and
$\left[ \begin{smallmatrix} 0 & 1 \\ 1 & 0 \end{smallmatrix} \right]$,
are invertible.

On the other hand, the system $x' = x^2$, $y' = y^2$ has an isolated
critical point at $(0,0)$, however the Jacobian matrix
\begin{equation*}
\begin{bmatrix} 2x & 0 \\ 0 & 2y \end{bmatrix}
\end{equation*}
is zero when $(x,y) = (0,0)$.  So the system is not almost
linear. 
Even a worse example is the system $x' = x$, $y' = x^2$, which does not have 
isolated critical points; $x'$ and $y'$ are both zero
whenever $x=0$, that is, the entire $y$-axis.

Fortunately, most often critical
points are isolated, and the system is almost linear at the critical
points.  So if we learn what happens there, we will have figured out the majority
of situations that arise in applications.



\subsection{Stability and classification of isolated critical points}

Once we have an isolated critical point, the system is almost linear at
that critical point, and we computed the
associated linearized system, we can classify what happens to the 
solutions.  We more or less use the classification for linear
two-variable systems from \sectionref{sec:twodimaut}, with one minor
caveat.
Let us list the behaviors depending on the eigenvalues of
the Jacobian matrix at the critical point in \tablevref{pln:behtab2}.
This table is very similar to \tablevref{pln:behtab}, with
the exception of missing \myquote{center} points.
The repeated eigenvalue cases are also missing. They behave similarly to the real eigenvalue descriptions
in the table below, but similar to centers, the behavior can change slightly.
It can behave like either a spiral or a node, but will be either a source or sink based on the sign of the repeated eigenvalue. 
%There is also a new column
%that we will discuss.
We will discuss centers later, as they are more complicated.

\begin{table}[h!t]
\mybeginframe
\capstart
\begin{center}
\begin{tabular}{@{}lll@{}}
\toprule
Eigenvalues of the Jacobian matrix & Behavior & Stability \\
\midrule
real and both positive & source / unstable node & unstable \\
real and both negative & sink / stable node & asymptotically stable \\
real and opposite signs & saddle & unstable \\
complex with positive real part & spiral source & unstable \\
complex with negative real part & spiral sink & asymptotically stable \\
\bottomrule
\end{tabular}
\end{center}
\caption{Behavior of an almost linear system near an isolated critical
point.  \label{pln:behtab2}}
\myendframe
\end{table}

In the third column,
we mark points as \emph{asymptotically stable} or \emph{unstable}.  Formally, a
\emph{\myindex{stable critical point}} $(x_0,y_0)$ is one where given any small distance $\epsilon$ to
$(x_0,y_0)$, and any initial condition within a perhaps smaller radius
around $(x_0,y_0)$, the trajectory
of the system never goes further away from $(x_0,y_0)$ than $\epsilon$.
An \emph{\myindex{unstable critical point}} is one that is not stable.
Informally, a point is stable if we start close to a critical point and
follow a trajectory we either go towards, or at least not away
from,
this critical point.

A stable critical point $(x_0,y_0)$ is called \emph{\myindex{asymptotically stable}} if
given any initial condition sufficiently close to $(x_0,y_0)$ and any
solution $\bigl( x(t), y(t) \bigr)$ satisfying that condition, then
\begin{equation*}
\lim_{t \to \infty} \bigl( x(t), y(t) \bigr) = (x_0,y_0) .
\end{equation*}
That is, the critical point is asymptotically stable
if any trajectory for a sufficiently close initial condition
goes towards the critical point $(x_0,y_0)$.

\begin{example} \label{example:nlin-xplusy}
Consider
$x'=-y-x^2$,
$y'=-x+y^2$.
See \figurevref{fig:nlin-ex813-new} for the phase diagram.
Let us find the critical points.  These are the points where
$-y-x^2 = 0$ and $-x+y^2=0$.  The first equation means $y = -x^2$, and
so $y^2 = x^4$.  Plugging into the second equation we obtain 
$-x+x^4 = 0$.  Factoring we obtain $x(1-x^3)=0$.  Since we are looking only
for real solutions we get either $x=0$ or $x=1$.  Solving for the
corresponding $y$ using $y = -x^2$, we get two critical points, one being $(0,0)$
and the other being $(1,-1)$.  Clearly the critical points are isolated.

\begin{myfig}
\capstart
\diffyincludegraphics{width=3in}{width=4.5in}{nlin-ex813-new}
\caption{The phase portrait with few sample trajectories of 
$x'=-y-x^2$, $y'=-x+y^2$.  \label{fig:nlin-ex813-new}}
\end{myfig}


Let us compute the Jacobian matrix:
\begin{equation*}
\begin{bmatrix}
-2x & -1 \\
-1 & 2y
\end{bmatrix} .
\end{equation*}
At the point $(0,0)$ we get the matrix
$\left[ \begin{smallmatrix} 0 & -1 \\ -1 & 0 \end{smallmatrix} \right]$ and
so the two eigenvalues are $1$ and $-1$.  As the matrix is invertible, the system is almost linear
at $(0,0)$.  As the eigenvalues are real
and of opposite signs, we get a saddle point, which is an unstable
equilibrium point.

At the point $(1,-1)$ we get the matrix
$\left[ \begin{smallmatrix} -2 & -1 \\ -1 & -2 \end{smallmatrix} \right]$ and
computing the eigenvalues we get $-1$, $-3$.
The matrix is invertible, and so the system is almost linear at $(1,-1)$.
As we have real eigenvalues and both negative, the critical
point is a sink, and therefore an asymptotically stable equilibrium point.
That is, if we start with any point $(x_i,y_i)$ close to $(1,-1)$ as
an initial condition and plot a trajectory, it approaches $(1,-1)$.
In other words,
\begin{equation*}
\lim_{t \to \infty} \bigl( x(t), y(t) \bigr) = (1,-1) .
\end{equation*}
As you can 
see from the diagram, this behavior is true even for some
initial points quite far from $(1,-1)$, but it is definitely not true for all
initial points.
\end{example}

\begin{example} \label{example:nlin-withexp}
Let us look at
$x'=y+y^2e^x$,
$y'=x$.  First let us find the critical points.  These are the points where
$y+y^2e^x = 0$ and $x=0$.  Simplifying we get $0=y+y^2 = y(y+1)$.  So the
critical points are $(0,0)$ and $(0,-1)$, and hence are isolated.  Let us
compute the Jacobian matrix:
\begin{equation*}
\begin{bmatrix}
y^2e^x & 1+2ye^x \\
1 & 0
\end{bmatrix}.
\end{equation*}

At the point $(0,0)$ we get the matrix
$\left[ \begin{smallmatrix} 0 & 1 \\ 1 & 0 \end{smallmatrix} \right]$ and
so the two eigenvalues are $1$ and $-1$.  As the matrix is invertible, the system is almost linear
at $(0,0)$.  And, as the eigenvalues are real
and of opposite signs, we get a saddle point, which is an unstable
equilibrium point.

At the point $(0,-1)$ we get the matrix
$\left[ \begin{smallmatrix} 1 & -1 \\ 1 & 0 \end{smallmatrix} \right]$ whose
eigenvalues are $\frac{1}{2} \pm i \frac{\sqrt{3}}{2}$.
The matrix is invertible, and so the system is almost linear at $(0,-1)$.
As we have complex eigenvalues with positive real part, the critical
point is a spiral source, and therefore an unstable equilibrium point.

\begin{myfig}
\capstart
\diffyincludegraphics{width=3in}{width=4.5in}{nlin-ex813}
\caption{The phase portrait with few sample trajectories of 
$x'=y+y^2e^x$, $y'=x$.  \label{fig:nlin-ex813}}
\end{myfig}

See \figurevref{fig:nlin-ex813} for the phase diagram.  Notice the two
critical points, and the behavior of the arrows in the vector field around
these points.
\end{example}

\subsection{The trouble with centers}

Recall, a linear system with a center means that trajectories
travel in closed elliptical orbits
in some direction around the critical point.  Such
a critical point we call a \emph{\myindex{center}} or
a \emph{\myindex{stable center}}.  It is not an asymptotically 
stable critical point, as the trajectories never approach the critical
point, but at least if you start sufficiently close to the critical point,
you stay close to the critical point.  The simplest example of such
behavior is the linear system with a center.  Another
example is the critical point $(0,0)$ in
\examplevref{example:nlin-1b-example}.

The trouble with a center in a nonlinear system is that whether the
trajectory goes towards or away from the critical point is governed by the
sign of the real part of the eigenvalues of the Jacobian matrix, and the Jacobian
matrix
in a nonlinear system changes from point to point.  Since this real
part is zero at the critical point itself, it can have either sign nearby,
meaning the trajectory could be pulled towards or away from the critical
point.

\begin{example}
An example of such a problematic behavior is the system
$x'=y, y' = -x+y^3$.  The only critical point
is the origin $(0,0)$.  The Jacobian matrix is 
\begin{equation*}
\begin{bmatrix}
0 & 1 \\
-1 & 3 y^2 \\
\end{bmatrix} .
\end{equation*}
At 
$(0,0)$ the Jacobian matrix is
$\left[ \begin{smallmatrix}
0 & 1 \\
-1 & 0 \\
\end{smallmatrix} \right]$, which has eigenvalues $\pm i$.  So the
linearization has a center.

Using the quadratic equation, the eigenvalues of the
Jacobian matrix at any point $(x,y)$ are
\begin{equation*}
\lambda = 
\frac{3}{2}y^2 \pm
i
\frac{\sqrt{4-9y^4}}{2} .
\end{equation*}
At any point where $y \not= 0$ (so at most points near the origin), the eigenvalues have a positive real part ($y^2$ can
never be negative).  This positive real part 
pulls the trajectory away from the origin.  A sample trajectory for an
initial condition near the origin is given in
\figurevref{fig:nlin-unstable-center}.
\begin{myfig}
\capstart
\diffyincludegraphics{width=3in}{width=4.5in}{nlin-unstable-centerfig}
\caption{An unstable critical point (spiral source) at the origin
for $x'=y, y' = -x+y^3$, even if the linearization has a center.  \label{fig:nlin-unstable-center}}
\end{myfig}
\end{example}

The same process could be carried out with the system $x'=y, y' = -x-y^3$. This one will also have a center as the linearization at the origin, but the non-linear system will have a spiral sink at the origin. The moral of the example is that further analysis is needed when the
linearization has a center.  The analysis will in general be more
complicated than in the example above, and is more likely to involve
case-by-case consideration.  Such a complication should not be
surprising to you.  By now in your mathematical career, you have
seen many places where a simple test is inconclusive, recall for example
the second derivative test for maxima or minima, and requires more careful,
and perhaps ad hoc analysis of the situation.

\subsection{Conservative equations}

An equation of the form
\begin{equation*}
x'' + f(x) = 0
\end{equation*}
for an arbitrary function $f(x)$ is called a
\emph{\myindex{conservative equation}}.  For example the pendulum equation
is a conservative equation.  The equations are conservative as there is no
friction in the system so the energy in the system is \myquote{conserved.}
Let us write this equation as a
system of nonlinear ODE.
\begin{equation*}
x' = y, \qquad y' = -f(x) .
\end{equation*}
These types of equations have the
advantage that we can solve for their trajectories easily.

The trick is to first think of $y$ as a function of $x$ for a moment.  Then
use the chain rule
\begin{equation*}
x'' = y' = \frac{dy}{dx} x' = y \frac{dy}{dx} ,
\end{equation*}
where the prime indicates a derivative with respect to $t$.  
We obtain $y \frac{dy}{dx} + f(x) = 0$.  We integrate with respect to
$x$ to get
$\int y \frac{dy}{dx} \,dx + \int f(x)\, dx = C$.  In other words
\begin{equation*}
\frac{1}{2} y^2  + \int f(x)\, dx = C .
\end{equation*}
We obtained an implicit equation for the trajectories, with different $C$
giving different trajectories.  The value of
$C$ is conserved on any trajectory.  This expression is
sometimes called the \emph{\myindex{Hamiltonian}} or the energy of the
system.
If you look back to \sectionref{exact:section}, you will notice
that $y\frac{dy}{dx} + f(x) = 0$ is an exact equation, and
we just found a potential function.

\begin{example}
Let us find the trajectories for the equation $x'' + x-x^2 = 0$,
which is the equation from
\examplevref{example:nlin-1b-example}.  The corresponding
first order system is
\begin{equation*}
x' = y , \qquad y' = -x+x^2 .
\end{equation*}
Trajectories satisfy
\begin{equation*}
\frac{1}{2} y^2  + \frac{1}{2} x^2 - \frac{1}{3} x^3  = C .
\end{equation*}
We solve for $y$
\begin{equation*}
y = \pm \sqrt{-x^2 + \frac{2}{3} x^3  + 2C} .
\end{equation*}

Plotting these graphs we get exactly the trajectories in 
\figurevref{fig:nlin-1b}.  In particular we notice that near the origin
the trajectories are \emph{\myindex{closed curves}}: they keep going
around the origin, never spiraling in or out.  Therefore we discovered a way
to verify that the critical point at $(0,0)$ is a stable center.
The critical point at $(0,1)$ is a saddle as we already noticed.
This example is typical for conservative equations.
\end{example}

Consider an arbitrary
conservative equation $x'' + f(x) = 0$.
All critical points occur when $y=0$ (the
$x$-axis), that is when $x' = 0$.  The critical points are 
those points on the $x$-axis where $f(x) = 0$.
The trajectories are given by
\begin{equation*}
y = \pm \sqrt{ - 2 \int f(x)\, dx + 2C} .
\end{equation*}
So all trajectories are mirrored across the $x$-axis.  In particular,
there can be no spiral sources nor sinks.
The Jacobian matrix is
\begin{equation*}
\begin{bmatrix}
0 & 1 \\
-f'(x) & 0
\end{bmatrix} .
\end{equation*}
The critical point is almost linear if $f'(x) \not= 0$ at the critical 
point.  Let $J$ denote the Jacobian matrix.
The eigenvalues of $J$ are solutions to
\begin{equation*}
0 = \det(J - \lambda I) = \lambda^2 + f'(x) .
\end{equation*}
Therefore $\lambda = \pm \sqrt{-f'(x)}$.  In other words, either we get
real eigenvalues of opposite signs (if $f'(x) < 0$),
or we get purely imaginary eigenvalues (if $f'(x) > 0$).
There are only two possibilities for critical points, either an \emph{unstable
saddle point}, or a \emph{stable center}.
There are never any sinks or sources.

\subsection{Hamiltonian Systems}

A generalization of conservative equations to systems is a \myindex{Hamiltonian} system. This type of system has all of the nice properties of conservative equations when converted into systems, but allows for more general interactions between $x$ and $y$. For these systems, the point is that the solution has a conserved quantity called a Hamiltonian, which does not change as the system evolves in time, which generally represents the energy of the system. Calling this function $H(x,y)$, this means that
\[ \frac{d}{dt}H(x,y) = 0. \] By the chain rule, this is equivalent to
\[ \frac{\partial H}{\partial x} \frac{dx}{dt} + \frac{\partial H}{\partial y} \frac{dy}{dt} = 0.\]

One way to satisfy this is with 
\begin{equation}
\begin{split}
\frac{dx}{dt} &= -\frac{\partial H}{\partial y} \\
\frac{dy}{dt} &= \frac{\partial H}{\partial x}
\end{split}
\label{eq:HamilDef}
\end{equation} 

and this gives the definition of a \emph{Hamiltonian system}. That is, the system
\begin{equation}
\begin{split}
\frac{dx}{dt} &=f(x,y) \\
\frac{dy}{dt} &= g(x,y)
\end{split}
\end{equation} 
is Hamiltonian if there is a function $H(x,y)$ so that $f(x,y) = -\frac{\partial H}{\partial y}$ and $g(x,y) = \frac{\partial H}{\partial x}$. 

For solving these sorts of systems, we know that 
\[ \frac{d}{dt}H(x,y) = 0, \]
since that's how we defined the system. This means that the trajectories of this system are given by
\[ H(x,y) = C \] for a constant $C$ determined by initial conditions. So if we can find the function $H$ that expresses the system in the form \eqref{eq:HamilDef}, then we are done.

Finding this $H$ is a lot similar to finding solutions to exact equations in \sectionref{exact:section}. First, we need to determine if the system is Hamiltonian. Since we want to have that
\[ f(x,y) = -\frac{\partial H}{\partial y} \qquad g(x,y) = \frac{\partial H}{\partial x} \]
we know that
\[ f_x(x,y) = -\frac{\partial^2 H}{\partial x \partial y} \qquad g_y(x,y) = \frac{\partial^2 H}{\partial x \partial y} \] which shows that
\[ f_x + g_y = 0. \] This is what we can use to check if a system is Hamiltonian; compare to Theorem \ref{thm:Poincare} for exact equations.

Once we know that a system is Hamiltonian, we can integrate the different components of the equation to find the function $H$. Since $f = -\frac{\partial H}{\partial y}$, then we can write
\[ H(x,y) = -\int f(x,y)\ dy + A(x) \] where $A(x)$ is an unknown function, which can be determined by differentiating this in $x$ and setting equal to $g(x,y)$.

\begin{example}
Consider the system of differential equations given by
\[ x' = -4x + 3y \qquad y' = 2x + 4y. \]
Determine if this system is Hamiltonian and, if so, find the trajectories of the solution.

We first check if $f_x + g_y = 0$ to see if the system is Hamiltonian. Since $f_x = -4$ and $g_y = 4$, this means we have a Hamiltonian system. In order to find the function $H$, we use that
\[ \frac{\partial H}{\partial y} = -f(x,y) = 4x - 3y. \]
Integrating both sides in $y$ gives that
\[ H(x,y) = 4xy - \frac{3}{2}y^2 + A(x) \] for an unknown function $A(x)$. Differentiating this in $x$ gives 
\[ \frac{\partial H}{\partial x} = 4y + A'(x) \] which we want to equal $2x + 4y$. This gives that $A'(x) = 2x$ so $A(x) = x^2$. Thus, the Hamiltonian is given by
\[ H(x,y) = x^2 + 4xy - \frac{3}{2}y^2 \] so that the trajectories are defined by 
\[ x^2 + 4xy - \frac{3}{2}y^2 = C\] for any constant $C$. These are sketched in \figureref{fig:HamilPlot}.

\begin{myfig}
\capstart
\myincludegraphics{width=3in}{width=4.5in}{HamilPlot}
\caption{Vector field and trajectories for a Hamiltonian System \label{fig:HamilPlot}}
\end{myfig}
\end{example}


\subsection{Exercises}

\begin{exercise}
For the systems below, find and classify the critical points, also indicate
if the equilibria are stable, asymptotically stable, or unstable.
\begin{tasks}(2)
\task $x'=-x+3x^2, y'=-y$
\task $x'=x^2+y^2-1$, $y'=x$
\task $x'=ye^x$, $y'=y-x+y^2$
\end{tasks}
\end{exercise}

\begin{exercise}
\pagebreak[2]
Find the implicit equations of the trajectories of the following
conservative systems.  Next find their critical points (if any) and classify them.
\begin{tasks}(2)
\task $x''+ x+x^3 = 0$
\task $\theta''+\sin \theta = 0$
\task $z''+ (z-1)(z+1) = 0$
\task $x''+ x^2+1 = 0$
\end{tasks}
\end{exercise}

\begin{exercise}
Find and classify the critical point(s) of $x' = -x^2$, $y' = -y^2$.
\end{exercise}

\begin{samepage}
\begin{exercise}
Suppose $x'=-xy$, $y'=x^2-1-y$.
\begin{tasks}
\task
Show there are two spiral sinks at
$(-1,0)$ and $(1,0)$.
\task
For any initial point of the form $(0,y_0)$, find what is the trajectory.
\task
Can a trajectory starting at $(x_0,y_0)$ where $x_0 > 0$ spiral into 
the critical point at $(-1,0)$?  Why or why not?
\end{tasks}
\end{exercise}
\end{samepage}

\begin{exercise} \label{exercise:increasing}
In the example $x'=y$, $y'=y^3-x$ show that for any trajectory, the distance
from the origin is an increasing function.
Conclude
that the origin behaves like is a spiral source.
Hint: Consider $f(t) =
{\bigl(x(t)\bigr)}^2 + 
{\bigl(y(t)\bigr)}^2$ and show it has positive derivative.
\end{exercise}


\begin{exercise}
Suppose $f$ is always positive.
Find the trajectories of $x''+f(x') = 0$.
Are there any critical points?
\end{exercise}

\begin{exercise}
Suppose that $x' = f(x,y)$, $y' = g(x,y)$.  Suppose that $g(x,y) > 1$ for
all $x$ and $y$.  Are there any critical points?  What can we say about the
trajectories at $t$ goes to infinity?
\end{exercise}

\setcounter{exercise}{100}

\begin{exercise}
For the systems below, find and classify the critical points.
\begin{tasks}(3)
\task $x'=-x+x^2$, $y'=y$
\task $x'=y-y^2-x$, $y'=-x$
\task $x'=xy$, $y'=x+y-1$
\end{tasks}
\end{exercise}
\exsol{%
a) $(0,0)$: saddle (unstable), $(1,0)$: source (unstable), \qquad
b) $(0,0)$: spiral sink (asymptotically stable), $(0,1)$: saddle (unstable), \qquad
c) $(1,0)$: saddle (unstable), $(0,1)$: saddle (unstable)
}

\begin{exercise}
Find the implicit equations of the trajectories of the following
conservative systems.  Next find their critical points (if any) and classify them.
\begin{tasks}(3)
\task $x''+ x^2 = 4$
\task $x''+ e^x = 0$
\task $x''+ (x+1)e^x = 0$
\end{tasks}
\end{exercise}
\exsol{%
a) $\frac{1}{2}y^2 + \frac{1}{3}x^3 -4x = C$, critical points:
$(-2,0)$, an unstable saddle, and $(2,0)$, a stable center. \quad
b) $\frac{1}{2}y^2 + e^x = C$, no critical points. \quad
c) $\frac{1}{2}y^2 + xe^x = C$, critical point at $(-1,0)$ is a stable center.
}

\begin{exercise}
The conservative system $x''+x^3 = 0$ is not almost linear.  Classify
its critical point(s) nonetheless.
\end{exercise}
\exsol{%
Critical point at $(0,0)$.
Trajectories are $y = \pm \sqrt{2C-(\nicefrac{1}{2})x^4}$, for $C > 0$, these give closed
curves around the origin, so the critical point is a stable center.
}

\begin{exercise}
Derive an analogous classification of critical points for equations in one dimension,
such as $x'= f(x)$ based on the derivative.  A point $x_0$ is critical when $f(x_0) = 0$ and
almost linear if in addition $f'(x_0) \not= 0$.  Figure out if the critical point is stable or unstable
depending on the sign of $f'(x_0)$.  Explain.  Hint: see \sectionref{auteq:section}.
\end{exercise}
\exsol{%
A critical point $x_0$ is stable if $f'(x_0) < 0$ and unstable when $f'(x_0)
> 0$.
}

\TODO{Work on exercises if we change these topics? Like trajectories or Hamiltonians.}
