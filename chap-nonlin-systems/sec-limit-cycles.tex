\section{Limit cycles}
\label{limitcycles:section}

\LAtt{8.4}

\LO{
\item Identify differential equations that have limit cycles from slope fields and
\item Find and classify limit cycles of systems of differential equations by converting the system to depend on radius.
}

For nonlinear systems, trajectories do
not simply need to approach or leave a single point.  They may in fact approach a
larger set, such as a circle or another closed curve.

\begin{example}
The \emph{\myindex{Van der Pol oscillator}}\footnote{Named for the
Dutch physicist 
\href{https://en.wikipedia.org/wiki/Balthasar_van_der_Pol}{Balthasar van der
Pol} (1889--1959).}
is the following equation
\begin{equation*}
x''-\mu(1-x^2) x' + x = 0,
\end{equation*}
where $\mu$ is some positive constant.  The Van der Pol oscillator
originated with electrical circuits, but finds applications
in diverse fields such as biology, seismology, and
other physical sciences.

For simplicity, let us use $\mu = 1$.  A
phase diagram is given in the left-hand plot in
\figurevref{fig:nlin-van-der-fig}.  Notice how the
trajectories seem to very quickly settle on a closed curve.  On the
right-hand side is the plot of a single solution for $t=0$ to $t=30$ with
initial conditions $x(0) = 0.1$ and $x'(0) = 0.1$.  The solution
quickly tends to a periodic solution.
\begin{myfig}
\capstart
%original files nlin-van-der-phase nlin-van-der-plot1
\diffyincludegraphics{width=6.24in}{width=9in}{nlin-van-der-phase-plot1}
\caption{The phase portrait (left) and a graph of a sample solution
of the Van der Pol oscillator.\label{fig:nlin-van-der-fig}}
\end{myfig}

The Van der Pol oscillator is an 
example of so-called \emph{\myindex{relaxation oscillation}}.  The word
relaxation comes from the sudden jump (the very steep part of the solution).
For larger $\mu$ the steep part becomes even more pronounced, for small $\mu$ 
the limit cycle looks more like a circle.  In fact, setting
$\mu = 0$, we get $x''+x=0$, which is a linear system with a
center and all trajectories become circles.
\end{example}

What we see in this example is a curve to which many solution seem to head towards as $t$ gets larger. This motivates the following definition.

\begin{definition}
\begin{enumerate}[1.]
\item A trajectory in the phase portrait that is a closed 
curve (a curve that is a loop) is called a 
\emph{\myindex{closed trajectory}}.

\item A \emph{\myindex{limit cycle}}
is a closed trajectory such
that at least one other trajectory spirals into it.
\item If all trajectories that start near the limit cycle spiral into it, the
limit cycle is called
\emph{asymptotically stable}\index{asymptotically stable limit cycle}.
\end{enumerate}
\end{definition}
For example, the closed curve in the phase portrait for the Van der Pol
equation is a limit cycle, and the limit cycle in the Van der Pol oscillator is 
asymptotically stable.

%The fact that the Van der Pol oscillator has a limit cycle follows from
%the following theorem.
%
%\begin{theorem}[Li\'enard\footnote{FIXME}]
%An equation of the form (a so-called \emph{\myindex{Li\'enard equation}})
%\begin{equation*}
%x'' + f(x) x' + g(x) = 0
%\end{equation*}
%for continuously differentiable $f(x)$ and $g(x)$, where $f(x)$ is an even
%function and $g(x)$ an odd function, has a unique limit cycle if $g(x) > 0$
%for all $x >0$ and $F(x) = \int_0^x f(s)\,ds$ satisfies
%satisfies
%\begin{equation*}
%\lim_{x \to \infty} F(x) = \infty
%\end{equation*}
%and there is some unique number $x_0$ such that
%$F(x) < 0$ for $0 < x < x_0$ and $F(x)$
%\end{theorem}

Given a closed trajectory on an autonomous system,
any solution that starts on it is periodic.
Such a curve is called a
\emph{\myindex{periodic orbit}}.
More precisely, if
$\bigl(x(t),y(t)\bigr)$
is a solution such that for some $t_0$ the point
$\bigl(x(t_0),y(t_0)\bigr)$ lies on a periodic orbit, then both $x(t)$ and $y(t)$
are periodic functions (with the same period).  That is, there is some
number $P$ such that $x(t) = x(t+P)$ and $y(t) = y(t+P)$.

We would like to be able to identify when these sorts of periodic orbits can or can't happen to understand more about these systems. Thankfully, we have a theorem that gives us some help here. 

%Consider the system
%\begin{equation} \label{nlin:gensys}
%x' = f(x,y), \qquad y' = g(x,y) ,
%\end{equation}
%where the functions $f$ and $g$ have continuous derivatives in some region
%$R$ in the plane.

\begin{theorem1}{Poincar\'e--Bendixson}
\index{Poincar\'e--Bendixson Theorem}
Consider the system
\begin{equation} \label{nlin:gensys}
x' = f(x,y), \qquad y' = g(x,y) ,
\end{equation}
where the functions $f$ and $g$ have continuous derivatives in some region
$R$ in the plane. 

Suppose $R$ is a closed bounded region (a region in the plane that includes
its boundary and does not have points arbitrarily far from the origin).
Suppose $\bigl(x(t), y(t)\bigr)$ is a solution of
\eqref{nlin:gensys} in $R$ that exists
for all $t \geq t_0$.  Then either the solution is a periodic function,
or the solution tends towards a periodic solution in $R$.
\end{theorem1}

The main point of the theorem\footnote{%
\href{https://en.wikipedia.org/wiki/Ivar_Otto_Bendixson}{Ivar Otto Bendixson}
(1861--1935) was a Swedish mathematician.}% 
is that if you find one solution that exists
for all $t$ large enough (that is, as $t$ goes to infinity) and stays
within a bounded region, then
you have found either a periodic orbit, or a solution that spirals towards a
limit cycle or tends to a critical point.
That is, in the long term, the
behavior is very close to a periodic function.
Note that a constant solution at a critical point is periodic (with
any period).
The theorem is more a qualitative statement rather than
something to help us in computations.  In practice it is hard to find
analytic solutions and so hard to show rigorously that they exist for all
time.
But if we think the solution exists we numerically solve for a
large time to approximate the limit cycle.
Another caveat is that the theorem only works in two
dimensions.  In three dimensions and higher, there is simply too much room.

The theorem applies to all solutions in the Van der Pol oscillator.
Solutions that start at any point except the origin $(0,0)$ will tend to the
periodic solution around the limit cycle, and if the initial condition
of $(0,0)$ will lead to the constant solution $x=0$, $y=0$.

\begin{example}
Consider
\begin{equation*}
x' = y + {(x^2+y^2-1)}^2 x, \qquad
y' = -x + {(x^2+y^2-1)}^2 y.
\end{equation*}
A vector field along with solutions with initial conditions
$(1.02,0)$, $(0.9,0)$, and $(0.1,0)$ are drawn in
\figurevref{fig:nlin-unstable-limit-cycle}. Analyze this system to determine what will happen to the solution for a variety of initial conditions.


%\begin{mywrapfig}{3.25in}
\begin{myfig}
\capstart
\diffyincludegraphics{width=3in}{width=4.5in}{nlin-unstable-limit-cycle}
\caption{Semistable limit cycle example.\label{fig:nlin-unstable-limit-cycle}}
\end{myfig}
%\end{mywrapfig}
\end{example}

\begin{exampleSol}
Notice that points on the unit circle (distance one from the origin)
satisfy $x^2+y^2-1=0$.  And $x(t) = \sin(t)$, $y = \cos(t)$ is a solution
of the system.  Therefore we have a closed trajectory.
For points off the unit circle, the second term in
$x'$ pushes the solution further
away from the $y$-axis than the system $x' = y$, $y' = -x$,
and $y'$ pushes the solution further away from the $x$-axis
than the linear system $x'=y$, $y' = -x$.  In other words for all
other initial conditions the trajectory will spiral out.

This means that for initial conditions inside the unit circle, the solution
spirals out towards the periodic solution on the unit circle, and
for initial conditions outside the unit circle the solutions spiral off
towards infinity.  Therefore the unit circle is a limit cycle, but
not an asymptotically stable one. In relation to the terms used for autonomous equations in \sectionref{auteq:section}, we could refer to this as a semistable limit cycle, since on one side (inside) the solutions spiral towards the periodic orbit, while on the other side (outside) the solutions move away. The Poincar\'e--Bendixson Theorem applies to the initial points inside
the unit circle, as those solutions stay bounded, but not to those outside,
as those solutions go off to infinity.
\end{exampleSol}

A very similar analysis applies to the system
\begin{equation*}
x' = y + {(x^2+y^2-1)} x, \qquad
y' = -x + {(x^2+y^2-1)} y.
\end{equation*}
We still obtain a closed trajectory on the unit circle, and
points outside the unit circle spiral out to infinity, but now points
inside the unit circle spiral towards the critical point at the origin.
So this system does not have a limit cycle, even though it has a closed
trajectory.

One way to see this more explicitly is by trying to write this all in terms of 
\[ r = \sqrt{x^2 + y^2}. \] For simplicity here, we will determine everything in terms of \[ s = r^2 = x^2 + y^2 \] because as long as $r > 0$, $r$ and $s$ always have the same behavior (in terms of increasing and decreasing), and it is easier to compute with $s$. 

Using the first example
\begin{equation*}
x' = y + {(x^2+y^2-1)}^2 x, \qquad
y' = -x + {(x^2+y^2-1)}^2 y.
\end{equation*}
we see that
\[ \begin{split}
 s' &= 2xx' + 2yy' \\
 &= 2x(y + {(x^2+y^2-1)}^2 x + 2y(-x + {(x^2+y^2-1)}^2 y) \\
 &= 2xy + 2x^2(x^2 + y^2 - 1)^2 - 2xy + 2y^2(x^2 + y^2 - 1)^2 \\
s' &= 2s(s-1)^2
\end{split}
\]

Thus, we are left with the equation
\[ \frac{ds}{dt} = 2s(s-1)^2 \]
which is an autonomous first-order equation that we can analyze. We have two equilibrium solutions in terms of $s$ at $s=0$, which corresponds to the origin, and $s=1$, which corresponds to the unit circle. We can then plug in values to see that for $s = \frac{1}{2}$, $\frac{ds}{dt} > 0$, so that the solutions will increase out to the unit circle. For $s>1$, $\frac{ds}{dt} > 0$ as well, so solutions move away from the circle outside it. This is the same as the result we obtained in the first example.

For the second example, we end up with the autonomous equation
\[ \frac{ds}{dt} = 2s(s-1) \] which is negative for $0 < s < 1$ and positive for $1 < s$, giving that solutions that start inside the unit circle will converge to the origin, and solutions that start outside the circle will move away from it.

Due to the Picard theorem (\thmvref{sys:picardthm}) we find that no matter
where we are in the plane we can always find a solution a little bit
further in time,
as long as $f$ and $g$ have continuous derivatives.  So 
if we find a closed trajectory in an autonomous system,
then for every initial point inside
the closed trajectory, the solution will exist for all time and it will stay
bounded (it will stay inside the closed trajectory). Since the closed trajectory is a solution, we can not cross it (by Picard theorem), and so we have to stay trapped inside. So the moment
we found the solution above going around the unit circle, we knew that for
every initial point inside the circle, the solution exists for all time and
the Poincar\'e--Bendixson theorem applies.

\medskip

Let us next look for conditions when limit cycles (or periodic orbits) do not exist.
We assume
the equation \eqref{nlin:gensys} is defined on a
\emph{\myindex{simply connected region}}, that is, a region with no holes
we can go around.  For example the entire plane is a simply
connected region, and so is the inside of the unit disc.  However,
the entire plane minus a point is not a simply connected domain as it has a
\myquote{hole} at the origin.

\begin{theorem1}[thm:BD]{Bendixson--Dulac%
}%
\index{Bendixson--Dulac Theorem}
Suppose $R$ is a simply connected region,
and the expression%
%Suppose $f$ and $g$ are functions with continuous derivatives in
%a simply connected region $R$, and
%If the expression%
\footnote{Usually the expression in the Bendixson--Dulac Theorem is
$\frac{\partial (\varphi f)}{\partial x} + \frac{\partial (\varphi
g)}{\partial y}$
for some continuously differentiable function $\varphi$.  For simplicity,
let us just consider the case $\varphi = 1$.}
\begin{equation*}
\frac{\partial f}{\partial x} + \frac{\partial g}{\partial y}
\end{equation*}
is either always positive or always negative
on $R$ (except perhaps a small set such as on isolated points or curves)
then the system \eqref{nlin:gensys}
has no closed trajectory inside $R$.
\end{theorem1}

The theorem\footnote{%
\href{https://en.wikipedia.org/wiki/Henri_Dulac}{Henri Dulac} (1870--1955)
was a French mathematician.} gives us a way of ruling out the existence of a closed
trajectory, and hence a
way of ruling out limit cycles.
The exception about points or curves 
means that we can allow the expression to be zero at a few points,
or perhaps on a curve, but not on any larger set.

\begin{example}
Let us look at $x'=y+y^2e^x$, $y'=x$ in the entire plane (see
\examplevref{example:nlin-withexp}) and try to apply \thmref{thm:BD}.
\end{example}
\begin{exampleSol}
The entire plane
is simply connected and so we can apply the theorem.  We compute
$\frac{\partial f}{\partial x} + \frac{\partial g}{\partial y} =
y^2e^x+ 0$.  The function $y^2e^x$ is always positive except on the line
$y=0$.  Therefore, via the theorem, the system has no closed trajectories.
\end{exampleSol}

In some books (or the internet) the theorem is not stated carefully
and it concludes there are no periodic solutions.  That is not quite
right.  The example above has two critical points and hence it has
constant solutions, and constant functions are periodic.  The conclusion of
the theorem should be that there exist no trajectories that form closed
curves.  Another way to state the conclusion of the theorem would be to
say that there exist no nonconstant periodic solutions that stay in $R$.

Let us look at a somewhat more complicated example.

\begin{example}
Take the system $x'=-y-x^2$, $y'=-x+y^2$ (see
\examplevref{example:nlin-xplusy}) and look at how \thmref{thm:BD} works here.
\end{example}
\begin{exampleSol}
We compute
$\frac{\partial f}{\partial x} + \frac{\partial g}{\partial y} =
-2x + 2y=2(-x+y)$.  This expression takes on both signs, so if we are talking about the
whole plane we cannot simply apply the theorem.  However, we could apply it
on the set where $-x+y \geq 0$.  Via the theorem, there is no
closed trajectory in that set.  Similarly, there is no closed trajectory
in the set $-x+y \leq 0$.  We cannot conclude (yet) that there is no closed
trajectory in the entire plane.  Perhaps half of it is in the set where
$-x+y \geq 0$ and the other half is in the set where $-x+y \leq 0$.

The key is to look at the line where $-x+y=0$, or $x=y$.  On this line
$x' = -y-x^2 = -x-x^2$ and $y' = -x+y^2 = -x+x^2$.  In particular,
when $x=y$ then $x' \leq y'$.  That means that the arrows, the vectors
$(x',y')$, always point
into the set where $-x+y \geq 0$.  There is no way we can start in the
set where $-x+y \geq 0$
and go into the set where $-x+y \leq 0$.  Once we are in
the set where $-x+y \geq 0$, we stay there.  So no closed trajectory can
have points in both sets.
%
%The key is to look at the set $x+y=0$, or $x=-y$.  Let us
%make a substitution $x=z$ and $y=-z$ (so that $x=-y$).  
%Both equations become
%$z'=z-z^2$.  So any solution of $z'=z-z^2$, gives us a solution
%$x(t)=z(t)$, $y(t)=-z(t)$.  In particular, any solution that starts
%out on the line $x+y=0$, stays on the line $x+y = 0$.  In other words,
%there cannot be a closed trajectory that starts on the set where $x+y > 0$
%and goes through the set where $x+y < 0$, as it would have to pass through
%$x+y = 0$.
\end{exampleSol}

\begin{example}
Consider
$x' = y+(x^2+y^2-1)x$, 
$y' = -x +(x^2+y^2-1)y$, and consider the region $R$ given by $x^2+y^2 >
\frac{1}{2}$.  That is, $R$ is the region outside a circle of radius
$\frac{1}{\sqrt{2}}$ centered at the origin.  Then
there is a closed trajectory in $R$, namely $x=\cos(t)$, $y=\sin(t)$.
Furthermore,
\begin{equation*}
\frac{\partial f}{\partial x} + 
\frac{\partial g}{\partial x} = 4x^2+4y^2-2 ,
\end{equation*}
which is always positive on $R$.  So what is going on?  The Bendixson--Dulac theorem does not
apply since the region $R$ is not simply connected---it has a hole, the
circle we cut out!
\end{example}

% \TODO{This feels overly complicated. It probably needs some examples converting to r and some concrete calculations. The same goes for exercises.}

\subsection{Exercises}

\begin{exercise}
Consider the two-dimensional system of differential equation written in polar coordinates as
\[ \frac{dr}{dt} = r(r-1)(r-4)^2 \qquad \frac{d\theta}{dt} = 1. \] Determine all limit cycles, periodic solutions, and classify the stability of each of these solutions. 
\end{exercise}

\begin{exercise}
Consider the two-dimensional system of differential equation written in polar coordinates as
\[ \frac{dr}{dt} = r^2(r-1)^2(r-3) \qquad \frac{d\theta}{dt} = -1. \] Determine all limit cycles, periodic solutions, and classify the stability of each of these solutions. 
\end{exercise}

\begin{exercise}\ansMark%
Consider the system of differential equation given by
\[ \frac{dx}{dt} = x(3- 2y^2 - x^2) \qquad \frac{dy}{dt} = y(3-y^2) .\]
Find and classify all limit cycles by converting to an autonomous equation in $r = \sqrt{x^2 + y^2}$ or $s = x^2 + y^2$. 
\end{exercise}
\exsol{%
$(0,0)$, unstable, $r = \sqrt{3}$, asymptotically stable.
}

\begin{exercise}\ansMark%
Consider the system of differential equation given by
\[ \frac{dx}{dt} = -x(x^2 + y^2)^2 + 6x(x^2 + y^2) - 8x + 6y \qquad \frac{dy}{dt} = -y(x^2 + y^2)^2 + 6y(x^2 + y^2) - 8y - 6x .\]
Find and classify all limit cycles by converting to an autonomous equation in $r = \sqrt{x^2 + y^2}$ or $s = x^2 + y^2$. 
\end{exercise}
\exsol{%
$(0,0)$, asymptotically stable, $r = \sqrt{2}$, unstable, $r = 2$, asymptotically stable. 
}

\begin{exercise}
Consider the system %5.4
\begin{equation}
\begin{bmatrix}\dfrac{dx}{dt}=x+2y+x(x^2+y^2-2\sqrt{x^2+y^2})\\[6pt]
  \dfrac{dy}{dt}=-2x+y+y(x^2+y^2-2\sqrt{x^2+y^2})
\end{bmatrix}. \label{eq:LimitCycleExercise1}
\end{equation}
\begin{tasks}
\task Use polar coordinates to write $\dfrac{dr}{dt}$ as a function of $r$.
\task Draw the phase line of the DE $\dfrac{dr}{dt}=f(r)$, where $f(r)$ is the function from part a.
\task Does the system \eqref{eq:LimitCycleExercise1} have a limit cycle? If so, find it. If not, explain why not.
For each positive root of $f(r)$, decide whether the corresponding trajectory one is stable, unstable, or semistable.
\end{tasks}
\end{exercise}

\begin{exercise}
Show that the following systems have no closed trajectories.
\begin{tasks}(2)
\task $x'=x^3+y,\quad y'=y^3+x^2$,
\task $x'=e^{x-y},\quad y'=e^{x+y}$,
\task $x'=x+3y^2-y^3,\quad y'=y^3+x^2$.
\end{tasks}
\end{exercise}

\begin{exercise}\ansMark%
Show that the following systems have no closed trajectories.
\begin{tasks}(2)
\task $x'=x+y^2,\quad y'=y+x^2$,
\task $x'=-x\sin^2(y),\quad y'=e^x$,
\task $x'=xy,\quad y'=x+x^2$.
\end{tasks}
\end{exercise}
\exsol{%
Use Bendixson--Dulac Theorem.
a) $f_x+g_y = 1+1 > 0$, so no closed trajectories.
b) $f_x+g_y = -\sin^2(y)+0 < 0$ for all $x,y$ except the lines
given by $y=k\pi$ (where we get zero), so no closed trajectories.
c) $f_x+g_y = y + 0 > 0$ for all $x,y$ except the line
given by $y=0$ (where we get zero), so no closed trajectories.
}

\begin{exercise}\ansMark%
Suppose an autonomous system in the plane has a solution
$x=\cos(t)+e^{-t}$, $y=\sin(t)+e^{-t}$.  What can you say
about the system (in particular about limit cycles and periodic solutions)?
\end{exercise}
\exsol{%
Using Poincar\'e--Bendixson Theorem,
the system has a limit cycle, which is the unit circle centered at the origin as
$x=\cos(t)+e^{-t}$, $y=\sin(t)+e^{-t}$ gets closer and closer to the unit
circle.  Thus we also have that $x=\cos(t)$, $y=\sin(t)$ is the periodic
solution.
}

\begin{exercise}
Formulate a condition for a 2-by-2 linear system
${\vec{x}}' = A \vec{x}$ to not be a center using the Bendixson--Dulac theorem.
That is, the theorem says something about certain elements of $A$.
\end{exercise}

\begin{exercise}
Explain why the Bendixson--Dulac Theorem does not apply for any conservative
system $x''+h(x) = 0$.
\end{exercise}

\begin{exercise}
A system such as $x'=x, y'=y$ has solutions that exist for all time $t$,
yet there are no closed trajectories.  Explain
why the Poincar\'e--Bendixson Theorem does not apply.
\end{exercise}

\begin{exercise}\ansMark%
Show that the limit cycle
of the 
Van der Pol oscillator (for $\mu > 0$) must not lie completely in the set
where 
$-1 < x < 1$.
Compare with \figurevref{fig:nlin-van-der-fig}.
\end{exercise}
\exsol{%
$f(x,y) = y$, $g(x,y) = \mu(1-x^2)y-x$.  So
$f_x+g_y = \mu(1-x^2)$.  The Bendixson--Dulac Theorem
says there is no closed trajectory lying entirely in the set $x^2 < 1$.
}

\begin{exercise}
Differential equations can also be given in different coordinate systems.  
Suppose we have the system $r' = 1-r^2$, $\theta' = 1$ given
in polar coordinates.  Find all the closed trajectories and check if they
are limit cycles and if so, if they are asymptotically stable or not.
\end{exercise}

\begin{exercise}\ansMark%
Suppose we have the system $r' = \sin(r)$, $\theta' = 1$ given
in polar coordinates.  Find all the closed trajectories.
\end{exercise}
\exsol{%
The closed trajectories are those where $\sin(r) = 0$, therefore,
all the circles centered at the origin with radius that
is a multiple of $\pi$ are closed
trajectories.
}



\setcounter{exercise}{100}

