\section{Chaos} \label{sec:chaos}

\LAtt{8.5}

\LO{
\item Identify chaotic behavior and how it is distinct from other types of equations.
}

You have surely heard the idea of the ``butterfly effect,'' that the
flap of a butterfly wing in the Amazon can cause hurricanes in the North
Atlantic.  In a prior section, we mentioned that a small change in
initial conditions of the planets can lead to very different configuration
of the planets in the long term.  These are examples of
\emph{\myindex{chaotic systems}}.
Mathematical chaos is not really chaos, there is precise order behind the
scenes.  Everything is still deterministic.  However a chaotic system is extremely
sensitive to initial conditions.  This also means even small errors induced 
via numerical approximation create large errors very quickly, so it is
almost impossible to numerically approximate for long times.
This is a large part of
the trouble, as chaotic systems cannot be in general solved analytically.

Take the weather, the most well-known chaotic system.
A small change in the initial conditions
(the temperature at every point of the atmosphere for example) produces
drastically different predictions in relatively short time, and so we cannot
accurately predict weather.  And we do not actually know the
exact initial conditions.  We measure temperatures at a few points with some
error, and then we somehow estimate what is in between.
There is no way we
can accurately measure the effects of every butterfly wing.
Then
we solve the equations numerically introducing new errors.  You
should not trust weather prediction more than a few days out.

Chaotic behavior was first noticed by Edward
Lorenz\footnote{\href{https://en.wikipedia.org/wiki/Edward_Norton_Lorenz}{Edward Norton Lorenz} (1917--2008) was
an American mathematician and meteorologist.}
in the
1960s when trying to model thermally induced air convection (movement).
Lorentz was looking at the relatively simple system:
\begin{equation*}
x' = -10x +10y, \qquad y' = 28x-y-xz, \qquad z'=-\frac{8}{3}z + xy .
\end{equation*}
A small change in the initial conditions yields a very different solution
after a reasonably short time.

\begin{mywrapfigsimp}{0.95in}{1.25in}
\noindent
\inputpdft{chaos-pend}
\end{mywrapfigsimp}
A simple example the reader can experiment with, and which displays
chaotic behavior, is a double pendulum.  The equations for this
setup are somewhat complicated, and their derivation is quite
\myindex{tedious}, so we
will not bother to write them down.  The idea is to put a pendulum on the
end of another pendulum.  The movement of the bottom mass
will appear chaotic.  This type of chaotic system is a basis for a
whole number of office novelty desk toys.  It is simple to build a
version.  Take a piece of a string.  Tie two heavy nuts at different
points of the string; one at the end, and one a bit above.  Now give the
bottom nut a little push.  As long as the swings are not too big and the
string stays tight, you have a double pendulum system.

\subsection{Duffing equation and strange attractors}

Let us study the so-called \emph{\myindex{Duffing equation}}:
\begin{equation*}
x'' + a x' + bx + cx^3 = C \cos(\omega t) .
\end{equation*}
Here $a$, $b$, $c$, $C$, and $\omega$ are constants.
Except for the $c x^3$ term, this equation looks like
a forced mass-spring system.  The $c x^3$ means the spring does
not exactly obey Hooke's law (which no real-world spring actually does obey
exactly).  When $c$ is not zero, the equation does not have a closed
form solution, so we must resort to numerical solutions, as is usual for
nonlinear systems.  Not all choices of constants and initial conditions
exhibit chaotic behavior.  Let us study
\begin{equation*}
x''+0.05 x' + x^3 = 8\cos(t) .
\end{equation*}

The equation is not autonomous, so we cannot 
draw the vector field in the phase plane.
We can still draw the
trajectories.   
In \figurevref{nlin:duf-two-traj} we plot trajectories for $t$ going from 0
to 15, for two very close initial conditions
$(2,3)$ and $(2,2.9)$, and also the solutions in the $(x,t)$ space.  The two
trajectories are close at first, but after a while diverge significantly.
This sensitivity to initial conditions is precisely what
we mean by the system behaving chaotically.

\begin{myfig}
\capstart
%original files nlin-duf-two-traj nlin-duf-two-sols
\diffyincludegraphics{width=6.24in}{width=9in}{nlin-duf-two-traj-sols}
\caption{On left, two trajectories in phase space for $0 \leq t \leq 15$, for the Duffing equation
one with initial conditions $(2,3)$ and the other with $(2,2.9)$.  On
right the two solutions in $(x,t)$-space. \label{nlin:duf-two-traj}}
\end{myfig}

\begin{myfig}
\capstart
\diffyincludegraphics{width=6.24in}{width=9in}{nlin-duf-long}
\caption{The solution to the given Duffing equation for $t$ from 0 to 100.
\label{nlin:duf-long}}
\end{myfig}

\pagebreak[2]
Let us see the long term behavior.
In \figurevref{nlin:duf-long},
we plot the behavior of the system for initial conditions $(2,3)$ for
a longer period of time.
It is hard to see any particular pattern
in the shape of the solution except that it seems to oscillate, but each
oscillation appears quite unique.  The oscillation
is expected due to the forcing term.
We mention that to produce the picture accurately,
a ridiculously large number of steps\footnote{In
fact for reference, 30,000 steps were used with the Runge--Kutta
algorithm, see exercises in \sectionref{numer:section}.}
had to be used in the numerical
algorithm,
as even small errors quickly propagate in a chaotic system.


It is very difficult to analyze chaotic systems, or to find the
order behind the madness, but let us try to do something that we did for
the standard mass-spring system.  One way we analyzed
the system is that we figured out what was the long term behavior (not
dependent on initial conditions).  From the figure above, it is clear
that we will not get a nice exact description of the long term behavior
for this chaotic system, but
perhaps we can find some order to what happens on each
\myquote{oscillation}
and what do these oscillations have in common.

The concept we explore
is that of a \emph{\myindex{Poincar\'e section}}%
\footnote{Named for the French polymath
\href{https://en.wikipedia.org/wiki/Henri_Poincar\%C3\%A9}{Jules Henri
Poincar\'e} (1854--1912).}.  Instead of 
looking at $t$ in a certain interval, we look at where the
system is at a certain sequence of points in time.
Imagine flashing a strobe at a
fixed frequency and drawing the points where the solution is during the flashes.
The right strobing frequency
depends on the system in question.
The
correct frequency for the forced Duffing equation (and other similar
systems) is the frequency of the forcing term.
For the Duffing equation above, find a solution
$\bigl(x(t),y(t)\bigr)$, and look at the points
\begin{equation*}
\bigl(x(0),y(0)\bigr), \quad
\bigl(x(2\pi),y(2\pi)\bigr), \quad
\bigl(x(4\pi),y(4\pi)\bigr), \quad
\bigl(x(6\pi),y(6\pi)\bigr), \quad \ldots
\end{equation*}
As we are really not interested in the transient part of the solution, that
is, the part of the solution that depends on the initial condition, we 
skip some number of steps in the beginning.  For example, we might skip the first 100 such
steps and start plotting points at $t = 100(2\pi)$, that is
\begin{equation*}
\bigl(x(200\pi),y(200\pi)\bigr), \quad
\bigl(x(202\pi),y(202\pi)\bigr), \quad
\bigl(x(204\pi),y(204\pi)\bigr), \quad \ldots
\end{equation*}
The plot of these points is the Poincar\'e section.
After plotting enough points, a curious pattern emerges in
\figurevref{nlin:strange} (the left-hand picture), a so-called
\emph{\myindex{strange attractor}}.

\begin{myfig}
\capstart
%original files nlin-strange nlin-strange2
\diffyincludegraphics{width=6.24in}{width=9in}{nlin-strange-strange2}
\caption{Strange attractor.  The left plot 
is with no phase shift, the right plot has phase shift
$\nicefrac{\pi}{4}$. \label{nlin:strange}}
\end{myfig}

Given a sequence of points, 
an \emph{\myindex{attractor}} is a set towards which the points
in the sequence
eventually get closer and closer to, that is, they are attracted.  The
Poincar\'e section is not really the attractor itself, but as
the points are very close to it, we see its shape.  The strange
attractor is a very complicated set.   It has
fractal structure, that is, if you zoom in as far as you want, you
keep seeing the same complicated structure.

The initial condition makes no difference.  If
we start with a different initial condition, the points eventually
gravitate towards the attractor, and so as long as we throw away the first
few points, we get the same picture.
Similarly small errors in the numerical approximations do not matter here.

An amazing thing is that a chaotic system such as the Duffing equation is
not random at all.  There is a very complicated order to it, and the strange
attractor says something about this order.  We cannot quite say what state
the system will be in eventually, but given the fixed strobing frequency we
narrow it down to the points on the attractor.

If we use a phase shift, for example $\nicefrac{\pi}{4}$, and look at the
times
\begin{equation*}
\nicefrac{\pi}{4}, \quad
2\pi+\nicefrac{\pi}{4}, \quad
4\pi+\nicefrac{\pi}{4}, \quad
6\pi+\nicefrac{\pi}{4}, \quad
\ldots
\end{equation*}
we obtain a slightly different attractor.
The picture is the right-hand side of 
\figurevref{nlin:strange}.
It is as if we had
rotated, moved, and slightly distorted the original.
For each phase shift you can find the
set of points towards which the system periodically keeps coming back to.

Study the pictures and notice especially the scales---where are
these attractors located in the phase plane.  Notice the
regions where the strange attractor lives and compare it to the plot of the
trajectories in \figurevref{nlin:duf-two-traj}.

Let us
compare this section to the discussion in \sectionref{forcedo:section} about forced
oscillations.  Take the equation
\begin{equation*}
x''+2p x' + \omega_0^2 x = \frac{F_0}{m} \cos (\omega t) .
\end{equation*}
This is like the Duffing equation, but with no $x^3$ term.
The steady periodic solution is of the form
\begin{equation*}
x = C \cos (\omega t + \gamma) .
\end{equation*}
Strobing using the frequency $\omega$, we obtain a single point in the
phase space.  The attractor in this setting is a single point---an
expected result as the system is not chaotic.  It was the opposite
of chaotic:  Any difference induced by the initial conditions dies away very
quickly, and we settle into always the same steady periodic motion.

\subsection{The Lorenz system}

In two dimensions to find chaotic behavior,
we must study forced, or non-autonomous, systems such as the Duffing
equation.
The Poincar\'e--Bendixson Theorem says that
a solution to an autonomous
two-dimensional system that exists for all time in the future
and does not go towards infinity
is periodic or tends towards a periodic solution.  Hardly the chaotic
behavior we are looking for.

In three dimensions even autonomous systems can be chaotic.
Let us very briefly return to the \myindex{Lorenz system}
\begin{equation*}
x' = -10x +10y, \qquad y' = 28x-y-xz, \qquad z'=-\frac{8}{3}z + xy .
\end{equation*}
The Lorenz system is an autonomous system in three dimensions
exhibiting chaotic behavior.
See the \figurevref{nlin:lorenz} for a sample trajectory,
which is now a curve in three-dimensional space.
\begin{myfig}
\capstart
\diffyincludegraphics{width=3in}{width=4.5in}{nlin-lorenz}
\caption{A trajectory in the Lorenz system. \label{nlin:lorenz}}
\end{myfig}

The solutions tend to an \emph{attractor} in space,
the so-called \emph{\myindex{Lorenz attractor}}.
In this case no strobing is
necessary.
Again we cannot quite see the attractor itself, but if we try to follow a solution
for long enough, as in the figure,
we get a pretty good picture of what the attractor looks
like.
The Lorenz attractor is also a strange attractor and has a complicated
fractal structure.  And, just as for the Duffing equation, what we want to
draw is not the whole trajectory, but start drawing the trajectory after a
while, once it is close to the attractor.

The path of the trajectory is not simply a repeating figure-eight.
The trajectory spins some
seemingly random number of times on the left, then spins a number of times on
the right, and so on.  As this system arose in weather prediction, one can
perhaps imagine a few days of warm weather and then a few days of cold
weather, where it is not easy to predict when the weather will change,
just as it is not really easy to predict far in advance when the solution
will jump onto the other side.  See \figurevref{nlin:lorenz-graphx} for a
plot of the $x$ component of the solution drawn above.  A negative $x$
corresponds to the left \myquote{loop} and a positive $x$
corresponds to the right \myquote{loop}.

Most of the mathematics we studied in this book is quite classical and
well understood.
On the other hand, chaos, including the Lorenz system, continues to be the
subject of current research.
Furthermore, chaos has found applications not just in the sciences, but
also in art.

\begin{myfig}
\capstart
\diffyincludegraphics{width=3in}{width=4.5in}{nlin-lorenz-graphx}
\caption{Graph of the $x(t)$ component of the solution.
\label{nlin:lorenz-graphx}}
\end{myfig}

\subsection{Exercises}

\begin{exercise}[*]
Find critical points of the Lorenz system and the associated linearizations.
\end{exercise}
\exsol{%
Critical points: $(0,0,0)$, $(3\sqrt{8},3 \sqrt{8}, 27)$,
$(-3 \sqrt{8},-3 \sqrt{8}, 27)$.
Linearization at $(0,0,0)$ using $u=x$, $v=y$, $w=z$ is
$u' = -10u+10v$, $v'=28u-v$, $w'=-(\nicefrac{8}{3})w$.
Linearization at $(3 \sqrt{8},3\sqrt{8},27)$ using $u=x-3\sqrt{8}$,
$v=y-3\sqrt{8}$, $w=z-27$ is
$u' = -10u+10v$, $v'=u-v-3\sqrt{8}w$, $w'=3\sqrt{8}u+3\sqrt{8}v-(\nicefrac{8}{3})w$.
Linearization at $(-3 \sqrt{8},-3\sqrt{8},27)$ using $u=x+3\sqrt{8}$,
$v=y+3\sqrt{8}$, $w=z-27$ is
$u' = -10u+10v$, $v'=u-v+3\sqrt{8}w$, $w'=-3\sqrt{8}u-3\sqrt{8}v-(\nicefrac{8}{3})w$.
}

\begin{exercise}
For the non-chaotic equation
$x''+2p x' + \omega_0^2 x = \frac{F_0}{m} \cos (\omega t)$, suppose we
strobe with frequency $\omega$ as we mentioned above.  Use the known
steady periodic solution to find precisely the point which is the attractor
for the Poincar\'e section.
\end{exercise}
\comboSol{%
}
{%
$(A, \omega B)$ for $A \cos(\omega t) + B\sin(\omega t)$, or $\frac{F_0/m}{(\omega_0^2 - \omega^2)^2 + (2p\omega)^2}(\omega_0^2 - \omega^2, 2p\omega^2)$
}

\begin{exercise}[project]
Construct the double pendulum described in the text with a string and two
nuts (or heavy beads).  Play around with the position of the middle nut, and
perhaps use different weight nuts.  Describe what you find.
\end{exercise}

\begin{samepage}
\begin{exercise}[project]
A simple fractal attractor can be drawn via the following chaos game.  Draw
the three
vertices of a triangle and label them, say $p_1$, $p_2$
and $p_3$.  Draw some
random point $p$ (it does not have to be one of the three points above).
Roll a die to pick of the $p_1$, $p_2$, or $p_3$
randomly (for example 1 and 4 mean $p_1$, 2 and 5 mean $p_2$, and 3 and 6
mean $p_3$).  Suppose we picked $p_2$, then let $p_{\text{new}}$ be the
point exactly halfway between $p$ and $p_2$.  Draw this point and let $p$
now refer to this new point $p_{\text{new}}$.  Rinse, repeat.  Try to be
precise and draw as many iterations as possible.  Your points will be
attracted to the so-called \emph{\myindex{Sierpinski triangle}}.  A computer
was used to run the game for 10,000 iterations to obtain the picture in
\figurevref{nlin:sierpinski}.
\end{exercise}
\end{samepage}

\begin{myfig}
\capstart
\diffyincludegraphics{width=3in}{width=4.5in}{nlin-sierpinski}
\caption{10,000 iterations of the chaos game producing the 
Sierpinski triangle. \label{nlin:sierpinski}}
\end{myfig}


\begin{exercise}[computer project]
Use a computer software (such as Matlab, Octave, or
perhaps even a spreadsheet), plot the solution
of the given forced Duffing equation with Euler's method.  Plotting the
solution for $t$ from 0 to 100 with several different (small) step sizes.
Discuss.
\end{exercise}
\comboSol{%
}
{%
You should get very different behavior for similar (small) step sizes becasue the equation is chaotic. 
}


