
\section{Dirichlet problem in the circle and the Poisson kernel}
\label{dirichdisc:section}

\LAtt{4.10}

\LO{
\item Solve the Laplace equation in a circle,
\item Use series solutions in polar coordinates to solve the Laplace equation, and
\item Use the Poisson kernel to solve boundary value problems for the Laplacian in a circle.
}

% \sectionnotes{Verbatim from Lebl}

% \sectionnotes{2 lectures\EPref{, \S9.7 in \cite{EP}}\BDref{,
% \S10.8 in \cite{BD}}}

\subsection{Laplace in polar coordinates}

A more natural setting for the Laplace equation $\Delta u = 0$
is a circle rather than a rectangle.  On the other hand, what makes the
problem somewhat more difficult is that we need polar coordinates.

\begin{mywrapfigsimp}{1.3in}{1.5in}
\diffypdfversion{\vspace*{5pt}}
\noindent
\inputpdft{polarcoords}
\diffypdfversion{\vspace*{5pt}}
\end{mywrapfigsimp}
Recall that the polar coordinates for the $(x,y)$-plane are $(r,\theta)$: 
\begin{equation*}
x = r \cos \theta , \quad y = r \sin \theta ,
\end{equation*}
where $r \geq 0$ and $-\pi < \theta \leq \pi$.  So the point $(x,y)$ is
distance $r$ from the origin at an angle $\theta$ from the positive
$x$-axis.

Now that we know our coordinates, let us give the problem we wish
to solve.  We have a circular region of radius 1, and we are interested
in the Dirichlet problem for the Laplace equation for this region.  Let
$u(r,\theta)$ denote the temperature at the point $(r,\theta)$ in polar
coordinates.

\begin{mywrapfigsimp}{2.4in}{2.7in}
\noindent
\inputpdft{dirichdiscsetup}
\end{mywrapfigsimp}
We have the problem:
\begin{equation} \label{dirichdisc:theprobeq}
\begin{aligned}
& \Delta u = 0 , & & \text{for } \; r < 1, \\
& u(1,\theta) = g(\theta), & & \text{for } \; {-\pi} < \theta \leq \pi.
\end{aligned}
\end{equation}

The first issue we face is that we do not know the Laplacian
in polar coordinates.
Normally we would find $u_{xx}$ and $u_{yy}$ in terms of
the derivatives in $r$ and $\theta$.  We would need to solve
for $r$ and $\theta$ in terms of $x$ and $y$.  In this case
it is more convenient to work in
reverse.  We compute derivatives in $r$ and $\theta$ in terms
of derivatives in $x$ and $y$ and then we solve.  The
computations are easier this way.  First
\begin{equation*}
\begin{aligned}
& x_r = \cos \theta, & &
x_\theta = - r \sin \theta, \\
& y_r = \sin \theta, & &
y_\theta = r \cos \theta.
\end{aligned}
\end{equation*}
Next by chain rule we obtain
\begin{align*}
u_r & = u_x x_r + u_y y_r = \cos(\theta) u_x + \sin(\theta) u_y ,
\\
u_{rr} & =
\cos(\theta) ( u_{xx} x_r +u_{xy} y_r )
+ \sin(\theta) ( u_{yx} x_r +u_{yy} y_r )
\\
&
=
\cos^2(\theta) u_{xx} +
2 \cos(\theta)\sin(\theta) u_{xy} +
\sin^2(\theta) u_{yy} .
\end{align*}
Similarly for the $\theta$ derivative.  Note that we have to use
the product rule for the second derivative.
\begin{align*}
u_\theta & = u_x x_\theta + u_y y_\theta =
-r\sin(\theta) u_x + r\cos(\theta) u_y ,
\\
u_{\theta\theta} & =
-r\cos(\theta) u_x
-r\sin(\theta) (u_{xx} x_\theta + u_{xy} y_\theta)
-r\sin(\theta) u_y
+
r\cos(\theta) (u_{yx} x_\theta + u_{yy} y_\theta)
%\\
%& = 
%-r\cos(\theta) u_x
%-r\sin(\theta) (u_{xx} (-r\sin(\theta)) + u_{xy} (r \cos(\theta)))
%-r\sin(\theta) u_y
%+
%r\cos(\theta) (u_{yx} (-r\sin(\theta)) + u_{yy} (r \cos (\theta)))
\\
& = 
-r\cos(\theta) u_x
-r\sin(\theta) u_y
+r^2 \sin^2(\theta) u_{xx}
-r^2 2\sin(\theta)\cos(\theta) u_{xy}
+r^2 \cos^2(\theta) u_{yy} .
\end{align*}
Let us now try to solve for $u_{xx} + u_{yy}$.  We start with
$\frac{1}{r^2} u_{\theta\theta}$ to get rid of those pesky $r^2$.
If we add $u_{rr}$
and use the fact that $\cos^2(\theta) +\sin^2(\theta) = 1$, we get
\begin{equation*}
\frac{1}{r^2} u_{\theta\theta}
+
u_{rr}
=
u_{xx} + u_{yy} - \frac{1}{r} \cos(\theta) u_x - \frac{1}{r} \sin(\theta)
u_y .
\end{equation*}
We're not quite there yet, but all we are lacking is 
$\frac{1}{r} u_r$.  Adding it we obtain the
\emph{\myindex{Laplacian in polar coordinates}}:
\begin{equation*}
\mybxbg{~~
\Delta u 
=
u_{xx} + u_{yy} =
\frac{1}{r^2} u_{\theta\theta}
+
\frac{1}{r} u_{r}
+
u_{rr} .
~~}
\end{equation*}

Notice that the Laplacian in polar coordinates no longer has constant
coefficients.
%It looks more complicated than the original, but in fact
%the computations get easier from now on.

\subsection{Series solution}

Let us separate variables as usual.  That is let us try
$u(r,\theta) = R(r)\Theta(\theta)$.  Then
\begin{equation*}
0 = \Delta u = 
\frac{1}{r^2} R \Theta''
+
\frac{1}{r} R' \Theta
+
R'' \Theta .
\end{equation*}
Let us put $R$ on one side and $\Theta$ on the other and conclude
that both sides must be constant.
\begin{align*}
\frac{1}{r^2} R \Theta''
& =
-
\left(\frac{1}{r} R' + R''\right) \Theta 
\\
\frac{\Theta''}{\Theta}
& =
-
\frac{r R' + r^2 R''}{R} = -\lambda
\end{align*}
We get two equations:
\begin{align*}
& \Theta'' + \lambda \Theta = 0 ,
\\
& r^2 R'' + r R' -\lambda R = 0.
\end{align*}
Let us first focus on $\Theta$.  We know that $u(r,\theta)$ ought to be
$2\pi$-periodic in $\theta$, that is,
$u(r,\theta) = u(r,\theta+2\pi)$.  Therefore, the solution to
$\Theta'' + \lambda \Theta = 0$ must be $2\pi$-periodic.
We have seen such a problem in \exampleref{bvp-periodic:example}.
We conclude
that
%$\lambda = 0,1,4,9,\ldots$.  That is,
$\lambda = n^2$ for a
nonnegative integer $n=0,1,2,3,\ldots$.  The equation becomes
$\Theta'' + n^2 \Theta = 0$.  When $n=0$ the equation is just
$\Theta'' = 0$, so we have the general solution $A \theta + B$.  As
$\Theta$ is periodic,
$A=0$.
For convenience we write this solution as
\begin{equation*}
\Theta_0 = \frac{a_0}{2}
\end{equation*}
for some constant $a_0$.  For positive $n$,
the 
solution to
$\Theta'' + n^2 \Theta = 0$ is
\begin{equation*}
\Theta_n = a_n \cos(n\theta) + b_n \sin(n\theta) ,
\end{equation*}
for some constants $a_n$ and $b_n$.

Next, we consider the equation for $R$,
\begin{equation*}
r^2 R'' + r R' - n^2 R = 0.
\end{equation*}
This equation appeared in exercises before---we
solved it in \exerciseref{sol:eulerex}
and \exercisevref{sol:eulerexln}.  The idea is to try a solution
$r^s$ and if that does not give us two solutions, also try a solution of the form
$r^s \ln r$.  Let us name the solution for $R_n$.  When $n=0$ we obtain
\begin{equation*}
R_0 = A r^0 + B r^0 \ln r = A + B \ln r ,
\end{equation*}
and if $n > 0$, we get
\begin{equation*}
R_n = A r^n + B r^{-n} .
\end{equation*}
The function $u(r,\theta)$ must be finite at the origin, that is, when $r=0$.
So $B=0$ in both
cases.  Set $A=1$ in both cases as well; the constants in $\Theta_n$
will pick up the slack so nothing is lost.  Let
\begin{equation*}
R_0 = 1 , \qquad \text{and} \qquad
R_n = r^n .
\end{equation*}
Hence our building block solutions are
\begin{align*}
& u_0(r,\theta) = \frac{a_0}{2} ,
& u_n(r,\theta) = a_n r^n \cos(n \theta) + b_n r^n \sin(n \theta) .
\end{align*}
Putting everything together our solution is:
\begin{equation*}
\mybxbg{~~
u(r,\theta)
=
\frac{a_0}{2} +
\sum_{n=1}^\infty
a_n r^n \cos(n \theta) + b_n r^n \sin(n \theta) .
~~}
\end{equation*}

We look at the boundary condition in \eqref{dirichdisc:theprobeq},
\begin{equation*}
g(\theta) = u(1,\theta)
=
\frac{a_0}{2} +
\sum_{n=1}^\infty
a_n \cos(n \theta) + b_n \sin(n \theta) .
\end{equation*}
Therefore, to solve \eqref{dirichdisc:theprobeq}
we expand $g(\theta)$, which is 
a $2\pi$-periodic function, as a Fourier series, and then 
multiply the $n^{\text{th}}$ term by $r^n$.  To
find the $a_n$ and the $b_n$ we compute
\begin{equation*}
a_n =
\frac{1}{\pi} \int_{-\pi}^\pi g(\theta) \cos (n\theta) ~ d\theta , \qquad
\text{and} \qquad
b_n =
\frac{1}{\pi} \int_{-\pi}^\pi g(\theta) \sin (n\theta) ~ d\theta.
\end{equation*}

\begin{example}
Suppose we wish to solve
\begin{align*}
& \Delta u = 0 , \qquad 0 \leq r < 1, \quad -\pi < \theta \leq \pi,\\
& u(1,\theta) = \cos(10\,\theta), \qquad -\pi < \theta \leq \pi.
\end{align*}

The solution is
\begin{equation*}
u(r,\theta) = r^{10} \cos(10\,\theta) .
\end{equation*}

See the plot in \figurevref{dirichdisc:tenspeedfig}.
The thing to notice in this example is that the effect of a high frequency
is mostly felt at the boundary.  In the middle of the disc, the solution
is very close to zero.  That is because $r^{10}$ is rather small when $r$
is close to 0.
\begin{myfig}
\capstart
\diffyincludegraphics{width=5in}{width=7.5in}{dirichdisc-tenspeed}
\caption{The solution of the Dirichlet problem in the disc with
$\cos(10\,\theta)$ as boundary data.\label{dirichdisc:tenspeedfig}}
\end{myfig}
\end{example}

\begin{example}
Let us solve a more difficult problem.  Consider a long
rod with circular cross section of radius 1.  Suppose we wish to solve the
steady state heat problem in the rod.
If the rod is long enough, we simply need to solve
the Laplace equation in two dimensions.  Let us put the center of the rod at
the origin and we have exactly the region we are currently
studying---a circle of radius 1.  For the boundary conditions, suppose in
Cartesian coordinates $x$ and
$y$, the temperature on the boundary is 0 when $y < 0$, and it is $2y$ when $y > 0$.

Let us set the problem up.
As $y = r\sin(\theta)$, then on the
circle of radius 1, that is, where $r=1$, we have $2y = 2\sin(\theta)$.  So
%The problem becomes
\begin{align*}
& \Delta u = 0 , \qquad 0 \leq r < 1, \quad -\pi < \theta \leq \pi,\\
& u(1,\theta) = 
\begin{cases}
2\sin(\theta) & \text{if } \; \phantom{-}0 \leq \theta \leq \pi, \\
0 & \text{if } \; {-\pi} < \theta < 0.
\end{cases}
\end{align*}

We must now compute the Fourier series for the boundary
condition.  By now the reader has plentiful experience in computing
Fourier series and so we simply state that 
\begin{equation*}
u(1,\theta) = 
\frac{2}{\pi}
+
\sin(\theta)
+
\sum_{n=1}^\infty \frac{-4}{\pi(4n^2-1)} \cos(2n\theta) .
\end{equation*}

\begin{exercise}
Compute the series for $u(1,\theta)$ and verify that it really is what
we have just claimed.  Hint: Be careful, make sure not to divide by zero.
\end{exercise}

We now simply write the solution (see \figurevref{dirichdisc:zero2yfig}) by multiplying by $r^n$ in the right places.
\begin{equation*}
u(r,\theta) = 
\frac{2}{\pi}
+
r\sin(\theta)
+
\sum_{n=1}^\infty \frac{-4r^{2n}}{\pi(4n^2-1)} \cos(2n\theta) .
\end{equation*}
\begin{myfig}
\capstart
\diffyincludegraphics{width=5in}{width=7.5in}{dirichdisc-zero2y}
\caption{The solution of the Dirichlet problem with
boundary data 0 for $y < 0$ and $2y$ for $y > 0$.\label{dirichdisc:zero2yfig}}
\end{myfig}
%The plot of the solution is given in \figurevref{dirichdisc:zero2yfig}.
\end{example}

%Note that the formulas are not difficult to generalize
%for a circle of
%any radius and this generalization is left to the reader in the exercises.

\subsection{Poisson kernel}

There is another way to solve the Dirichlet problem with the help of an
integral kernel.  That is, we will find a function $P(r,\theta,\alpha)$
called the \emph{\myindex{Poisson kernel}}\footnote{%
Named for the French mathematician
\href{https://en.wikipedia.org/wiki/Sim\%C3\%A9on_Denis_Poisson}{Sim\'eon
Denis Poisson}
(1781--1840).} such that
\begin{equation*}
u(r,\theta) = 
\frac{1}{2\pi}
\int_{-\pi}^{\pi}
P(r,\theta,\alpha) \, g(\alpha) ~d\alpha .
\end{equation*}
While the integral will generally not be solvable analytically, it can
be evaluated numerically.   In fact, unless the boundary data is given
as a Fourier series already, it may be much easier to numerically
evaluate this formula as there is only one integral to evaluate.

The formula also has theoretical applications.
For instance, as $P(r,\theta,\alpha)$ 
will have infinitely many derivatives, then
via differentiating under the integral we find
that the solution $u(r,\theta)$ has infinitely many derivatives, at least
when inside the circle, $r < 1$.  By \myquote{having infinitely many
derivatives,} what you
should think of is that $u(r,\theta)$ has \myquote{no corners} and all of its
partial derivatives of all orders exist and also have \myquote{no corners.}

%A similar integral formula and an integral kernel
%always exists no matter what the shape of the region is,
%however it is much more difficult to compute when the region is
%very irregular.  For a circle, the formula we will obtain in the
%end is quite simple and has a nice geometric interpretation.

We will compute
the formula for $P(r,\theta,\alpha)$ from the series
solution, and this idea can be applied anytime you have a convenient
series solution where the coefficients are obtained via integration.
Hence you can apply this reasoning to obtain such integral kernels
for other equations, such as the heat equation.
The computation is long and \myindex{tedious}, but not overly difficult.
Since the ideas are often applied in similar contexts, it is good to
understand how this computation works.

What we do is start with the series solution and replace the coefficients
with the integrals that compute them.  Then we try to write everything as
a single integral.  We must use a different dummy variable for the
integration and hence we use $\alpha$ instead of $\theta$.
\begin{equation*}
\begin{split}
u(r,\theta)
& =
\frac{a_0}{2} +
\sum_{n=1}^\infty
a_n r^n \cos(n \theta) + b_n r^n \sin(n \theta)
\\
& =
\underbrace{
 \left(
  \frac{1}{2\pi} \int_{-\pi}^\pi g(\alpha) ~ d\alpha
 \right)
}_{\frac{a_0}{2}}
+
%\\
%& ~~~~~~~~ +
\sum_{n=1}^\infty
\underbrace{
 \left(
  \frac{1}{\pi} \int_{-\pi}^\pi g(\alpha) \cos (n\alpha) ~ d\alpha
 \right)
}_{a_n}
r^n \cos(n \theta) +
\\
& ~~~~~~~~ +
\underbrace{
 \left(
  \frac{1}{\pi} \int_{-\pi}^\pi g(\alpha) \sin (n\alpha) ~ d\alpha
 \right)
}_{b_n}
r^n \sin(n \theta)
\\
& =
\frac{1}{2\pi}
\int_{-\pi}^\pi
\left(  g(\alpha)
+
2
\sum_{n=1}^\infty
g(\alpha) \cos (n\alpha) 
\, r^n \cos(n \theta) +
g(\alpha) \sin (n\alpha)
\, r^n \sin(n \theta)
\right) ~d\alpha
\\
& =
\frac{1}{2\pi}
\int_{-\pi}^\pi
\underbrace{
 \left( 1
 +
 2
 \sum_{n=1}^\infty
 r^n 
 \bigl(
 \cos (n\alpha) 
 \cos(n \theta) +
 \sin (n\alpha)
 \sin(n \theta) \bigr)
 \right)
}_{P(r,\theta,\alpha)}
g(\alpha) ~d\alpha
%\\
%& =
%\frac{1}{2\pi}
%\int_{-\pi}^\pi
%\left( 1
%+
%2
%\sum_{n=1}^\infty
%r^n 
%\cos \bigl(n(\theta-\alpha)\bigr)
%\right) g(\alpha) ~d\alpha .
\end{split}
\end{equation*}
OK\@, so we have what we wanted, the expression in the parentheses is the
Poisson kernel, $P(r,\theta,\alpha)$.  However, we can do a lot better.  It is still given as a
series, and we would really like to have a nice simple expression for it. 
We must work a little harder.  The trick is to rewrite everything in terms of
complex exponentials.  Let us work
just on the kernel.
\begin{equation*}
\begin{split}
P(r,\theta,\alpha)
& =
1
+
2
\sum_{n=1}^\infty
r^n 
\bigl(
\cos (n\alpha) 
\cos(n \theta) +
\sin (n\alpha)
\sin(n \theta) \bigr)
\\
& =
1
+
2
\sum_{n=1}^\infty
r^n 
\cos \bigl(n(\theta-\alpha)\bigr)
\\
& =
1
+
\sum_{n=1}^\infty
r^n 
\bigl(
e^{in(\theta-\alpha)} +
e^{-in(\theta-\alpha)} \bigr)
\\
& =
1
+
\sum_{n=1}^\infty
{\bigl(
re^{i(\theta-\alpha)}\bigr)}^{n}
+
\sum_{n=1}^\infty
{\bigl(
re^{-i(\theta-\alpha)}\bigr)}^{n} .
\end{split}
\end{equation*}
In the expression above, we recognize the
\emph{\myindex{geometric series}}.
Recall from calculus that if $z$ is a complex number where $\lvert z \rvert < 1$, then
\begin{equation*}
\sum_{n=1}^\infty z^n = \frac{z}{1-z} .
\end{equation*}
Note that $n$ starts at $1$ and that is why we have the $z$ in the numerator.
It is the standard geometric series multiplied by $z$.
We can use $z = re^{i(\theta-\alpha)}$, as
lo and behold $\lvert re^{i(\theta-\alpha)} \rvert = r < 1$.
Let us
continue with the computation.
\begin{equation*}
\begin{split}
P(r,\theta,\alpha)
& =
1
+
\sum_{n=1}^\infty
{\bigl(
re^{i(\theta-\alpha)}\bigr)}^{n}
+
\sum_{n=1}^\infty
{\bigl(
re^{-i(\theta-\alpha)}\bigr)}^{n}
\\
& =
1
+
\frac{re^{i(\theta-\alpha)}}{1-re^{i(\theta-\alpha)}}
+
\frac{re^{-i(\theta-\alpha)}}{1-re^{-i(\theta-\alpha)}}
\\
& = 
\frac{
\bigl(1-re^{i(\theta-\alpha)}\bigr)\bigl(1-re^{-i(\theta-\alpha)}\bigr)
+
\bigl(1-re^{-i(\theta-\alpha)}\bigr)re^{i(\theta-\alpha)} +
\bigl(1-re^{i(\theta-\alpha)}\bigr)re^{-i(\theta-\alpha)}}
{\bigl(1-re^{i(\theta-\alpha)}\bigr)\bigl(1-re^{-i(\theta-\alpha)}\bigr)}
\\
& = 
\frac{1 -r^2}{1 - re^{i(\theta-\alpha)} - re^{-i(\theta-\alpha)} +r^2}
\\
& = 
\frac{1 -r^2}{1 - 2r\cos(\theta-\alpha) +r^2} .
\end{split}
\end{equation*}
That's a formula we can live with.  The
solution to the Dirichlet problem using the Poisson kernel is
\begin{equation*}
\mybxbg{~~
u(r,\theta) = 
\frac{1}{2\pi} \int_{-\pi}^{\pi}
\frac{1 -r^2}{1 - 2r\cos(\theta-\alpha) +r^2} g(\alpha) ~ d\alpha .
~~}
\end{equation*}
Sometimes the formula for the Poisson kernel is
given together with the constant $\frac{1}{2\pi}$, in which case we should
of course not leave it in front of the integral.
Also, often the limits
of the integral are given as 0 to $2\pi$; everything inside is
$2\pi$-periodic in $\alpha$, so this does not change the integral.

%12 is the number of lines, must be adjusted
\begin{mywrapfigsimp}[12]{2.1in}{2.4in}
\diffypdfversion{\vspace*{5pt}}
\noindent
\inputpdft{poisson}
\end{mywrapfigsimp}
Let us not leave the Poisson kernel without explaining its geometric
meaning.  Let $s$ be the distance from $(r,\theta)$ to
$(1,\alpha)$.
You may recall from calculus that
this distance $s$ in polar coordinates is given precisely by the square root
of $1 - 2r\cos(\theta-\alpha) +r^2$.  That is, the Poisson kernel is really
the formula
\begin{equation*}
\frac{1-r^2}{s^2} .
\end{equation*}
%See the figure on the right.

One final note we make about the formula is that it is really
a weighted average of the boundary values.
First let us look
at what happens at the origin,
that is when $r=0$. %(and then $\theta$ can be anything at all)
\begin{equation*}
\begin{split}
u(0,0) &= 
\frac{1}{2\pi} \int_{-\pi}^{\pi}
\frac{1 -0^2}{1 - 2(0)\cos(\theta-\alpha) +0^2} g(\alpha) ~ d\alpha
\\
& =
\frac{1}{2\pi} \int_{-\pi}^{\pi}
g(\alpha) ~ d\alpha .
\end{split}
\end{equation*}
So $u(0,0)$ is precisely the average value of $g(\theta)$ and
therefore the average value of $u$ on the boundary.  This is
a general feature of harmonic functions, the value at some point $p$
is equal to the average of the values on a circle centered at $p$.

What the formula says is that the value of the solution at any point in the
circle is a weighted average of the boundary data $g(\theta)$.  The kernel
is bigger when $(1,\alpha)$ is closer to $(r,\theta)$.  Therefore when
computing $u(r,\theta)$ we
give more weight to the values $g(\alpha)$ when $(1,\alpha)$ is closer to $(r,\theta)$ and less
weight to the values $g(\alpha)$ when $(1,\alpha)$ far from $(r,\theta)$.

\subsection{Exercises}

\begin{exercise}
Using series solve
$\Delta u = 0$, $u(1,\theta) = \lvert \theta \rvert$, for $-\pi < \theta
\leq \pi$.
\end{exercise}

\begin{exercise}\ansMark%
Using series solve
$\Delta u = 0$, $u(1,\theta) = 1+ \sum\limits_{n=1}^\infty \frac{1}{n^2}\sin(n\theta)$.
\end{exercise}
\exsol{%
$u = 1+ \sum\limits_{n=1}^\infty \frac{1}{n^2}r^n\sin(n\theta)$
}

\begin{exercise}\ansMark%
Using the series solution find the solution to
$\Delta u = 0$, $u(1,\theta) = 1- \cos(\theta)$.  Express the solution
in Cartesian coordinates (that is, using $x$ and $y$).
\end{exercise}
\exsol{%
$u = 1-x$
}

\begin{exercise}
Using series solve $\Delta u = 0$, $u(1,\theta) = g(\theta)$ for the
following data.  Hint: trig identities.
\begin{tasks}(2)
\task
$g(\theta) = 
\nicefrac{1}{2} + 3\sin(\theta) + \cos(3\theta)$
\task
$g(\theta) = 
3\cos(3\theta) + 3\sin(3\theta) + \sin(9\theta)$
\task
$g(\theta) = 2 \cos(\theta+1)$
\task
$g(\theta) = \sin^2(\theta)$
\end{tasks}
\end{exercise}

\begin{exercise}
Using the Poisson kernel, give the solution to
$\Delta u = 0$, where $u(1,\theta)$ is zero for $\theta$ outside
the interval $[-\nicefrac{\pi}{4},\nicefrac{\pi}{4}]$ and 
$u(1,\theta)$ is 1 for $\theta$ on the interval
$[-\nicefrac{\pi}{4},\nicefrac{\pi}{4}]$.
\end{exercise}

\begin{exercise}
\pagebreak[2]
\leavevmode
\begin{tasks}
\task Draw a graph for the Poisson kernel as a function of $\alpha$
when $r=\nicefrac{1}{2}$ and $\theta = 0$.
\task Describe what happens to the graph when you make $r$ bigger (as it
approaches 1).
\task Knowing that the solution $u(r,\theta)$ is the weighted average
of $g(\theta)$ with Poisson kernel as the weight, explain what your answer
to part b) means.
\end{tasks}
\end{exercise}

\begin{exercise} \label{exercise:dirichproblemxy}
Let $g(\theta)$ be the function $xy = \cos \theta \sin
\theta$ on the boundary.  Use the series solution to find a solution
to the Dirichlet problem $\Delta u = 0$, $u(1,\theta) = g(\theta)$.  Now
convert the solution to Cartesian coordinates $x$ and $y$.  Is this
solution surprising?  Hint: use your trig identities.
\end{exercise}

\begin{exercise}\ansMark%
\leavevmode
\begin{tasks}
\task
Try and guess a solution to $\Delta u = -1$, $u(1,\theta) = 0$.
Hint: try a solution that only depends on $r$.  Also first, don't worry
about the boundary condition.
\task
Now solve $\Delta u = -1$, $u(1,\theta) = \sin(2\theta)$ using
superposition.
\end{tasks}
\end{exercise}
\exsol{%
a) $u = \frac{-1}{4} r^2 + \frac{1}{4}$
b) $u = \frac{-1}{4} r^2 + \frac{1}{4} + r^2 \sin(2\theta)$
}

\begin{exercise}
Carry out the computation we needed in the separation of variables and solve
$r^2 R'' + r R' - n^2 R = 0$, for $n=0,1,2,3,\ldots$.
\end{exercise}

\begin{exercise}[challenging]
Derive the series solution to the Dirichlet problem if the region is a
circle of radius $\rho$ rather
than 1.
That is, solve $\Delta u = 0$, $u(\rho,\theta) = g(\theta)$.
\end{exercise}

\begin{exercise}[challenging]
\leavevmode
\begin{tasks}
\task
Find the solution for
$\Delta u = 0$, $u(1,\theta) = x^2y^3 + 5 x^2$.  Write the answer in Cartesian coordinates.
\task
Now solve
$\Delta u = 0$, $u(1,\theta) = x^k y^\ell$.
Write the solution in Cartesian coordinates.
\task
Suppose you have a polynomial $P(x,y) = \sum_{j=0}^m \sum_{k=0}^n c_{j,k}
x^j y^k$, solve $\Delta u = 0$, $u(1,\theta) = P(x,y)$ (that is, write down
the formula for the answer).  Write the answer
in Cartesian coordinates.
\end{tasks}
Notice the answer is again a polynomial in $x$ and $y$.
See also \exerciseref{exercise:dirichproblemxy}.
\end{exercise}

\begin{exercise}[challenging]\ansMark%
Derive the Poisson kernel solution
if the region is a circle of radius $\rho$ rather
than 1.  That is, solve $\Delta u = 0$, $u(\rho,\theta) = g(\theta)$.
\end{exercise}
\exsol{%
$\displaystyle
u(r,\theta) = 
\frac{1}{2\pi} \int_{-\pi}^{\pi}
\frac{\rho^2 -r^2}{\rho - 2r\rho\cos(\theta-\alpha) +r^2} g(\alpha) ~ d\alpha$
}

\setcounter{exercise}{100}

