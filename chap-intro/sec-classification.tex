\section{Classification of differential equations}
\label{classification:section}

%Perhaps no [EP] ref?
\LAtt{0.3}

\LO{
\item Classify equation as ordinary or partial differential equations,
\item Identify whether an equation is linear or non-linear, and 
\item Classify linear equations as homogenoeus, non-homogenoeus, or constant coefficient, as appropriate.
}

% \sectionnotes{less than 1 lecture or left as reading\BDref{, \S1.3 in \cite{BD}}}

There are many types of differential equations, and we classify them into
different categories based on their properties.  Let us quickly go over
the most basic classification.  We already saw the distinction
between ordinary and partial differential equations:

\begin{definition}
\begin{itemize}
\item
\emph{Ordinary differential equations}
\index{Ordinary differential equations}\index{ODE} or (ODE) are
equations where the derivatives are taken with respect to only one variable.
That is, there is only one independent variable.
\item
\emph{Partial differential equations}
\index{Partial differential equations}\index{PDE} or (PDE) are
equations that depend on partial derivatives of several variables.
That is, there are several independent variables.
\end{itemize}
\end{definition}

Let us see some examples of ordinary differential equations:
\begin{align*}
& \frac{d y}{dt} = ky , & & \text{(Exponential growth\index{exponential growth})} \\
& \frac{d y}{dt} = k(A-y) , & & \text{(\myindex{Newton's law of cooling})} \\
& m \frac{d^2 x}{dt^2} + c \frac{dx}{dt} + kx = f(t) . & &
\text{(Mechanical vibrations\index{mechanical vibrations})}
\end{align*}
And of partial differential equations:
\begin{align*}
& \frac{\partial y}{\partial t} + c \frac{\partial y}{\partial x} = 0 , & & 
\text{(Transport equation\index{transport equation})} \\
& \frac{\partial u}{\partial t} = \frac{\partial^2 u}{\partial x^2} , & & 
\text{(Heat equation\index{heat equation})} \\
& \frac{\partial^2 u}{\partial t^2} = \frac{\partial^2 u}{\partial x^2} +
\frac{\partial^2 u}{\partial y^2} . & & 
\text{(Wave equation in 2 dimensions\index{wave equation in 2 dimensions})}
\end{align*}

If there are several equations working together, we have a so-called
\emph{\myindex{system of differential equations}}.  For example,
\begin{equation*}
y' = x , \qquad x' = y
\end{equation*}
is a simple system of ordinary differential equations.
\myindex{Maxwell's equations} for electromagnetics,
\begin{align*}
& \nabla \cdot \vec{D} = \rho, & & \nabla \cdot \vec{B} = 0 , \\
& \nabla \times \vec{E} = - \frac{\partial \vec{B}}{\partial t}, &
& \nabla \times \vec{H} = \vec{J} + \frac{\partial \vec{D}}{\partial t} ,
\end{align*}
are a system of partial differential equations. 
The divergence operator $\nabla \cdot$ and the
curl operator $\nabla \times$ can be written out in partial derivatives of
the functions involved in the $x$, $y$, and $z$ variables.

\medskip

%The next bit of information is the \emph{\myindex{order}} of the
%equation (or system).  The order is simply the order of the largest
%derivative that appears.  If the highest derivative that appears is
%the first derivative, the equation is of first order.  If the highest
%derivative that appears is the second derivative, then the equation is of second
%order.  For example, Newton's law of cooling above is a first order
%equation, while the mechanical vibrations equation is a second order equation.
%The equation governing transversal vibrations in a beam,
%\begin{equation*}
%a^4 \frac{\partial^4 y}{\partial x^4} + \frac{\partial^2 y}{\partial t^2} = 0,
%\end{equation*}
%is a fourth order partial differential equation.  It is
%fourth order as at least one derivative is the fourth derivative.  It
%does not matter that the derivative in $t$ is only of second order.

In the first chapter, we will start attacking first order ordinary
differential equations, that is, equations of the form $\frac{dy}{dx} = f(x,y)$.
In general, lower order equations are easier to work with and have simpler
behavior, which is why we start with them.

\medskip

We also distinguish how the dependent variables appear in the equation (or
system).  
\begin{definition}
We say an equation is
\emph{linear}\index{linear equation} if the
dependent variable (or variables) and their derivatives appear linearly,
that is only as first powers, they are not multiplied together, and no other functions of the dependent
variables appear.  
Otherwise, the equation is called
\emph{nonlinear}\index{nonlinear equation}.
\end{definition}

Another way to determine if a differential equation is linear is if the equation is a sum of terms,
where each term is
some function of the independent variables
or 
some function of the independent variables
multiplied by a dependent variable
or its derivative. That is,
an ordinary differential equation is linear if it can be
put into the form
\begin{equation} \label{classification:eqlingen}
a_n(x) \frac{d^n y}{dx^n} + 
a_{n-1}(x) \frac{d^{n-1} y}{dx^{n-1}} + 
\cdots
+
a_{1}(x) \frac{dy}{dx}
+
a_{0}(x) y = b(x) .
\end{equation}
The functions $a_0$, $a_1$, \ldots, $a_n$ are called the
\emph{\myindex{coefficients}}.
The equation is allowed to depend arbitrarily on the independent variable.
So 
\begin{equation} \label{classification:eqlinex}
e^x \frac{d^2 y}{dx^2} + 
\sin(x) \frac{d y}{dx} + 
x^2 y
=
\frac{1}{x}
\end{equation}
is still a linear equation as $y$ and its derivatives only appear linearly. The equation
\[ \cos(x) \frac{d^2y}{dx^2} - xy + \frac{e^x}{x} = 0 \] is also linear, even though it
is not initially in the correct form. From this equation, we can move the last term over to the right-hand
side as a $-\frac{e^x}{x}$, and then it is in the correct form, with the $\frac{dy}{dx}$ term missing (or has coefficient zero).

All the equations and systems above as examples are linear.  
It may not be immediately obvious for Maxwell's equations unless you write out
the divergence and curl in terms of partial derivatives.  Let us see some
nonlinear equations.  For example \myindex{Burger's equation},
\begin{equation*}
\frac{\partial y}{\partial t} + 
y \frac{\partial y}{\partial x} =
\nu \frac{\partial^2 y}{\partial x^2} ,
\end{equation*}
is a nonlinear second order partial differential equation.  It is nonlinear
because $y$ and $\frac{\partial y}{\partial x}$ are multiplied together.
The equation
\begin{equation} \label{classification:eqnonlinode}
\frac{dx}{dt} = x^2
\end{equation}
is a nonlinear first order differential equation as there is a second power of
the dependent variable $x$.

\medskip

\begin{definition}
A linear equation may further be called \emph{\myindex{homogeneous}} if
all terms depend on the dependent variable.  That is, if no
term is a function of the independent variables alone.  Otherwise, the
equation is called \emph{\myindex{nonhomogeneous}} or
\emph{\myindex{inhomogeneous}}.
\end{definition}

For example,
the exponential growth equation, the wave equation, or the transport equation above
are homogeneous. The mechanical vibrations equation above is nonhomogeneous
as long as $f(t)$ is not the zero function.  Similarly, if the ambient temperature $A$ is nonzero,
Newton's law of cooling is nonhomogeneous.
A homogeneous linear ODE can be put into the form
\begin{equation*}
a_n(x) \frac{d^n y}{dx^n} + 
a_{n-1}(x) \frac{d^{n-1} y}{dx^{n-1}} + 
\cdots
+
a_{1}(x) \frac{dy}{dx}
+
a_{0}(x) y = 0 .
\end{equation*}
Compare to \eqref{classification:eqlingen} and notice there is no
function $b(x)$.

\medskip

If the coefficients of a linear equation are actually constant functions,
then the equation is said to have
\emph{constant coefficients}\index{constant coefficient}.
The coefficients are the functions multiplying the dependent
variable(s) or one of its derivatives, not the function $b(x)$ standing alone.
A constant coefficient nonhomogeneous ODE is an equation of the form
\begin{equation*}
a_n \frac{d^n y}{dx^n} + 
a_{n-1} \frac{d^{n-1} y}{dx^{n-1}} + 
\cdots
+
a_{1} \frac{dy}{dx}
+
a_{0} y = b(x) ,
\end{equation*}
where $a_0, a_1, \ldots, a_n$ are all constants,
but $b$ may depend on 
the independent variable $x$.
The mechanical vibrations equation
above is a constant coefficient nonhomogeneous second order ODE\@.
The same nomenclature applies to PDEs, so the transport equation,
heat equation and wave equation are all examples of constant coefficient
linear PDEs.

\medskip

Finally, an equation (or system) is called \emph{\myindex{autonomous}}
if the equation does not explicitly depend on the independent variable.
For autonomous ordinary differential equations, the
independent variable is then thought of as time.  Autonomous equation
means an equation that does not change with time.
For example, Newton's law of cooling is autonomous, so is equation
\eqref{classification:eqnonlinode}.  On the other hand, mechanical
vibrations or 
\eqref{classification:eqlinex} are not autonomous.

\subsection{Exercises}

\begin{exercise}
Classify the following equations.  Are they ODE or PDE\@?  Is it an equation
or a system?  What is the order?  Is it linear or nonlinear, and if it is
linear, is it homogeneous, constant coefficient?  If it is an ODE\@, is it
autonomous?
\begin{tasks}(2)
\task $\displaystyle \sin(t) \frac{d^2 x}{dt^2} + \cos(t) x = t^2$
\task $\displaystyle \frac{\partial u}{\partial x} + 3 \frac{\partial u}{\partial y} = xy$
\task $\displaystyle y''+3y+5x=0, \quad x''+x-y=0$
\task $\displaystyle \frac{\partial^2 u}{\partial t^2} + u\frac{\partial^2 u}{\partial s^2} =
0$
\task $\displaystyle x''+tx^2=t$
\task $\displaystyle \frac{d^4 x}{dt^4} = 0$
\end{tasks}
\end{exercise}

\begin{exercise}\ansMark%
Classify the following equations.  Are they ODE or PDE\@?  Is it an equation
or a system?  What is the order?  Is it linear or nonlinear, and if it is
linear, is it homogeneous, constant coefficient?  If it is an ODE\@, is it
autonomous?
\begin{tasks}(2)
\task $\displaystyle \frac{\partial^2 v}{\partial x^2} + 3 \frac{\partial^2
v}{\partial y^2} = \sin(x)$
\task $\displaystyle \frac{d x}{dt} + \cos(t) x = t^2+t+1$
\task $\displaystyle \frac{d^7 F}{dx^7} = 3F(x)$
\task $\displaystyle y''+8y'=1$
\task $\displaystyle x''+tyx'=0, \quad y''+txy = 0$
\task $\displaystyle \frac{\partial u}{\partial t} = \frac{\partial^2 u}{\partial s^2} + u^2$
\end{tasks}
\end{exercise}
\exsol{%
a) 
PDE\@, equation, second order, linear, nonhomogeneous, constant coefficient.\\
b) 
ODE\@, equation, first order, linear, nonhomogeneous, not constant coefficient, not autonomous.\\
c) 
ODE\@, equation, seventh order, linear, homogeneous, constant coefficient, autonomous.\\
d) 
ODE\@, equation, second order, linear, nonhomogeneous, constant coefficient, autonomous.\\
e) 
ODE\@, system, second order, nonlinear.\\
f) 
PDE\@, equation, second order, nonlinear.
}

\begin{exercise}
If $\vec{u} = (u_1,u_2,u_3)$ is a vector, we have the divergence
$\nabla \cdot \vec{u} =
\frac{\partial u_1}{\partial x} +
\frac{\partial u_2}{\partial y} +
\frac{\partial u_3}{\partial z}$ and curl
$\nabla \times \vec{u} =
\Bigl(
\frac{\partial u_3}{\partial y} - \frac{\partial u_2}{\partial z} , ~
\frac{\partial u_1}{\partial z} - \frac{\partial u_3}{\partial x} , ~
\frac{\partial u_2}{\partial x} - \frac{\partial u_1}{\partial y} \Bigr)$.
Notice that curl of a vector is still a vector.  Write out Maxwell's
equations in terms of partial derivatives and classify the system.
\end{exercise}

\begin{exercise}
Suppose $F$ is a linear function, that is,
$F(x,y) = ax+by$ for constants $a$ and $b$.  What is the
classification of equations of the form $F(y',y) = 0$.
\end{exercise}

\begin{exercise}
Write down an explicit example of a third order, linear, nonconstant coefficient,
nonautonomous, nonhomogeneous system of two ODE such that every derivative
that could appear, does appear.
\end{exercise}

\begin{exercise}\ansMark%
Write down the general \emph{zero}th order linear ordinary differential
equation.  Write down the general solution.
\end{exercise}
\exsol{%
equation: $a(x) y = b(x)$, solution: $y = \frac{b(x)}{a(x)}$.
}

\begin{exercise}\ansMark%
For which $k$ is $\frac{dx}{dt}+x^k = t^{k+2}$ linear.  Hint: there are two answers.
\end{exercise}
\exsol{%
$k=0$ or $k=1$
}

\begin{exercise}
Write out an explicit example of a non-homogeneous fourth order, linear, constant coefficient differential equation. where all possible derivatives of the unknown function $y$ appear. 
\end{exercise}

\begin{exercise}\ansMark%
Let $x$, $y$, and $z$ be three functions of $t$ defined by the system of differential equations
\begin{equation*} 
x' = y \quad y' = z \quad z' = 3x - 2y + 5z + e^t
\end{equation*}

with initial conditions $x(0) = 3$, $y(0) = -2$ and $z(0) = 1$, and let $u(t)$ be the function defined by the solution to
\begin{equation*}
u''' - 5u'' + 2u' - 3u = e^t
\end{equation*}
with initial conditions $u(0) = 3$, $u'(0) = -2$, and $u''(0) = 1$. 
\begin{tasks}
\task Use the substitution $u=x$, $u' = y$, and $u'' = z$ to verify that $x(t) = u(t)$ because they solve the same initial value problem.
\task What is the order of the system defining $x$, $y$, and $z$ and how many components does it have?
\task What is the order of the equation defining $u$? How many components does that have?
\task How do these numbers relate to each other?
\end{tasks}
\end{exercise}
\exsol{%
b) First order with three components.\\
c) Third order with one component. \\
d) The product is three in both cases. ($1 \times 3 = 3 \times 1$).
}

\setcounter{exercise}{100}







