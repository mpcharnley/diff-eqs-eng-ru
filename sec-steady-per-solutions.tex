\section{Steady periodic solutions}
\label{sps:section}

\sectionnotes{Verbatim from Lebl}

\sectionnotes{1--2 lectures\EPref{, \S10.3 in \cite{EP}}\BDref{,
not in \cite{BD}}}

\subsection{Forced vibrating string}

Consider a guitar string of length $L$.  We studied this 
setup in \sectionref{we:section}.
Let $x$ be the position on the string, $t$ the time, and $y$ the displacement of the string.  See
\figurevref{sps:vibstrfig}.

\begin{myfig}
\capstart
\inputpdft{sps-vibstr}
\caption{Vibrating string.\label{sps:vibstrfig}}
\end{myfig}

The problem is governed by the wave equation
\begin{equation} \label{sps:freevib}
\begin{array}{ll}
y_{tt} = a^2 y_{xx} , & \\
\noalign{\smallskip}
y(0,t) = 0 , & y(L,t) = 0 , \\
y(x,0) = f(x) , & y_t(x,0) = g(x) .
\end{array}
\end{equation}
We found that the solution is of the form
\begin{equation*}
y = 
\sum_{n=1}^\infty \left( A_n \cos \left( \frac{n\pi a}{L} t \right) +
B_n \sin \left( \frac{n\pi a}{L} t \right) \right)
\sin \left( \frac{n\pi}{L} x \right) ,
\end{equation*}
where $A_n$ and $B_n$ are determined by the initial conditions.  The natural
frequencies of the system are the (angular) frequencies $\frac{n \pi a}{L}$
for integers $n \geq 1$.

But these are free vibrations.  What if there is an external force acting on
the string.  Let us assume say air vibrations (noise), for example from a second
string.  Or perhaps a jet engine.  For simplicity, assume nice pure
sound and assume the force is uniform at every position on the string.
Let us say $F(t) = F_0 \cos (\omega t)$ as force per unit mass.  Then our wave
equation becomes (remember force is mass times acceleration)
\begin{equation} \label{sps:forcedeq}
y_{tt} = a^2 y_{xx} + F_0 \cos ( \omega t) ,
\end{equation}
with the same boundary conditions of course.

We want to find the solution here that satisfies the equation above and
\begin{equation} \label{sps:forcedinitcond}
y(0,t) = 0, \qquad y(L,t) = 0, \qquad
y(x,0) = 0, \qquad y_t(x,0) = 0.
\end{equation}
That is, the string is initially at rest.  First we find a particular
solution $y_p$ of \eqref{sps:forcedeq} that satisfies
$y(0,t) = y(L,t) = 0$.  We define the functions $f$ and $g$ as
\begin{equation*}
f(x) = -y_p(x,0), \qquad g(x) = -\frac{\partial y_p}{\partial t} (x,0) .
\end{equation*}
We then find solution $y_c$ of \eqref{sps:freevib}.  If we add the two
solutions, we find that $y = y_c + y_p$ solves \eqref{sps:forcedeq} with
the initial conditions.

\begin{exercise}
Check that $y = y_c + y_p$ solves \eqref{sps:forcedeq} and the
side conditions \eqref{sps:forcedinitcond}.
\end{exercise}

So the big issue here is to find the particular solution $y_p$.
We look at the equation and we make an educated guess
\begin{equation*}
y_p(x,t) = X(x) \cos (\omega t) .
\end{equation*}
We plug in to get
\begin{equation*}
-\omega^2 X \cos ( \omega t) = a^2 X'' \cos ( \omega t) +
F_0 \cos ( \omega t ) ,
\end{equation*}
or
$-\omega^2 X = a^2 X'' + F_0$ after canceling the cosine.
We know how to find a general solution to this equation (it is a
nonhomogeneous constant coefficient equation).
The general solution is
\begin{equation*}
X(x) = A \cos \left( \frac{\omega}{a} x \right)
+ B \sin \left( \frac{\omega}{a} x \right) -
\frac{F_0}{\omega^2} .
\end{equation*}
The endpoint conditions imply $X(0) = X(L) = 0$.  So
\begin{equation*}
0 = X(0) = A - \frac{F_0}{\omega^2} ,
\end{equation*}
or $A = \frac{F_0}{\omega^2}$, and also
\begin{equation*}
0 = X(L)
= \frac{F_0}{\omega^2} \cos \left( \frac{\omega L}{a} \right)
+ B \sin \left( \frac{\omega L}{a} \right) -
\frac{F_0}{\omega^2} .
\end{equation*}
Assuming that $\sin ( \frac{\omega L}{a} )$ is not zero we can solve for $B$ to
get
\begin{equation} \label{natfreq:Beq}
B = 
\frac{-F_0 \left( \cos \left( \frac{\omega L}{a} \right) - 1 \right)}%
{\omega^2 \sin \left( \frac{\omega L}{a} \right)}.
\end{equation}
Therefore,
\begin{equation*}
X(x) =
\frac{F_0}{\omega^2} \left(
\cos \left(  \frac{\omega}{a} x \right) -
\frac{\cos \left( \frac{\omega L}{a} \right) - 1}%
{\sin \left( \frac{\omega L}{a} \right)}
\sin \left( \frac{\omega}{a} x \right)
- 1
\right) .
\end{equation*}
The particular solution $y_p$ we are looking for is
\begin{equation*}
\mybxbg{~~
y_p(x,t) =
\frac{F_0}{\omega^2} \left(
\cos \left( \frac{\omega}{a} x \right) -
\frac{\cos \left( \frac{\omega L}{a} \right) - 1}%
{\sin \left( \frac{\omega L}{a}\right)}
\sin \left( \frac{\omega}{a} x \right)
-1
\right)
\cos ( \omega t) .
~~}
\end{equation*}

\begin{exercise}
Check that $y_p$ works.
\end{exercise}

Now we get to the point that we skipped.  Suppose 
$\sin ( \frac{\omega L}{a} ) = 0$.  What this means is that
$\omega$ is equal to one of the natural frequencies of the system,
i.e.\ a multiple of $\frac{\pi a}{L}$.  We notice that if $\omega$
is not equal to a multiple of the base frequency, but is very close,
then the coefficient $B$ in \eqref{natfreq:Beq} seems to
become very large.  But let us not jump to conclusions just yet.
When $\omega = \frac{n \pi a}{L}$
for $n$ even, then $\cos (\frac{\omega L}{a}) = 1$ and hence we really get that
$B=0$.  So resonance occurs only when 
both $\cos (\frac{\omega L}{a}) = -1$ and
$\sin (\frac{\omega L}{a}) = 0$.  That is when $\omega = \frac{n \pi a }{L}$
for \emph{odd} $n$.

We could again solve for the resonance solution if we wanted to, but it is, in the right sense, the limit of the solutions as $\omega$ gets
close to a resonance frequency.
In real life, pure resonance never occurs anyway.
%You would get a particular solution that would have a term
%such as $t \cos (\omega t)$ in it (though it would be harder (but possible)
%to find the solution using the method above).

The calculation above explains why a string begins to vibrate if the
identical string is plucked close by.  In the absence of friction this vibration
would get louder and louder as time goes on.
On the other hand, you are unlikely to get large vibration if the forcing 
frequency is not close to a resonance frequency even if you have a jet engine
running close to
the string.  That is, the amplitude does not keep
increasing unless you tune to just the right frequency.

Similar resonance phenomena occur when you break a wine glass using human
voice (yes
this is possible, but not easy%
\footnote{\emph{Mythbusters}, episode 31, Discovery Channel, originally aired
may 18th 2005.}) if you happen to hit just the right
frequency.  Remember a glass has much purer sound, i.e.\ it is more like a
vibraphone, so there are far fewer resonance frequencies to hit.

When the forcing function is more complicated, you decompose it in terms of
the Fourier series and apply the result above.  You may also need to solve
the problem above if the forcing function is a sine rather than a cosine,
but if you think about it, the solution is almost the same.
%  That is, the
%particular solution has $\sin$ instead of cosine and otherwise is identical.
%Your complementary solution will be slightly different as $f$ and $g$
%will differ.

\begin{example}
Let us do the computation for specific values.
Suppose $F_0 = 1$ and $\omega = 1$ and $L=1$ and $a=1$.  Then 
\begin{equation*}
y_p(x,t) =
\left(
\cos (x) -
\frac{\cos (1) - 1}{\sin (1)}
\sin (x)
-1
\right)
\cos (t) .
\end{equation*}
Write $B = \frac{\cos (1) - 1}{\sin (1)}$ for simplicity.

Then plug in $t=0$ to get
\begin{equation*}
f(x) =- y_p(x,0) = 
- \cos x +
B \sin x
+1 ,
\end{equation*}
and after differentiating in $t$ we see that 
$g(x) = -\frac{\partial y_p}{\partial t}(x,0) = 0$.

Hence to find $y_c$ we need to solve the problem
\begin{align*}
& y_{tt} = y_{xx} , \\
& y(0,t) = 0 , \quad y(1,t) = 0 , \\
& y(x,0) = - \cos x + B \sin x +1 , \\
& y_t(x,0) = 0 .
\end{align*}
The formula that we use to define $y(x,0)$ is not odd,
hence it is not a simple matter of plugging in the expression for $y(x,0)$
to the d'Alembert
formula directly!  You must define $F$ to be the odd, 2-periodic
extension of $y(x,0)$.  Then our solution is
\begin{equation} \label{natfreq:exsol}
y(x,t) = 
\frac{F(x+t) + F(x-t)}{2} + 
\left(
\cos (x) -
\frac{\cos (1) - 1}{\sin (1)}
\sin (x)
-1
\right)
\cos (t) .
\end{equation}

It is not hard to compute specific values
for an odd periodic extension of a function and
hence \eqref{natfreq:exsol} is a wonderful solution to the problem.
For example, it is very easy to have a computer do it, unlike a series solution.
A plot is given in \figurevref{natfreq:forcedvibfig}.
\begin{myfig}
\capstart
\diffyincludegraphics{width=5in}{width=7.5in}{natfreq-forcedvib}
\caption{Plot of $y(x,t) = \frac{F(x+t) + F(x-t)}{2} + \left( \cos (x) -
\frac{\cos (1) - 1}{\sin (1)} \sin (x) -1 \right) \cos (t)$.%
\label{natfreq:forcedvibfig}}
\end{myfig}
\end{example}

\subsection{Underground temperature oscillations}

Let $u(x,t)$ be the temperature at a certain location at depth $x$
underground at time $t$.  See \figurevref{sps:groundtempfig}.

\begin{mywrapfig}{2.65in}
\capstart
\inputpdft{sps-groundtemp}
\caption{Underground temperature.\label{sps:groundtempfig}}
\end{mywrapfig}

The temperature $u$ satisfies the heat equation $u_t = ku_{xx}$, where $k$
is the diffusivity of the soil.
We know the temperature at the surface $u(0,t)$ from weather
records.  Let us assume for simplicity that
\begin{equation*}
u(0,t) = T_0 + A_0 \cos (\omega t) ,
\end{equation*}
where $T_0$ is the yearly mean
temperature, and
$t=0$ is midsummer (you can put
negative sign above to make it midwinter if you wish).  $A_0$ gives 
the typical variation for the year.  That is,
the hottest temperature is $T_0 + A_0$ and the coldest is $T_0 - A_0$.
For simplicity, we assume that $T_0 = 0$.
The frequency $\omega$ is picked depending on the units of $t$, such that
when $t=\unit[1]{year}$, then $\omega t = 2 \pi$.  For example if $t$ is
in years, then $\omega = 2\pi$.

It seems reasonable that the temperature at depth $x$ also oscillates
with the same frequency.  This, in fact, is the steady periodic
solution, a solution independent of the initial conditions.
So we are looking for a solution of the form
\begin{equation*}
u(x,t) = V(x) \cos (\omega t) + W (x) \sin ( \omega t)
\end{equation*}
for the problem
\begin{equation} \label{sps:ueq}
u_t = k u_{xx}, \qquad u(0,t) = A_0 \cos ( \omega t) .
\end{equation}

We employ the complex exponential here to make calculations simpler.
Suppose we have a complex-valued function
\begin{equation*}
h(x,t) = X(x)\, e^{i\omega t} .
\end{equation*}
We look for an $h$ such that $\operatorname{Re} h = u$.
To find an $h$, whose real part satisfies \eqref{sps:ueq}, we look for
an $h$ such that
\begin{equation} \label{sps:heq}
h_t = k h_{xx}, \qquad h(0,t) = A_0 e^{i\omega t} .
\end{equation}

\begin{exercise}
Suppose $h$ satisfies \eqref{sps:heq}.
Use \hyperref[eulersformula]{Euler's formula} for the complex exponential to
check that $u = \operatorname{Re} h$ satisfies \eqref{sps:ueq}.
\end{exercise}

Substitute $h$ into \eqref{sps:heq}.
\begin{equation*}
i\omega X e^{i\omega t} = k X'' e^{i \omega t} .
\end{equation*}
Hence,
\begin{equation*}
k X''  - i \omega X = 0 ,
\end{equation*}
or 
\begin{equation*}
X''  - \alpha^2 X = 0 ,
\end{equation*}
where $\alpha = \pm \sqrt{\frac{i\omega}{k}}$.  Note that $\pm \sqrt{i} = \pm
\frac{1+i}{\sqrt{2}}$ so you could simplify to
$\alpha = \pm (1+i)\sqrt{\frac{\omega}{2k}}$.
Hence the general solution is
\begin{equation*}
X(x) = A e^{-(1+i)\sqrt{\frac{\omega}{2k}} \, x}
+ B e^{(1+i)\sqrt{\frac{\omega}{2k}} \, x} .
\end{equation*}
We assume that an $X(x)$ that solves the problem must be bounded as $x \to
\infty$ since $u(x,t)$ should be bounded (we are not worrying about the earth
core!).
If you use \hyperref[eulersformula]{Euler's formula} to expand the complex exponentials, 
note that the second term is unbounded (if $B \not = 0$),
while the first term is always bounded.  Hence $B=0$.

\begin{exercise}
Use \hyperref[eulersformula]{Euler's formula} to show that
$e^{(1+i)\sqrt{\frac{\omega}{2k}} \, x}$ is unbounded as $x \to \infty$,
while $e^{-(1+i)\sqrt{\frac{\omega}{2k}} \, x}$ is bounded
as $x \to \infty$.
\end{exercise}

Furthermore, $X(0) = A_0$ since $h(0,t) = A_0 e^{i \omega t}$.
Thus $A=A_0$.  This means that
\begin{equation*}
h(x,t) = A_0 e^{-(1+i)\sqrt{\frac{\omega}{2k}} \, x} e^{i \omega t}
=
A_0 e^{-(1+i)\sqrt{\frac{\omega}{2k}} \, x + i \omega t}
=
A_0 e^{-\sqrt{\frac{\omega}{2k}} \, x}
e^{i(\omega t - \sqrt{\frac{\omega}{2k}} \, x)} .
\end{equation*}
We need to get the real part of $h$, so
we apply \hyperref[eulersformula]{Euler's formula} to get
\begin{equation*}
h(x,t) =
A_0 e^{-\sqrt{\frac{\omega}{2k}} \, x}
\left(\cos \left(\omega t - \sqrt{\frac{\omega}{2k}}\, x\right) + 
i \sin \left(\omega t - \sqrt{\frac{\omega}{2k}}\, x\right) \right) .
\end{equation*}
Then finally
\begin{equation*}
u(x,t) = \operatorname{Re} h(x,t) =
A_0 e^{-\sqrt{\frac{\omega}{2k}}\, x}
\cos \left(\omega t - \sqrt{\frac{\omega}{2k}}\, x\right) .
\end{equation*}
Yay!

Notice the phase is different at different depths.  At depth $x$ the
phase is delayed by $x \sqrt{\frac{\omega}{2k}}$.
For example in cgs units (centimeters-grams-seconds)\index{cgs units}
we have $k=0.005$ (typical value for soil),
$\omega = \frac{2\pi}{\text{seconds in a year}}
= \frac{2\pi}{31,557,341} \approx 1.99 \times {10}^{-7}$.   Then
if we compute where the phase shift $x \sqrt{\frac{\omega}{2k}} = \pi$
we find the depth in centimeters where the seasons are reversed.  That is,
we get the depth at which summer is the coldest and winter is the warmest.
We get
approximately 700 centimeters, which is approximately 23 feet below ground.

Be careful not to jump to conclusions.  The temperature swings decay rapidly as you dig deeper.  The
amplitude of the temperature swings is
$A_0 e^{-\sqrt{\frac{\omega}{2k}} x}$.  This function decays \emph{very}
quickly as $x$ (the depth) grows.
Let us again take
typical parameters as above.  We also assume that
our surface temperature swing is $\pm {15}^\circ$ Celsius, that is,
$A_0 = 15$.  Then the maximum temperature variation at 700 centimeters
is only $\pm {0.66}^\circ$ Celsius.

You need not dig very deep to get an effective
\myquote{refrigerator,} with nearly constant temperature.  That is why wines are kept in a cellar; you need consistent
temperature.
The temperature differential could also be used for energy.  A home could
be heated or cooled by taking advantage of the fact above.
Even without the earth core you could heat a home in the winter and cool it
in the summer.  The earth core makes the
temperature higher the deeper you dig, although you need to dig somewhat
deep to feel a difference.
We did not take that into account above.

\subsection{Exercises}

\begin{exercise} \label{sps:sinforceexr}
Suppose that the forcing function for the vibrating string
is $F_0 \sin (\omega t)$.  Derive the particular solution $y_p$.
\end{exercise}

\begin{exercise}
Take the forced vibrating string.
Suppose that $L=1$, $a=1$.  Suppose that the forcing function
is the square wave that is 1 on the interval $0 < x < 1$ and
$-1$ on the interval $-1 < x< 0$.
Find the particular solution.  Hint: You may want to use result
of \exerciseref{sps:sinforceexr}.
\end{exercise}

\begin{exercise}
The units are cgs (centimeters-grams-seconds)\index{cgs units}.
For $k=0.005$, $\omega = 1.991 \times {10}^{-7}$, $A_0 = 20$.
Find the depth at which the temperature variation is half ($\pm 10$
degrees) of what it is on the surface.
\end{exercise}

\begin{exercise}
Derive the solution for underground temperature oscillation without assuming
that $T_0 = 0$.
\end{exercise}

\setcounter{exercise}{100}

\begin{exercise}
Take the forced vibrating string.
Suppose that $L=1$, $a=1$.  Suppose that the forcing function
is a sawtooth, that is $\lvert x \rvert -\frac{1}{2}$
on $-1 < x < 1$ extended periodically.
Find the particular solution.
\end{exercise}
\exsol{%
$y_p(x,t) =
\sum\limits_{\substack{n=1 \\ n \text{ odd}}}^\infty
\frac{-4}{n^4 \pi^4}
\left(
\cos(n \pi x ) -
\frac{\cos ( n \pi ) - 1}%
{\sin( n \pi)}
\sin( n \pi x)
-1
\right)
\cos (n \pi t) .
$
}

\begin{exercise}
The units are cgs (centimeters-grams-seconds)\index{cgs units}.
For $k=0.01$, $\omega = 1.991 \times {10}^{-7}$, $A_0 = 25$.
Find the depth at which the summer is again the hottest point.
\end{exercise}
\exsol{%
%$x \sqrt{\frac{\omega}{2k}} = 2\pi$
%$x = 2\pi\sqrt{\frac{2k}{\omega}}$
%$x = 2\pi\sqrt{\frac{0.02}{1.991 x 10^-7}}$
Approximately 1991 centimeters
}
