\section{Guide to this book} \label{guide:section}

\sectionnotes{Verbatim from Lebl}

This book originated from my class notes for Math 286
at the \href{https://www.math.uiuc.edu/}{University of Illinois at
Urbana-Champaign} (UIUC)
in Fall 2008 and
Spring 2009.
It is a first course on differential equations for engineers.
Using this book, I also taught Math 285 at UIUC\@,
Math 20D at
\href{https://www.math.ucsd.edu/}{University of California, San Diego} (UCSD),
and Math 4233 at 
\href{https://math.okstate.edu/}{Oklahoma State University} (OSU).
Normally these courses are taught with
Edwards and Penney, \emph{Differential
Equations and Boundary Value Problems: Computing and Modeling}~\cite{EP}, or
Boyce and DiPrima's
\emph{Elementary
Differential Equations and Boundary Value Problems}~\cite{BD},
and this book aims to be more or less a drop-in replacement.
Other books I used as sources of information and inspiration
are E.L.\ Ince's classic (and inexpensive)
\emph{Ordinary Differential Equations}~\cite{I},
Stanley Farlow's \emph{Differential Equations and Their
Applications}~\cite{F}, now available from Dover,
Berg and McGregor's
\emph{Elementary Partial Differential Equations}~\cite{BM},
and William Trench's free book
\emph{Elementary
Differential Equations with Boundary Value Problems}~\cite{T}.
See the \hyperref[furtherreading:chapter]{Further Reading} chapter at the end of the book.

\subsection{Organization}

The organization of this book to some degree
requires chapters be done in order.
Later chapters can be dropped.
The dependence of the material covered is roughly:

% If changing make sure to also update figures/chapterdiagram.pdf_t
% That's a hopefully short term hack before I figure out how to do it
% better so that it also gets links and such
%mbxSTARTIGNORE
\begin{equation*}
\begin{tikzcd}[cramped, row sep=small]
& {\text{\hyperref[intro:chapter]{Introduction}}} \arrow[d] \\
{\text{\Appendixref{linalg:appendix}}} \arrow[dd, dotted]
& {\text{\Chapterref{fo:chapter}}} \arrow[d] \\
& {\text{\Chapterref{ho:chapter}}} \arrow[dddr] \arrow[dd] \arrow[dl] \arrow[dr] \\
{\text{\Chapterref{sys:chapter}}} \arrow[dr, dotted] \arrow[d] & &
  {\text{\Chapterref{ps:chapter}}} \\
{\text{\Chapterref{nlin:chapter}}} & {\text{\Chapterref{FS:chapter}}} \arrow[d]
\arrow[dr,dotted] \\
& {\text{\Chapterref{SL:chapter}}}
& {\text{\Chapterref{LT:chapter}}}
\end{tikzcd}
\end{equation*}
%mbxENDIGNORE
%mbxlatex \begin{center}
%mbxlatex \inputpdft{chapterdiagram}
%mbxlatex \end{center}

There are a few references in chapters \ref{FS:chapter} and \ref{SL:chapter}
to \chapterref{sys:chapter} (some linear algebra), but these
references are not essential and can be skimmed over,
so \chapterref{sys:chapter}
can safely be dropped, while still covering
chapters \ref{FS:chapter} and \ref{SL:chapter}.
\Chapterref{LT:chapter} does not depend on 
\chapterref{FS:chapter} except that the
PDE section \ref{laplacepde:section} makes a
few references to
\chapterref{FS:chapter},
although it could, in theory, be covered
separately.
The more in-depth \appendixref{linalg:appendix} on linear algebra
can replace the short review \sectionref{sec:matrix}
for a course that combines linear algebra and ODE\@.

%\medskip
\subsection{Typical types of courses}

Several typical types of courses can be run with the book.
There are the two original courses at UIUC\@,
both cover ODE as well some PDE\@.
Either, there is the 4 hours-a-week for a semester (Math 286 at UIUC):

\medskip

\noindent
\hyperref[intro:chapter]{Introduction} (\ref{introde:section}),
\chapterref{fo:chapter} (\ref{integralsols:section}--\ref{numer:section}),
\chapterref{ho:chapter},
\chapterref{sys:chapter},
\chapterref{FS:chapter} (\ref{bvp:section}--\ref{dirich:section}),
\chapterref{SL:chapter} (or
\ref{LT:chapter} or \ref{ps:chapter} or \ref{nlin:chapter}).

\medskip

Or, the second course at UIUC is at 3 hours-a-week (Math 285 at UIUC):

\medskip

\noindent
\hyperref[intro:chapter]{Introduction} (\ref{introde:section}),
\chapterref{fo:chapter} (\ref{integralsols:section}--\ref{numer:section}),
\chapterref{ho:chapter},
\chapterref{FS:chapter} (\ref{bvp:section}--\ref{dirich:section}),
(and maybe \chapterref{SL:chapter},
\ref{LT:chapter}, or \ref{ps:chapter}).

\medskip

A semester-long course at 3 hours a week that doesn't cover either systems or PDE
will cover, beyond the introduction,
%\sectionref{introde:section},
\chapterref{fo:chapter},
\chapterref{ho:chapter},
\chapterref{LT:chapter}, and \chapterref{ps:chapter},
(with sections skipped as above).
On the other hand, a typical course that covers 
systems will probably need to skip Laplace and power series
and cover
%\sectionref{introde:section},
\chapterref{fo:chapter},
\chapterref{ho:chapter},
\chapterref{sys:chapter}, and \chapterref{nlin:chapter}.

\medskip

If sections need to be skipped in the beginning, a good core of the 
sections on single ODE is:
\ref{introde:section},
\ref{integralsols:section}--\ref{intfactor:section},
\ref{auteq:section},
\ref{solinear:section},
\ref{sec:ccsol},
\ref{sec:mv}--\ref{forcedo:section}.

\medskip


The complete book can be covered at a reasonably
fast pace at approximately 76 lectures
(without \appendixref{linalg:appendix})
or 86 lectures (with \appendixref{linalg:appendix} replacing
\sectionref{sec:matrix}).
This is not accounting for exams, review,
or time spent in a computer lab. % (if using IODE for example).
A two-quarter or a two-semester course can be easily run with the material.
For example (with some sections perhaps strategically skipped):

\medskip

\noindent
Semester 1:
\hyperref[intro:chapter]{Introduction},
\chapterref{fo:chapter},
\chapterref{ho:chapter},
\chapterref{LT:chapter},
\chapterref{ps:chapter}.
\\
Semester 2: 
\Chapterref{sys:chapter},
\chapterref{nlin:chapter},
\chapterref{FS:chapter},
\chapterref{SL:chapter}.

\medskip

A combined course on ODE with linear algebra can run as:

\medskip

\noindent
\hyperref[intro:chapter]{Introduction},
\chapterref{fo:chapter} (\ref{integralsols:section}--\ref{numer:section}),
\chapterref{ho:chapter},
\appendixref{linalg:appendix},
\chapterref{sys:chapter} (w/o \sectionref{sec:matrix}), (possibly 
\chapterref{nlin:chapter}).

\medskip

The chapter on
the Laplace transform (\chapterref{LT:chapter}),
the chapter on Sturm--Liouville (\chapterref{SL:chapter}),
the chapter on power series (\chapterref{ps:chapter}),
and the chapter on nonlinear systems (\chapterref{nlin:chapter}),
are more or less interchangeable and can be treated as \myquote{topics}.
If \chapterref{nlin:chapter} is covered, it may be best to place it right 
after \chapterref{sys:chapter},
and \chapterref{SL:chapter} is best covered right after
\chapterref{FS:chapter}.
If time is short, the first two sections of
\chapterref{ps:chapter} make a reasonable self-contained unit.

\subsection{Computer resources}

The book's
website \url{https://www.jirka.org/diffyqs/}
contains the following resources:
\begin{enumerate}
\item Interactive SAGE demos.
\item Online WeBWorK homeworks
(using either your own WeBWorK installation or Edfinity)
for most sections, customized for this book.
\item The PDFs of the figures used in this book.
\end{enumerate}

I taught the UIUC courses using IODE\index{IODE software}
(\url{https://faculty.math.illinois.edu/iode/}).
IODE is a free software package that
works with Matlab (proprietary) or Octave (free software).
%Unfortunately IODE is not kept up to date at this point, and may have
%trouble running on newer versions of Matlab.
The graphs in the book were made with
the Genius\index{Genius software} software
(see \url{https://www.jirka.org/genius.html}).  I use Genius
in class to show these (and other) graphs.

The \LaTeX\ source of the book is also available
for possible modification and customization
at github (\url{https://github.com/jirilebl/diffyqs}).

%\medskip

%\textbf{Acknowlegements:}

\subsection{Acknowledgments}

Firstly, I would like to acknowledge Rick Laugesen.  I used his handwritten
class notes
the first time I taught
Math 286.  My organization of this book through chapter 5,
and the choice of
material covered, is heavily influenced by his notes.  Many
examples and computations are taken from his notes.  I am also heavily
indebted to Rick for all the advice he has given me, not just on teaching
Math 286.
For spotting errors and other suggestions,
I would also like to acknowledge (in no particular order):
John P.\ D'Angelo,
Sean Raleigh, Jessica Robinson, Michael Angelini, Leonardo Gomes, Jeff
Winegar, Ian Simon, Thomas Wicklund, Eliot Brenner, Sean Robinson,
Jannett Susberry, Dana Al-Quadi, Cesar Alvarez, Cem Bagdatlioglu,
Nathan Wong, Alison Shive, Shawn White, Wing Yip Ho, Joanne Shin,
Gladys Cruz, Jonathan Gomez, Janelle Louie, Navid Froutan,
Grace Victorine, Paul Pearson, Jared Teague, Ziad Adwan,
Martin Weilandt, S\"{o}nmez \c{S}ahuto\u{g}lu,
Pete Peterson, Thomas Gresham, Prentiss Hyde, Jai Welch,
Simon Tse, Andrew Browning, James Choi, Dusty Grundmeier,
John Marriott,
Jim Kruidenier,
Barry Conrad,
Wesley Snider,
Colton Koop,
Sarah Morse,
Erik Boczko,
Asif Shakeel,
Chris Peterson,
Nicholas Hu,
Paul Seeburger,
Jonathan McCormick,
David Leep,
William Meisel,
Shishir Agrawal,
Tom Wan,
Andres Valloud,
and probably others I
have forgotten.
Finally, I would like
to acknowledge NSF grants DMS-0900885 and DMS-1362337.