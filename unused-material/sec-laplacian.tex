\section{Steady state temperature and the Laplacian}
\label{dirich:section}

\LAtt{4.9}

\LO{
\item Relate the heat equation independent of time to the Laplace equation and
\item Use separation of variables to solve the Laplace equation on rectangular regions.
}

% \sectionnotes{Verbatim from Lebl}

% \sectionnotes{1 lecture\EPref{, \S9.7 in \cite{EP}}\BDref{,
% \S10.8 in \cite{BD}}}

Consider an insulated wire, a plate, or a 3-dimensional object.
We apply
certain fixed temperatures on the ends of the wire, the edges of the plate,
or on all sides of the 3-dimensional object.  We wish to find out what is the
\emph{\myindex{steady state temperature}} distribution.  That is, we wish to know what will
be the temperature after long enough period of time.

We are really looking for a solution to the heat equation that is not
dependent on time.  Let us first solve the problem in one space variable.  We are
looking for a function $u$ that satisfies
\begin{equation*}
u_t = k u_{xx} ,
\end{equation*}
but such that $u_t = 0$ for all $x$ and $t$.  Hence, we are looking for a
function of $x$ alone that satisfies $u_{xx} = 0$.  It is easy to solve this
equation by integration and we see that $u = Ax+B$ for some constants $A$ and
$B$.

Consider an insulated wire where we apply constant temperature $T_1$
at one end (say where $x=0$) and $T_2$ on the other end (at $x=L$ where $L$
is the length of the wire).  Our steady state solution is
\begin{equation*}
u(x) = \frac{T_2-T_1}{L} x + T_1 .
\end{equation*}
This solution agrees with our common sense intuition with how the heat should be
distributed in the wire.  So in one dimension, the steady state solutions
are basically just straight lines.

Things are more complicated in two or more space dimensions.  Let us
restrict to two space dimensions for simplicity.  The heat equation in two
space variables is
\begin{equation} \label{dirich:heateq}
u_t = k(u_{xx} + u_{yy}) ,
\end{equation}
or more commonly written as
$u_t = k \Delta u$ or
$u_t = k \nabla^2 u$.  Here the $\Delta$ and $\nabla^2$ symbols
mean $\frac{\partial^2}{\partial x^2} +
\frac{\partial^2}{\partial y^2}$.  We will use $\Delta$
from now on.  The reason for using such a notation is that you
can define $\Delta$ to be the right thing for any number of space
dimensions and then the heat equation is always
$u_t = k \Delta u$.  The operator $\Delta$ is called the \emph{\myindex{Laplacian}}.

OK\@, now that we have notation out of the way, let us see what does an equation
for the steady state solution look like.  We are looking for a solution to
\eqref{dirich:heateq} that does not depend on $t$, or in other words $u_t =
0$.  Hence we are looking for a
function $u(x,y)$ such that
\begin{equation*}
\mybxbg{~~
\Delta u = 
u_{xx} + u_{yy} = 0 .
~~}
\end{equation*}
This equation is called the \emph{\myindex{Laplace equation}}%
\footnote{Named after the French mathematician
\href{https://en.wikipedia.org/wiki/Laplace}{Pierre-Simon, marquis de Laplace}
(1749--1827).}, and is an example of an elliptic equation.
Solutions to the Laplace equation
are called \emph{harmonic functions\index{harmonic function}}
and have many nice properties and
applications far beyond the steady state heat problem. One of these main applications is in electrostatics, as the electric potential $V$ in a region also solves the Laplace equation
\[ \Delta V = 0. \]

Harmonic functions in two variables are no longer just linear (plane
graphs).  For example, you can check that the functions
$x^2-y^2$ and $xy$ are harmonic.  However, note that if $u_{xx}$ is positive, $u$ is concave
up in the $x$ direction, then $u_{yy}$ must be negative and $u$ must be
concave down in the $y$ direction.  A harmonic function can never
have any \myquote{hilltop} or \myquote{valley} on the graph.  This observation is
consistent with our intuitive idea of steady state heat distribution;
the hottest or coldest spot will not be inside.

Commonly the Laplace equation is part of a so-called
\emph{\myindex{Dirichlet problem}}%
\footnote{Named after the German mathematician
\href{https://en.wikipedia.org/wiki/Dirichlet}{Johann Peter Gustav Lejeune Dirichlet}
(1805--1859).}.
That
is, we have a region in the $xy$-plane and we specify certain values along
the boundaries of the region.  We then try to find a solution $u$ to the
Laplace equation defined on
this region such that $u$ agrees with the values we specified on the
boundary.

In this section we consider a rectangular region.  For simplicity
we specify boundary values to be zero at 3 of the four edges and only
specify an arbitrary function at one edge.  As we still have the
principle of superposition, we can use this simpler
solution to derive the general
solution for arbitrary boundary values by solving 4 different problems,
one for each edge, and adding those solutions together.
This setup is left as an exercise.

We wish to solve the following problem.  Let $h$ and $w$
be the height and width of our rectangle, with one corner at the origin and
lying in the first quadrant.

% FIXME: numbering does not work now since there are no hard coded numbers
% here!
%mbx <mdn>
%mbx   <mrow xml:id="dirich_eq1" number="%MBXEQNNUMBER%">
%mbx     &amp; \Delta u = 0 ,
%mbx   </mrow>
%mbx   <mrow xml:id="dirich_eq2" number="%MBXEQNNUMBER%">
%mbx     &amp; u(0,y) = 0 \quad \text{for } 0 &lt; y &lt; h,
%mbx   </mrow>
%mbx   <mrow xml:id="dirich_eq3" number="%MBXEQNNUMBER%">
%mbx     &amp; u(x,h) = 0 \quad \text{for } 0 &lt; x &lt; w,
%mbx   </mrow>
%mbx   <mrow xml:id="dirich_eq4" number="%MBXEQNNUMBER%">
%mbx     &amp; u(w,y) = 0 \quad \text{for } 0 &lt; y &lt; h,
%mbx   </mrow>
%mbx   <mrow xml:id="dirich_eq5" number="%MBXEQNNUMBER%">
%mbx     &amp; u(x,0) = f(x) \quad \text{for } 0 &lt; x &lt; w.
%mbx   </mrow>
%mbx </mdn>

\begin{center}
%mbxSTARTIGNORE
\begin{minipage}[b]{2.8in}
\vspace{\fill}
\begin{align}
& \Delta u = 0 , & &  \label{dirich:eq1} \\
& u(0,y) = 0 & & \text{for }  0 < y < h,\label{dirich:eq2} \\
& u(x,h) = 0 & & \text{for }  0 < x < w,\label{dirich:eq3} \\
& u(w,y) = 0 & & \text{for }  0 < y < h,\label{dirich:eq4} \\
& u(x,0) = f(x) & & \text{for }  0 < x < w.\label{dirich:eq5}
\end{align}
\vspace{\fill}
\end{minipage}
\qquad
%mbxENDIGNORE
\inputpdft{dirichsetup}
%mbxSTARTIGNORE
\qquad
%mbxENDIGNORE
\end{center}

The method we apply is separation of variables.  Again, we will
come up with enough building-block
solutions satisfying all the homogeneous boundary conditions
(all conditions except \eqref{dirich:eq5}).  We notice that superposition
still works for the equation and all the homogeneous conditions.
Therefore,
we can use the Fourier series for $f(x)$ to solve the
problem as before.

We try $u(x,y) = X(x)Y(y)$.  We plug $u$ into the equation to get
\begin{equation*}
X''Y + XY'' = 0 .
\end{equation*}
We put the $X$s on one side and the $Y$s on the other to get
\begin{equation*}
- \frac{X''}{X} = \frac{Y''}{Y} .
\end{equation*}
The left-hand side only depends on $x$ and the right-hand side only depends
on $y$.  Therefore, there is some constant $\lambda$ such that
$\lambda = \frac{-X''}{X} = \frac{Y''}{Y}$.
And we get two equations
\begin{align*}
& X'' + \lambda X = 0 , \\
& Y'' - \lambda Y = 0 .
\end{align*}
Furthermore, the homogeneous boundary conditions imply that
$X(0) = X(w) = 0$ and $Y(h) = 0$.  Taking the equation for $X$
we have already seen that we have a nontrivial solution if and only if
$\lambda = \lambda_n = \frac{n^2 \pi^2}{w^2}$ and the solution is
a multiple of
\begin{equation*}
X_n(x) = \sin \left( \frac{n \pi}{w} x \right) .
\end{equation*}
For these given $\lambda_n$,
the general solution for $Y$ (one for each $n$) is
\begin{equation} \label{dirich:Yngensol}
Y_n(y) = A_n \cosh \left( \frac{n \pi}{w} y \right)
+ B_n \sinh \left( \frac{n \pi}{w} y \right) .
\end{equation}
We only have one condition on $Y_n$ and hence we can pick one of $A_n$
or $B_n$
to be something convenient.
It will be useful to have $Y_n(0) = 1$, so we let $A_n=1$.
Setting $Y_n(h) = 0$ and solving for $B_n$ we get that
\begin{equation*}
B_n = \frac{- \cosh \left( \frac{n \pi h }{w} \right)}%
{\sinh \left( \frac{n \pi h }{w} \right)} .
\end{equation*}
After we plug the $A_n$ and $B_n$ we
into \eqref{dirich:Yngensol} and simplify by using
the identity $\sinh(\alpha-\beta) =
\sinh(\alpha) \cosh(\beta) -
\cosh(\alpha) \sinh(\beta)$, we find
\begin{equation*}
Y_n(y) =
\frac{\sinh \left( \frac{n \pi (h-y) }{w} \right)}%
{\sinh \left( \frac{n \pi h }{w} \right)} .
\end{equation*}
We define $u_n(x,y) = X_n(x)Y_n(y)$.
And note that $u_n$
satisfies \eqref{dirich:eq1}--\eqref{dirich:eq4}.


Observe that
\begin{equation*}
u_n(x,0) = X_n(x)Y_n(0) = \sin \left( \frac{n \pi}{w} x \right) .
\end{equation*}
Suppose
\begin{equation*}
f(x) =
%\frac{a_0}{2} +
\sum_{n=1}^\infty
%a_n \cos \left( \frac{n \pi x }{w} \right)
%+
b_n \sin \left( \frac{n \pi x }{w} \right) .
\end{equation*}
Then we get a solution of \eqref{dirich:eq1}--\eqref{dirich:eq5} of the
following form.
\begin{equation*}
\mybxbg{
~~
u(x,y) =
\sum_{n=1}^\infty
b_n u_n(x,y)
=
\sum_{n=1}^\infty
b_n 
\sin \left( \frac{n \pi}{w} x \right)
\left( \frac{\sinh \left( \frac{n \pi (h-y) }{w} \right)}%
{\sinh \left( \frac{n \pi h }{w} \right)} \right)
.
~~
}
\end{equation*}
As $u_n$ satisfies \eqref{dirich:eq1}--\eqref{dirich:eq4} and any linear
combination (finite or infinite) of $u_n$ also satisfies 
\eqref{dirich:eq1}--\eqref{dirich:eq4}, then $u$ satisfies
\eqref{dirich:eq1}--\eqref{dirich:eq4}.
By plugging in $y=0$, we see $u$
satisfies 
\eqref{dirich:eq5} as well.

\begin{example}
Take $w=h=\pi$ and let $f(x) = \pi$.  Let us compute the sine
series for the function $\pi$ (same as the series for the square wave).
For $0 < x < \pi$, we have
\begin{equation*}
f(x) =
\sum_{\substack{n=1 \\ n \text{ odd}}}^\infty
\frac{4}{n}
\sin (n x) .
\end{equation*}
Therefore the solution $u(x,y)$, see \figurevref{dirichsquareplot:fig},
to the corresponding Dirichlet problem is
given as
\begin{equation*}
u(x,y) =
\sum_{\substack{n=1 \\ n \text{ odd}}}^\infty
\frac{4}{n}
\sin (n x)
\left( \frac{\sinh \bigl( n (\pi-y) \bigr) }{\sinh (n \pi)} \right)
.
\end{equation*}

\begin{myfig}
\capstart
\diffyincludegraphics{width=5in}{width=7.5in}{dirichsquareplot}
\caption{Steady state temperature of a square plate, three sides
held at zero and one side held at $\pi$.\label{dirichsquareplot:fig}}
\end{myfig}
\end{example}

This scenario
corresponds to the steady state temperature on a square plate of width $\pi$
with 3 sides held at 0 degrees and one side held at $\pi$ degrees.
If we have arbitrary initial data on all sides, then we solve four problems,
each using one piece of nonhomogeneous data.  Then we use the principle of
superposition to add up all four solutions to have a solution to the
original problem.

A different
way to visualize solutions of the Laplace equation is to
take a wire and bend
it so that it corresponds to the graph of the
temperature above the boundary of your region.  Cut a rubber sheet in
the shape of your region---a square in our case---and stretch it
fixing the edges of the sheet to the wire.
The rubber sheet is a good approximation of the graph of the solution to
the Laplace equation with the given boundary data.

\subsection{Exercises}

\begin{exercise}
Let $R$ be the region described by $0 < x < \pi$ and $0 < y < \pi$.
Solve the problem
\begin{equation*}
\Delta u = 0, \quad u(x,0) = \sin x, \quad u(x,\pi) = 0,
\quad u(0,y) = 0, 
\quad u(\pi,y) = 0 .
\end{equation*}
\end{exercise}

\begin{exercise}
Let $R$ be the region described by $0 < x < 1$ and $0 < y < 1$.
Solve the problem
\begin{align*}
& u_{xx} + u_{yy} = 0, \\
& u(x,0) = \sin (\pi x) - \sin (2\pi x), \quad u(x,1) = 0, \\
& u(0,y) = 0, \quad u(1,y) = 0 .
\end{align*}
\end{exercise}

\begin{exercise}
Let $R$ be the region described by $0 < x < 1$ and $0 < y < 1$.
Solve the problem
\begin{align*}
& u_{xx} + u_{yy} = 0, \\
& u(x,0) = u(x,1) = u(0,y) = u(1,y) = C .
\end{align*}
for some constant $C$.  Hint: Guess, then check your intuition.
\end{exercise}

\begin{exercise} \label{dirich:diffsepexr}
Let $R$ be the region described by $0 < x < \pi$ and $0 < y < \pi$.
Solve
\begin{equation*}
\Delta u = 0,
\quad u(x,0) = 0,
\quad u(x,\pi) = \pi,
\quad u(0,y) = y,
\quad u(\pi,y) = y .
\end{equation*}
Hint: Try a solution of the form $u(x,y) = X(x) + Y(y)$ (different separation
of variables).
\end{exercise}

\begin{exercise}
Use the solution of \exerciseref{dirich:diffsepexr} to solve
\begin{equation*}
\Delta u = 0,
\quad u(x,0) = \sin x,
\quad u(x,\pi) = \pi,
\quad u(0,y) = y,
\quad u(\pi,y) = y .
\end{equation*}
Hint: Use superposition.
\end{exercise}

\begin{exercise}
Let $R$ be the region described by $0 < x < w$ and $0 < y < h$.
Solve the problem
\begin{align*}
& u_{xx} + u_{yy} = 0, \\
& u(x,0) = 0, \quad u(x,h) = f(x), \\
& u(0,y) = 0, \quad u(w,y) = 0.
\end{align*}
The solution should be in series form using the Fourier series coefficients
of $f(x)$.
\end{exercise}

\begin{exercise}\ansMark%
Let $R$ be the region described by $0 < x < 1$ and $0 < y < 1$.
Solve the problem
\begin{equation*}
\Delta u = 0, \quad u(x,0) = \sum_{n=1}^\infty \frac{1}{n^2} \sin (n \pi x),
\quad u(x,1) = 0,
\quad u(0,y) = 0, 
\quad u(1,y) = 0 .
\end{equation*}
\end{exercise}
\exsol{%
$u(x,y) =
\sum\limits_{n=1}^\infty
\frac{1}{n^2}
\sin ( n \pi x )
\left( \frac{\sinh ( n \pi (1-y) )}{\sinh ( n \pi )} \right)
$
}

\begin{exercise}
Let $R$ be the region described by $0 < x < w$ and $0 < y < h$.
Solve the problem
\begin{align*}
& u_{xx} + u_{yy} = 0, \\
& u(x,0) = 0, \quad u(x,h) = 0, \\
& u(0,y) = f(y), \quad u(w,y) = 0.
\end{align*}
The solution should be in series form using the Fourier series coefficients
of $f(y)$.
\end{exercise}

\begin{exercise}
Let $R$ be the region described by $0 < x < w$ and $0 < y < h$.
Solve the problem
\begin{align*}
& u_{xx} + u_{yy} = 0, \\
& u(x,0) = 0, \quad u(x,h) = 0, \\
& u(0,y) = 0, \quad u(w,y) = f(y).
\end{align*}
The solution should be in series form using the Fourier series coefficients
of $f(y)$.
\end{exercise}

\begin{exercise}
Let $R$ be the region described by $0 < x < 1$ and $0 < y < 1$.
Solve the problem
\begin{align*}
& u_{xx} + u_{yy} = 0, \\
& u(x,0) = \sin (9 \pi x), \quad u(x,1) = \sin (2 \pi x), \\
& u(0,y) = 0, \quad u(1,y) = 0 .
\end{align*}
Hint: Use superposition.
\end{exercise}

\begin{exercise}\ansMark%
Let $R$ be the region described by $0 < x < 1$ and $0 < y < 2$.
Solve the problem
\begin{equation*}
\Delta u = 0, \quad u(x,0) = 0.1 \sin (\pi x),
\quad u(x,2) = 0,
\quad u(0,y) = 0, 
\quad u(1,y) = 0 .
\end{equation*}
\end{exercise}
\exsol{%
$u(x,y) =
0.1
\sin ( \pi x )
\left( \frac{\sinh ( \pi (2-y)  )}%
{\sinh ( 2 \pi )} \right)$
}


\begin{exercise}
Let $R$ be the region described by $0 < x < 1$ and $0 < y < 1$.
Solve the problem
\begin{align*}
& u_{xx} + u_{yy} = 0, \\
& u(x,0) = \sin (\pi x), \quad u(x,1) = \sin (\pi x), \\
& u(0,y) = \sin (\pi y), \quad u(1,y) = \sin (\pi y) .
\end{align*}
Hint: Use superposition.
\end{exercise}

\begin{exercise}[challenging]
Using only your intuition find $u(\nicefrac{1}{2},\nicefrac{1}{2})$,
for the problem
$\Delta u = 0$, where $u(0,y) = u(1,y) = 100$ for $0 < y < 1$, and
$u(x,0) = u(x,1) = 0$ for $0 < x < 1$.  Explain.
\end{exercise}

\setcounter{exercise}{100}




%\begin{equation*}
%f(x) =
%%\frac{a_0}{2} +
%\sum_{n=1}^\infty
%%a_n \cos \left( \frac{n \pi x }{w} \right)
%%+
%b_n \sin \left( \frac{n \pi x }{w} \right) .
%\end{equation*}
%
%u(x,y) =
%\sum_{n=1}^\infty
%b_n u_n(x,y)
%=
%\sum_{n=1}^\infty
%b_n 
%\sin \left( \frac{n \pi}{w} x \right)
%\left( \frac{\sinh \left( \frac{n \pi (h-y) }{w} \right)}%
%{\sinh \left( \frac{n \pi h }{w} \right)} \right)