
\section{Series solutions of linear second order ODEs}
\label{seriessols:section}

\LAtt{7.2}

\LO{
\item Use power series methods to solve second order linear ODEs near ordinary points and
\item Write a recurrence relation for the coefficients in a power series solution to an ODE.
}

% \sectionnotes{Verbatim from Lebl}

% \sectionnotes{1 or 1.5 lecture\EPref{, \S8.2 in \cite{EP}}\BDref{,
% \S5.2 and \S5.3 in \cite{BD}}}

Suppose we have a linear second order homogeneous ODE of the form
\begin{equation*}
p(x) y'' + q(x) y' + r(x) y = 0 .
\end{equation*}
Suppose that $p(x)$, $q(x)$, and $r(x)$ are polynomials.  We will 
try a solution of the form
\begin{equation*}
y = \sum_{k=0}^\infty a_k {(x-x_0)}^k
\end{equation*}
and solve for the $a_k$ to try to obtain a solution defined in some
interval around $x_0$.

The point $x_0$ is called an \emph{\myindex{ordinary point}}
if $p(x_0) \not= 0$.  That is, the functions
\begin{equation*}
\frac{q(x)}{p(x)} \qquad \text{and} \qquad \frac{r(x)}{p(x)}
\end{equation*}
are defined for $x$ near $x_0$.  If $p(x_0) = 0$, then we say $x_0$
is a \emph{\myindex{singular point}}.  Handling singular points is
harder than ordinary points and so we now focus only on ordinary points.

\begin{example}
Let us start with a very simple example
\begin{equation*}
y'' - y = 0 .
\end{equation*}
\end{example}

\begin{exampleSol}
Let us try a power series solution near $x_0 = 0$,
which is an ordinary point.  Every point is an ordinary
point in fact, as the equation is constant coefficient.  We already know
we should obtain exponentials or the hyperbolic sine and cosine,
but let us pretend we do not know this.

We try
\begin{equation*}
y = \sum_{k=0}^\infty a_k x^k .
\end{equation*}
If we differentiate, the $k=0$ term is a constant and hence disappears.
We therefore get
\begin{equation*}
y' = \sum_{k=1}^\infty k a_k x^{k-1} .
\end{equation*}
We differentiate yet again to obtain (now the $k=1$ term disappears)
\begin{equation*}
y'' = \sum_{k=2}^\infty k(k-1) a_k x^{k-2} .
\end{equation*}
We reindex the series (replace $k$ with $k+2$) to obtain
\begin{equation*}
y'' = \sum_{k=0}^\infty (k+2)\,(k+1) \, a_{k+2} x^k .
\end{equation*}
Now we plug $y$ and $y''$ into the differential equation
\begin{equation*}
\begin{split}
0 = y''-y & = 
\Biggl( \sum_{k=0}^\infty (k+2)\,(k+1) \, a_{k+2} x^k  \Biggr)
-
\Biggl( \sum_{k=0}^\infty a_k x^k \Biggr)
\\
& =
\sum_{k=0}^\infty \,\Bigl( (k+2)\,(k+1) \, a_{k+2} x^k 
-
a_k x^k \Bigr)
\\
& =
\sum_{k=0}^\infty \,\bigl( (k+2)\,(k+1) \,a_{k+2} - a_k \bigr) \, x^k  .
\end{split}
\end{equation*}
As $y'' - y$ is supposed to be equal to 0, we know that the
coefficients of the resulting series must be equal to 0.  Therefore,
\begin{equation*}
(k+2)\,(k+1) \,a_{k+2} - a_k = 0 ,
\qquad
\text{or}
\qquad
a_{k+2} = \frac{a_k}{(k+2)(k+1)} .
\end{equation*}
The equation above is called a \emph{\myindex{recurrence relation}}
for the coefficients of the power series.
It did not matter what $a_0$ or $a_1$ was.  They can be arbitrary.
But once we pick $a_0$ and $a_1$, then all other coefficients are
determined by the recurrence relation.

Let us see what the coefficients
must be.  First, $a_0$ and $a_1$ are arbitrary.  Then,
\begin{equation*}
a_2 = \frac{a_0}{2}, \quad
a_3 = \frac{a_1}{(3)(2)}, \quad
a_4 = \frac{a_2}{(4)(3)} = \frac{a_0}{(4)(3)(2)}, \quad
a_5 = \frac{a_3}{(5)(4)} = \frac{a_1}{(5)(4)(3)(2)}, \quad \ldots
\end{equation*}
So for even $k$, that is $k=2n$,
we have
\begin{equation*}
a_k = a_{2n} = \frac{a_0}{(2n)!} ,
\end{equation*}
and for odd $k$, that is $k=2n+1$, we have
\begin{equation*}
a_k = a_{2n+1} = \frac{a_1}{(2n+1)!} .
\end{equation*}
Let us write down the series
\begin{equation*}
y =
\sum_{k=0}^\infty
a_k x^k
=
\sum_{n=0}^\infty
\left(
\frac{a_0}{(2n)!} \,x^{2n}
+
\frac{a_1}{(2n+1)!} \,x^{2n+1}
\right)
=
a_0
\sum_{n=0}^\infty
\frac{1}{(2n)!} \,x^{2n}
+
a_1
\sum_{n=0}^\infty
\frac{1}{(2n+1)!} \,x^{2n+1} .
\end{equation*}
We recognize the two series as the hyperbolic sine and cosine.
Therefore,
\begin{equation*}
y =
a_0 \cosh x + a_1 \sinh x .
\end{equation*}
\end{exampleSol}

Of course, in general we will not be able to recognize 
the series that appears, since usually there will not be
any elementary function that matches it.  In that case we will be
content with the series.

\begin{example}
Let us do a more complex example.  Consider
\emph{\myindex{Airy's equation}}%
\footnote{Named after the English mathematician
\href{http://en.wikipedia.org/wiki/George_Biddell_Airy}{Sir George Biddell Airy}
(1801--1892).}:
\begin{equation*}
y'' - xy = 0 ,
\end{equation*}
near the point $x_0 = 0$.  Note that $x_0 = 0$ is an ordinary point.
\end{example}

\begin{exampleSol}
\pagebreak[2]
We try
\begin{equation*}
y = \sum_{k=0}^\infty a_k x^k .
\end{equation*}
We differentiate twice (as above) to obtain
\begin{equation*}
y'' = \sum_{k=2}^\infty k\,(k-1) \, a_k x^{k-2} .
\end{equation*}
We plug $y$ into the equation
\begin{equation*}
\begin{split}
0 = y''-xy &= 
\Biggl( \sum_{k=2}^\infty k\,(k-1) \, a_k x^{k-2}  \Biggr)
-
x
\Biggl( \sum_{k=0}^\infty a_k x^k \Biggr)
\\
&=
\Biggl( \sum_{k=2}^\infty k\,(k-1) \, a_k x^{k-2}  \Biggr)
-
\Biggl( \sum_{k=0}^\infty a_k x^{k+1} \Biggr) .
\end{split}
\end{equation*}
We reindex to make things easier to sum
\begin{equation*}
\begin{split}
0 = y''-xy
&= 
\Biggl( 2 a_2 + \sum_{k=1}^\infty (k+2)\,(k+1) \, a_{k+2} x^k  \Biggr)
-
\Biggl( \sum_{k=1}^\infty a_{k-1} x^k \Biggr)
\\
&= 
2 a_2 + 
\sum_{k=1}^\infty \Bigl( (k+2)\,(k+1) \, a_{k+2} - a_{k-1} \Bigr) \, x^k .
\end{split}
\end{equation*}
Again $y''-xy$ is supposed to be 0, so $a_2 = 0$, and
\begin{equation*}
(k+2)\,(k+1) \,a_{k+2} - a_{k-1} = 0 ,
\qquad
\text{or}
\qquad
a_{k+2} = \frac{a_{k-1}}{(k+2)(k+1)} .
\end{equation*}
We jump in steps of three.  First, since $a_2 = 0$
we must have , $a_5 = 0$, $a_8 = 0$, $a_{11}=0$, etc.
In general, $a_{3n+2} = 0$.

The constants $a_0$ and $a_1$ are arbitrary and we obtain
\begin{equation*}
a_3 = \frac{a_0}{(3)(2)}, \quad
a_4 = \frac{a_1}{(4)(3)}, \quad
a_6 = \frac{a_3}{(6)(5)} = \frac{a_0}{(6)(5)(3)(2)}, \quad
a_7 = \frac{a_4}{(7)(6)} = \frac{a_1}{(7)(6)(4)(3)}, \quad \ldots
\end{equation*}
For $a_k$ where $k$ is a multiple of $3$, that is $k=3n$ we notice
that
\begin{equation*}
a_{3n} = \frac{a_0}{(2)(3)(5)(6) \cdots (3n-1)(3n)} .
\end{equation*}
For $a_k$ where $k = 3n+1$, we notice
\begin{equation*}
a_{3n+1} = \frac{a_1}{(3)(4)(6)(7) \cdots (3n)(3n+1)} .
\end{equation*}
In other words, if we write down the series for $y$,
it has two parts
\begin{equation*}
\begin{split}
y &=
\left(
a_0 + \frac{a_0}{6} x^3 + \frac{a_0}{180} x^6 + \cdots +
\frac{a_0}{(2)(3)(5)(6) \cdots (3n-1)(3n)} x^{3n} + \cdots
\right)
\\
&\phantom{=}
+
\left(
a_1 x + \frac{a_1}{12} x^4 + \frac{a_1}{504} x^7 + \cdots +
\frac{a_1}{(3)(4)(6)(7) \cdots (3n)(3n+1)} x^{3n+1} + \cdots
\right)
\\
& =
a_0
\left(
1 + \frac{1}{6} x^3 + \frac{1}{180} x^6 + \cdots +
\frac{1}{(2)(3)(5)(6) \cdots (3n-1)(3n)} x^{3n} + \cdots
\right)
\\
&\phantom{=}
+
a_1
\left(
x + \frac{1}{12} x^4 + \frac{1}{504} x^7 + \cdots +
\frac{1}{(3)(4)(6)(7) \cdots (3n)(3n+1)} x^{3n+1} + \cdots
\right) .
\end{split}
\end{equation*}
We define
\begin{align*}
y_1(x) &= 
1 + \frac{1}{6} x^3 + \frac{1}{180} x^6 + \cdots +
\frac{1}{(2)(3)(5)(6) \cdots (3n-1)(3n)} x^{3n} + \cdots, \\
y_2(x) &= 
x + \frac{1}{12} x^4 + \frac{1}{504} x^7 + \cdots +
\frac{1}{(3)(4)(6)(7) \cdots (3n)(3n+1)} x^{3n+1} + \cdots ,
\end{align*}
and write the general solution to the equation as
$y(x)= a_0 y_1(x) + a_1 y_2(x)$.  If we plug in $x=0$ into the
power series for $y_1$ and $y_2$, we find
$y_1(0) = 1$ and $y_2(0) = 0$.  Similarly,
$y_1'(0) = 0$ and $y_2'(0) = 1$.  Therefore $y = a_0 y_1 + a_1 y_2$
is a solution
that satisfies the initial conditions $y(0) = a_0$ and $y'(0) = a_1$.

\begin{myfig}
\capstart
\diffyincludegraphics{width=3in}{width=4.5in}{ps-airy}
\caption{The two solutions $y_1$ and $y_2$ to Airy's equation.\label{ps:airyfig}}
\end{myfig}
\end{exampleSol}
The functions $y_1$ and $y_2$ cannot be written in terms of the elementary
functions that you know.  See \figurevref{ps:airyfig} for the plot of
the solutions $y_1$ and $y_2$.  These functions have many interesting
properties.  For example, they are oscillatory for negative $x$
(like solutions to $y''+y=0$) and
for positive $x$ they grow without bound (like solutions to $y''-y=0$).

\medskip

Sometimes a solution may turn out to be a polynomial.

\begin{example}
Find a solution to the so-called
\emph{\myindex{Hermite's equation of order $n$}}%
\footnote{Named after the French mathematician
\href{http://en.wikipedia.org/wiki/Hermite}{Charles Hermite}
(1822--1901).}:
\begin{equation*}
y'' -2xy' + 2n y = 0 .
\end{equation*}
\end{example}

\begin{exampleSol}
Let us find a solution around the point $x_0 = 0$.
We try
\begin{equation*}
y = \sum_{k=0}^\infty a_k x^k .
\end{equation*}
We differentiate (as above) to obtain
\begin{align*}
y' &= \sum_{k=1}^\infty k a_k x^{k-1} ,
\\
y'' &= \sum_{k=2}^\infty k\,(k-1) \, a_k x^{k-2} .
\end{align*}

Now we plug into the equation
\begin{equation*}
\begin{split}
0 = y''-2xy'&+2ny \\
 &= 
\Biggl( \sum_{k=2}^\infty k(k-1) a_k x^{k-2}  \Biggr)
-
2x
\Biggl( \sum_{k=1}^\infty k a_k x^{k-1} \Biggr)
+
2n
\Biggl( \sum_{k=0}^\infty a_k x^k \Biggr)
\\
&=
\Biggl( \sum_{k=2}^\infty k(k-1) a_k x^{k-2}  \Biggr)
-
\Biggl( \sum_{k=1}^\infty 2k a_k x^k \Biggr)
+
\Biggl( \sum_{k=0}^\infty 2n a_k x^k \Biggr)
\\
&=
\Biggl(2a_2+
 \sum_{k=1}^\infty (k+2)(k+1) a_{k+2} x^k  \Biggr)
-
\Biggl( \sum_{k=1}^\infty 2k a_k x^k \Biggr)
+
\Biggl(
2na_0 + 
\sum_{k=1}^\infty 2n a_k x^k \Biggr)
\\
&=
2a_2+2na_0+
\sum_{k=1}^\infty \bigl( (k+2)(k+1)  a_{k+2} - 2ka_k + 2n a_k \bigr) x^k .
\end{split}
\end{equation*}
As $y''-2xy'+2ny = 0$ we have
\begin{equation*}
(k+2)(k+1)  a_{k+2} + ( - 2k+ 2n) a_k = 0 ,
\qquad
\text{or}
\qquad
a_{k+2} = \frac{(2k-2n)}{(k+2)(k+1)} a_k .
\end{equation*}
This recurrence relation actually includes
$a_2 = -na_0$ (which comes about from $2a_2+2na_0 = 0$).
Again $a_0$ and $a_1$ are arbitrary.
\begin{align*}
& a_2 = \frac{-2n}{(2)(1)}a_0, \qquad
a_3 = \frac{2(1-n)}{(3)(2)} a_1,
\displaybreak[0]\\
& a_4 = \frac{2(2-n)}{(4)(3)} a_2 = \frac{2^2(2-n)(-n)}{(4)(3)(2)(1)} a_0 ,
\displaybreak[0]\\
&
a_5 = \frac{2(3-n)}{(5)(4)} a_3 = \frac{2^2(3-n)(1-n)}{(5)(4)(3)(2)} a_1 ,
\quad \ldots
\end{align*}
Let us separate the even and odd coefficients.
We find that 
\begin{align*}
a_{2m} &=\frac{2^m(-n)(2-n)\cdots(2m-2-n)}{(2m)!} , \\
a_{2m+1} &=\frac{2^m(1-n)(3-n)\cdots(2m-1-n)}{(2m+1)!} .
\end{align*}

Let us write down the two series, one with the even powers and one with the
odd.
\begin{align*}
y_1(x) & = 
1+\frac{2(-n)}{2!} x^2 + \frac{2^2(-n)(2-n)}{4!} x^4 + 
\frac{2^3(-n)(2-n)(4-n)}{6!} x^6 + \cdots ,
\\
y_2(x) & = 
x+\frac{2(1-n)}{3!} x^3 + \frac{2^2(1-n)(3-n)}{5!} x^5 + 
\frac{2^3(1-n)(3-n)(5-n)}{7!} x^7 + \cdots .
\end{align*}
We then write
\begin{equation*}
y(x) = a_0 y_1(x) + a_1 y_2(x) .
\end{equation*}

We remark that if $n$ is a positive even integer, then $y_1(x)$ is a
polynomial as all the coefficients in the series beyond a certain
degree are zero.  If $n$ is a positive odd integer, then $y_2(x)$ is
a polynomial.  For example, if $n=4$, then
\begin{equation*}
y_1(x) = 1 + \frac{2(-4)}{2!} x^2 + \frac{2^2(-4)(2-4)}{4!} x^4
= 1 - 4x^2 + \frac{4}{3} x^4 .
\end{equation*}
\end{exampleSol}

\subsection{Exercises}

In the following exercises, when asked to solve an equation using power
series methods, you should find the first few terms of the series,
and if possible find a general formula for the $k^{\text{th}}$ coefficient.

\begin{exercise}
Use power series methods to solve $y''+y = 0$ at the point $x_0 = 1$.
\end{exercise}

\begin{exercise}
Use power series methods to solve $y''+4xy = 0$ at the point $x_0 = 0$.
\end{exercise}

\begin{exercise}\ansMark%
Use power series methods to solve $y'' + 2 x^3 y = 0$ at the point $x_0 =
0$.
\end{exercise}
\exsol{%
%\begin{equation*}
%\begin{split}
%0 = y''+2 x^3 y &= 
%\Biggl( \sum_{k=2}^\infty k\,(k-1) \, a_k x^{k-2}  \Biggr)
%+
%2 x^3
%\Biggl( \sum_{k=0}^\infty a_k x^k \Biggr)
%\\
%&=
%\Biggl( \sum_{k=2}^\infty k\,(k-1) \, a_k x^{k-2}  \Biggr)
%+
%\Biggl( \sum_{k=0}^\infty 2 a_k x^{k+3} \Biggr) .
%\\
%&=
%\Biggl( \sum_{k=0}^\infty (k+2)\,(k+1) \, a_{k+2} x^k  \Biggr)
%+
%\Biggl( \sum_{k=3}^\infty 2 a_{k-3} x^k \Biggr) .
%\\
%&=
%2 a_2 +
%6 a_3 x +
%12 a_4 x^2 +
%\Biggl( \sum_{k=3}^\infty (k+2)\,(k+1) \, a_{k+2} x^k  \Biggr)
%+
%\Biggl( \sum_{k=3}^\infty 2 a_{k-3} x^k \Biggr) .
%\end{split}
%\end{equation*}
$a_2 = 0$, $a_3 = 0$, $a_4 = 0$, recurrence relation (for $k \geq 5$): $a_k
= \frac{- 2 a_{k-5}}{k(k-1)}$,
so:\\
$y(x) = a_0 + a_1 x -\frac{a_0}{10} x^5 - \frac{a_1}{15} x^6
+ \frac{a_0}{450} x^{10} + \frac{a_1}{825} x^{11}
- \frac{a_0}{47250} x^{15} - \frac{a_1}{99000} x^{16}
+
\cdots$
}


\begin{exercise}
Use power series methods to solve $y''-xy = 0$ at the point $x_0 = 1$.
\end{exercise}

\begin{exercise}
Use power series methods to solve $y''+x^2y = 0$ at the point $x_0 = 0$.
\end{exercise}

\begin{exercise}
The methods work for other orders than second order.  Try the methods
of this section to solve the first order system $y'-xy = 0$ at
the point $x_0 = 0$.
\end{exercise}

\begin{exercise}\ansMark%
Attempt to solve $x^2 y'' - y = 0$ at $x_0 = 0$ using the power series
method of this section ($x_0$ is a singular point).
Can you find at least one solution?  Can you find more than one solution?
\end{exercise}
\exsol{%
%\begin{equation*}
%\begin{split}
%0 = x^2 y''-y &= 
%x^2 \Biggl( \sum_{k=2}^\infty k\,(k-1) \, a_k x^{k-2}  \Biggr)
%-
%\Biggl( \sum_{k=0}^\infty a_k x^k \Biggr)
%\\
%&=
%\Biggl( \sum_{k=2}^\infty k\,(k-1) \, a_k x^k  \Biggr)
%-
%a_0
%-
%a_1 x
%-
%\Biggl( \sum_{k=2}^\infty a_k x^k \Biggr) .
%\end{split}
%\end{equation*}
%so $a_0 = 0$, $a_1 = 0$, $k(k-1) a_k = a_k$
Applying the method of this section directly we obtain $a_k = 0$ for
all $k$ and so $y(x) = 0$ is the only solution we find.
}

\begin{exercise}[\myindex{Chebyshev's equation of order $p$}]
\leavevmode
\begin{tasks}
\task Solve $(1-x^2)y''-xy' + p^2y = 0$ using power series methods at $x_0=0$.
\task For what $p$ is there a polynomial solution?
\end{tasks}
\end{exercise}

\begin{exercise}
Find a polynomial solution to $(x^2+1) y''-2xy'+2y = 0$ using
power series methods.
\end{exercise}

\begin{exercise}
\leavevmode
\begin{tasks}
\task Use power series methods to solve $(1-x)y''+y = 0$ at the point $x_0 = 0$.
\task Use the solution to part a) to find a solution
for $xy''+y=0$ around the point $x_0=1$.
\end{tasks}
\end{exercise}

\pagebreak[2]
\begin{exercise}[challenging]\ansMark%
Power series methods also work for nonhomogeneous equations.
\begin{tasks}
\task Use power series methods to solve $y'' - x y = \frac{1}{1-x}$
at the point $x_0 = 0$. Hint: Recall the geometric series.
\task Now solve for the initial condition $y(0)=0$, $y'(0) = 0$.
\end{tasks}
\end{exercise}
\exsol{%
%\begin{equation*}
%\begin{split}
%\frac{1}{1-x} =
%\sum_{k=0}^\infty x^k
%=
%y''-x y &= 
%\Biggl( \sum_{k=2}^\infty k\,(k-1) \, a_k x^{k-2}  \Biggr)
%-
%x
%\Biggl( \sum_{k=0}^\infty a_k x^k \Biggr)
%\\
%&=
%\Biggl( \sum_{k=2}^\infty k\,(k-1) \, a_k x^{k-2}  \Biggr)
%-
%\Biggl( \sum_{k=0}^\infty a_k x^{k+1} \Biggr) .
%\\
%&=
%\Biggl( \sum_{k=0}^\infty (k+2)\,(k+1) \, a_{k+2} x^k  \Biggr)
%-
%\Biggl( \sum_{k=1}^\infty a_{k-1} x^k \Biggr) .
%\\
%&=
%2 a_2 + 
%\Biggl( \sum_{k=1}^\infty (k+2)\,(k+1) \, a_{k+2} x^k  \Biggr)
%-
%\Biggl( \sum_{k=1}^\infty a_{k-1} x^k \Biggr) .
%\end{split}
%\end{equation*}
a) $a_2 = \frac{1}{2}$, and for $k \geq 1$ we have
$a_k = \frac{a_{k-3} + 1}{k(k-1)}$, so \\
$y(x) = a_0 + a_1 x + \frac{1}{2} x^2
+ \frac{a_0 + 1}{6} x^3
+ \frac{a_1 + 1}{12} x^4
+ \frac{3}{40} x^5
+ \frac{a_0 + 2}{30} x^6
+ \frac{a_1 + 2}{42} x^7
+ \frac{5}{112} x^8
+ \frac{a_0 + 3}{72} x^9
+ \frac{a_1 + 3}{90} x^{10} +
\cdots$
\\
b)
$y(x) = \frac{1}{2} x^2
+ \frac{1}{6} x^3
+ \frac{1}{12} x^4
+ \frac{3}{40} x^5
+ \frac{1}{15} x^6
+ \frac{1}{21} x^7
+ \frac{5}{112} x^8
+ \frac{1}{24} x^9
+ \frac{1}{30} x^{10} +
\cdots$
}

\setcounter{exercise}{100}

