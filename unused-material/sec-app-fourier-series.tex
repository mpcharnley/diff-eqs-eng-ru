
\section{Applications of Fourier series}
\label{appoffourier:section}

\LAtt{4.5}

\LO{
\item Apply Fourier Series to solve forced oscillation problems, and
\item Understand how resonance shows up in these types of problems.
}

% \sectionnotes{Verbatim from Lebl}

% \sectionnotes{2 lectures\EPref{, \S9.4 in \cite{EP}}\BDref{,
% not in \cite{BD}}}

\subsection{Periodically forced oscillation}

\begin{mywrapfigsimp}{2.0in}{2.3in}
\noindent
\inputpdft{massfigforce}
\end{mywrapfigsimp}
Let us return to the forced oscillations.  Consider a mass-spring system as
before, where we have a mass $m$
on a spring with spring constant $k$,
with damping $c$, and a force $F(t)$ applied to the mass.  Suppose 
the forcing function $F(t)$ is $2L$-periodic for some $L > 0$.
We saw
this problem in \chapterref{ho:chapter} with $F(t) = F_0 \cos (\omega t)$.  The
equation that governs this particular setup is
\begin{equation} \label{afs:eq}
mx''(t) + cx'(t) + kx(t) = F(t) .
\end{equation}

The general solution of \eqref{afs:eq} consists of the complementary solution $x_c$, which
solves the associated homogeneous equation $mx'' + cx' + kx = 0$, and
a particular solution of \eqref{afs:eq} we call $x_p$.  For $c > 0$,
the complementary solution $x_c$ will decay as time goes by.
Therefore,
we are mostly interested
in a particular solution $x_p$ that does not decay
and is periodic with the same period as $F(t)$.  We call this particular
solution
the \emph{\myindex{steady periodic solution}} and we write it as $x_{sp}$ as before.
What is new in this section is that we consider an arbitrary
forcing function $F(t)$ instead of a simple cosine.

For simplicity, suppose $c=0$.  The problem with $c > 0$ is very
similar.
The equation
\begin{equation*}
mx'' + kx = 0 
\end{equation*}
has the general solution
\begin{equation*}
x(t) = A \cos (\omega_0 t) + 
B \sin (\omega_0 t) ,
\end{equation*}
where $\omega_0 = \sqrt{\frac{k}{m}}$.
Any solution to
$mx''(t) + kx(t) = F(t)$ is of the form
$A \cos (\omega_0 t) + B \sin (\omega_0 t) + x_{sp}$.
The steady
periodic solution $x_{sp}$ has the same period as $F(t)$.

In the spirit of the last section and the idea of undetermined coefficients
we first write
\begin{equation*}
F(t) = \frac{c_0}{2} + \sum_{n=1}^\infty
c_n \cos \left( \frac{n \pi}{L} t \right) +
d_n \sin \left( \frac{n \pi}{L} t \right) .
\end{equation*}
Then we write a proposed steady periodic solution $x$ as
\begin{equation*}
x(t) = \frac{a_0}{2} + \sum_{n=1}^\infty
a_n \cos \left( \frac{n \pi}{L} t \right) +
b_n \sin \left( \frac{n \pi}{L} t \right) ,
\end{equation*}
where $a_n$ and $b_n$ are unknowns.
We plug $x$ into the differential equation and solve for $a_n$ and
$b_n$ in terms of $c_n$ and $d_n$.  This process
is perhaps best understood by example.
\pagebreak[2]

\begin{example} \label{afs:steadyex}
Suppose that $k=2$, and $m=1$.
The units are again the mks units\index{mks units}
(meters-kilograms-seconds).
There is a jetpack strapped to the mass, which fires with a force of 1
newton for 1
second and then is off for 1 second, and so on.  We want to find the steady periodic
solution.
\end{example}

\begin{exampleSol}
The equation is, therefore,
\begin{equation*}
x'' + 2 x = F(t) ,
\end{equation*}
where $F(t)$ is the step function
\begin{equation*}
F(t) =
\begin{cases}
0 & \text{if } \; {-1} < t < 0 , \\
1 & \text{if } \; \phantom{-}0 < t < 1 ,
\end{cases}
\end{equation*}
extended periodically.
We write
\begin{equation*}
F(t) = \frac{c_0}{2} + \sum_{n=1}^\infty
c_n \cos (n \pi t) +
d_n \sin (n \pi t) .
\end{equation*}
We compute
\begin{align*}
c_n & = \int_{-1}^1 F(t) \cos (n \pi t) \, dt = 
\int_{0}^1 \cos (n \pi t) \, dt = 0 \qquad \text{for } \; n \geq 1,
\\
c_0 & = \int_{-1}^1 F(t) \, dt = 
\int_{0}^1 \, dt = 1 ,
\\
d_n & = \int_{-1}^1 F(t) \sin (n \pi t) \, dt
\\
& = \int_{0}^1 \sin (n \pi t) \, dt
\\
& = \left[ \frac{-\cos (n \pi t)}{n \pi} \right]_{t=0}^1
\\
& = \frac{1-{(-1)}^n}{\pi n} =
\begin{cases}
\frac{2}{\pi n} & \text{if } n \text{ odd} , \\
0 & \text{if } n \text{ even} .
\end{cases}
\end{align*}
So
\begin{equation*}
F(t) = \frac{1}{2} + \sum_{\substack{n=1 \\ n \text{ odd}}}^\infty
\frac{2}{\pi n} \sin (n \pi t) .
\end{equation*}

We want to try
\begin{equation*}
x(t) = \frac{a_0}{2} + \sum_{n=1}^\infty
a_n \cos (n \pi t) +
b_n \sin (n \pi t) .
\end{equation*}
Once we plug $x$ into the differential equation $x''+2x = F(t)$,
it is clear that $a_n = 0$ for $n \geq 1$ as there are no corresponding terms
in the series for
$F(t)$.  Similarly $b_n = 0$ for $n$ even.  Hence we try
\begin{equation*}
x(t) = \frac{a_0}{2} +
\sum_{\substack{n=1 \\ n \text{ odd}}}^\infty
b_n \sin (n \pi t) .
\end{equation*}
We plug into the differential equation and obtain
\begin{equation*}
\begin{split}
x'' + 2 x & =
\sum_{\substack{n=1 \\ n \text{ odd}}}^\infty
\Bigl[ - b_n n^2 \pi^2 \sin (n \pi t) \Bigr] + 
a_0 +
2
\sum_{\substack{n=1 \\ n \text{ odd}}}^\infty
\Bigl[ b_n \sin (n \pi t) \Bigr]
\\
& =
a_0 +
\sum_{\substack{n=1 \\ n \text{ odd}}}^\infty
b_n (2 - n^2 \pi^2 ) \sin (n \pi t)
\\
& =
F(t) = \frac{1}{2} + \sum_{\substack{n=1 \\ n \text{ odd}}}^\infty
\frac{2}{\pi n} \sin (n \pi t) .
\end{split}
\end{equation*}
So $a_0 = \frac{1}{2}$, $b_n = 0$ for even $n$, and for odd $n$ we
get
\begin{equation*}
b_n = 
\frac{2}{\pi n (2 - n^2 \pi^2 )} .
\end{equation*}

The steady periodic solution has the Fourier series
\begin{equation*}
x_{sp}(t) = \frac{1}{4} + \sum_{\substack{n=1 \\ n \text{ odd}}}^\infty
\frac{2}{\pi n (2 - n^2 \pi^2 )}
\sin (n \pi t) .
\end{equation*}
We know this is the steady periodic solution as it contains no terms 
of the complementary solution and it is periodic with the same period as
$F(t)$ itself.  See \figurevref{afs:steadyexfig} for the plot of this solution.
\begin{myfig}
\capstart
\diffyincludegraphics{width=3in}{width=4.5in}{afs-steadyex}
\caption{Plot of the steady periodic solution $x_{sp}$ of
\exampleref{afs:steadyex}.%
\label{afs:steadyexfig}}
\end{myfig}
\end{exampleSol}

\subsection{Resonance}

Just as when the forcing function was a simple cosine, we may encounter
resonance.  Assume $c=0$ and let us discuss only pure resonance.
Let $F(t)$ be $2L$-periodic and consider
\begin{equation*}
m x''(t) + k x (t) = F(t) .
\end{equation*}
When we expand $F(t)$ and find that some of its terms coincide with the
complementary solution to $mx''+kx=0$, we cannot use those terms in the
guess.  Just like before, they disappear when we plug them into the left-hand
side and we get a contradictory equation (such as $0=1$).   That is,
suppose
\begin{equation*}
x_c = A \cos (\omega_0 t) + B \sin (\omega_0 t), 
\end{equation*}
where $\omega_0 = \frac{N \pi}{L}$ for some positive integer $N$.
We have
to modify our guess and try
\begin{equation*}
x(t) = \frac{a_0}{2} +
t \left(
a_N \cos \left( \frac{N \pi}{L} t \right) +
b_N \sin \left( \frac{N \pi}{L} t \right) \right) +
\sum_{\substack{n=1\\n\not= N}}^\infty
a_n \cos \left( \frac{n \pi}{L} t \right) +
b_n \sin \left( \frac{n \pi}{L} t \right) .
\end{equation*}
In other words, we multiply the offending term by $t$.  From then on, we
proceed as before.

Of course, the solution is not a Fourier series (it is not even
periodic) since it contains these terms multiplied by $t$.  Further, the
terms
$t \left( a_N \cos \left( \frac{N \pi}{L} t \right) +
b_N \sin \left( \frac{N \pi}{L} t \right) \right)$ eventually dominate and lead to
wild oscillations.  As before, this behavior is called \emph{\myindex{pure
resonance}} or just \emph{\myindex{resonance}}.

Note that there now may be infinitely many resonance frequencies to hit.
That is, as we change the frequency of $F$ (we change $L$), different
terms from the Fourier series of $F$ may interfere with the complementary
solution and cause resonance.
However, we should note that since everything is an approximation and in
particular $c$ is never actually zero but something very close to zero,
only the first
few resonance frequencies matter in real life.

\begin{example}
We want to solve the equation
\begin{equation} \label{afs:eq-resonance}
2 x'' + 18 \pi^2 x = F(t) ,
\end{equation}
where
\begin{equation*}
F(t) =
\begin{cases}
-1 & \text{if } \; {-1} < t < 0 , \\
1 & \text{if } \; \phantom{-}0 < t < 1 ,
\end{cases}
\end{equation*}
extended periodically. 
\end{example}

\begin{exampleSol}
We note that
\begin{equation*}
F(t) =
\sum_{\substack{n=1 \\ n \text{ odd}}}^\infty
\frac{4}{\pi n}
\sin (n \pi t) . 
\end{equation*}

\begin{exercise}
Compute the Fourier series of $F$ to verify the equation above.
\end{exercise}

As $\sqrt{\frac{k}{m}} = \sqrt{\frac{18\pi^2}{2}} = 3\pi$,
the solution to \eqref{afs:eq-resonance} is
\begin{equation*}
x(t) = c_1 \cos  (3\pi t) + c_2 \sin (3\pi t) + x_p (t)
\end{equation*}
for some particular solution $x_p$.

If we just try an $x_p$ given as a Fourier series with $\sin (n\pi t)$ as usual,
the complementary equation, $2x''+18\pi^2x=0$, eats our $3^\text{rd}$ harmonic.  That is, the term
with $\sin(3 \pi t)$
is already in
in our complementary solution.
Therefore, we pull that term out and
multiply it by $t$.  We also add a cosine term to get everything right.
That is, we try
\begin{equation*}
x_p(t) =
a_3
t \cos (3 \pi t )
+
b_3
t \sin (3 \pi t)
+
\sum_{\substack{n=1 \\ n~\text{odd} \\ n\not= 3}}^\infty
b_n
\sin (n \pi t) . 
\end{equation*}
Let us compute the second derivative.
\begin{multline*}
x_p''(t) =
- 6 a_3
\pi \, \sin (3 \pi t) - 9\pi^2 a_3 \, t \, \cos (3 \pi t)
+
6 b_3
\pi \, \cos (3 \pi t) - 9\pi^2 b_3 \, t \, \sin (3 \pi t)
\\
{} +
\sum_{\substack{n=1 \\ n~\text{odd} \\ n\not= 3}}^\infty
(-n^2 \pi^2 b_n ) \,
\sin (n \pi t) . 
\end{multline*}
We now plug into the left-hand side of the differential equation.
\begin{align*}
2x_p'' + 18\pi^2 x_p 
= & 
- 12 a_3 \pi \sin (3 \pi t)
- 18\pi^2 a_3 t \cos (3 \pi t)
+ 12 b_3 \pi \cos (3 \pi t)
- 18\pi^2 b_3 t \sin (3 \pi t)
\\
& \phantom{\, - 12 a_3 \pi \sin (3 \pi t)} ~
{} + 18 \pi^2 a_3 t \cos (3 \pi t)
\phantom{\, + 12 b_3 \pi \cos (3 \pi t)} ~
{} + 18 \pi^2 b_3 t \sin (3 \pi t)
\\
& {} + \sum_{\substack{n=1 \\ n~\text{odd} \\ n\not= 3}}^\infty
(-2n^2 \pi^2 b_n + 18\pi^2 b_n) \,
\sin (n \pi t) . 
\end{align*}
We simplify,
\begin{equation*}
2x_p'' + 18\pi^2 x_p =
- 12 a_3
\pi \sin (3 \pi t)
+
12 b_3
\pi \cos (3 \pi t)
+
\sum_{\substack{n=1 \\ n~\text{odd} \\ n\not= 3}}^\infty
(-2n^2 \pi^2 b_n + 18\pi^2 b_n)
\sin (n \pi t) . 
\end{equation*}
This series has to equal to the series for $F(t)$.
We equate the coefficients and solve for $a_3$ and $b_n$.
\begin{align*}
& a_3 = \frac{4/(3\pi)}{-12\pi} = \frac{-1}{9\pi^2} , \\
& b_3 = 0 , \\
& b_n = \frac{4}{n\pi(18\pi^2 - 2n^2 \pi^2)} 
= \frac{2}{\pi^3 n(9 - n^2)} \qquad \text{for } n \text{ odd and } n\not=3 .
\end{align*}

That is,
\begin{equation*}
x_p(t) =
\frac{-1}{9\pi^2}
\,
t \, \cos (3 \pi t)
+
\sum_{\substack{n=1 \\ n~\text{odd} \\ n\not= 3}}^\infty
\frac{2}{\pi^3 n(9 - n^2)}
\sin (n \pi t) . 
\end{equation*}
\end{exampleSol}

When $c > 0$, you do not have to worry about pure resonance.  That is,
there are never any conflicts and you do not need to multiply any
terms by $t$.  There is a corresponding concept of
\myindex{practical resonance}
and it is very similar to the ideas we already explored in
\chapterref{ho:chapter}.
Basically what happens in practical resonance is that one of the
coefficients in the series for $x_{sp}$ can get very big.  Let us not go
into details here.

\subsection{Exercises}

\begin{exercise}
Let $F(t) = \frac{1}{2} + \sum_{n=1}^\infty \frac{1}{n^2} \cos (n \pi t)$.
Find
the steady periodic solution to
$x'' + 2 x = F(t)$.  Express your solution as a Fourier series.
\end{exercise}

\begin{exercise}\ansMark%
Let $F(t) = \sin(2\pi t) + 0.1 \cos(10 \pi t)$.
Find the steady periodic solution to $x'' + \sqrt{2}\, x = F(t)$.
Express your solution as a Fourier series.
\end{exercise}
\exsol{%
$x = \frac{1}{\sqrt{2}-4 \pi^2} \sin(2\pi t) + \frac{0.1}{\sqrt{2}-100 \pi^2} \cos(10 \pi t)$
}

\begin{exercise}
Let $F(t) = \sum_{n=1}^\infty \frac{1}{n^3} \sin (n \pi t)$.  Find
the steady periodic solution to
$x'' + x' + x = F(t)$.  Express your solution as a Fourier series.
\end{exercise}

\begin{exercise}\ansMark%
Let $F(t) = \sum_{n=1}^\infty e^{-n} \cos(2 n t)$.
Find the steady periodic solution to $x'' + 3 x = F(t)$.
Express your solution as a Fourier series.
\end{exercise}
\exsol{%
$x =
\sum\limits_{n=1}^\infty
\frac{e^{-n}}{3-{(2n)}^2} \cos(2n t)$
}


\begin{exercise}
Let $F(t) = \sum_{n=1}^\infty \frac{1}{n^2} \cos (n \pi t)$.  Find
the steady periodic solution to
$x'' + 4 x = F(t)$.  Express your solution as a Fourier series.
\end{exercise}

\begin{exercise}
Let $F(t) = t$ for $-1 < t < 1$ and extended periodically.
Find the steady periodic solution to
$x'' + x = F(t)$.  Express your solution as a series.
\end{exercise}

\begin{exercise}\ansMark%
Let $F(t) = \lvert t \rvert$ for $-1 \leq t \leq 1$ extended periodically.
Find the steady periodic solution to $x'' + \sqrt{3}\, x = F(t)$.
Express your solution as a series.
\end{exercise}
\exsol{%
$x =
\frac{1}{2\sqrt{3}} + 
\sum\limits_{\substack{n=1 \\ n \text{ odd}}}^\infty \frac{-4}{n^2 \pi^2
(\sqrt{3}-n^2 \pi^2)} \cos (n \pi t)$
}

\begin{exercise}
Let $F(t) = t$ for $-1 < t < 1$ and extended periodically.
Find the steady periodic solution to
$x'' + \pi^2 x = F(t)$.  Express your solution as a series.
\end{exercise}

\begin{exercise}\ansMark%
Let $F(t) = \lvert t \rvert$ for $-1 \leq t \leq 1$ extended periodically.
Find the steady periodic solution to $x'' + \pi^2 x = F(t)$.
Express your solution as a series.
\end{exercise}
\exsol{%
$x =
\frac{1}{2\sqrt{3}} -
\frac{2}{\pi^3} t \sin(\pi t) + 
\sum\limits_{\substack{n=3 \\ n \text{ odd}}}^\infty \frac{-4}{n^2 \pi^4
(1-n^2)} \cos (n \pi t)$
}

\setcounter{exercise}{100}







