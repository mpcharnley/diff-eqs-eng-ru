\section{More on the Fourier series}
\label{moreonfourier:section}

\LAtt{4.3}

\LO{
\item Discuss Fourier series over intervals of different lengths,
\item Discuss the convergence of Fourier series, and
\item Compute derivatives and integrals of functions written as Fourier series.
}

% \sectionnotes{Verbatim from Lebl}

% \sectionnotes{2 lectures\EPref{, \S9.2--\S9.3 in \cite{EP}}\BDref{,
% \S10.3 in \cite{BD}}}

%Before reading the lecture, it may be good to first try
%Project IV (Fourier series)\index{IODE software!Project IV} from the
%IODE website: \url{http://www.math.uiuc.edu/iode/}.  After reading the
%lecture it may be good to continue with 
%Project V (Fourier series again)\index{IODE software!Project V}.

\subsection{$2L$-periodic functions}

We have computed the Fourier series for a $2\pi$-periodic function, but what
about functions of different periods.  Well, fear not, the computation is a
simple case of change of variables.  We just rescale the independent
axis.  Suppose we have a $2L$-periodic function $f(t)$.  Then $L$ is called
the \emph{\myindex{half period}}.  Let $s = \frac{\pi}{L}  t$.
Then the function
\begin{equation*}
g(s) = f\left(\frac{L}{\pi} s \right)
\end{equation*}
is $2\pi$-periodic.  We must also rescale all our sines and cosines.
In the series we use $\frac{\pi}{L} t$ as the variable.  That is, we
want to write
\begin{equation*}
\mybxbg{~~
f(t) = 
\frac{a_0}{2} +
\sum_{n=1}^\infty a_n \cos \left( \frac{n \pi}{L} t \right)
+ b_n \sin \left(\frac{n \pi}{L} t \right) .
~~}
\end{equation*}
If we change variables to $s$, we see that
\begin{equation*}
g(s) = 
\frac{a_0}{2} +
\sum_{n=1}^\infty a_n \cos (n s)
+ b_n \sin (n s) .
\end{equation*}
We compute $a_n$ and $b_n$ as before.  After we write down the
integrals, we change variables from $s$ back to $t$, noting also
that $ds = \frac{\pi}{L} \, dt$.
\begin{equation*}
\mybxbg{~~
\begin{aligned}
& a_0 =
\frac{1}{\pi}
\int_{-\pi}^\pi
g(s) \, ds
=
\frac{1}{L}
\int_{-L}^L
f(t) \, dt , \\
& a_n =
\frac{1}{\pi}
\int_{-\pi}^\pi
g(s) \, \cos (n s) \, ds
=
\frac{1}{L}
\int_{-L}^L
f(t) \, \cos \left( \frac{n \pi}{L} t \right) \, dt , \\
& b_n =
\frac{1}{\pi}
\int_{-\pi}^\pi
g(s) \, \sin (n s) \, ds
=
\frac{1}{L}
\int_{-L}^L
f(t) \, \sin \left( \frac{n \pi}{L} t \right) \, dt .
\end{aligned}
~~}
\end{equation*}

The two most common half periods that show up in examples
are $\pi$ and 1 because of the simplicity of the formulas.  We should stress that we have
done no new mathematics, we have only changed variables.  If you understand 
the Fourier series for $2\pi$-periodic functions, you understand it for
$2L$-periodic functions.  You can think of it as just using
different units for time.  All that we are doing is moving some constants
around, but all the mathematics is the same.

\begin{example}
Let
\begin{equation*}
f(t) =
\lvert t \rvert
\qquad \text{for } \; {-1} < t \leq 1,
\end{equation*}
extended periodically.  The plot of the
periodic extension is given in \figurevref{gfs:sawcontfig}.
Compute the Fourier series of $f(t)$.

\begin{myfig}
\capstart
\diffyincludegraphics{width=3in}{width=4.5in}{gfs-sawcont}
\caption{Periodic extension of the function $f(t)$.\label{gfs:sawcontfig}}
\end{myfig}
\end{example}
\begin{exampleSol}
We want to
write $f(t) = \frac{a_0}{2} + \sum_{n=1}^\infty a_n \cos (n \pi t) + b_n
\sin (n \pi t)$.  For $n \geq 1$ we note that $\lvert t \rvert \cos (n \pi t)$
is even and hence
\begin{equation*}
\begin{split}
a_n & = \int_{-1}^1 f(t) \cos (n \pi t) \, dt \\
& = 2 \int_{0}^1 t \cos (n \pi t) \, dt \\
 & = 2 \left[ \frac{t}{n \pi} \sin (n \pi t) \right]_{t=0}^1 -
2 \int_{0}^1 \frac{1}{n \pi} \sin (n \pi t) \, dt \\
& =  0 + \frac{1}{n^2 \pi^2} \Bigl[ \cos (n \pi t) \Bigr]_{t=0}^1
 =  \frac{2 \bigl( {(-1)}^n -1 \bigr) }{n^2 \pi^2}
=
\begin{cases}
0 & \text{if } n \text{ is even} , \\
\frac{-4 }{n^2 \pi^2} & \text{if } n \text{ is odd}  .
\end{cases}
\end{split}
\end{equation*}
Next we find $a_0$:
\begin{equation*}
a_0 = \int_{-1}^1 \lvert t \rvert \, dt 
=
1 .
\end{equation*}
You should be able to find this integral by thinking about the integral
as the area under the graph without doing any computation at all.
Finally we can find $b_n$.  Here, we notice that
$\lvert t \rvert \sin (n \pi t)$ is odd and, therefore,
\begin{equation*}
b_n = \int_{-1}^1 f(t) \sin (n \pi t) \, dt = 0 .
\end{equation*}
Hence,
the series is 
\begin{equation*}
\frac{1}{2} + 
\sum_{\substack{n=1 \\ n \text{ odd}}}^\infty \frac{-4}{n^2 \pi^2} \cos (n \pi t) .
\end{equation*}

Let us explicitly write down the first few terms of the series
up to the $3^{\text{rd}}$ harmonic.
\begin{equation*}
\frac{1}{2} -
\frac{4}{\pi^2} \cos (\pi t)
-
\frac{4}{9 \pi^2} \cos (3 \pi t)
- \cdots
\end{equation*}
The plot of these few terms and also a plot up to the ${20}^{\text{th}}$
harmonic is given in
\figurevref{gfs:sawcontfsfig}.  You should notice how close the graph is
to the real function.  You should also notice that there is no
\myquote{Gibbs phenomenon} present as there are no discontinuities.

\begin{myfig}
\capstart
%original files gfs-sawcontfs3 gfs-sawcont-fs20
\diffyincludegraphics{width=6.24in}{width=9in}{gfs-sawcont-fs3-fs20}
\caption{Fourier series of $f(t)$ up to the $3^{\text{rd}}$ harmonic (left
graph)
and up to the ${20}^{\text{th}}$ harmonic (right graph).\label{gfs:sawcontfsfig}}
\end{myfig}
\end{exampleSol}

\subsection{Convergence}

We will need the one sided limits of functions.
We will use the following notation
\begin{equation*}
f(c-) = \lim_{t \uparrow c} f(t),
\qquad \text{and} \qquad
f(c+) = \lim_{t \downarrow c} f(t).
\end{equation*}
If you are unfamiliar with this notation,
$\lim_{t \uparrow c} f(t)$ means we are taking a limit of $f(t)$
as $t$ approaches $c$ from below (i.e.\ $t < c$) and
$\lim_{t \downarrow c} f(t)$ means we are taking a limit of $f(t)$
as $t$ approaches $c$ from above (i.e.\ $t > c$).
For example, for the square wave function
\begin{equation} \label{gfs:sqwaveeq}
f(t) =
\begin{cases}
0 & \text{if } \; {-\pi} < t \leq 0 , \\
\pi & \text{if } \; \phantom{-}0 < t \leq \pi ,
\end{cases}
\end{equation}
we have $f(0-) = 0$ and $f(0+) = \pi$.

Let $f(t)$ be a function defined on an interval $[a,b]$.  Suppose
that we find finitely many points
$a=t_0$, $t_1$, $t_2$, \ldots, $t_k=b$ in
the interval, such that $f(t)$ is continuous
on the intervals
$(t_0,t_1)$, 
$(t_1,t_2)$, \ldots, 
$(t_{k-1},t_k)$.
Also suppose that all the one sided limits exist, that is,
all of
$f(t_0+)$,
$f(t_1-)$,
$f(t_1+)$,
$f(t_2-)$,
$f(t_2+)$,
\ldots,
$f(t_k-)$
exist and are finite.
Then
we say $f(t)$ is \emph{\myindex{piecewise continuous}}.

If moreover, $f(t)$ is differentiable at all but finitely many points,
and $f'(t)$ is piecewise continuous, then 
$f(t)$ is said to be \emph{\myindex{piecewise smooth}}.

\begin{example}
The square wave function \eqref{gfs:sqwaveeq}
is piecewise smooth on $[-\pi,\pi]$ or any other interval.  In such a
case we simply say that the function is piecewise smooth.
\end{example}

\begin{example}
The function $f(t) = \lvert t \lvert$
is piecewise smooth.
\end{example}

\begin{example}
The function $f(t) = \frac{1}{t}$ is not piecewise smooth on
$[-1,1]$ (or any other interval containing zero).  In fact, it is not
even piecewise continuous.
\end{example}

\begin{example}
The function $f(t) = \sqrt[3]{t}$ is not piecewise smooth on
$[-1,1]$ (or any other interval containing zero).  $f(t)$ is continuous, but
the derivative of $f(t)$ is unbounded near zero and hence not piecewise
continuous.
\end{example}

Piecewise smooth functions have an easy answer on the convergence
of the Fourier series.

\begin{theorem1}{}
Suppose $f(t)$ is a $2L$-periodic piecewise smooth function.
Let
\begin{equation*}
\frac{a_0}{2} + \sum_{n=1}^\infty a_n \cos \left( \frac{n \pi}{L} t
\right)
+ b_n \sin \left( \frac{n \pi}{L} t \right)
\end{equation*}
be the Fourier series for $f(t)$.  Then the series converges
for all $t$.  If $f(t)$ is continuous
at $t$, then
\begin{equation*}
f(t) = \frac{a_0}{2} + \sum_{n=1}^\infty
a_n \cos \left( \frac{n \pi}{L} t \right)
+ b_n \sin \left( \frac{n \pi}{L} t \right) .
\end{equation*}
Otherwise,
\begin{equation*}
\frac{f(t-)+f(t+)}{2} =
\frac{a_0}{2} + \sum_{n=1}^\infty a_n \cos \left( \frac{n \pi}{L}  t
\right)
+ b_n \sin \left( \frac{n \pi}{L} t \right) .
\end{equation*}
\end{theorem1}

If we happen to have that
$f(t) = \frac{f(t-)+f(t+)}{2}$ at all the discontinuities, the Fourier series
converges to $f(t)$ everywhere.  We can always just redefine $f(t)$
by changing the value at each discontinuity appropriately.  Then we can write
an equals sign between $f(t)$ and the series without any worry.
We mentioned this fact
briefly at the end last section.

The theorem does not say how fast the series converges.
Think back to the discussion of the Gibbs phenomenon in the last section.
The closer you get to the discontinuity, the more terms you need to take
to get an accurate approximation to the function.

\subsection{Differentiation and integration of Fourier series}

Not only does Fourier series converge nicely, but it is easy to differentiate
and integrate the series.  We can do this just by differentiating or
integrating term by term.

\begin{theorem1}{}
Suppose
\begin{equation*}
f(t) = \frac{a_0}{2} + \sum_{n=1}^\infty a_n \cos \left( \frac{n \pi}{L} t
\right)
+ b_n \sin \left( \frac{n \pi}{L} t \right)
\end{equation*}
is a piecewise smooth continuous function and the derivative $f'(t)$ is
piecewise smooth.  Then the derivative can be
obtained by differentiating term by term,
\begin{equation*}
f'(t) = \sum_{n=1}^\infty \frac{-a_n n \pi}{L} 
\sin \left( \frac{n \pi}{L} t \right)
+ \frac{b_n n \pi}{L} \cos \left( \frac{n \pi}{L} t \right) .
\end{equation*}
\end{theorem1}

It is important that the function is continuous.  It can have corners, but no
jumps.  Otherwise, the differentiated series will fail to converge.  For an
exercise, take the series obtained for the square wave and try to
differentiate the series.  Similarly, we can also integrate a Fourier series.

\begin{theorem1}{}
Suppose
\begin{equation*}
f(t) = \frac{a_0}{2} + \sum_{n=1}^\infty
a_n \cos \left( \frac{n \pi}{L} t \right)
+ b_n \sin \left( \frac{n \pi}{L} t \right)
\end{equation*}
is a piecewise smooth function.  Then the antiderivative is
obtained by antidifferentiating term by term and so
\begin{equation*}
F(t) = \frac{a_0 t}{2} + C + \sum_{n=1}^\infty
\frac{a_n L}{n \pi} \sin \left( \frac{n \pi}{L} t \right)
+ \frac{-b_n L}{n \pi}  \cos \left( \frac{n \pi}{L} t \right) ,
\end{equation*}
where $F'(t) = f(t)$ and $C$ is an arbitrary constant.
\end{theorem1}

Note that the series for $F(t)$ is no longer a Fourier series as it contains
the $\frac{a_0 t}{2}$ term.  The antiderivative
of a periodic function need no longer be periodic and so we should not
expect a Fourier series.

\subsection{Rates of convergence and smoothness}

Let us do an example of a periodic function with one derivative everywhere.

\begin{example}
Take the function
\begin{equation*}
f(t) =
\begin{cases}
(t+1)\,t & \text{if } \; {-1} < t \leq 0 , \\
(1-t)\,t & \text{if } \; \phantom{-}0 < t \leq 1 ,
\end{cases}
\end{equation*}
and extend to a
2-periodic function.  The plot is given in
\figurevref{gfs:smoothexfig}.

\begin{myfig}
\capstart
\diffyincludegraphics{width=3in}{width=4.5in}{gfs-smoothex}
\caption{Smooth 2-periodic function.\label{gfs:smoothexfig}}
\end{myfig}

This function has one derivative everywhere, but it
does not have a second derivative whenever $t$ is an integer.

\begin{exercise}
Compute $f''(0+)$ and $f''(0-)$.
\end{exercise}

Let us compute the Fourier series coefficients.  The actual computation
involves several integration by parts and is left to student.
\begin{align*}
a_0 & = 
\int_{-1}^1
f(t) \, dt = 
\int_{-1}^0
(t+1)\,t \, dt +
\int_0^1
(1-t)\,t \, dt = 0 , \\
a_n & = 
\int_{-1}^1
f(t) \, \cos (n\pi t) \, dt = 
\int_{-1}^0
(t+1)\,t
\, \cos (n \pi t) \, dt +
\int_0^1
(1-t)\,t
\, \cos (n \pi t) \, dt = 0, \\
b_n & = 
\int_{-1}^1
f(t) \, \sin (n\pi t) \, dt = 
\int_{-1}^0
(t+1)\,t
\, \sin (n \pi t) \, dt +
\int_0^1
(1-t)\,t
\, \sin (n \pi t) \, dt \\
& =
\frac{4 ( 1-{(-1)}^n)}{\pi^3 n^3} 
=
\begin{cases}
\frac{8}{\pi^3 n^3} & \text{if } n \text{ is odd} , \\
0 & \text{if } n \text{ is even} .
\end{cases}
\end{align*}
That is, the series is
\begin{equation*}
\sum_{\substack{n=1 \\ n \text{ odd}}}^\infty \frac{8}{\pi^3 n^3} \sin (n \pi t) .
\end{equation*}

This series converges very fast.
If you plot up
to the third harmonic, that is the function
\begin{equation*}
\frac{8}{\pi^3} \sin (\pi t) + 
\frac{8}{27 \pi^3} \sin (3 \pi t) ,
\end{equation*}
it is almost indistinguishable from the plot of $f(t)$ in
\figurevref{gfs:smoothexfig}.
In fact, the coefficient 
$\frac{8}{27 \pi^3}$ is already just 0.0096 (approximately).
The reason for this behavior is the $n^3$ term in the denominator.
The coefficients $b_n$ in this case go to zero as fast as
$\nicefrac{1}{n^3}$ goes to
zero.
\end{example}

For functions constructed piecewise from polynomials as above,
it is generally true that
if you have one derivative, the Fourier
coefficients will go to zero approximately like $\nicefrac{1}{n^3}$.  If you
have only a continuous function, then the Fourier coefficients will go to
zero as $\nicefrac{1}{n^2}$.  If you have discontinuities, then 
the Fourier coefficients will go to zero approximately as $\nicefrac{1}{n}$.
For more general functions the story is somewhat more complicated but the
same idea holds, the more derivatives you have, the faster the coefficients
go to zero.  Similar reasoning works in reverse.  If the coefficients go to
zero like $\nicefrac{1}{n^2}$, you always obtain a continuous function.  If
they go to zero like $\nicefrac{1}{n^3}$, you obtain an everywhere differentiable
function.
%Therefore, we can tell a lot about the smoothness of a function by looking
%at its Fourier coefficients.

To justify this behavior, take for example the function defined by
the Fourier series
\begin{equation*}
f(t) = \sum_{n=1}^\infty \frac{1}{n^3} \sin (n t) .
\end{equation*}
When we differentiate term by term we notice
\begin{equation*}
f'(t) = \sum_{n=1}^\infty \frac{1}{n^2} \cos (n t) .
\end{equation*}
Therefore, the coefficients now go down like $\nicefrac{1}{n^2}$, which 
means that we have a continuous function.
The derivative 
of $f'(t)$ is defined
at most points, but there are points where $f'(t)$ is not differentiable.
It has corners, but no jumps.
If we
differentiate again (where we can), we find that the function
$f''(t)$,
now fails to be continuous (has jumps)
\begin{equation*}
f''(t) = \sum_{n=1}^\infty \frac{-1}{n} \sin (n t) .
\end{equation*}
This function is similar to the sawtooth.  If we tried to differentiate
the series again, we would obtain
\begin{equation*}
\sum_{n=1}^\infty -\cos (n t) ,
\end{equation*}
which does not converge!

\begin{exercise}
Use a computer to plot the series we obtained for $f(t)$, $f'(t)$ and
$f''(t)$.  That is, plot say the first 5 harmonics of the functions.  At what
points does $f''(t)$ have the discontinuities?
\end{exercise}

\subsection{Exercises}

\begin{exercise}
Let
\begin{equation*}
f(t) =
\begin{cases}
0 & \text{if } \; {-1} < t \leq 0 , \\
t & \text{if } \; \phantom{-}0 < t \leq  1 ,
\end{cases}
\end{equation*}
extended periodically.
\begin{tasks}
\task Compute the Fourier series for $f(t)$.
\task Write out the series explicitly up to the $3^{\text{rd}}$ harmonic.
\end{tasks}
\end{exercise}

\begin{exercise}\ansMark%
Let
\begin{equation*}
f(t) = t^2 \qquad \text{for } \; {-2} < t \leq 2
\end{equation*}
extended periodically.
\begin{tasks}
\task Compute the Fourier series for $f(t)$.
\task Write out the series explicitly up to the $3^{\text{rd}}$ harmonic.
\end{tasks}
\end{exercise}
\exsol{%
a) $\frac{8}{6} +
\sum\limits_{n=1}^\infty
\frac{16{(-1)}^n}{\pi^2 n^2}
\cos\bigl(\frac{n\pi}{2} t\bigr)$
\quad
b) $\frac{8}{6}
-
\frac{16}{\pi^2 }
\cos\bigl(\frac{\pi}{2} t\bigr)
+
\frac{4}{\pi^2}
\cos\bigl(\pi t\bigr)
-
\frac{16}{9\pi^2}
\cos\bigl(\frac{3\pi}{2} t\bigr) + \cdots$
}

\begin{exercise}
Let
\begin{equation*}
f(t) =
\begin{cases}
-t & \text{if } \; {-1} < t \leq 0 , \\
t^2 & \text{if } \; \phantom{-}0 < t \leq  1 ,
\end{cases}
\end{equation*}
extended periodically.
\begin{tasks}
\task Compute the Fourier series for $f(t)$.
\task Write out the series explicitly up to the $3^{\text{rd}}$ harmonic.
\end{tasks}
\end{exercise}

\begin{exercise}\ansMark%
Let
\begin{equation*}
f(t) = t \qquad \text{for } \; {-\lambda} < t \leq \lambda \; \text{ (for some } \lambda > 0 \text{)}
\end{equation*}
extended periodically.
\begin{tasks}
\task Compute the Fourier series for $f(t)$.
\task Write out the series explicitly up to the $3^{\text{rd}}$ harmonic.
\end{tasks}
\end{exercise}
\exsol{%
a)
$\sum\limits_{n=1}^\infty
\frac{{(-1)}^{n+1}2\lambda}{n \pi}
\sin\bigl(\frac{n\pi}{\lambda} t\bigr)$
\quad
b)
$\frac{2\lambda}{\pi}
\sin\bigl(\frac{\pi}{\lambda} t\bigr)
-
\frac{\lambda}{\pi}
\sin\bigl(\frac{2\pi}{\lambda} t\bigr)
+
\frac{2\lambda}{3\pi}
\sin\bigl(\frac{3\pi}{\lambda} t\bigr) - \cdots$
}

\begin{exercise}
Let
\begin{equation*}
f(t) =
\begin{cases}
\frac{-t}{10} & \text{if } \; {-10} < t \leq 0 , \\
\frac{t}{10} & \text{if } \; \phantom{-1}0 < t \leq  10 ,
\end{cases}
\end{equation*}
extended periodically (period is 20).
\begin{tasks}
\task Compute the Fourier series for $f(t)$.
\task Write out the series explicitly up to the $3^{\text{rd}}$ harmonic.
\end{tasks}
\end{exercise}

\begin{exercise}\ansMark%
Let
\begin{equation*}
f(t) = \frac{1}{2} + \sum_{n=1}^\infty
\frac{1}{n(n^2+1)}
\sin(n\pi t) .
\end{equation*}
Compute $f'(t)$.
\end{exercise}
\exsol{%
$f'(t) = \sum\limits_{n=1}^\infty
\frac{\pi}{n^2+1}
\cos(n\pi t)$
}

\begin{exercise}
Let $f(t) = \sum_{n=1}^\infty \frac{1}{n^3} \cos (n t)$.  Is $f(t)$
continuous and differentiable everywhere?  Find the derivative (if it exists
everywhere)
or justify why $f(t)$ is not differentiable everywhere.
\end{exercise}

\begin{exercise}
Let $f(t) = \sum_{n=1}^\infty \frac{{(-1)}^n}{n} \sin (n t)$.  Is $f(t)$
differentiable everywhere?  Find the derivative (if it exists everywhere) or
justify why $f(t)$ is not differentiable everywhere.
\end{exercise}

\begin{exercise}\ansMark%
Let
\begin{equation*}
f(t) = \frac{1}{2} + \sum_{n=1}^\infty
\frac{1}{n^3}
\cos(n t) .
\end{equation*}
\begin{tasks}
\task Find the antiderivative.
\task Is the antiderivative periodic?
\end{tasks}
\end{exercise}
\exsol{%
a)
$F(t) = \frac{t}{2} + C + \sum\limits_{n=1}^\infty
\frac{1}{n^4}
\sin(nt)$
\qquad
b) no
}

\begin{exercise}
Let
\begin{equation*}
f(t) =
\begin{cases}
0 & \text{if } \; {-2} < t \leq 0, \\
t & \text{if } \; \phantom{-}0 < t \leq 1, \\
-t+2 & \text{if } \; \phantom{-}1 < t \leq 2,
\end{cases}
\end{equation*}
extended periodically.
\begin{tasks}
\task Compute the Fourier series for $f(t)$.
\task Write out the series explicitly up to the $3^{\text{rd}}$ harmonic.
\end{tasks}
\end{exercise}

\begin{exercise}
Let
\begin{equation*}
f(t) = e^t \qquad \text{for } \; {-1} < t \leq 1
\end{equation*}
extended periodically.
\begin{tasks}
\task Compute the Fourier series for $f(t)$.
\task Write out the series explicitly up to the $3^{\text{rd}}$ harmonic.
\task What does the series converge to at $t=1$.
\end{tasks}
\end{exercise}

\begin{exercise}
Let
\begin{equation*}
f(t) = t^2 \qquad \text{for } \; {-1} < t \leq 1
\end{equation*}
extended periodically.
\begin{tasks}
\task Compute the Fourier series for $f(t)$.
\task By plugging in $t=0$,
evaluate $\displaystyle \sum_{n=1}^\infty \frac{{(-1)}^n}{n^2} = 1 - \frac{1}{4} +
\frac{1}{9} - \cdots$.
\task Now evaluate $\displaystyle \sum_{n=1}^\infty \frac{1}{n^2} = 1 + \frac{1}{4} +
\frac{1}{9} + \cdots$.
\end{tasks}
\end{exercise}

\begin{exercise}\ansMark%
Let
\begin{equation*}
f(t) = \nicefrac{t}{2} \qquad \text{for } \; {-\pi} < t < \pi
\end{equation*}
extended periodically.
\begin{tasks}
\task Compute the Fourier series for $f(t)$.
\task Plug in $t=\nicefrac{\pi}{2}$ to find a series representation
for $\nicefrac{\pi}{4}$.
\task Using the first 4 terms of the result from part b) approximate
$\nicefrac{\pi}{4}$.
\end{tasks}
\end{exercise}
\exsol{%
a)
$\sum\limits_{n=1}^\infty
\frac{{(-1)}^{n+1}}{n} \sin(nt)$
\qquad
b) $f$ is continuous at $t=\nicefrac{\pi}{2}$ so the
Fourier series converges to $f(\nicefrac{\pi}{2}) = \nicefrac{\pi}{4}$.
Obtain
$\nicefrac{\pi}{4} = \sum\limits_{n=1}^\infty
\frac{{(-1)}^{n+1}}{2n-1} = 1 - \nicefrac{1}{3} + \nicefrac{1}{5}-
\nicefrac{1}{7} + \cdots$.
\qquad
c) Using the first 4 terms get $\nicefrac{76}{105}\approx 0.72$ (quite a bad
approximation, you would have to take about 50 terms to start to get to
within $0.01$ of $\nicefrac{\pi}{4}$).
}

\begin{exercise}
Let
\begin{equation*}
f(t) =
\begin{cases}
0 & \text{if } \; {-3} < t \leq 0, \\
t & \text{if } \; \phantom{-}0 < t \leq 3,
\end{cases}
\end{equation*}
extended periodically.  Suppose $F(t)$ is the function given
by the Fourier series of $f$.  Without computing the Fourier series
evaluate
\begin{tasks}(3)
\task $F(2)$
\task $F(-2)$
\task $F(4)$
\task $F(-4)$
\task $F(3)$
\task $F(-9)$
\end{tasks}
\end{exercise}

\begin{exercise}\ansMark%
Let
\begin{equation*}
f(t) = 
\begin{cases}
0 & \text{if } \; {-2} < t \leq 0, \\
2 & \text{if } \; \phantom{-}0 < t \leq 2,
\end{cases}
\end{equation*}
extended periodically.  Suppose $F(t)$ is the function given
by the Fourier series of $f$.  Without computing the Fourier series
evaluate
\begin{tasks}(3)
\task $F(0)$
\task $F(-1)$
\task $F(1)$
\task $F(-2)$
\task $F(4)$
\task $F(-8)$
\end{tasks}
\end{exercise}
\exsol{%
a) $F(0) = 1$, 
b) $F(-1) = 0$, 
c) $F(1) = 2$, 
d) $F(-2) = 1$, 
e) $F(4) = 1$, 
f) $F(-9) = 0$ 
}

\setcounter{exercise}{100}

