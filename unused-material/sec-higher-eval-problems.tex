\section{Higher order eigenvalue problems}
\label{sec:appeig}

\sectionnotes{Verbatim from Lebl}

\sectionnotes{1 lecture\EPref{, \S10.2 in \cite{EP}}\BDref{,
exercises in \S11.2 in \cite{BD}}}

The eigenfunction series can arise even from higher order equations.
Consider an elastic beam (say made of steel).  We will study the
transversal vibrations of the beam.  That is, suppose the beam lies along
the $x$-axis and let $y(x,t)$ measure the displacement of the point $x$
on the beam at time $t$.  See \figurevref{appeig:transbeamfig}.

\begin{myfig}
\capstart
\inputpdft{trans-beam}
\caption{Transversal vibrations of a beam.\label{appeig:transbeamfig}}
\end{myfig}

The equation that governs this setup is
\begin{equation*}
a^4 \frac{\partial^4 y}{\partial x^4} + \frac{\partial^2 y}{\partial t^2} = 0,
\end{equation*}
for some constant $a > 0$, let us not worry about the physics\footnote{If
you are interested, $a^4 = \frac{EI}{\rho}$, where $E$ is the elastic
modulus, $I$ is the second moment of area of the cross section,
and $\rho$ is linear density.}.

Suppose the beam is of length 1 simply supported (hinged) at the ends.
The beam is displaced by some function $f(x)$ at time $t=0$ and then
let go (initial velocity is 0).  Then $y$ satisfies:
\begin{equation} \label{appeig:beameq}
\begin{aligned}
& a^4 y_{xxxx} + y_{tt} = 0 \qquad (0 < x < 1, \enspace t > 0), \\
& y(0,t) = y_{xx}(0,t) = 0 , \\
& y(1,t) = y_{xx}(1,t) = 0 , \\
& y(x,0) = f(x), \qquad y_{t}(x,0) = 0 .
\end{aligned}
\end{equation}

Again we try $y(x,t) = X(x)T(t)$ and plug in to get
$a^4 X^{(4)}T + XT'' = 0$ or 
\begin{equation*}
\frac{X^{(4)}}{X} = \frac{- T''}{a^4T} = \lambda .
\end{equation*}
The equations are
\begin{equation*}
T'' + \lambda a^4 T = 0, \qquad
X^{(4)} - \lambda X = 0 .
\end{equation*}
The boundary conditions
$y(0,t) = y_{xx}(0,t) = 0$ and $y(1,t) = y_{xx}(1,t) = 0$ imply
\begin{equation*}
X(0) =  X''(0) = 0, \qquad \text{and} \qquad
X(1) =  X''(1) = 0 .
\end{equation*}
and the initial homogeneous condition $y_t(x,0) = 0$ implies
\begin{equation*}
T'(0) = 0 .
\end{equation*}
As usual, we leave the nonhomogeneous $y(x,0) = f(x)$ for later.

Considering the equation for $T$, that is,
$T'' + \lambda a^4 T = 0$, and physical intuition leads us
to the fact that if $\lambda$ is an eigenvalue then $\lambda > 0$.
This is because we expect vibration and not exponential growth nor
decay in the $t$ direction (there is no friction in our model for instance).
So there are no negative eigenvalues.
Similarly $\lambda = 0$ is not an eigenvalue.

\begin{exercise}
Justify $\lambda > 0$ just from the equation for $X$ and the boundary
conditions.
\end{exercise}

Let $\omega = \sqrt[4]{\lambda}$, that is $\omega^4 = \lambda$,
so that we do not need to write the fourth root
all the time.  Notice $\omega > 0$.
The equation $X^{(4)} - \omega^4 X = 0$ has the
general solution is
\begin{equation*}
X(x) = A e^{\omega x} + B e^{-\omega x} + C \sin (\omega x) +
D \cos (\omega x) .
\end{equation*}
Now $0 = X(0) = A+B+D$, $0 = X''(0) = \omega^2 (A + B - D)$.  Hence, $D = 0$ and $A+B = 0$, or $B = - A$.  So we have
\begin{equation*}
X(x) = A e^{\omega x} - A e^{-\omega x} + C \sin (\omega x) .
\end{equation*}
Also $0 = X(1) = A (e^{\omega} - e^{-\omega}) + C \sin \omega$, and
$0 = X''(1) = A \omega^2 (e^{\omega} - e^{-\omega}) - C \omega^2 \sin \omega$.
This means that $C \sin \omega  = 0$ and 
$A (e^{\omega} - e^{-\omega}) = 2 A \sinh \omega = 0$.  If $\omega > 0$, then
$\sinh \omega \not= 0$ and so $A = 0$.  This means that $C \not=0$ otherwise
$\lambda$ is not an
eigenvalue.  Also $\omega$ must be an integer multiple of
$\pi$.   Hence $\omega = n \pi$ and $n \geq 1$ (as $\omega > 0$).  We can take
$C=1$.  So the eigenvalues are $\lambda_n = n^4 \pi^4$ and corresponding eigenfunctions
are $\sin (n \pi x)$.

Now 
$T'' + n^4 \pi^4 a^4 T = 0$.  The general solution is $T(t) =
A \sin (n^2 \pi^2 a^2 t) + B \cos (n^2 \pi^2 a^2 t)$.  But $T'(0) = 0$ and hence
we must have $A=0$.  We can take $B=1$ to make $T(0) = 1$ for convenience.
So our solutions are $T_n(t) = \cos (n^2 \pi^2 a^2 t)$.

As eigenfunctions are just sines again, we decompose the function
$f(x)$ on $0 < x < 1$ using the sine series.
We find numbers $b_n$ such that for
$0 < x < 1$ we have
\begin{equation*}
f(x) = \sum_{n=1}^\infty b_n \sin (n \pi x) .
\end{equation*}
Then the solution to \eqref{appeig:beameq} is
\begin{equation*}
y(x,t) = \sum_{n=1}^\infty b_n
X_n(x) T_n(t)
= \sum_{n=1}^\infty b_n
\sin (n \pi x)  \cos ( n^2 \pi^2 a^2 t ) .
\end{equation*}
The point is that $X_nT_n$ is a solution that satisfies all the homogeneous
conditions (that is, all conditions except the initial position).  And since
$T_n(0) = 1$, we have
\begin{equation*}
y(x,0) = \sum_{n=1}^\infty b_n X_n(x) T_n(0) = 
\sum_{n=1}^\infty b_n X_n(x) =
\sum_{n=1}^\infty b_n
\sin (n \pi x) = f(x) .
\end{equation*}
So $y(x,t)$ solves \eqref{appeig:beameq}.

The natural (angular) frequencies of the system are $n^2 \pi^2 a^2$.
These frequencies are all integer multiples of the fundamental frequency
$\pi^2 a^2$, so we get a nice musical note.  The exact frequencies
and their amplitude
are what musicians call the \emph{\myindex{timbre}} of the note (outside
of music it is called the spectrum).

The timbre of a beam
is different than for a vibrating string where we get \myquote{more}
of the lower frequencies since we get all integer multiples,
$1,2,3,4,5,\ldots$.  For a steel beam we get
only the square multiples $1,4,9,16,25,\ldots$.  That is why when you hit a
steel beam you hear a very pure sound.  The sound of a
xylophone or vibraphone is, therefore, very different from a guitar or piano.

\begin{example}
Let us assume that $f(x) = \frac{x(x-1)}{10}$.  
On $0 < x < 1$ we have (you know how to do this by now)
\begin{equation*}
f(x) = \sum_{\substack{n=1\\n \text{~odd}}}^\infty \frac{4}{5\pi^3 n^3}
\sin (n \pi x) .
\end{equation*}
Hence, the solution to \eqref{appeig:beameq} with the given initial
position $f(x)$ is
\begin{equation*}
y(x,t) = \sum_{\substack{n=1\\n \text{~odd}}}^\infty \frac{4}{5\pi^3 n^3}
\sin (n \pi x) \cos ( n^2 \pi^2 a^2 t ) .
\end{equation*}
\end{example}

There are other boundary conditions than just hinged ends.  There are
three basic possibilities: hinged, free, or fixed.  Let us consider
the end at $x=0$.  For the other end, it is the same idea.
If the end is \emph{hinged}\index{hinged end of beam}, then
\begin{equation*}
u(0,t) = u_{xx}(0,t) = 0 .
\end{equation*}
If the end is \emph{free}\index{free end of beam}, that is, it is just
floating in air, then
\begin{equation*}
u_{xx}(0,t) = u_{xxx}(0,t) = 0 .
\end{equation*}
And finally, if the end is
\emph{clamped}\index{clamped end of beam}
or  
\emph{fixed}\index{fixed end of beam}, for example it is welded to a
wall,
then
\begin{equation*}
u(0,t) = u_{x}(0,t) = 0 .
\end{equation*}


\subsection{Exercises}

\begin{exercise}
Suppose you have a beam of length 5 with free ends.  Let $y$ be the
transverse deviation of the beam at position $x$ on the beam ($0 < x < 5$).
You know that the
constants are such that this satisfies the equation $y_{tt} + 4 y_{xxxx} =
0$.   Suppose you know that the initial shape of the beam is the graph of
$x(5-x)$, and the initial velocity is uniformly equal to 2 (same for each $x$)
in the positive $y$ direction.  Set up the equation together with the
boundary and initial conditions.  Just set up, do not solve.
\end{exercise}

\begin{exercise}
Suppose you have a beam of length 5 with one end free and one end fixed
(the fixed end is at $x=5$).
Let $u$ be the
longitudinal deviation of the beam at position $x$ on the beam ($0 < x < 5$).
You know that the
constants are such that this satisfies the equation $u_{tt} = 4 u_{xx}$.
Suppose you know that the initial displacement of the beam
is $\frac{x-5}{50}$, and the initial velocity is $\frac{-(x-5)}{100}$
in the positive $u$ direction.  Set up the equation together with the
boundary and initial conditions.  Just set up, do not solve.
\end{exercise}

\begin{exercise}
Suppose the beam is $L$ units long, everything else kept the same
as in \eqref{appeig:beameq}.  What is the equation and the series
solution?
\end{exercise}

\begin{exercise}
Suppose you have 
\begin{equation*}
\begin{aligned}
& a^4 y_{xxxx} + y_{tt} = 0 \quad (0 < x < 1, t > 0) , \\
& y(0,t) = y_{xx}(0,t) = 0,\\
& y(1,t) = y_{xx}(1,t) = 0 ,\\
& y(x,0) = f(x), \quad y_{t}(x,0) = g(x) .
\end{aligned}
\end{equation*}
That is, you have also an initial velocity.  Find a series solution.  Hint:
Use the same idea as we did for the wave equation.
\end{exercise}

\setcounter{exercise}{100}

\begin{exercise}
Suppose you have a beam of length 1 with hinged ends.  Let $y$ be the
transverse deviation of the beam at position $x$ on the beam ($0 < x < 1$).
You know that the
constants are such that this satisfies the equation $y_{tt} + 4 y_{xxxx} =
0$.   Suppose you know that the initial shape of the beam is the graph of
$\sin (\pi x)$, and the initial velocity is 0.  Solve for $y$.
\end{exercise}
\exsol{%
$y(x,t) = \sin(\pi x) \cos (2 \pi^2 t)$
}

\begin{exercise}
Suppose you have a beam of length 10 with two fixed ends.  Let $y$ be the
transverse deviation of the beam at position $x$ on the beam ($0 < x < 10$).
You know that the
constants are such that this satisfies the equation $y_{tt} + 9 y_{xxxx} =
0$.   Suppose you know that the initial shape of the beam is the graph of
$\sin(\pi x)$, and the initial velocity is uniformly equal to $x(10-x)$.
Set up the equation together with the
boundary and initial conditions.  Just set up, do not solve.
\end{exercise}
\exsol{%
$9 y_{xxxx} + y_{tt} = 0 \quad (0 < x < 10, t > 0)$, \quad
$y(0,t) = y_{x}(0,t) = 0$, \quad
$y(10,t) = y_{x}(10,t) = 0$, \quad
$y(x,0) = \sin(\pi x), \quad y_{t}(x,0) = x(10-x)$.
}

