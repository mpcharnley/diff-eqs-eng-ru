\section{Solving PDEs with the Laplace transform}
\label{laplacepde:section}

\LAtt{6.5}

\LO{
\item Use the Laplace transform in one variable to solve PDE
}

% \sectionnotes{Verbatim from Lebl}

% \sectionnotes{1--1.5 lecture, can be skipped}

The Laplace transform comes from the same family of transforms as does the
Fourier series\footnote{There is a corresponding Fourier transform
on the real line as well that looks sort of like the Laplace transform.},
which we used in \chapterref{FS:chapter} to solve partial differential
equations (PDEs).
It is therefore not surprising that we can also solve PDEs with the Laplace
transform.

Given a PDE in two independent variables $x$ and $t$,  we use the
Laplace transform on one of the variables (taking the transform
of everything in sight), and derivatives in that variable become
multiplications by the transformed variable $s$.
The PDE becomes an ODE, which we solve.
Afterwards we invert the transform to find a solution to the original
problem.
It is best to see the procedure on an example.

\begin{example}
Consider the first order PDE
\begin{equation*}
y_t = - \alpha y_x, \qquad \text{for } x > 0, \enspace t > 0,
\end{equation*}
with side conditions
\begin{equation*}
y(0,t) = C, \qquad y(x,0) = 0 .
\end{equation*}
This equation is 
called the \emph{\myindex{convection equation}} or sometimes
the \emph{\myindex{transport equation}}. %, and it already made
%an appearance in \sectionref{fopde:section}, with different conditions.
See \figureref{lt:half-infinite-goo-river} for a diagram of the setup.

\begin{mywrapfig}{3.1in}
\capstart
\inputpdft{half-infinite-goo-river}
\caption{Transport equation on a half line.\label{lt:half-infinite-goo-river}}
\end{mywrapfig}

A physical setup of this equation is a river of solid goo,
as we do not want anything to diffuse.  The function
$y$ is the concentration of
some toxic substance\footnote{It's a river of goo already,
we're not hurting the environment much more.}.
The variable $x$ denotes position where $x=0$
is the location of a factory spewing the toxic substance into the
river.  The toxic substance flows into the river so that at $x=0$ the
concentration is always $C$.  We wish to see what happens past the factory,
that is at $x > 0$.  Let $t$ be the time, and assume
the factory started operations at $t=0$, so that at $t=0$ the river is just
pure goo.

Consider a function of two variables $y(x,t)$.
Let us fix $x$ and transform the $t$ variable.
For convenience, we treat the transformed $s$
variable as a parameter, since there are no derivatives in $s$.
That is, we write $Y(x)$ for the transformed function,
and treat it as a function of $x$, leaving $s$ as a parameter.
\begin{equation*}
Y(x)
= {\mathcal L} \bigl\{ y(x,t) \bigr\}
= \int_0^\infty y(x,t) e^{-st} \,ds .
\end{equation*}
The transform of a derivative with respect to $x$ is just differentiating 
the transformed function:
\begin{equation*}
{\mathcal L} \bigl\{ y_x(x,t) \bigr\} =
\int_0^\infty y_x(x,t) e^{-st} \,ds
=
\frac{d}{dx} \left[\int_0^\infty y(x,t) e^{-st} \,ds \right]
=
Y'(x) .
\end{equation*}
To transform the derivative in $t$ (the variable being transformed),
we use the rules from \sectionref{transformsofders:section}:
\begin{equation*}
{\mathcal L} \bigl\{ y_t(x,t) \bigr\} 
=
sY(x) - y(x,0) .
\end{equation*}

In our specific case,
$y(x,0)=0$, and so
${\mathcal L} \bigl\{ y_t(x,t) \bigr\} = sY(x)$.  We transform the equation
to find
\begin{equation*}
sY(x) = -\alpha Y'(x) .
\end{equation*}
This ODE needs an initial condition.  The initial condition is the
other side condition of the PDE\@, the one that depends on $x$.  Everything
is transformed, so we must also transform this condition
\begin{equation*}
Y(0) = 
{\mathcal L} \bigl\{ y(0,t) \bigr\} 
=
{\mathcal L} \bigl\{ C \bigr\} 
=
\frac{C}{s} .
\end{equation*}

We solve the ODE problem $sY(x) = -\alpha Y'(x)$, $Y(0) = \frac{C}{s}$, to find
\begin{equation*}
Y(x) = \frac{C}{s} e^{-\frac{s}{\alpha} x} .
\end{equation*}
We are not done, we have $Y(x)$, but we really want $y(x,t)$.  We 
transform the $s$ variable back to $t$.
Let 
\begin{equation*}
u(t) = \begin{cases} 0 & \text{if $t < 0$}, \\ 1 & \text{otherwise}
\end{cases}
\end{equation*}
be the Heaviside function.
As
\begin{equation*}
{\mathcal L} \bigl\{ u(t-a) \bigr\} =
\int_0^\infty u(t-a) \, e^{-st} \,dt
=
\int_a^\infty e^{-st} \,dt
=
\frac{e^{-as}}{s} ,
\end{equation*}
then
\begin{equation*}
y(x,t) = {\mathcal L}^{-1} \left\{
\frac{C}{s} e^{-\frac{s}{\alpha} x}
\right\}
=
C u\bigl(t-\nicefrac{x}{\alpha}\bigr) .
\end{equation*}
In other words,
\begin{equation*}
y(x,t) =
\begin{cases}
0 & \text{if $t < \nicefrac{x}{\alpha}$}, \\
C & \text{otherwise.}
\end{cases}
\end{equation*}
See \figurevref{lt:half-infinite-goo-river-wavefront} for a diagram of this
solution.  The line of slope $\nicefrac{1}{\alpha}$ indicates
the wavefront of the toxic substance in the picture as it is leaving the
factory.
What the equation does is simply move the initial condition to the right at
speed $\alpha$.

\begin{myfig}
\capstart
\inputpdft{half-infinite-goo-river-wavefront}
\caption{Wavefront of toxic substance is a line of slope
$\nicefrac{1}{\alpha}$.\label{lt:half-infinite-goo-river-wavefront}}
\end{myfig}


Shhh\ldots $y$ is not differentiable, it is not even continuous
(nobody ever seems to notice).  How could we plug something that's not
differentiable into the equation?
Well, just think of a differentiable function very very close to $y$.
Or, if you recognize the derivative of the Heaviside function as the delta
function, then all is well too:
\begin{equation*}
y_t (x,t) = \frac{\partial}{\partial t} \left[
C  u\bigl(t-\nicefrac{x}{\alpha}\bigr)
\right]
=
C  u'\bigl(t-\nicefrac{x}{\alpha}\bigr)
=
C \delta\bigl(t-\nicefrac{x}{\alpha}\bigr)
\end{equation*}
and
\begin{equation*}
y_x (x,t) = \frac{\partial}{\partial x} \left[
C  u\bigl(t-\nicefrac{x}{\alpha}\bigr)
\right]
=
- \frac{C}{\alpha}  u'\bigl(t-\nicefrac{x}{\alpha}\bigr)
=
- \frac{C}{\alpha} \delta\bigl(t-\nicefrac{x}{\alpha}\bigr) .
\end{equation*}
So $y_t = - \alpha y_x$.
\end{example}

Laplace equation is very good with constant coefficient equations.  One
advantage of Laplace is that it easily handles
nonhomogeneous side conditions.
Let us try a more complicated example.

\begin{example}
Use the Laplace transform to solve
\begin{align*}
& y_t + y_x + y = 0, \qquad \text{for } x > 0, \enspace t > 0,
\\
& y(0,t) = \sin(t), \qquad y(x,0) = 0 .
\end{align*}
\end{example}

\begin{exampleSol}
Again, we transform $t$, and we write $Y(x)$ for the
transformed function.  As $y(x,0) = 0$, we find
\begin{equation*}
sY(x) + Y'(x) + Y(x) = 0, \qquad 
Y(0) = \frac{1}{s^2+1} .
\end{equation*}
The solution of the transformed equation is
\begin{equation*}
Y(x) =
\frac{1}{s^2+1} e^{-(s+1) x}
=
\frac{1}{s^2+1} e^{-xs}
e^{-x}
.
\end{equation*}
Using the second shifting property \eqref{ltd:sseq}
and linearity of the transform,
we obtain the solution
\begin{equation*}
y(x,t) 
=
e^{-x}
\sin(t-x)
u(t-x) .
\end{equation*}
\end{exampleSol}

We can also detect when the problem is
\emph{\myindex{ill-posed}}
in the sense that it has no solution.
Let us change the equation to
\begin{align*}
& -y_t + y_x = 0, \qquad \text{for } x > 0, \enspace t > 0,
\\
& y(0,t) = \sin(t), \qquad y(x,0) = 0 .
\end{align*}
Then the problem
has no solution. In the language of first order PDE, which will be discussed in 
\sectionref{fopde:section}, this can be seen as follows.
The characteristic curves are $t=-x+C$.  If $\tau$ is the
the characteristic coordinate, then we find the equation $y_\tau = 0$
along the curve,
meaning a solution is constant along characteristic curves.
But these curves intersect both the $x$-axis and the $t$-axis.
For example,
the curve $t=-x+1$ intersects at $(1,0)$ and $(0,1)$.  The solution is
constant along the curve so $y(1,0)$ should equal $y(0,1)$.  But
$y(1,0) = 0$ and $y(0,1) = \sin(1) \not= 0$.
See \figurevref{lt:half-infinite-ill-posed}.

\begin{myfig}
\capstart
\inputpdft{half-infinite-ill-posed}
\caption{Ill-posed problem.\label{lt:half-infinite-ill-posed}}
\end{myfig}

Now consider the transform.  The transformed problem is
\begin{equation*}
-sY(x) + Y'(x) = 0, \enspace Y(0) = \frac{1}{s^2+1} ,
\end{equation*}
and the solution ought to be
\begin{equation*}
Y(x) = \frac{1}{s^2+1} e^{sx} .
\end{equation*}
Importantly, this Laplace transform does not decay to zero at infinity!  That is,
since $x > 0$ in the region of interest, then
\begin{equation*}
\lim_{s \to \infty}
\frac{1}{s^2+1} e^{sx}
= \infty \not= 0 .
\end{equation*}
It almost
looks as if we could use the shifting property, but notice that the shift
is in the wrong direction.

\medskip

Of course, we need not restrict ourselves to first order equations, although
the computations become more involved for higher order equations.

\begin{example}
Let us use Laplace for the following problem:
\begin{align*}
& y_t = y_{xx}, \qquad 0 < x < \infty, \enspace t > 0,\\
& y_x(0,t) = f(t), \\
%& \text{for every $t$, $y(x,t)$ is bounded: } \lvert y(x,t) \rvert < M \\
%FIXME: perhaps limit as $x \to \infty$?
& y(x,0) = 0 .
\end{align*}
Really we also impose other conditions on the solution so that for example
the Laplace transform exists.  For example, we might impose that $y$ is
bounded for each fixed time $t$.
\end{example}

\begin{exampleSol}
Transform the equation in the $t$ variable to find
\begin{equation*}
sY(x) = Y''(x) .
\end{equation*}
The general solution to this ODE is
\begin{equation*}
Y(x) = A e^{\sqrt{s}x} + B e^{-\sqrt{s} x} .
\end{equation*}
First $A=0$, since otherwise $Y$ does not decay to zero as $s \to \infty$.

Now consider the boundary condition.
Transform $Y'(0) = F(s)$ and so $-\sqrt{s}B = F(s)$.
In other words,
\begin{equation*}
Y(x) = - F(s) \frac{1}{\sqrt{s}} e^{-\sqrt{s} x} .
\end{equation*}

If we look up the inverse transform in a table such as the one in
\Appendixref{laplacelist:appendix} (or we spend the afternoon
doing calculus), we find 
\begin{equation*}
{\mathcal L}^{-1}\left[e^{-\sqrt{s} x}\right] =
\frac{x}{\sqrt{4\pi t^3}} e^{\frac{-x^2}{4t}} ,
\end{equation*}
or
\begin{equation*}
{\mathcal L}^{-1}\left[\frac{1}{\sqrt{s}} e^{-\sqrt{s} x}\right] =
\frac{1}{\sqrt{\pi t}} e^{\frac{-x^2}{4t}} .
\end{equation*}
So
\begin{equation*}
y(x,t) =
{\mathcal L}^{-1} \left[
F(s) e^{-\sqrt{s} x}\right]
=
\int_0^t
f(\tau) 
\frac{1}{\sqrt{\pi {(t-\tau)}}} e^{\frac{-x^2}{4(t-\tau)}} \, d\tau .
\end{equation*}
\end{exampleSol}

Laplace can solve problems where separation of variables fails.
Laplace does not mind nonhomogeneity, but it is essentially only useful for
constant coefficient equations.

\subsection{Exercises}

\begin{exercise}
Solve
\begin{align*}
& y_t + y_x = 1, \qquad 0 < x < \infty, \enspace t > 0,
\\
& y(0,t) = 1, \quad y(x,0) = 0 .
\end{align*}
\end{exercise}

\begin{exercise}\ansMark%
Solve
\begin{align*}
& y_t + y_x = 1, \qquad 0 < x < \infty, \enspace t > 0,
\\
& y(0,t) = 0, \quad y(x,0) = 0 .
\end{align*}
\end{exercise}
\exsol{%
$y=(x-t)u(t-x)+t$
}

\begin{exercise}
Solve
\begin{align*}
& y_t + \alpha y_x = 0, \qquad 0 < x < \infty, \enspace t > 0,
\\
& y(0,t) = t, \quad y(x,0) = 0 .
\end{align*}
\end{exercise}

\begin{exercise}
Solve
\begin{align*}
& y_t + 2 y_x = x+t, \qquad 0 < x < \infty, \enspace t > 0,
\\
& y(0,t) = 0, \quad y(x,0) = 0 .
\end{align*}
\end{exercise}

\begin{exercise}
For an $\alpha > 0$, solve
\begin{align*}
& y_t + \alpha y_x + y = 0, \qquad 0 < x < \infty, \enspace t > 0,
\\
& y(0,t) = \sin(t), \quad y(x,0) = 0 .
\end{align*}
\end{exercise}

\begin{exercise}\ansMark
For a $c > 0$, solve
\begin{align*}
& y_t + y_x + c y = 0, \qquad 0 < x < \infty, \enspace t > 0,
\\
& y(0,t) = \sin(t), \quad y(x,0) = 0 .
\end{align*}
\end{exercise}
\exsol{%
$y=e^{-cx}\sin(t-x)u(t-x)$
}

\begin{exercise}
Find the corresponding ODE problem for $Y(x)$, after transforming the $t$
variable
\begin{align*}
& y_{tt} + 3y_{xx} + y_{xt} + 3 y_x + y = \sin(x) + t, \qquad 0 < x < 1, \enspace t > 0,
\\
& y(0,t) = 1, \quad y(1,t) =  t, \quad y(x,0) = 1-x, \quad y_t(x,0) = 1 .
\end{align*}
Do not solve the problem.
\end{exercise}

\begin{exercise}\ansMark%
Find the corresponding ODE problem for $Y(x)$, after transforming the $t$
variable
\begin{align*}
& y_{tt} + 3y_{xx} + y = x+t, \qquad -1 < x < 1, \enspace t > 0,
\\
& y(-1,t) = 0, \quad y(1,t) = 0, \quad y(x,0) = (1-x^2) , \quad y_t(x,0) = 0.
\end{align*}
Do not solve the problem.
\end{exercise}
\exsol{%
$s^2Y(x) - s(1-x^2) + 3Y''(x) + Y(x) = \frac{x}{s} + \frac{1}{s^2}, \quad
Y(-1) = 0, \quad Y(1)=0$.
}

\begin{exercise}
Write down a solution to
\begin{align*}
& y_t = y_{xx}, \qquad 0 < x < \infty, \enspace t > 0,\\
& y_x(0,t) = e^{-t}, \quad y(x,0) = 0 ,
\end{align*}
as an definite integral (convolution).
\end{exercise}

\begin{exercise}
Use the Laplace transform in $t$ to solve
\begin{align*}
& y_{tt} = y_{xx}, \qquad -\infty < x < \infty, \enspace t > 0,\\
& y_t(x,0) = \sin(x), \quad  y(x,0) = 0 .
\end{align*}
Hint: Note that
$e^{sx}$ does not go to zero as $s \to \infty$ for positive
$x$, and 
$e^{-sx}$ does not go to zero as $s \to \infty$ for negative $x$.
\end{exercise}

\begin{exercise}\ansMark%
Use the Laplace transform in $t$ to solve
\begin{align*}
& y_{tt} = y_{xx}, \qquad -\infty < x < \infty, \enspace t > 0,\\
& y_t(x,0) = x^2, \quad  y(x,0) = 0 .
\end{align*}
Hint: Note that
$e^{sx}$ does not go to zero as $s \to \infty$ for positive
$x$, and 
$e^{-sx}$ does not go to zero as $s \to \infty$ for negative $x$.
\end{exercise}
\exsol{%
$y=tx^2+\frac{t^3}{3}$
}

\setcounter{exercise}{100}

