
\section{Sine and cosine series}
\label{sec:scs}

\LAtt{4.4}

\LO{
\item Use sine and cosine series to represent odd and even periodic extensions of functions and
\item Understand the connection between Fourier series and boundary value problems.
}

% \sectionnotes{Verbatim from Lebl}

% \sectionnotes{2 lectures\EPref{, \S9.3 in \cite{EP}}\BDref{,
% \S10.4 in \cite{BD}}}

\subsection{Odd and even periodic functions}

You may have noticed by now that an odd function has no cosine terms in the 
Fourier series and an even function has no sine terms in the Fourier series.
This observation is not a coincidence.  Let us look at even and odd periodic
function in more detail.

Recall that a function $f(t)$ is \emph{odd}\index{odd function} if $f(-t) =
-f(t)$.  A function $f(t)$ is \emph{even}\index{even function} if
$f(-t) = f(t)$.  For example, $\cos (n t)$ is even and $\sin (n t)$ is odd.
Similarly the function $t^k$ is even if $k$ is even and odd when $k$ is odd.

\begin{exercise}
Take two functions $f(t)$ and $g(t)$ and define their product $h(t) =
f(t)g(t)$.
\begin{tasks}
\task Suppose both $f(t)$ and $g(t)$ are odd.  Is $h(t)$ odd or even?
\task Suppose one is even and one is odd.  Is $h(t)$ odd or even?
\task Suppose both are even.  Is $h(t)$ odd or even?
\end{tasks}
\end{exercise}

If $f(t)$ and $g(t)$ are both odd, then $f(t)+g(t)$ is odd.  Similarly for
even functions.  On the other hand,
if $f(t)$ is odd and $g(t)$ even, then we cannot say anything about
the sum
$f(t) + g(t)$.  In fact, the Fourier series of any function is a sum of
an odd (the sine terms) and an even (the cosine terms) function.

In this section we consider odd and even periodic
functions.  We have previously defined the $2L$-periodic extension
of a function defined on the interval $[-L,L]$.  Sometimes we are only
interested in the function on the range $[0,L]$ and it would be convenient
to have an odd (resp.\ even) function.  If the function is odd (resp.\ even),
all the cosine (resp.\ sine) terms disappear.
What we will do is
take the
odd (resp.\ even) extension of the function to $[-L,L]$ and then 
extend periodically to a $2L$-periodic function.

Take a function $f(t)$ defined on $[0,L]$.  On $(-L,L]$ define the functions
\begin{align*}
F_{\text{odd}}(t) & \overset{\text{def}}{=}
\begin{cases}
f(t) & \text{if } \; \phantom{-}0 \leq t \leq L , \\
-f(-t) & \text{if } \; {-L} < t < 0 ,
\end{cases}
\\
F_{\text{even}}(t) & \overset{\text{def}}{=}
\begin{cases}
f(t) & \text{if } \; \phantom{-}0 \leq t \leq L , \\
f(-t) & \text{if } \; {-L} < t < 0 .
\end{cases}
\end{align*}
Extend $F_{\text{odd}}(t)$ and $F_{\text{even}}(t)$ to be $2L$-periodic.
Then
$F_{\text{odd}}(t)$ is called
the \emph{\myindex{odd periodic extension}} of $f(t)$, and
$F_{\text{even}}(t)$ is called the
\emph{\myindex{even periodic extension}} of $f(t)$.
For the odd extension we generally assume that $f(0) = f(L) = 0$.

\begin{exercise}
Check that $F_{\text{odd}}(t)$ is odd and $F_{\text{even}}(t)$ is even.
For $F_{\text{odd}}$,
assume $f(0) = f(L) = 0$.
\end{exercise}

\begin{example}
Take the function $f(t) = t\,(1-t)$ defined on $[0,1]$. 
\figurevref{scs:oddevenextfig}
shows the plots of the odd and even periodic extensions of $f(t)$.

\begin{myfig}
\capstart
%original files scs-oddext scs-evenext
\diffyincludegraphics{width=6.24in}{width=9in}{scs-ext-odd-even}
\caption{Odd and even 2-periodic extension of $f(t) =
t\,(1-t)$, $0 \leq t \leq 1$.\label{scs:oddevenextfig}}
\end{myfig}
\end{example}

\subsection{Sine and cosine series}

Let $f(t)$ be an odd $2L$-periodic function.  We write 
the Fourier series for $f(t)$.  First, we compute the coefficients $a_n$ (including
$n=0$) and get
\begin{equation*}
a_n = \frac{1}{L} \int_{-L}^L f(t) \cos \left( \frac{n \pi}{L} t \right)
\, dt = 0 .
\end{equation*}
That is, there are no cosine terms in the Fourier series of an odd function.
The integral is zero
because $f(t) \cos \left( {n \pi}{L} t \right)$
is an odd function (product of an odd and an
even function is odd) and the integral of an odd function over a symmetric
interval is always zero.
The integral of an even function over a symmetric interval
$[-L,L]$ is twice the integral of the function over the interval $[0,L]$.
The function $f(t) \sin \left( \frac{n \pi}{L} t \right)$ is the product of two odd
functions and hence is even.
\begin{equation*}
b_n = 
\frac{1}{L} \int_{-L}^L f(t) \sin \left( \frac{n \pi}{L} t \right) \, dt =
\frac{2}{L} \int_{0}^L f(t) \sin \left( \frac{n \pi}{L} t \right) \, dt .
\end{equation*}
We now write the Fourier series of $f(t)$ as
\begin{equation*}
\sum_{n=1}^\infty b_n \sin \left( \frac{n \pi}{L} t \right) .
\end{equation*}

Similarly, if $f(t)$ is an even $2L$-periodic function.  For the same exact
reasons as above, we find that $b_n = 0$ and
\begin{equation*}
a_n = 
\frac{2}{L} \int_{0}^L f(t) \cos \left( \frac{n \pi}{L} t \right) \, dt .
\end{equation*}
The formula still works for $n=0$, in which case it becomes
\begin{equation*}
a_0 = 
\frac{2}{L} \int_{0}^L f(t) \, dt .
\end{equation*}
The Fourier series is then
\begin{equation*}
\frac{a_0}{2}
+
\sum_{n=1}^\infty a_n \cos \left( \frac{n \pi}{L} t \right) .
\end{equation*}

An interesting consequence is that the coefficients of the Fourier series of
an odd (or even) function can be computed by just integrating over the half
interval $[0,L]$.  Therefore, we can compute the Fourier series of
the odd (or even) extension of a
function by computing certain integrals over the interval
where the original function is defined.

\begin{theorem1}{}
Let $f(t)$ be a piecewise smooth function defined on $[0,L]$.
Then the odd periodic extension
of $f(t)$ has the Fourier series
\begin{equation*}
%\mybxbg{~~
F_{\text{odd}}(t) = \sum_{n=1}^\infty b_n \sin \left( \frac{n \pi}{L} t
\right) ,
%~~}
\end{equation*}
where
\begin{equation*}
%\mybxbg{~~
b_n = 
\frac{2}{L} \int_{0}^L f(t)\, \sin \left( \frac{n \pi}{L} t \right) \, dt .
%~~}
\end{equation*}
The even periodic extension of $f(t)$ has the Fourier series
\begin{equation*}
%\mybxbg{~~
F_{\text{even}}(t) = \frac{a_0}{2} + \sum_{n=1}^\infty a_n \cos \left(
\frac{n \pi}{L} t \right) ,
%~~}
\end{equation*}
where
\begin{equation*}
%\mybxbg{~~
a_n = 
\frac{2}{L} \int_{0}^L f(t)\, \cos \left( \frac{n \pi}{L} t \right) \, dt .
%~~}
\end{equation*}
\end{theorem1}

\begin{definition}
We call the series $\sum_{n=1}^\infty b_n \sin \left( \frac{n \pi}{L} t\right)$ 
the \emph{\myindex{sine series}} of $f(t)$ and we call the series
$\frac{a_0}{2} + \sum_{n=1}^\infty a_n \cos \left( \frac{n \pi}{L} t
\right)$
the \emph{\myindex{cosine series}} of $f(t)$.  
\end{definition}
We often do not actually care what happens outside of $[0,L]$.  In this case,
we pick whichever series fits our problem better.

It is not necessary to start with the full Fourier series to obtain
the sine and cosine series.
The sine series is really the eigenfunction expansion of $f(t)$ using 
eigenfunctions of the eigenvalue problem $x''+\lambda x = 0$, $x(0) = 0$,
$x(L) = L$.  The cosine series is the eigenfunction expansion of $f(t)$
using 
eigenfunctions of the eigenvalue problem $x''+\lambda x = 0$, $x'(0) = 0$,
$x'(L) = L$.  We could have, therefore, gotten the same formulas
by defining the inner product
\begin{equation*}
\langle f(t), g(t) \rangle = \int_0^L f(t) g(t) \, dt ,
\end{equation*}
and following the procedure of \sectionref{ts:section}.  This point of view is
useful, as we commonly use a specific series that arose because our underlying
question 
led to a certain eigenvalue problem.  If the eigenvalue 
problem is not one of the three we covered so far, you can still do an
eigenfunction expansion, generalizing the results of this chapter.  %We will
%deal with such a generalization in \chapterref{SL:chapter}.

%f(t) = \frac{a_0}{2} + \sum_{n=1}^\infty a_n \cos \left( \frac{n \pi}{L} 
%t \right)
%+ b_n \sin \left( \frac{n \pi}{L} t \right) ,

\begin{example}
Find the Fourier series of the even periodic extension of 
the function $f(t) = t^2$ for $0 \leq t \leq \pi$.
\end{example}

\begin{exampleSol}
We want to write
\begin{equation*}
f(t) = \frac{a_0}{2} + \sum_{n=1}^\infty a_n \cos (n t) ,
\end{equation*}
where
\begin{equation*}
a_0 = \frac{2}{\pi}
\int_0^\pi t^2 \, dt = \frac{2 \pi^2}{3} ,
\end{equation*}
and
\begin{equation*}
\begin{split}
a_n & = \frac{2}{\pi}
\int_0^\pi t^2 \cos (n t) \, dt
= \frac{2}{\pi} \left[ t^2 \frac{1}{n} \sin (nt) \right]_0^\pi -
\frac{4}{n\pi}
\int_0^\pi t \sin (n t) \, dt \\
& = 
\frac{4}{n^2\pi}
\Bigl[ t \cos (n t) \Bigr]_0^\pi
+
\frac{4}{n^2\pi}
\int_0^\pi \cos (n t) \, dt
= 
\frac{4{(-1)}^n}{n^2} .
\end{split}
\end{equation*}
Note that we have \myquote{detected} the continuity of the extension since the
coefficients decay as $\frac{1}{n^2}$.  That is, the even periodic extension
of $t^2$ has no jump discontinuities.  It does have corners, since
the derivative, which is an odd function and a sine series, has jumps; it has
a Fourier series whose coefficients decay only as $\frac{1}{n}$.

Explicitly, the first few terms of the series are
\begin{equation*}
\frac{\pi^2}{3} - 4 \cos (t) + \cos (2t) - \frac{4}{9} \cos (3t) + \cdots
\end{equation*}
\end{exampleSol}

\begin{exercise}
\leavevmode
\begin{tasks}
\task Compute the derivative of the even periodic extension of $f(t)$ above and verify it
has jump discontinuities.  Use the actual definition of $f(t)$, not its cosine
series!
\task Why is it that the derivative of the even periodic extension of $f(t)$ is the
odd periodic extension of $f'(t)$?
\end{tasks}
\end{exercise}

\subsection{Application}

Fourier series ties in to the boundary value problems
we studied earlier.  Let us see this connection in an application.

Consider the boundary value problem for $0 < t < L$,
\begin{equation*}
x''(t) + \lambda\, x(t) = f(t) ,
\end{equation*}
for the \emph{\myindex{Dirichlet boundary conditions}}
$x(0) = 0$, $x(L) = 0$.
The Fredholm alternative (\thmvref{thm:fredholmsimple})
says that
as long as $\lambda$ is not an eigenvalue of the underlying homogeneous
problem, there exists a unique solution.
Eigenfunctions of this eigenvalue problem are the functions
$\sin \left( \frac{n \pi}{L} t \right)$.
Therefore,
to find the solution,
we first find the Fourier sine series for $f(t)$.
We write $x$ also as a sine series, but with unknown coefficients.  
We substitute the series for $x$ into the equation and solve for the unknown
coefficients.
If we have
the \emph{\myindex{Neumann boundary conditions}}
$x'(0) = 0$, $x'(L) = 0$, we do the same procedure using the cosine
series.

Let us see how this method works on examples.

\begin{example}
Take the boundary value problem for $0 < t < 1$,
\begin{equation*}
x''(t) + 2 x(t) = f(t) ,
\end{equation*}
where $f(t) = t$ on $0 < t < 1$, and 
satisfying the Dirichlet boundary conditions
$x(0) = 0$, $x(1)=0$.
\end{example}

\begin{exampleSol}
We write $f(t)$ as a sine series
\begin{equation*}
f(t) = \sum_{n=1}^\infty c_n \sin (n \pi t) .
\end{equation*}
Compute
\begin{equation*}
c_n = 2 \int_0^1 t \sin (n \pi t) \,dt = \frac{2 \, {(-1)}^{n+1}}{n \pi} .
\end{equation*}
We write $x(t)$ as
\begin{equation*}
x(t) = \sum_{n=1}^\infty b_n \sin (n \pi t) .
\end{equation*}
We plug in to obtain 
\begin{equation*}
\begin{split}
x''(t) + 2 x(t) & =
\underbrace{
\sum_{n=1}^\infty - b_n n^2 \pi^2 \sin (n \pi t)
}_{x''}
\,
+
\,
2
\underbrace{
\sum_{n=1}^\infty b_n \sin (n \pi t)
}_{x}
\\
& =
\sum_{n=1}^\infty b_n (2 - n^2 \pi^2 ) \sin (n \pi t)
\\
& = f(t)
=
\sum_{n=1}^\infty  \frac{2\, {(-1)}^{n+1}}{n \pi} \sin (n \pi t) .
\end{split}
\end{equation*}
Therefore,
\begin{equation*}
b_n (2 - n^2 \pi^2)
=
\frac{2\,{(-1)}^{n+1}}{n \pi}
\end{equation*}
or
\begin{equation*}
b_n
=
\frac{2\,{(-1)}^{n+1}}{n \pi (2 - n^2 \pi^2)} .
\end{equation*}
That $2-n^2\pi^2$ is not zero for any $n$, and that we can
solve for $b_n$, is precisely because
$2$ is not an eigenvalue of the problem.
We have thus obtained a Fourier series for the solution
\begin{equation*}
x(t) = 
\sum_{n=1}^\infty
\frac{2\,{(-1)}^{n+1}}{n \pi \,(2 - n^2 \pi^2)}
\sin (n \pi t) .
\end{equation*}
See \figurevref{bnd-dirich-graph:fig} for a graph of the solution.
Notice that because the eigenfunctions satisfy the boundary conditions, 
and $x$ is written in terms of the boundary conditions, then $x$
satisfies the boundary conditions.
\begin{myfig}
\capstart
\diffyincludegraphics{width=3in}{width=4.5in}{bnd-dirich-graph}
\caption{Plot of the solution of $x''+2x=t$, $x(0)=0$, $x(1)=0$.%
\label{bnd-dirich-graph:fig}}
\end{myfig}
\end{exampleSol}

\begin{example}
Similarly we handle the Neumann conditions.
Take the boundary value problem for $0 < t < 1$,
\begin{equation*}
x''(t) + 2 x(t) = f(t) ,
\end{equation*}
where again $f(t) = t$ on $0 < t < 1$, but now satisfying
the Neumann boundary conditions
$x'(0) = 0$, $x'(1)=0$.
\end{example}

\begin{exampleSol}
We write $f(t)$ as a cosine series
\begin{equation*}
f(t) = \frac{c_0}{2} + \sum_{n=1}^\infty c_n \cos (n \pi t) ,
\end{equation*}
where
\begin{equation*}
c_0 = 2 \int_0^1 t \,dt = 1 ,
\end{equation*}
and
\begin{equation*}
c_n = 2 \int_0^1 t \cos (n \pi t) \,dt =
\frac{2\bigl({(-1)}^n-1\bigr)}{\pi^2 n^2} = 
\begin{cases}
\frac{-4}{\pi^2 n^2} & \text{if } n \text{ odd} , \\
0 & \text{if } n \text{ even}.
\end{cases}
\end{equation*}
We write $x(t)$ as a cosine series
\begin{equation*}
x(t) = \frac{a_0}{2} + \sum_{n=1}^\infty a_n \cos (n \pi t) .
\end{equation*}
We plug in to obtain 
\begin{equation*}
\begin{split}
x''(t) + 2 x(t) & =
\sum_{n=1}^\infty \Bigl[ - a_n n^2 \pi^2 \cos (n \pi t) \Bigr]
+
a_0 +
2
\sum_{n=1}^\infty \Bigl[ a_n \cos (n \pi t) \Bigr]
\\
& =
a_0 +
\sum_{n=1}^\infty a_n (2 - n^2 \pi^2 ) \cos (n \pi t)
\\
& = f(t)
=
\frac{1}{2} +
\sum_{\substack{n=1\\n~\text{odd}}}^\infty
\frac{-4}{\pi^2 n^2} \cos (n \pi t) .
\end{split}
\end{equation*}
Therefore, $a_0 = \frac{1}{2}$, $a_n = 0$ for $n$ even ($n \geq 2$) and for
$n$ odd we have
\begin{equation*}
a_n (2 - n^2 \pi^2)
=
\frac{-4}{\pi^2 n^2} ,
\end{equation*}
or
\begin{equation*}
a_n
=
\frac{-4}{n^2 \pi^2 (2 - n^2 \pi^2)} .
\end{equation*}
The Fourier series for the solution $x(t)$ is
\begin{equation*}
x(t) = 
\frac{1}{4} +
\sum_{\substack{n=1\\n~\text{odd}}}^\infty
\frac{-4}{n^2 \pi^2 (2 - n^2 \pi^2)} 
\cos (n \pi t) .
\end{equation*}
\end{exampleSol}

\subsection{Exercises}

\begin{exercise}
Take $f(t) = {(t-1)}^2$ defined on $0 \leq t \leq 1$.
\begin{tasks}
\task Sketch the plot of the even periodic extension of $f$.
\task Sketch the plot of the odd periodic extension of $f$.
\end{tasks}
\end{exercise}

\begin{exercise}
Find the Fourier series of both the odd and even
periodic extension of 
the function $f(t) = {(t-1)}^2$ for $0 \leq t \leq 1$.
Can you tell which extension is continuous from the Fourier series
coefficients?
\end{exercise}

\begin{exercise}
Find the Fourier series of both the odd and even periodic extension of 
the function $f(t) = t$ for $0 \leq t \leq \pi$.
\end{exercise}

\begin{exercise}\ansMark%
Let $f(t) = \nicefrac{t}{3}$ on $0 \leq t < 3$.
\begin{tasks}
\task Find the Fourier series of the even periodic extension.
\task Find the Fourier series of the odd periodic extension.
\end{tasks}
\end{exercise}
\exsol{%
a)
$\nicefrac{1}{2}
+
\sum\limits_{\substack{n=1\\n\text{ odd}}}^\infty
\frac{-4}{\pi^2 n^2}
\cos\bigl(\frac{n\pi}{3} t \bigr)$
\qquad
b) 
$\sum\limits_{n=1}^\infty
\frac{2{(-1)}^{n+1}}{\pi n}
\sin\bigl(\frac{n\pi}{3} t \bigr)$
}

\begin{exercise}
Find the Fourier series of the even periodic extension of 
the function $f(t) = \sin t$ for $0 \leq t \leq \pi$.
\end{exercise}

\begin{exercise}\ansMark%
Let $f(t) = \cos(2t)$ on $0 \leq t < \pi$.
\begin{tasks}
\task Find the Fourier series of the even periodic extension.
\task Find the Fourier series of the odd periodic extension.
\end{tasks}
\end{exercise}
\exsol{%
a)
$\cos(2t)$
\qquad
b) 
$\sum\limits_{\substack{n=1 \\n \text{ odd}}}^\infty
\frac{-4n}{\pi n^2 - 4 \pi}
\sin(n t)$
}

\begin{exercise}\ansMark%
Let $f(t)$ be defined on $0 \leq t < 1$.  Now take
the average of the two extensions
$g(t) = \frac{F_{\text{odd}}(t)+ F_{\text{even}}(t)}{2}$.
\begin{tasks}(2)
\task What is $g(t)$ if $0 \leq t < 1$ (Justify!)
\task What is $g(t)$ if $-1 < t < 0$ (Justify!)
\end{tasks}
\end{exercise}
\exsol{%
a) $f(t)$
\qquad
b) $0$
}


\begin{exercise}
\pagebreak[2]
Consider
\begin{equation*}
x''(t) + 4 x(t) = f(t) ,
\end{equation*}
where $f(t) = 1$ on $0 < t < 1$.
\begin{tasks}
\task Solve for the Dirichlet conditions $x(0)=0, x(1) = 0$.
\task Solve for the Neumann conditions $x'(0)=0, x'(1) = 0$.
\end{tasks}
\end{exercise}

\begin{exercise}
Consider
\begin{equation*}
x''(t) + 9 x(t) = f(t) ,
\end{equation*}
for $f(t) = \sin (2\pi t)$ on $0 < t < 1$.
\begin{tasks}
\task Solve for the Dirichlet conditions $x(0)=0, x(1) = 0$.
\task Solve for the Neumann conditions $x'(0)=0, x'(1) = 0$.
\end{tasks}
\end{exercise}

\begin{exercise}\ansMark%
Let $f(t) = \sum_{n=1}^\infty \frac{1}{n^2} \sin(nt)$.  Solve
$x''- x = f(t)$ for the Dirichlet conditions $x(0) = 0$
and $x(\pi) = 0$.
\end{exercise}
\exsol{%
$\sum\limits_{n=1}^\infty \frac{-1}{n^2(1+n^2)} \sin(nt)$
}

\begin{exercise}
Consider
\begin{equation*}
x''(t) + 3 x(t) = f(t) , \quad x(0) = 0, \quad x(1) = 0,
\end{equation*}
where $f(t) = \sum_{n=1}^\infty b_n \sin (n \pi t)$.  Write the solution $x(t)$
as a Fourier series, where the coefficients are given in terms of $b_n$.
\end{exercise}

\begin{exercise}
Let $f(t) = t^2(2-t)$ for $0 \leq t \leq 2$.  Let $F(t)$ be the odd periodic
extension.  Compute $F(1)$, $F(2)$, $F(3)$, $F(-1)$, $F(\nicefrac{9}{2})$,
$F(101)$, $F(103)$.  Note: Do \textbf{not} compute using the sine series.
\end{exercise}

\begin{exercise}[challenging]\ansMark%
Let $f(t) = t + \sum_{n=1}^\infty \frac{1}{2^n} \sin(nt)$.  Solve
$x'' + \pi x = f(t)$ for the Dirichlet conditions $x(0) = 0$
and $x(\pi) = 1$.  Hint:  Note that $\frac{t}{\pi}$ satisfies the
given Dirichlet conditions.
\end{exercise}
\exsol{%
$\frac{t}{\pi} + \sum\limits_{n=1}^\infty \frac{1}{2^n(\pi-n^2)} \sin(nt)$
}

\setcounter{exercise}{100}

