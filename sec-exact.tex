\section{Exact equations}
\label{exact:section}

\LAtt{1.8}

\LO{
\item Determine if a first order differential equation is exact,
\item Find the general solution to an exact equation,
\item Solve initial value problems for exact equations, and
\item Use integrating factors to make some non-exact equations exact in order to solve them.
}

% \sectionnotes{1--2 lectures, can safely be skipped\EPref{, \S1.6 in \cite{EP}}\BDref{, \S2.6 in \cite{BD}}}

Another type of equation that comes up quite often in physics and
engineering is an
\emph{\myindex{exact equation}}.
%  To truly make sense out of these
%equations requires a little bit of multivariable calculus and a tiny
%bit of physics.
Suppose $F(x,y)$ is a function of two variables, which we call the
\emph{\myindex{potential function}}.  The naming should suggest 
potential energy, or electric potential.  Exact equations and potential
functions appear when there is a conservation law at play, such as 
conservation of energy.
Let us make up a simple example.  Let
\begin{equation*}
F(x,y) = x^2+y^2 .
\end{equation*}

%17 is the number of lines, must be adjusted
\begin{mywrapfig}[17]{3.25in}
\capstart
\diffyincludegraphics{width=3in}{width=4.5in}{circlesfig}
\caption{Solutions to $F(x,y) = x^2+y^2 = C$ for various
$C$.\label{exact:circlesfig}}
\end{mywrapfig}
We are interested in the lines of constant energy, that is lines where
the energy is conserved;  we want curves where $F(x,y) = C$,
for some constant $C$, since $F$ represents the energy of the system.  In
our example, the curves $x^2+y^2=C$ are circles.  See
\figurevref{exact:circlesfig}.

We take the
\emph{\myindex{total derivative}} of
$F$:
\begin{equation*}
dF = \frac{\partial F}{\partial x} dx + \frac{\partial F}{\partial y} dy .
\end{equation*}
For convenience,
we will make use of the notation of
$F_x = \frac{\partial F}{\partial x}$ and
$F_y = \frac{\partial F}{\partial y}$.
In our example,
\begin{equation*}
dF = 2x \, dx + 2y \, dy .
\end{equation*}
We apply the total derivative to $F(x,y) = C$, to find
the differential equation $dF = 0$.  The differential equation we obtain in such a way
has the form
\begin{equation*}
M \, dx + N \, dy = 0, \qquad
\text{or} \qquad
M + N \, \frac{dy}{dx} = 0 .
\end{equation*}

\begin{definition}
An equation of the form
\[ M(x,y) + N(x, y)\ \frac{dy}{dx} = 0 \]
is called \emph{exact} if it was obtained as $dF = 0$ for some potential
function $F$.
\end{definition}

In our simple example, we obtain the equation
\begin{equation*}
2x \, dx + 2y \, dy = 0, \qquad
\text{or} \qquad
2x + 2y \, \frac{dy}{dx} = 0 .
\end{equation*}
Since we obtained this equation by differentiating $x^2+y^2=C$, 
the equation is exact.
We often wish to solve for $y$ in terms of $x$.  In our example,
\begin{equation*}
y = \pm \sqrt{C^2-x^2} .
\end{equation*}

An interpretation of the setup is that at each point in the plane $\vec{v} = (M,N)$ is
a vector, that is, a direction and a magnitude.
As $M$ and $N$ are functions of $(x,y)$, we have a \emph{\myindex{vector field}}.  The particular 
vector field $\vec{v}$ that comes from an exact equation is a so-called
\emph{\myindex{conservative vector field}}, that is, a vector field that comes with a
potential function $F(x,y)$, such that
\begin{equation*}
\vec{v} = \left( \frac{\partial F}{\partial x} ,\frac{\partial F}{\partial
y} \right) .
\end{equation*} This is something that you may have seen in your Calculus 3 course, and if so, the process for solving exact equations is basically identical to the process of finding a potential function for a conservative vector field.  The physical interpretation of conservative vector fields is as follows. Let
$\gamma$ be a path in the plane starting at $(x_1,y_1)$ and ending at
$(x_2,y_2)$.
If we think of $\vec{v}$ as force, then the work required to
move along $\gamma$ is
\begin{equation*}
\int_\gamma \vec{v}(\vec{r}) \cdot d\vec{r}
=
\int_\gamma M \, dx + N \, dy
=
F(x_2,y_2) - F(x_1,y_1) .
\end{equation*}
That is, the work done only depends on endpoints, that is where we start and
where we end.   For example, suppose $F$ is gravitational potential.  The
derivative of $F$ given by $\vec{v}$ is the gravitational force.
What we
are saying is that the work required to move a heavy box from the ground
floor to the roof only depends on the change in potential energy.  That is,
the work done is the same
no matter what path we took; if we took the stairs or the elevator.
Although if we took the elevator, the elevator is doing the work for us.
The curves $F(x,y) = C$ are those where no work need be done, such as
the heavy box sliding along without accelerating or breaking on a perfectly
flat roof, on a cart with incredibly well oiled wheels. Effectively, an exact equation is a conservative vector field, and the implicit solution of this equation is the potential function. 

\subsection{Solving exact equations}

Now you, the reader, should ask: Where did we solve a differential equation?
Well, in applications we generally know $M$ and $N$, but we do not
know $F$.  That is, we may have just started with
$2x + 2y \frac{dy}{dx} = 0$, or perhaps even
\begin{equation*}
x + y \frac{dy}{dx} = 0 .
\end{equation*}
It is up to us to find some potential $F$ that works.  Many different $F$
will work; adding a constant to $F$ does not change the equation.
Once we have a potential function $F$, the equation 
$F\bigl(x,y(x)\bigr) = C$
gives an implicit solution of the ODE\@.

\begin{example}
Let us find the general solution to
$2x + 2y \frac{dy}{dx} = 0$.  Forget we knew
what $F$ was.
\end{example}

\begin{exampleSol}
If we know that this is an exact equation, we start looking for a potential
function $F$.
We have $M = 2x$ and $N=2y$.
If $F$ exists, it must be such that
$F_x (x,y) = 2x$.
Integrate in the $x$ variable to find
\begin{equation} \label{eq:exact:fint}
F(x,y) = x^2 + A(y) ,
\end{equation}
for some function $A(y)$.  The function $A$ is the \myquote{constant of
integration}, though it is only constant as far as $x$ is concerned, and
may still depend on $y$.  Now differentiate \eqref{eq:exact:fint} in $y$ 
and set it equal to $N$, which is what $F_y$ is supposed to be:
\begin{equation*}
2y = F_y (x,y) = A'(y) .
\end{equation*}
Integrating, we find $A(y) = y^2$.  We could add a constant of integration
if we wanted to, but there is no need.  We found $F(x,y) = x^2+y^2$.
Next for a constant $C$, we solve
\begin{equation*}
F\bigl(x,y(x)\bigr) = C .
\end{equation*}
for $y$ in terms of $x$.  In this case, we obtain $y = \pm \sqrt{C^2-x^2}$
as we did before.
\end{exampleSol}

\begin{exercise}
Why did we not need to add a constant of integration when integrating $A'(y)
= 2y$?  Add a constant of integration, say $3$, and see what $F$ you get.
What is the difference from what we got above, and why does it not matter?
\end{exercise}

In the previous example, you may have also noticed that the equation $2x + 2y\frac{dy}{dx} = 0$ is separable, and we could have solved it via that method as well. This is not a coincidence, as every separable equation is exact (see \exerciseref{ex:separableExact} for the details) but there are many exact equations that are not separable, which we will see throughout the examples here. 

The procedure, once we know that the equation is exact, is:
\begin{enumerate}[(i)]
\item Integrate $F_x = M$ in $x$ resulting in $F(x,y) = \text{something} + A(y)$.
\item Differentiate this $F$ in $y$, and set that equal to
$N$, so that we may find $A(y)$ by integration.
\end{enumerate}
The procedure can also be done by first integrating in $y$ and then
differentiating in $x$.
Pretty easy huh?  Let's try this again.

\begin{example}
Consider now $2x+y + xy \frac{dy}{dx} = 0$.
\end{example}

\begin{exampleSol}
OK\@, so $M = 2x+y$ and $N=xy$.  We try to proceed as before.
Suppose 
$F$ exists.  Then $F_x (x,y) = 2x+y$.
We integrate:
\begin{equation*}
F(x,y) = x^2 + xy + A(y)
\end{equation*}
for some function $A(y)$.  Differentiate in $y$ and set equal to $N$:
\begin{equation*}
N = xy = F_y (x,y) = x+A'(y) .
\end{equation*}
But there is no way to satisfy this requirement!  The function $xy$ cannot
be written as $x$ plus a function of $y$.  The equation is not
exact; no potential function $F$ exists.
\end{exampleSol}

Is there an easier way to check for the existence of $F$, other than failing
in trying to
find it?  Turns out there is.  Suppose
$M = F_x$ and
$N = F_y$.  Then
as long as the second derivatives are continuous,
\begin{equation*}
\frac{\partial M}{\partial y}
=
\frac{\partial^2 F}{\partial y \partial x}
=
\frac{\partial^2 F}{\partial x \partial y}
=
\frac{\partial N}{\partial x} .
\end{equation*}
Let us state it as
a theorem.  Usually this is called the Poincar\'e Lemma\footnote{Named for the French polymath
\href{https://en.wikipedia.org/wiki/Henri_Poincar\%C3\%A9}{Jules Henri
Poincar\'e} (1854--1912).}.

\begin{theorem1}[thm:Poincare]{Poincar\'e}
If $M$ and $N$ are continuously differentiable functions of $(x,y)$, and
$\frac{\partial M}{\partial y} = \frac{\partial N}{\partial x}$,
then near any point there is a function $F(x,y)$
such that
$M = \frac{\partial F}{\partial x}$ and
$N = \frac{\partial F}{\partial y}$.
\end{theorem1}

The theorem doesn't give us a global $F$ defined everywhere.  In
general, we can only find the potential locally, near some initial point.
By this time, we have come to expect this from differential
equations.

Let us return to the example above where $M = 2x + y$ and $N = xy$.  
Notice 
$M_y = 1$ and  $N_x =
y$, which are clearly not equal.  The equation is not exact.

\begin{example}
Solve
\begin{equation*}
\frac{dy}{dx} = \frac{-2x-y}{x-1}, \qquad y(0) = 1.
\end{equation*}
\end{example}

\begin{exampleSol}
We write the equation as
\begin{equation*}
(2x+y) + (x-1)\frac{dy}{dx} = 0 ,
\end{equation*}
so $M = 2x+y$ and $N = x-1$.  Then
\begin{equation*}
M_y = 1 = N_x .
\end{equation*}
The equation is exact.
Integrating $M$ in $x$, we find
\begin{equation*}
F(x,y) = x^2+xy + A(y) .
\end{equation*}
Differentiating in $y$ and setting to $N$, we find
\begin{equation*}
x-1 = x + A'(y) .
\end{equation*}
So $A'(y) = -1$, and $A(y) = -y$ will work.  Take $F(x,y) = x^2+xy-y$.  We
wish to solve $x^2+xy-y = C$.  First let us find $C$.  As $y(0)=1$ then
$F(0,1) = C$.  Therefore $0^2+0\times 1 - 1 = C$, so $C=-1$.  Now we solve
$x^2+xy-y = -1$ for $y$ to get
\begin{equation*}
y = \frac{-x^2-1}{x-1} .
\end{equation*}
\end{exampleSol}

\begin{example}
Solve
\begin{equation*}
-\frac{y}{x^2+y^2} dx + \frac{x}{x^2+y^2} dy = 0 , \qquad y(1) = 2.
\end{equation*}
\end{example}

\begin{exampleSol}
We leave to the reader to check that
$M_y = N_x$.

This vector field $(M,N)$ is not conservative if considered as a vector
field of the entire plane minus the origin.  The problem is that if the curve $\gamma$
is a circle around the origin, say starting at $(1,0)$ and
ending at $(1,0)$ going counterclockwise, then if $F$ existed we would expect
\begin{equation*}
0 = F(1,0) - F(1,0) = \int_\gamma F_x \, dx + F_y \, dy = \int_\gamma \frac{-y}{x^2+y^2} \, dx +
\frac{x}{x^2+y^2} \, dy = 2\pi .
\end{equation*}
That is nonsense!
We leave the computation of the path integral to the interested reader, or
you can consult your multivariable calculus textbook.  So there is no
potential function $F$ defined everywhere outside the origin $(0,0)$.

If we think back to the theorem, it does not guarantee such a function
anyway.  It only guarantees a potential function locally, that is only in
some region near the initial point.  As $y(1) = 2$
we start at the point $(1,2)$.  Considering $x > 0$ and
integrating $M$ in $x$ or $N$ in $y$, we find
\begin{equation*}
F(x,y) = \operatorname{arctan} \bigl( \nicefrac{y}{x} \bigr) .
\end{equation*}
The implicit solution is 
$\operatorname{arctan} \bigl( \nicefrac{y}{x} \bigr) = C$.  Solving,
$y = \tan(C) x$.  That is, the solution is a straight line.  Solving $y(1) =
2$ gives us that $\tan(C) = 2$, and so $y= 2x$ is the desired solution.
See \figurevref{exact:y2x}, and note that the solution only exists for $x >
0$.
\begin{myfig}
\capstart
\diffyincludegraphics{width=3in}{width=4.5in}{exact-y2x}
\caption{Solution to 
$-\frac{y}{x^2+y^2} dx + \frac{x}{x^2+y^2} dy = 0$, $y(1) = 2$,
with initial point marked.\label{exact:y2x}}
\end{myfig}
\end{exampleSol}

\begin{example} \label{exact:exampleabove}
Solve
\begin{equation*}
x^2+y^2 + 2y(x+1) \frac{dy}{dx} = 0 .
\end{equation*}
\end{example}

\begin{exampleSol}
The reader should check that this equation is exact.
Let $M= x^2+y^2$ and $N=2y(x+1)$.
We follow the procedure for exact equations
\begin{equation*}
F(x,y) = \frac{1}{3}x^3 + xy^2 + A(y) ,
\end{equation*}
and
\begin{equation*}
2y(x+1) = 2xy + A'(y) .
\end{equation*}
Therefore $A'(y) = 2y$ or $A(y) = y^2$ and $F(x,y) = \frac{1}{3}x^3 + xy^2 +
y^2$.
We try to solve $F(x,y) = C$.  We easily solve for $y^2$ and then just take
the square root:
\begin{equation*}
y^2 = \frac{C-(\nicefrac{1}{3})x^3}{x+1},
\qquad \text{so} \qquad
y = \pm \sqrt{\frac{C-(\nicefrac{1}{3})x^3}{x+1}} .
\end{equation*}
When $x=-1$, the term in front of $\frac{dy}{dx}$ vanishes.  You can also
see that our solution is not valid in that case.  However, one could in that
case try to solve for $x$ in terms of $y$ starting from the implicit
solution $\frac{1}{3}x^3 + xy^2 + y^2 = C$.  The solution is 
somewhat messy and we leave it as implicit.
\end{exampleSol}

\subsection{Integrating factors}

Sometimes an equation $M\, dx + N \, dy = 0$ is not exact, but it can be
made exact by multiplying with a function $u(x,y)$.  That is, perhaps
for some nonzero function $u(x,y)$,
\begin{equation*}
u(x,y) M(x,y) \, dx + u(x,y) N(x,y) \, dy = 0
\end{equation*}
is exact.  Any solution to this new equation is also a solution to
$M\, dx + N \, dy = 0$.

In fact, a linear equation
\begin{equation*}
\frac{dy}{dx} + p(x) y = f(x), \qquad
\text{or} \qquad
\bigl( p(x) y - f(x) \bigr)\, dx +  dy  = 0
\end{equation*}
is always such an equation.  Let $r(x) = e^{\int p(x)\,dx}$ be the
integrating factor for a linear equation.  Multiply the equation by $r(x)$
and write it in the form of $M + N \frac{dy}{dx} = 0$.
\begin{equation*}
r(x) p(x) y - r(x) f(x) + r(x) \frac{dy}{dx} = 0 .
\end{equation*}
Then $M = r(x) p(x) y - r(x) f(x)$, so
$M_y = r(x) p(x)$, while $N = r(x)$, so
$N_x = r'(x) = r(x) p(x)$.  In other words, we have an exact equation.
Integrating
factors for linear functions are just a special case of integrating
factors for exact equations.

But how do we find the integrating factor $u$?  Well, given an equation
\begin{equation*}
M \, dx + N \, dy = 0 ,
\end{equation*}
$u$ should be a
function such that
\begin{equation*}
\frac{\partial}{\partial y} \bigl[ u M \bigr] = 
u_y M + u M_y = 
\frac{\partial}{\partial x} \bigl[ u N \bigr] = 
u_x N + u N_x .
\end{equation*}
Therefore,
\begin{equation*}
(M_y-N_x)u = u_x N - u_y M .
\end{equation*}
At first it may seem we replaced one differential equation by another.
True, but all hope is not lost.

A strategy that often works is to look for a $u$ that is a function
of $x$ alone, or a function of $y$ alone.  If $u$ is a function of $x$
alone,
that is $u(x)$, then we write $u'(x)$ instead of $u_x$, and $u_y$ is just
zero.
Then
\begin{equation*}
\frac{M_y-N_x}{N}u = u' .
\end{equation*}
In particular, $\frac{M_y-N_x}{N}$ ought to be a function of $x$ alone (not
depend on $y$).  If so, then we have a linear equation
\begin{equation*}
u' - \frac{M_y-N_x}{N} u = 0 .
\end{equation*}
Letting $p(x) = \frac{M_y-N_x}{N}$,
we solve using the standard integrating factor method,
to find $u(x) = C e^{\int p(x) \, dx}$.  The constant in the
solution is not relevant, we need any nonzero solution,
so we take $C=1$.
Then $u(x) = e^{\int p(x) \, dx}$ is the integrating factor.

Similarly we could try a function of the form $u(y)$.
Then
\begin{equation*}
\frac{M_y-N_x}{M} u = - u' .
\end{equation*}
In particular, $\frac{M_y-N_x}{M}$ ought to be a function of $y$ alone.
If so, then we have a linear equation
\begin{equation*}
u' + \frac{M_y-N_x}{M} u = 0 .
\end{equation*}
Letting $q(y) = \frac{M_y-N_x}{M}$,
we find $u(y) = C e^{-\int q(y) \, dy}$.  We
take $C=1$.  So $u(y) = e^{-\int q(y) \, dy}$ is the integrating factor.

\begin{example}
Solve
\begin{equation*}
\frac{x^2+y^2}{x+1} + 2y \frac{dy}{dx} = 0 .
\end{equation*}
\end{example}

\begin{exampleSol}
Let $M= \frac{x^2+y^2}{x+1}$ and $N=2y$.
Compute
\begin{equation*}
M_y-N_x = \frac{2y}{x+1} - 0 = \frac{2y}{x+1} .
\end{equation*}
As this is not zero, the equation is not exact.  We notice 
\begin{equation*}
P(x) = \frac{M_y-N_x}{N} = \frac{2y}{x+1} \frac{1}{2y} = \frac{1}{x+1} 
\end{equation*}
is a function of $x$ alone.    We compute the integrating factor
\begin{equation*}
e^{\int  P(x) \, dx}
=
e^{\ln |x+1|} = |x+1| .
\end{equation*}
Assuming that we want to look at $x > -1$, we multiply our given equation by $(x+1)$ to obtain
\begin{equation*}
x^2+y^2 + 2y(x+1) \frac{dy}{dx} = 0 ,
\end{equation*}
which is an exact equation that we solved in
\exampleref{exact:exampleabove}.  The solution was
\begin{equation*}
y = \pm \sqrt{\frac{C-(\nicefrac{1}{3})x^3}{x+1}} .
\end{equation*}
If, instead, we had wanted a solution with $x < -1$, we would have needed to multiply by $-(x+1)$, which would have given a very similar result. 
\end{exampleSol}

\begin{example}
Solve
\begin{equation*}
y^2 + (xy+1) \frac{dy}{dx} = 0 .
\end{equation*}
\end{example}

\begin{exampleSol}
First compute
\begin{equation*}
M_y-N_x = 2y-y = y .
\end{equation*}
As this is not zero, the equation is not exact.  We observe
\begin{equation*}
Q(y) = \frac{M_y-N_x}{M} = \frac{y}{y^2} = \frac{1}{y} 
\end{equation*}
is a function of $y$ alone.    We compute the integrating factor
\begin{equation*}
e^{-\int  Q(y) \, dy}
=
e^{-\ln y} = \frac{1}{y} .
\end{equation*}
Therefore we look at the exact equation
\begin{equation*}
y + \frac{xy+1}{y} \frac{dy}{dx} = 0 .
\end{equation*}
The reader should double check that this equation is exact.
We follow the procedure for exact equations
\begin{equation*}
F(x,y) = xy + A(y) ,
\end{equation*}
and
\begin{equation}
\frac{xy+1}{y} = x+\frac{1}{y} = x+ A'(y) .
\end{equation}
Consequently $A'(y) = \frac{1}{y}$ or $A(y) = \ln y$.  Thus $F(x,y) = xy + \ln y$.
It is not possible to solve $F(x,y)=C$ for $y$
in terms of elementary functions, so 
let us be content with the implicit solution:
\begin{equation*}
xy + \ln y = C .
\end{equation*}
We are looking for the general solution and we divided by
$y$ above.  We should check what happens when $y=0$, as the equation itself
makes perfect sense in that case.  We plug in $y=0$ to find the
equation is satisfied.  So $y(x)=0$ is also a solution.
\end{exampleSol}

\subsection{Exercises}

\begin{exercise}
Solve the following exact equations, implicit general solutions
will suffice:
\begin{tasks}(2)
\task
$(2 xy + x^2) \, dx + (x^2+y^2+1) \, dy = 0$
\task
$x^5 + y^5 \frac{dy}{dx} = 0$
\task
$e^x+y^3 + 3xy^2 \frac{dy}{dx} = 0$
\task
$(x+y)\cos(x)+\sin(x) + \sin(x)y' = 0$
\end{tasks}
\end{exercise}

\begin{exercise}\ansMark%
Solve the following exact equations, implicit general solutions
will suffice:
\begin{tasks}(2)
\task
$\cos(x)+ye^{xy} + xe^{xy} y' = 0$
\task
$(2x+y)\, dx + (x-4y) \, dy = 0$
\task
$e^x + e^y \frac{dy}{dx} = 0$
\task
$(3x^2+3y)\,dx + (3y^2+3x)\, dy = 0$
\end{tasks}
\end{exercise}
\exsol{%
a) $e^{xy}+\sin(x)=C$ \quad
b) $x^2+xy-2y^2=C$ \quad
c) $e^x+e^y=C$ \quad
d) $x^3 + 3xy+ y^3 = C$
}

\begin{exercise}
Solve the differential equation $(2ye^{2xy} - 2x) + (2xe^{2xy} + \cos(y))y' = 0$
\end{exercise}

\begin{exercise}
Solve the differential equation $(-y\sin(xy) - 2xe^{x^2}) + (-x\sin(xy) + 1)y' = 0$
\end{exercise}

\begin{exercise}
Solve the differential equation $(2x + 3y\sin(xy)) + (3x\sin(xy) - e^y)y' = 0$ with $y(2) = 0$. 
\end{exercise}

\begin{exercise}
Solve the differential equation $x + yy' = 0$ with $y(0) = 8$. Write this as an explicit function and determine the interval of $x$ values where the solution is valid. 
\end{exercise}

\begin{exercise}
Solve the differential equation $2x-2 + (8y+16)y' = 0$ with $y(2) = 0$. Write this as an explicit function and determine the interval of $x$ values where the solution is valid. 
\end{exercise}

\begin{exercise}
Find the integrating factor for the following equations making them into
exact equations:
\begin{tasks}(2)
\task
$e^{xy} \, dx + \frac{y}{x} e^{xy} \, dy = 0$
\task
$\frac{e^x+y^3}{y^2} \, dx + 3x \, dy = 0$
\task
$4(y^2+x) \, dx + \frac{2x+2y^2}{y} \, dy = 0$
\task
$2\sin(y) \, dx + x\cos(y)\, dy = 0$
\end{tasks}
\end{exercise}

\begin{samepage}
\begin{exercise}\ansMark%
Find the integrating factor for the following equations making them into
exact equations:
\begin{tasks}(2)
\task
$\frac{1}{y}\, dx + 3y \, dy = 0$
\task
$dx - e^{-x-y} \, dy = 0$
\task
$\bigl( \frac{\cos(x)}{y^2} + \frac{1}{y} \bigr) \, dx + \frac{x}{y^2} \, dy = 0$
\task
$\bigl( 2y + \frac{y^2}{x} \bigr) \, dx + ( 2y+x )\, dy = 0$
\end{tasks}
\end{exercise}
\end{samepage}
\exsol{%
a) Integrating factor is $y$, equation becomes $dx + 3y^2\,dy = 0$.
\quad
b) Integrating factor is $e^x$, equation becomes $e^x \, dx - e^{-y}\,dy
= 0$.
\quad
c) Integrating factor is $y^2$, equation becomes $(\cos(x)+y)\, dx +
x\,dy = 0$.
\quad
d) Integrating factor is $x$, equation becomes $( 2xy+y^2 )\, dx +
(x^2+2xy)\,dy = 0$.
}

\begin{exercise}
Suppose you have an equation of the form:
$f(x) + g(y) \frac{dy}{dx} = 0$.
\begin{tasks}
\task
Show it is exact.
\task
Find the form of the potential function in terms of $f$ and $g$.
\end{tasks}
\end{exercise}

\begin{exercise}
Suppose that we have the equation $f(x) \, dx - dy = 0$.
\begin{tasks}
\task
Is this equation exact?
\task
Find the general solution using a definite integral.
\end{tasks}
\end{exercise}

\begin{exercise}
Find the potential function $F(x,y)$ of the exact equation $\frac{1+xy}{x}\, dx +
\bigl(\nicefrac{1}{y} + x \bigr) \, dy = 0$ in two different ways.
\begin{tasks}
\task
Integrate $M$ in terms of $x$ and then differentiate in $y$ and set to
$N$.
\task
Integrate $N$ in terms of $y$ and then differentiate in $x$ and set to
$M$.
\end{tasks}
\end{exercise}

\begin{samepage}
\begin{exercise}
A function $u(x,y)$ is said to be a \emph{\myindex{harmonic function}} if
$u_{xx} + u_{yy} = 0$.
\begin{tasks}
\task
Show if $u$ is harmonic, $-u_y \, dx + u_x \, dy = 0$ is an exact
equation.  So there exists (at least locally)
the so-called \emph{\myindex{harmonic conjugate}} function
$v(x,y)$ such that $v_x = -u_y$ and $v_y = u_x$.
\end{tasks}
Verify that the following $u$ are harmonic and 
find the corresponding harmonic conjugates $v$:
\begin{tasks}[resume](3)
\task $u = 2xy$
\task $u = e^x \cos y$
\task $u = x^3-3xy^2$
\end{tasks}
\end{exercise}
\end{samepage}

\begin{exercise}\label{ex:separableExact}\ansMark%
\leavevmode
\begin{tasks}
\task Show that every
separable equation $y' = f(x)g(y)$ can be written as an exact equation,
and verify that it is indeed exact.
\task Using this rewrite $y' = xy$ as an exact equation, solve it and verify
that the solution is the same as it was in \exampleref{example:yprimeisxy}.
\end{tasks}
\end{exercise}
\exsol{%
a) The equation is $ - f(x) \, dx + \frac{1}{g(y)} \, dy$,
and this is exact
because $M = -f(x)$, $N = \frac{1}{g(y)}$, so $M_y = 0 = N_x$.
\quad
b) $-x \, dx + \frac{1}{y} \, dy = 0$, leads to
potential function $F(x,y) = -\frac{x^2}{2} + \ln \lvert y \rvert$, solving
$F(x,y) = C$ leads to the same solution as the example.
}

\setcounter{exercise}{100}

