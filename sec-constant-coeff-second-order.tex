\section{Constant coefficient second order linear ODEs}
\label{sec:ccsol}

\sectionnotes{Verbatim from Lebl}

\sectionnotes{more than 1 lecture\EPref{,
second part of \S3.1 in \cite{EP}}\BDref{,
\S3.1 in \cite{BD}}}

\subsection{Solving constant coefficient equations}

Consider the problem
\begin{equation*}
y''-6y'+8y = 0, \qquad y(0) = - 2, \qquad y'(0) = 6 .
\end{equation*}
This is a second order linear homogeneous equation with constant
coefficients.  \emph{Constant coefficients\index{constant coefficient}}
means that the functions 
in front of $y''$, $y'$, and $y$ are constants, they do not depend on $x$.

To guess a solution, think of a function that stays essentially the
same when we differentiate it, so that we can take the function and its
derivatives, add some multiples of these together, and end up with zero.
Yes, we are talking about the exponential.

Let us try\footnote{%
Making an educated guess with some parameters to solve for 
is such a central technique in differential equations, that people sometimes use
a fancy name for such a guess: \emph{\myindex{ansatz}}, German for \myquote{initial
placement of a tool at a work piece.}  Yes, the Germans have a word for that.}
a solution of the form $y = e^{rx}$.  Then $y' = r e^{rx}$ and
$y'' = r^2 e^{rx}$.  Plug in to get
\begin{align*}
y''-6y'+8y & = 0 , \\
\underbrace{r^2 e^{rx}}_{y''} -6 \underbrace{r e^{rx}}_{y'}+8 \underbrace{e^{rx}}_{y} & = 0 , \\
r^2 -6 r +8 & = 0 \qquad \text{(divide through by } e^{rx} \text{)},\\
(r-2)(r-4) & = 0 .
\end{align*}
Hence, if $r=2$ or $r=4$, then $e^{rx}$ is a solution.  So let $y_1 = e^{2x}$
and $y_2 = e^{4x}$.

\begin{exercise}
Check that $y_1$ and $y_2$ are solutions.
\end{exercise}

The functions $e^{2x}$ and $e^{4x}$ are linearly independent.  If they
were not linearly independent, we could write $e^{4x} = C e^{2x}$ for
some constant $C$,
implying that $e^{2x} = C$ for all $x$, which is clearly not possible. 
Hence, we can write the general solution as
\begin{equation*}
y = C_1 e^{2x} + C_2 e^{4x} .
\end{equation*}
We need to solve for $C_1$ and $C_2$.  To apply the initial conditions,
we first find $y' = 2 C_1 e^{2x} + 4 C_2 e^{4x}$.  We plug $x=0$ into
$y$ and $y'$ and solve.
\begin{align*}
-2 & = y(0) = C_1 + C_2 , \\
6 & = y'(0) = 2 C_1 + 4 C_2 .
\end{align*}
Either apply some matrix algebra, or just solve these by high school
math.  For example, divide the second equation by 2
to obtain $3 = C_1 + 2 C_2$, and subtract the two equations to
get $5 = C_2$.  Then $C_1 = -7$ as $-2 = C_1 + 5$.  Hence, the solution we
are
looking for is
\begin{equation*}
y = -7 e^{2x} + 5 e^{4x} .
\end{equation*}

\medskip

Let us generalize this example into a method.
Suppose that we have an equation
\begin{equation} \label{ccsol:eq}
a y'' + b y' + c y = 0 ,
\end{equation}
where $a, b, c$ are constants.  Try the solution $y = e^{rx}$ to obtain
\begin{equation*}
a r^2 e^{rx} + 
b r e^{rx} + 
c e^{rx} = 0 .
\end{equation*}
Divide by $e^{rx}$ to obtain the so-called
\emph{\myindex{characteristic equation}} of the ODE:
\begin{equation*}
a r^2 + 
b r + 
c = 0 .
\end{equation*}
Solve for the $r$ by using the \myindex{quadratic formula}.
\begin{equation*}
r_1, r_2 = \frac{-b \pm \sqrt{b^2 - 4ac}}{2a} .
\end{equation*}
So $e^{r_1 x}$ and $e^{r_2 x}$ are solutions.  There is
still a difficulty if $r_1 = r_2$, but it is not hard to overcome.

\begin{theorem}
Suppose that $r_1$ and $r_2$ are the roots of the characteristic equation.
\begin{enumerate}[(i)]
\item If $r_1$ and $r_2$ are distinct and real (when $b^2 - 4ac > 0$),
then \eqref{ccsol:eq} has the general solution
\begin{equation*}
y = C_1 e^{r_1 x} + C_2 e^{r_2 x} .
\end{equation*}
\item If $r_1 = r_2$ (happens when $b^2 - 4ac = 0$), 
then \eqref{ccsol:eq} has the general solution
\begin{equation*}
y = (C_1 + C_2 x)\, e^{r_1 x} .
\end{equation*}
\end{enumerate}
\end{theorem}

\begin{example} \label{example:expsecondorder}
Solve
\begin{equation*}
y'' - k^2 y = 0 .
\end{equation*}
The characteristic equation is $r^2 - k^2 = 0$ or 
$(r-k)(r+k) = 0$.  Consequently, $e^{-k x}$ and $e^{kx}$ are the two
linearly independent solutions, and the general solution is
\begin{equation*}
y = C_1 e^{kx} + C_2e^{-kx} .
\end{equation*}
Since
$\cosh s = \frac{e^s+e^{-s}}{2}$
and
$\sinh s = \frac{e^s-e^{-s}}{2}$,
we can also write the general solution
as
\begin{equation*}
y = D_1 \cosh(kx) + D_2 \sinh(kx) .
\end{equation*}
\end{example}

\begin{example}
Find the general solution of
\begin{equation*}
y'' -8 y' + 16 y = 0 .
\end{equation*}

The characteristic equation is $r^2 - 8 r + 16 = {(r-4)}^2 = 0$.
The equation has a 
double root $r_1 = r_2 = 4$.  The general solution is, therefore,
\begin{equation*}
y = (C_1 + C_2 x)\, e^{4 x} = C_1 e^{4x} + C_2 x e^{4x} .
\end{equation*}

\begin{exercise}
Check that $e^{4x}$ and $x e^{4x}$ are linearly independent.
\end{exercise}

That $e^{4x}$ solves the equation is clear.  If $x e^{4x}$ solves the
equation, then we know we are done.  Let us compute
$y' = e^{4x} + 4xe^{4x}$ and
$y'' = 8 e^{4x} + 16xe^{4x}$.  Plug in
\begin{equation*}
y'' - 8 y' + 16 y = 
8 e^{4x} + 16xe^{4x} - 8(e^{4x} + 4xe^{4x}) + 16 xe^{4x} = 
0 .
\end{equation*}
\end{example}

In some sense, a doubled root rarely happens.  If coefficients are 
picked randomly, a doubled root is unlikely.
There are, however, some natural phenomena (such as resonance as we will see)
where a doubled root does happen, so we cannot just dismiss this case.

Let us give a short argument for why the solution $x e^{r x}$ works when the
root is doubled.  This case is really a limiting case of when
the two roots are distinct and very close.  Note that 
$\frac{e^{r_2 x} - e^{r_1 x}}{r_2 - r_1}$ is a solution when the roots are
distinct.  When we take the limit as $r_1$ goes to $r_2$, we are really
taking the
derivative of $e^{rx}$ using $r$ as the variable.  Therefore, the limit is 
$x e^{rx}$, and hence this is a solution in the doubled root case.

\subsection{Complex numbers and Euler's formula}

A polynomial may have complex roots.  The
equation $r^2 + 1 = 0$ has no real roots, but it does have two complex roots.
Here we review some properties of complex numbers\index{complex number}.

Complex numbers may seem a strange concept, especially because of the
terminology.  There is nothing imaginary or really complicated about complex
numbers.
A complex number is simply a pair of real numbers, $(a,b)$.  
Think of a complex number as a point in the plane.  We add complex numbers
in the straightforward way: $(a,b)+(c,d)=(a+c,b+d)$.  We define
multiplication\index{multiplication of complex numbers} by
\begin{equation*}
(a,b) \times (c,d) \overset{\text{def}}{=} (ac-bd,ad+bc) .
\end{equation*}
It turns out that with this multiplication rule, all the standard properties
of arithmetic hold.  Further, and most importantly $(0,1) \times (0,1) =
(-1,0)$.

Generally we write $(a,b)$ as $a+ib$, and we treat $i$ as if it were an
unknown.  When $b$ is zero, then $(a,0)$ is just the number $a$.
We do arithmetic with complex numbers just as we would
with polynomials.
The property we just mentioned becomes $i^2 = -1$.
So whenever we see $i^2$, we replace it by $-1$.
For example,
\begin{equation*}
(2+3i)(4i) - 5i = 
(2\times 4)i + (3 \times 4) i^2 - 5i
=
8i + 12 (-1) - 5i
=
-12 + 3i .
\end{equation*}

The numbers
$i$ and $-i$ are the two roots of $r^2 + 1 = 0$.
Engineers often use the letter $j$ instead of $i$ for the square
root of $-1$.  We use the mathematicians' convention and use $i$.

\begin{exercise}
Make sure you understand (that you can justify)
the following identities:
\begin{tasks}(2)
\task $i^2 = -1$, $i^3 = -i$, $i^4 = 1$,
\task $\dfrac{1}{i} = -i$,
\task $(3-7i)(-2-9i) = \cdots = -69-13i$,
\task $(3-2i)(3+2i) = 3^2 - {(2i)}^2 = 3^2 + 2^2 = 13$,
\task $\frac{1}{3-2i} = \frac{1}{3-2i} \frac{3+2i}{3+2i} = \frac{3+2i}{13}
= \frac{3}{13}+\frac{2}{13}i$.
\end{tasks}
\end{exercise}

\pagebreak[2]
We also define the exponential $e^{a+ib}$ of a complex number.  We do
this by writing down the Taylor series and plugging in the complex
number.  Because most properties of the exponential can be proved by looking
at the Taylor series, these
properties still hold for the complex
exponential.  For example the very important property: $e^{x+y} = e^x e^y$.  This means that
$e^{a+ib} = e^a e^{ib}$.  Hence if we can compute $e^{ib}$, we can
compute $e^{a+ib}$.  For $e^{ib}$ we use the so-called
\emph{\myindex{Euler's formula}}.

\begin{theorem}[Euler's formula] \label{eulersformula}
\begin{equation*}
\mybxbg{~~
e^{i \theta} = \cos \theta + i \sin \theta
\qquad \text{ and } \qquad
e^{- i \theta} = \cos \theta - i \sin \theta .
~~}
\end{equation*}
\end{theorem}

In other words, $e^{a+ib} = e^a \bigl( \cos(b) + i \sin(b) \bigr) = e^a \cos(b) + i e^a \sin(b)$.

\begin{exercise}
Using Euler's formula, check the identities:
\begin{equation*}
\cos \theta = \frac{e^{i \theta} + e^{-i \theta}}{2}
\qquad \text{and} \qquad
\sin \theta = \frac{e^{i \theta} - e^{-i \theta}}{2i}.
\end{equation*}
\end{exercise}

\begin{exercise}
Double angle identities:
Start with $e^{i(2\theta)} = {\bigl(e^{i \theta} \bigr)}^2$.  Use Euler on
each side and deduce:
\begin{equation*}
\cos (2\theta) = \cos^2 \theta - \sin^2 \theta
\qquad \text{and} \qquad
\sin (2\theta) = 2 \sin \theta \cos \theta .
\end{equation*}
\end{exercise}

For a complex number $a+ib$ we call
$a$ the \emph{\myindex{real part}} and $b$ the \emph{\myindex{imaginary part}} of the number.
Often the following notation is used,
\begin{equation*}
\operatorname{Re}(a+ib) = a
\qquad \text{and} \qquad
\operatorname{Im}(a+ib) = b.
\end{equation*}

\subsection{Complex roots}

Suppose the equation $ay'' + by' + cy = 0$ has the 
characteristic equation
$a r^2 + b r + c = 0$ that has \myindex{complex roots}.
By the quadratic
formula, the roots are
$\frac{-b \pm \sqrt{b^2 - 4ac}}{2a}$.
These roots are complex if $b^2 - 4ac < 0$.  In this case the
roots are
\begin{equation*}
r_1, r_2 = \frac{-b}{2a} \pm i\frac{\sqrt{4ac - b^2}}{2a} .
\end{equation*}
As you can see, we always get a pair of roots of the form $\alpha \pm i
\beta$.  In this case we can still write the solution as
\begin{equation*}
y = C_1 e^{(\alpha+i\beta)x} + C_2 e^{(\alpha-i\beta)x} .
\end{equation*}
However, the exponential is now complex-valued.  We need to allow
$C_1$ and $C_2$ to be complex numbers to obtain a real-valued solution (which
is what we are after).  While there is nothing particularly wrong with this
approach,
it can make calculations harder and it is generally preferred
to find two real-valued
solutions.

Here we can use \hyperref[eulersformula]{Euler's formula}.  Let
\begin{equation*}
y_1 = e^{(\alpha+i\beta)x} \qquad \text{and} \qquad y_2 = e^{(\alpha-i\beta)x} .
\end{equation*}
Then 
\begin{align*}
y_1 & = e^{\alpha x} \cos (\beta x) + i e^{\alpha x} \sin (\beta x) , \\
y_2 & = e^{\alpha x} \cos (\beta x) - i e^{\alpha x} \sin (\beta x) .
\end{align*}

Linear combinations of solutions are also solutions.  Hence,
\begin{align*}
y_3 & = \frac{y_1 + y_2}{2} = e^{\alpha x} \cos (\beta x) , \\ 
y_4 & = \frac{y_1 - y_2}{2i} = e^{\alpha x} \sin (\beta x) ,
\end{align*}
are also solutions.  Furthermore, they are real-valued.  It is not hard to
see that they are linearly independent (not multiples of each other).
Therefore, we have the following theorem.

\begin{theorem}
Take the equation
\begin{equation*}
ay'' + by' + cy = 0 .
\end{equation*}
If the characteristic equation has the roots $\alpha \pm i \beta$
(when $b^2 - 4ac < 0$),
then the general solution is
\begin{equation*}
y = C_1 e^{\alpha x} \cos (\beta x) + C_2 e^{\alpha x} \sin (\beta x) .
\end{equation*}
\end{theorem}

\begin{example} \label{example:sincossecondorder}
Find the general solution of $y'' + k^2 y = 0$, for a constant
$k > 0$.

The characteristic equation is $r^2 + k^2 = 0$.  Therefore,
the roots are $r = \pm ik$, and by the theorem, we have the general solution
\begin{equation*}
y = C_1 \cos (kx) + C_2 \sin (kx) .
\end{equation*}
\end{example}

\begin{example}
Find the solution of $y'' - 6 y' + 13 y = 0$, $y(0) = 0$, $y'(0) =
10$.

The characteristic equation is $r^2 - 6 r + 13 = 0$.  By completing the
square we get ${(r-3)}^2 + 2^2 = 0$ and hence the roots are
$r = 3 \pm 2i$.
By the theorem we have the general solution
\begin{equation*}
y = C_1 e^{3x} \cos (2x) + C_2 e^{3x} \sin (2x) .
\end{equation*}
To find the solution satisfying the initial conditions, we first plug in zero
to get
\begin{equation*}
0 = y(0) = C_1 e^{0} \cos 0 + C_2 e^{0} \sin 0  = C_1 .
\end{equation*}
Hence, $C_1 = 0$ and $y = C_2 e^{3x} \sin (2x)$.  We differentiate,
\begin{equation*}
y' = 3C_2 e^{3x} \sin (2x) + 2C_2 e^{3x} \cos (2x) .
\end{equation*}
We again plug in the initial condition and obtain $10 = y'(0) = 2C_2$, or
$C_2 = 5$.  The solution we are seeking is
\begin{equation*}
y = 5 e^{3x} \sin (2x) .
\end{equation*}
\end{example}

\subsection{Exercises}

\begin{exercise}
Find the general solution of $2y'' + 2y' -4 y = 0$.
\end{exercise}

\begin{exercise}
Find the general solution of $y'' + 9y' - 10 y = 0$.
\end{exercise}

\begin{exercise}
Solve $y'' - 8y' + 16 y = 0$ for $y(0) = 2$, $y'(0) = 0$.
\end{exercise}

\begin{exercise}
Solve $y'' + 9y' = 0$ for $y(0) = 1$, $y'(0) = 1$.
\end{exercise}

\begin{exercise}
Find the general solution of $2y'' + 50y = 0$.
\end{exercise}

\begin{exercise}
Find the general solution of $y'' + 6 y' + 13 y = 0$.
\end{exercise}

\begin{exercise}
Find the general solution of $y'' = 0$ using the methods of this section.
\end{exercise}

\begin{exercise}
The method of this section applies to equations of other orders than two.
We will see
higher orders later.  Try to solve the first order equation
$2y' + 3y = 0$ using the methods of this section.
\end{exercise}

\begin{exercise}
Let us revisit the Cauchy--Euler equations\index{Cauchy--Euler equation} of
\exercisevref{sol:eulerex}.  Suppose now
that ${(b-a)}^2-4ac < 0$.  Find a formula for the general solution
of $a x^2 y'' + b x y' + c y = 0$.  Hint: Note that $x^r = e^{r \ln x}$.
\end{exercise}

\begin{exercise}
Find the solution to
$y''-(2\alpha) y' + \alpha^2 y=0$, $y(0) = a$, $y'(0)=b$,
where $\alpha$, $a$, and $b$ are real numbers.
\end{exercise}

\begin{exercise}
Construct an equation such that $y = C_1 e^{-2x} \cos(3x) + C_2 e^{-2x}
\sin(3x)$ is the general
solution.
\end{exercise}

\setcounter{exercise}{100}

\begin{exercise}
Find the general solution to
$y''+4y'+2y=0$.
\end{exercise}
\exsol{%
$y =
C_1 e^{(-2+\sqrt{2}) x}
+
C_2 e^{(-2-\sqrt{2}) x}$
}

\begin{exercise}
Find the general solution to
$y''-6y'+9y=0$.
\end{exercise}
\exsol{%
$y =
C_1 e^{3x}
+
C_2 x e^{3x}$
}

\begin{exercise}
Find the solution to
$2y''+y'+y=0$, $y(0) = 1$, $y'(0)=-2$.
\end{exercise}
\exsol{%
$y =
e^{-x/4} \cos\bigl((\nicefrac{\sqrt{7}}{4})x\bigr)
-
\sqrt{7}
e^{-x/4} \sin\bigl((\nicefrac{\sqrt{7}}{4})x\bigr)$
}

\begin{exercise}
Find the solution to
$2y''+y'-3y=0$, $y(0) = a$, $y'(0)=b$.
\end{exercise}
\exsol{%
$y = \frac{2(a-b)}{5} \, e^{-3x/2}+\frac{3 a+2 b}{5} \, e^x$
}

\begin{exercise}
Find the solution to
$z''(t) = -2z'(t)-2z(t)$, $z(0) = 2$, $z'(0)= -2$.
\end{exercise}
\exsol{%
$z(t) =
2e^{-t} \cos(t)$
}

\begin{exercise}
Find the solution to
$y''-(\alpha+\beta) y' + \alpha \beta y=0$, $y(0) = a$, $y'(0)=b$,
where $\alpha$, $\beta$, $a$, and $b$ are real numbers, and $\alpha \not=
\beta$.
\end{exercise}
\exsol{%
$y =
\frac{a \beta-b}{\beta-\alpha} e^{\alpha x} + 
\frac{b-a \alpha}{\beta-\alpha} e^{\beta x}$
}

\begin{exercise}
Construct an equation such that $y = C_1 e^{3x} + C_2 e^{-2x}$ is the general
solution.
\end{exercise}
\exsol{%
$y'' -y'-6y=0$
}
