
\section{Matrix algebra}
\label{matalg:section}

\LAtt{A.2}

\LO{
\item Perform addition and multiplication operations on matrices,
\item Compute inverses of $2\times 2$ matrices, and
\item Identify triangular, diagonal, and symmetric matrices.
}

\subsection{One-by-one matrices}

Let us motivate what we want to achieve with matrices. What do real-valued linear mappings look like? A linear function 
of real numbers that you have seen in calculus is of the form
\[ f(x) = mx + b. \] However, the properties of linear mappings discussed in the previous section are that
\[ f(x+y) = f(x) + f(y) \qquad f(a x) = a f(x). \] Plugging in the definition from above gives that
\[ 
\begin{split}
f(x+y) &= m(x+y) + b = mx + my + b \\
f(ax) &= m(ax) + b = a(mx) + b
\end{split}
\]
and neither of these match up appropriately, since
\[
\begin{split}
f(x) + f(y) &= mx + b + my + b = mx + my + 2b \\
af(x) &+ a(mx + b) = a(mx) + ab
\end{split}.
\]
In order for these to work, we need to have $b=0$. Therefore, real-valued linear mappings of the real line, linear functions
that eat numbers and spit out numbers, are just multiplications by a
number.  

Consider a mapping defined by multiplying by a
number.  Let's call this number $\alpha$.   The mapping then takes $x$ to
$\alpha x$.  What we can do is
to \emph{add} such mappings.
If we have another mapping $\beta$, then
\begin{equation*}
\alpha x + \beta x = (\alpha + \beta) x .
\end{equation*}
We get a new mapping $\alpha+\beta$ that multiplies $x$ by, well,
$\alpha+\beta$.  If $D$ is a mapping that doubles things, 
$Dx = 2x$, and $T$ is a mapping that triples, $Tx = 3x$, then
$D+T$ is a mapping that multiplies by $5$, $(D+T)x = 5x$.

Similarly we can \emph{compose} such mappings, that
is, we could apply one and then the other.  We take $x$, we run it through
the first mapping $\alpha$ to get $\alpha$ times $x$, then we run
$\alpha x$ through the second mapping $\beta$.  In other words,
\begin{equation*}
\beta ( \alpha x ) = (\beta \alpha) x .
\end{equation*}
We just multiply those two numbers.  Using our doubling
and tripling mappings, if we double and then triple, that is $T(Dx)$ then
we obtain $3(2x) = 6x$.  The composition $TD$ is the mapping that multiplies
by $6$.  For larger matrices, composition also ends up being a kind of
multiplication.

\subsection{Matrix addition and scalar multiplication}

The mappings that multiply numbers by numbers are just $1 \times 1$ matrices.  The
number $\alpha$ above could be written as a matrix $[\alpha]$.
So perhaps we would want to do the same things to all matrices that we
did to those $1 \times 1$ matrices at the start of this section above.
First, let us add matrices.
If we have a matrix $A$ and a matrix $B$ that are of the same size,
say $m \times n$, then they are mappings from
${\mathbb{R}}^n$ to ${\mathbb{R}}^m$.  The mapping $A+B$ should also be a mapping from
${\mathbb{R}}^n$ to ${\mathbb{R}}^m$, and it should do the following to
vectors:
\begin{equation*}
(A+B) \vec{x} = A\vec{x} + B \vec{x} .
\end{equation*}
It turns out you just add the matrices element-wise:  If the
$ij^{\text{th}}$ entry of $A$ is $a_{ij}$, and the
$ij^{\text{th}}$ entry of $B$ is $b_{ij}$, then the
$ij^{\text{th}}$ entry of $A+B$ is $a_{ij} + b_{ij}$.  If
\begin{equation*}
A = 
\begin{bmatrix}
a_{11} & a_{12} & a_{13}  \\
a_{21} & a_{22} & a_{23}
\end{bmatrix}
\qquad \text{and} \qquad
B = 
\begin{bmatrix}
b_{11} & b_{12} & b_{13}  \\
b_{21} & b_{22} & b_{23}
\end{bmatrix} ,
\end{equation*}
then
\begin{equation*}
A+B = 
\begin{bmatrix}
a_{11} + b_{11} & a_{12} + b_{12} & a_{13} + b_{13}  \\
a_{21} + b_{21} & a_{22} + b_{22} & a_{23} + b_{23}
\end{bmatrix} .
\end{equation*}
Let us illustrate on a more concrete example:
\begin{equation*}
\begin{bmatrix}
1 & 2 \\
3 & 4 \\
5 & 6
\end{bmatrix}
+
\begin{bmatrix}
7 & 8 \\
9 & 10 \\
11 & -1
\end{bmatrix}
=
\begin{bmatrix}
1+7 & 2+8 \\
3+9 & 4+10 \\
5+11 & 6-1
\end{bmatrix}
=
\begin{bmatrix}
8 & 10 \\
12 & 14 \\
16 & 5
\end{bmatrix} .
\end{equation*}
Let's check that this does the right thing to a vector.  Let's use some
of the vector algebra that we already know, and
regroup things:
\begin{equation*}
\begin{split}
\begin{bmatrix}
1 & 2 \\
3 & 4 \\
5 & 6
\end{bmatrix}
\begin{bmatrix}
2 \\ -1
\end{bmatrix}
+
\begin{bmatrix}
7 & 8 \\
9 & 10 \\
11 & -1
\end{bmatrix}
\begin{bmatrix}
2 \\ -1
\end{bmatrix}
& =
\left(
2
\begin{bmatrix}
1 \\
3 \\
5
\end{bmatrix}
-
\begin{bmatrix}
2 \\
4 \\
6
\end{bmatrix}
\right)
+
\left(
2
\begin{bmatrix}
7 \\
9 \\
11
\end{bmatrix}
-
\begin{bmatrix}
8 \\
10 \\
-1
\end{bmatrix}
\right)
\\
& = 
2
\left(
\begin{bmatrix}
1 \\
3 \\
5
\end{bmatrix}
+
\begin{bmatrix}
7 \\
9 \\
11
\end{bmatrix}
\right)
-
\left(
\begin{bmatrix}
2 \\
4 \\
6
\end{bmatrix}
+
\begin{bmatrix}
8 \\
10 \\
-1
\end{bmatrix}
\right)
\\
& = 
2
\begin{bmatrix}
1+7 \\
3+9 \\
5+11
\end{bmatrix}
-
\begin{bmatrix}
2+8 \\
4+10 \\
6-1
\end{bmatrix}
=
2
\begin{bmatrix}
8 \\
12 \\
16
\end{bmatrix}
-
\begin{bmatrix}
10 \\
14 \\
5
\end{bmatrix}
\\
& =
\begin{bmatrix}
8 & 10 \\
12 & 14 \\
16 & 5
\end{bmatrix}
\begin{bmatrix}
2 \\
-1
\end{bmatrix} 
\quad
\left(
=
\begin{bmatrix}
2(8)- 10 \\
2(12) - 14 \\
2(16) - 5
\end{bmatrix}
=
\begin{bmatrix}
6 \\
10 \\
27
\end{bmatrix}
\right) .
\end{split}
\end{equation*}
If we replaced the numbers by letters that would constitute a proof!
You'll notice that we didn't really have to even compute what the
result is to convince ourselves that the two expressions were equal.

If the sizes of the matrices do not match, then addition is not defined.
If $A$ is $3 \times 2$ and $B$ is $2 \times 5$, then we cannot add
these matrices.  We don't know what that could possibly mean.

\medskip

It is also useful to have a matrix that when added to any other matrix
does nothing.  This is the zero matrix, the matrix of all zeros:
\begin{equation*}
\begin{bmatrix}
1 & 2 \\
3 & 4
\end{bmatrix}
+
\begin{bmatrix}
0 & 0 \\
0 & 0
\end{bmatrix}
=
\begin{bmatrix}
1 & 2 \\
3 & 4
\end{bmatrix} .
\end{equation*}
We often denote the zero matrix
by $0$ without specifying size.  We would then just write $A + 0$, where we
just assume that $0$ is the zero matrix of the same size as $A$.

\medskip

There are really two things we can multiply matrices by.  We can multiply
matrices by scalars or we can multiply by other matrices.  Let us first
consider multiplication by scalars.
For a matrix $A$ and a scalar $\alpha$ we want $\alpha A$ to be the matrix
that accomplishes
\begin{equation*}
(\alpha A) \vec{x} = \alpha (A \vec{x}) .
\end{equation*}
That is just scaling the result by $\alpha$.  If you think about it,
scaling every term in $A$ by $\alpha$ accomplishes just that:
If
\begin{equation*}
A = 
\begin{bmatrix}
a_{11} & a_{12} & a_{13}  \\
a_{21} & a_{22} & a_{23}
\end{bmatrix},
\qquad\text{then} \qquad
\alpha A = 
\begin{bmatrix}
\alpha a_{11} & \alpha a_{12} & \alpha a_{13}  \\
\alpha a_{21} & \alpha a_{22} & \alpha a_{23}
\end{bmatrix} .
\end{equation*}
For example,
\begin{equation*}
2
\begin{bmatrix}
1 & 2 & 3 \\
4 & 5 & 6
\end{bmatrix} =
\begin{bmatrix}
2 & 4 & 6 \\
8 & 10 & 12
\end{bmatrix} .
\end{equation*}

Let us list some properties of matrix addition and scalar multiplication.
Denote by $0$ the zero matrix, by
$\alpha$, $\beta$ scalars, and by $A$, $B$, $C$ matrices.  Then:
\begin{align*}
A + 0 & = A = 0 + A , \\
A + B & = B + A , \\
(A + B) + C & = A + (B + C) , \\
\alpha(A+B) & = \alpha A+\alpha B, \\
(\alpha+\beta)A & = \alpha A + \beta A.
\end{align*}
These rules should look very familiar.

\subsection{Matrix multiplication}

As we mentioned above, composition of linear mappings is also a
multiplication of matrices.  Suppose $A$ is an $m \times n$ matrix,
that is, $A$ takes
${\mathbb R}^n$ to
${\mathbb R}^m$,
and $B$ is an $n \times p$ matrix, that is, $B$ takes
${\mathbb R}^p$ to
${\mathbb R}^n$.  The composition $AB$ should work as follows
\begin{equation*}
AB\vec{x} = A(B\vec{x}) .
\end{equation*}
First, a vector $\vec{x}$ in ${\mathbb R}^p$ gets taken to 
the vector $B\vec{x}$ in
${\mathbb R}^n$.  Then the mapping $A$ takes it to the vector $A(B\vec{x})$
in ${\mathbb R}^m$.  In other words, the composition $AB$ should be an $m
\times p$ matrix.  In terms of sizes we should have
\begin{equation*}
%mbxSTARTIGNORE
\text{``}
%mbxENDIGNORE
%mbxlatex \text{"}
\quad
[ m \times n ]
\,
[ n \times p ]
=
[ m \times p ] . \quad
%mbxSTARTIGNORE
\text{''}
%mbxENDIGNORE
%mbxlatex \text{"}
\end{equation*}
Notice how the middle size must match.

OK\@, now we know what sizes of matrices we should be able to multiply,
and what the product should be.
Let us see how to actually compute matrix multiplication.
We start with the so-called
\emph{\myindex{dot product}} (or \emph{\myindex{inner product}}) of two vectors.
Usually this is a row vector multiplied
with a column vector of the same size.  Dot product multiplies
each pair of entries from the first and the second vector and sums these
products.  The result is a single number.
For example,
\begin{equation*}
\begin{bmatrix}
a_1 & a_2 & a_3
\end{bmatrix}
\cdot
\begin{bmatrix}
b_1 \\
b_2 \\
b_3
\end{bmatrix}
= a_1 b_1 + a_2 b_2 + a_3 b_3 .
\end{equation*}
And similarly for larger (or smaller) vectors.
A dot product is really a product of two matrices: a $1 \times n$ matrix
and an $n \times 1$ matrix resulting in a $1 \times 1$ matrix, that is, a
number.

Armed with the dot product we define the
\emph{\myindex{product of matrices}}\index{matrix product}.
First let us denote by $\operatorname{row}_i(A)$ the $i^{\text{th}}$ row
of $A$ and by
$\operatorname{column}_j(A)$ the $j^{\text{th}}$ column of $A$.
For an $m \times n$ matrix $A$ and an $n \times p$ matrix $B$
we can compute the product $AB$.  The matrix $AB$ is an $m \times p$
matrix whose $ij^{\text{th}}$ entry is the dot product
\begin{equation*}
\operatorname{row}_i(A) \cdot
\operatorname{column}_j(B) .
\end{equation*}
For example, given a $2 \times 3$ and a $3 \times 2$ matrix
we should end up with a $2 \times 2$ matrix:
\begin{equation} \label{linalg:eqmatrixmulex}
\begin{bmatrix}
a_{11} & a_{12} & a_{13} \\
a_{21} & a_{22} & a_{23}
\end{bmatrix}
\begin{bmatrix}
b_{11} & b_{12} \\
b_{21} & b_{22} \\
b_{31} & b_{32}
\end{bmatrix}
=
\begin{bmatrix}
a_{11} b_{11} + 
a_{12} b_{21} + 
a_{13} b_{31} & &
a_{11} b_{12} + 
a_{12} b_{22} + 
a_{13} b_{32} \\
a_{21} b_{11} + 
a_{22} b_{21} + 
a_{23} b_{31} & &
a_{21} b_{12} + 
a_{22} b_{22} + 
a_{23} b_{32}
\end{bmatrix} ,
\end{equation}
or with some numbers:
\begin{equation*}
\begin{bmatrix}
1 & 2 & 3 \\
4 & 5 & 6
\end{bmatrix}
\begin{bmatrix}
-1 & 2 \\
-7 & 0 \\
1 & -1
\end{bmatrix}
=
\begin{bmatrix}
1\cdot (-1) + 2\cdot (-7) + 3 \cdot 1 &  &
1\cdot 2 + 2\cdot 0 + 3 \cdot (-1) \\
4\cdot (-1) + 5\cdot (-7) + 6 \cdot 1 &  &
4\cdot 2 + 5\cdot 0 + 6 \cdot (-1)
\end{bmatrix}
=
\begin{bmatrix}
-12 & -1 \\
-33 & 2
\end{bmatrix} .
\end{equation*}

A useful consequence of the definition is that the evaluation
$A \vec{x}$ for a matrix $A$
and a (column) vector $\vec{x}$ is also matrix multiplication.
That is really why we think of
vectors as column vectors, or $n \times 1$ matrices.
For example,
\begin{equation*}
\begin{bmatrix}
1 & 2 \\ 3 & 4
\end{bmatrix}
\begin{bmatrix}
2 \\ -1
\end{bmatrix} 
=
\begin{bmatrix}
1 \cdot 2 + 2 \cdot (-1) \\
3 \cdot 2 + 4 \cdot (-1)
\end{bmatrix}
=
\begin{bmatrix}
0 \\ 2
\end{bmatrix}  .
\end{equation*}
If you look at the last section, that is precisely the last
example we gave.

You should stare at the computation of multiplication of matrices $AB$
and
the previous definition of $A\vec{y}$ as a mapping
for a moment.
What we are doing with matrix multiplication is applying
the mapping $A$ to the columns of $B$.  This is usually written as follows.
Suppose we write the $n \times p$ matrix
$B = [ \vec{b}_1 ~ \vec{b}_2 ~ \cdots ~ \vec{b}_p ]$, where
$\vec{b}_1, \vec{b}_2, \ldots, \vec{b}_p$ are the columns of $B$.  Then
for an $m \times n$ matrix $A$,
\begin{equation*}
AB = 
A [ \vec{b}_1 ~ \vec{b}_2 ~ \cdots ~ \vec{b}_p ]
=
[ A\vec{b}_1 ~ A\vec{b}_2 ~ \cdots ~ A\vec{b}_p ] .
\end{equation*}
The columns of the $m \times p$ matrix $AB$
are the
vectors $A\vec{b}_1, A\vec{b}_2, \ldots, A\vec{b}_p$.
For example, in \eqref{linalg:eqmatrixmulex},
the columns of 
\begin{equation*}
\begin{bmatrix}
a_{11} & a_{12} & a_{13} \\
a_{21} & a_{22} & a_{23}
\end{bmatrix}
\begin{bmatrix}
b_{11} & b_{12} \\
b_{21} & b_{22} \\
b_{31} & b_{32}
\end{bmatrix}
\end{equation*}
are
\begin{equation*}
\begin{bmatrix}
a_{11} & a_{12} & a_{13} \\
a_{21} & a_{22} & a_{23}
\end{bmatrix}
\begin{bmatrix}
b_{11} \\
b_{21} \\
b_{31}
\end{bmatrix}
\qquad
\text{and}
\qquad
\begin{bmatrix}
a_{11} & a_{12} & a_{13} \\
a_{21} & a_{22} & a_{23}
\end{bmatrix}
\begin{bmatrix}
b_{12} \\
b_{22} \\
b_{32}
\end{bmatrix} .
\end{equation*}
This is a very useful way to understand what matrix multiplication is.
It should also make it easier to remember how to perform matrix multiplication.

\subsection{Some rules of matrix algebra}

For multiplication we want an analogue of a 1.  That is,
we desire a matrix that just leaves everything as it found it.
This analogue is the
so-called \emph{\myindex{identity matrix}}.
The identity matrix is a square matrix with 1s on the
main diagonal and zeros everywhere else.  It is usually denoted by $I$.
For each size we have a different identity matrix and so sometimes we may denote
the size as a subscript.  For example, the $I_3$ would be the $3 \times 3$
identity matrix
\begin{equation*}
I = I_3 =
\begin{bmatrix}
1 & 0 & 0 \\
0 & 1 & 0 \\
0 & 0 & 1
\end{bmatrix} .
\end{equation*}
Let us see how the matrix works on a smaller example,
\begin{equation*}
\begin{bmatrix}
a_{11} & a_{12} \\
a_{21} & a_{22} 
\end{bmatrix}
\begin{bmatrix}
1 & 0 \\
0 & 1
\end{bmatrix} =
\begin{bmatrix}
a_{11} \cdot 1 + a_{12} \cdot 0
& &
a_{11} \cdot 0 + a_{12} \cdot 1
\\
a_{21} \cdot 1 + a_{22} \cdot 0
& &
a_{21} \cdot 0 + a_{22} \cdot 1
\end{bmatrix}
=
\begin{bmatrix}
a_{11} & a_{12} \\
a_{21} & a_{22} 
\end{bmatrix} .
\end{equation*}
Multiplication by the identity from the left looks similar, and also
does not touch anything.

\medskip

We have the following rules for matrix multiplication.  Suppose that
$A$, $B$, $C$ are matrices of the correct sizes so that the following
make sense.  Let $\alpha$ denote a scalar (number).  Then
\begin{align*}
A(BC) & = (AB)C & & \text{(\myindex{associative law})} , \\
A(B+C) & = AB + AC & & \text{(\myindex{distributive law})} , \\
(B+C)A & = BA + CA & & \text{(distributive law)} , \\
\alpha(AB) & = (\alpha A)B = A(\alpha B) , & &  \\
IA & = A = AI & & \text{(identity)}.
\end{align*}

\begin{example}
Let us demonstrate a couple of these rules.  For example, the associative
law:
\begin{equation*}
\underbrace{
\begin{bmatrix}
-3 & 3 \\ 2 & -2
\end{bmatrix}
}_A
\biggl(
\underbrace{
\begin{bmatrix}
4 & 4 \\ 1 & -3
\end{bmatrix}
}_B
\underbrace{
\begin{bmatrix}
-1 & 4 \\ 5 & 2
\end{bmatrix}
}_C
\biggr)
=
\underbrace{
\begin{bmatrix}
-3 & 3 \\ 2 & -2
\end{bmatrix}
}_A
\underbrace{
\begin{bmatrix}
16 & 24 \\ -16 & -2
\end{bmatrix}
}_{BC}
=
\underbrace{
\begin{bmatrix}
-96 & -78 \\ 64 & 52
\end{bmatrix}
}_{A(BC)} ,
\end{equation*}
and
\begin{equation*}
\biggl(
\underbrace{
\begin{bmatrix}
-3 & 3 \\ 2 & -2
\end{bmatrix}
}_A
\underbrace{
\begin{bmatrix}
4 & 4 \\ 1 & -3
\end{bmatrix}
}_B
\biggr)
\underbrace{
\begin{bmatrix}
-1 & 4 \\ 5 & 2
\end{bmatrix}
}_C
=
\underbrace{
\begin{bmatrix}
-9 & -21 \\ 6 & 14
\end{bmatrix}
}_{AB}
\underbrace{
\begin{bmatrix}
-1 & 4 \\ 5 & 2
\end{bmatrix}
}_C
=
\underbrace{
\begin{bmatrix}
-96 & -78 \\ 64 & 52
\end{bmatrix}
}_{(AB)C} .
\end{equation*}
Or how about multiplication by scalars:
\begin{equation*}
10
\biggl(
\underbrace{
\begin{bmatrix}
-3 & 3 \\ 2 & -2
\end{bmatrix}
}_A
\underbrace{
\begin{bmatrix}
4 & 4 \\ 1 & -3
\end{bmatrix}
}_B
\biggr)
=
10
\underbrace{
\begin{bmatrix}
-9 & -21 \\ 6 & 14
\end{bmatrix}
}_{A B} 
=
\underbrace{
\begin{bmatrix}
-90 & -210 \\ 60 & 140
\end{bmatrix}
}_{10 (AB)} ,
\end{equation*}
\begin{equation*}
\biggl(
10
\underbrace{
\begin{bmatrix}
-3 & 3 \\ 2 & -2
\end{bmatrix}
}_A
\biggr)
\underbrace{
\begin{bmatrix}
4 & 4 \\ 1 & -3
\end{bmatrix}
}_B
=
\underbrace{
\begin{bmatrix}
-30 & 30 \\ 20 & -20
\end{bmatrix}
}_{10 A}
\underbrace{
\begin{bmatrix}
4 & 4 \\ 1 & -3
\end{bmatrix}
}_B
=
\underbrace{
\begin{bmatrix}
-90 & -210 \\ 60 & 140
\end{bmatrix}
}_{(10 A)B} ,
\end{equation*}
and
\begin{equation*}
\underbrace{
\begin{bmatrix}
-3 & 3 \\ 2 & -2
\end{bmatrix}
}_A
\biggl(
10
\underbrace{
\begin{bmatrix}
4 & 4 \\ 1 & -3
\end{bmatrix}
}_B
\biggr)
=
\underbrace{
\begin{bmatrix}
-3 & 3 \\ 2 & -2
\end{bmatrix}
}_{A}
\underbrace{
\begin{bmatrix}
40 & 40 \\ 10 & -30
\end{bmatrix}
}_{10B}
=
\underbrace{
\begin{bmatrix}
-90 & -210 \\ 60 & 140
\end{bmatrix}
}_{A(10B)} .
\end{equation*}
\end{example}

A multiplication rule you have used since primary school on numbers
is quite conspicuously missing for matrices.
That is, matrix multiplication is
not commutative.  Firstly, just because $AB$ makes sense, it may be
that $BA$ is not even defined.  For example, if $A$ is $2 \times 3$, and
$B$ is $3 \times 4$, the we can multiply $AB$ but not $BA$.

Even if $AB$ and $BA$ are both defined, does not mean that they are equal.
For example, take
$A = \left[ \begin{smallmatrix} 1 & 1 \\ 1 & 1 \end{smallmatrix} \right]$
and
$B = \left[ \begin{smallmatrix} 1 & 0 \\ 0 & 2 \end{smallmatrix} \right]$:
\begin{equation*}
AB = 
\begin{bmatrix} 1 & 1 \\ 1 & 1 \end{bmatrix}
\begin{bmatrix} 1 & 0 \\ 0 & 2 \end{bmatrix}
=
\begin{bmatrix} 1 & 2 \\ 1 & 2 \end{bmatrix}
\qquad
\not=
\qquad
\begin{bmatrix} 1 & 1 \\ 2 & 2 \end{bmatrix}
=
\begin{bmatrix} 1 & 0 \\ 0 & 2 \end{bmatrix}
\begin{bmatrix} 1 & 1 \\ 1 & 1 \end{bmatrix}
=
BA .
\end{equation*}

\subsection{Inverse}

A couple of other algebra rules you know for numbers do not quite
work on matrices:
\begin{enumerate}[(i)]
\item $AB = AC$ does not necessarily imply $B=C$, even if $A$ is not 0.
\item $AB = 0$ does not necessarily mean that $A=0$ or $B=0$.
\end{enumerate}
For example:
\begin{equation*}
\begin{bmatrix} 0 & 1 \\ 0 & 0 \end{bmatrix}
\begin{bmatrix} 0 & 1 \\ 0 & 0 \end{bmatrix}
=
\begin{bmatrix} 0 & 0 \\ 0 & 0 \end{bmatrix}
=
\begin{bmatrix} 0 & 1 \\ 0 & 0 \end{bmatrix}
\begin{bmatrix} 0 & 2 \\ 0 & 0 \end{bmatrix} .
\end{equation*}

To make these rules hold, we do not just need one of the matrices to not be zero,
we would need to \myquote{divide} by
a matrix.  This is where the \emph{\myindex{matrix inverse}} comes in.

\begin{definition}
Suppose that $A$ and $B$ are $n \times n$ matrices such that
\begin{equation*}
AB = I = BA .
\end{equation*}
Then we call $B$ the inverse of $A$ and we denote $B$ by $A^{-1}$.

If the inverse of $A$ exists, then we say $A$ is
\emph{invertible\index{invertible matrix}}.
If $A$ is not invertible, we say $A$ is
\emph{singular\index{singular matrix}}.
\end{definition}

Perhaps not surprisingly, ${(A^{-1})}^{-1} = A$, since if the inverse of $A$ 
is $B$, then the inverse of $B$ is $A$.

If $A = [a]$ is a $1 \times 1$ matrix, then $A^{-1}$ is $a^{-1} =
\frac{1}{a}$.  That is where the notation comes from.  The computation is not nearly as simple when $A$ is
larger.

%Singular matrices are those you cannot \myquote{divide by.}
%So the mistake of \myquote{division by zero} is replaced by the mistake
%\myquote{inverting a singular matrix.}

The proper formulation of the cancellation rule is:
\begin{center}
\emph{If $A$ is invertible,
then
$AB = AC$ implies $B=C$.}
\end{center}
%Similarly:
%\emph{If $A$ is invertible,
%then
%$BA = CA$ implies $B=C$.}
The computation is what you would do in regular algebra with numbers,
but you have to
be careful never to commute matrices:
\begin{align*}
AB & = AC , \\
A^{-1}AB & = A^{-1}AC , \\
IB & = IC , \\
B & = C .
\end{align*}
And similarly for cancellation on the right:
\begin{center}
\emph{If $A$ is invertible,
then $BA = CA$ implies $B=C$.}
\end{center}

The rule says, among other things, that the
inverse of a matrix is unique if it exists:  If $AB = I = AC$, then $A$ is
invertible and $B=C$.

We will see later how to compute an inverse of a matrix
in general.  For now,
let us note that there is a simple formula for the inverse of
a $2 \times 2$ matrix
\begin{equation*}
\begin{bmatrix}
a & b \\
c & d
\end{bmatrix}^{-1}
=
\frac{1}{ad-bc}
\begin{bmatrix}
d & -b \\
-c & a
\end{bmatrix} .
\end{equation*}

For example:
\begin{equation*}
\begin{bmatrix}
1 & 1 \\
2 & 4
\end{bmatrix}^{-1}
=
\frac{1}{1\cdot 4-1 \cdot 2}
\begin{bmatrix}
4 & -1 \\
-2 & 1
\end{bmatrix}
=
\begin{bmatrix}
2 & \nicefrac{-1}{2} \\
-1 & \nicefrac{1}{2}
\end{bmatrix} .
\end{equation*}
Let's try it:
\begin{equation*}
\begin{bmatrix}
1 & 1 \\
2 & 4
\end{bmatrix}
\begin{bmatrix}
2 & \nicefrac{-1}{2} \\
-1 & \nicefrac{1}{2}
\end{bmatrix}
=
\begin{bmatrix}
1 & 0 \\
0 & 1
\end{bmatrix}
\qquad
\text{and}
\qquad
\begin{bmatrix}
2 & \nicefrac{-1}{2} \\
-1 & \nicefrac{1}{2}
\end{bmatrix}
\begin{bmatrix}
1 & 1 \\
2 & 4
\end{bmatrix}
=
\begin{bmatrix}
1 & 0 \\
0 & 1
\end{bmatrix} .
\end{equation*}
Just as we cannot divide by every number, not every matrix is
invertible.  In the case of matrices however we may have singular
matrices that are not zero.  For example,
\begin{equation*}
\begin{bmatrix}
1 & 1 \\
2 & 2
\end{bmatrix}
\end{equation*}
is a singular matrix.  But didn't we just give a formula for an inverse?
Let us try it:
\begin{equation*}
\begin{bmatrix}
1 & 1 \\
2 & 2
\end{bmatrix}^{-1}
=
\frac{1}{1\cdot 2-1 \cdot 2}
\begin{bmatrix}
2 & -1 \\
-2 & 1
\end{bmatrix}
=
%mbxSTARTIGNORE
\text{\Huge ?}
%mbxENDIGNORE
%mbxlatex \text{???}
\end{equation*}
We get into a bit of trouble; we are trying to divide by zero.
%The matrix is not invertible, it is singular.

So a $2 \times 2$ matrix $A$ is invertible whenever
\begin{equation*}
ad - bc \not= 0
\end{equation*}
and otherwise it is singular.  The expression $ad-bc$ is called
the \emph{determinant} and we will look at it 
more carefully in a later section.
There is a similar expression for a square
matrix of any size.

\subsection{Special types of matrices}

A simple (and surprisingly useful) type of a square matrix is a so-called
\emph{\myindex{diagonal matrix}}.  It is a matrix whose entries are all zero
except those on the main diagonal from top left to bottom right.  For
example a $4 \times 4$ diagonal matrix is of the form
\begin{equation*}
\begin{bmatrix}
d_1 & 0 & 0 & 0 \\
0 & d_2 & 0 & 0 \\
0 & 0 & d_3 & 0 \\
0 & 0 & 0 & d_4
\end{bmatrix} .
\end{equation*}
Such matrices have nice properties when we multiply by them.  If we multiply
them by a vector, they multiply the $k^{\text{th}}$ entry by $d_k$.  For
example,
\begin{equation*}
\begin{bmatrix}
1 & 0 & 0 \\
0 & 2 & 0 \\
0 & 0 & 3
\end{bmatrix}
\begin{bmatrix}
4 \\ 5 \\ 6
\end{bmatrix}
=
\begin{bmatrix}
1 \cdot 4 \\ 2 \cdot 5 \\ 3 \cdot 6
\end{bmatrix}
=
\begin{bmatrix}
4 \\ 10 \\ 18
\end{bmatrix} .
\end{equation*}
Similarly, when they multiply another matrix from the left, they multiply
the $k^{\text{th}}$ row by $d_k$.  For example,
\begin{equation*}
\begin{bmatrix}
2 & 0 & 0 \\
0 & 3 & 0 \\
0 & 0 & -1
\end{bmatrix}
\begin{bmatrix}
1 & 1 & 1 \\
1 & 1 & 1 \\
1 & 1 & 1 
\end{bmatrix}
=
\begin{bmatrix}
2 & 2 & 2 \\
3 & 3 & 3 \\
-1 & -1 & -1 
\end{bmatrix} .
\end{equation*}
On the other hand, multiplying on the right, they multiply the columns:
\begin{equation*}
\begin{bmatrix}
1 & 1 & 1 \\
1 & 1 & 1 \\
1 & 1 & 1 
\end{bmatrix}
\begin{bmatrix}
2 & 0 & 0 \\
0 & 3 & 0 \\
0 & 0 & -1
\end{bmatrix}
=
\begin{bmatrix}
2 & 3 & -1 \\
2 & 3 & -1 \\
2 & 3 & -1 
\end{bmatrix} .
\end{equation*}
And it is really easy to multiply two diagonal matrices together:
\begin{equation*}
\begin{bmatrix}
1 & 0 & 0 \\
0 & 2 & 0 \\
0 & 0 & 3 
\end{bmatrix}
\begin{bmatrix}
2 & 0 & 0 \\
0 & 3 & 0 \\
0 & 0 & -1
\end{bmatrix}
=
\begin{bmatrix}
1 \cdot 2 & 0 & 0 \\
0 & 2 \cdot 3 & 0 \\
0 & 0 & 3 \cdot (-1) 
\end{bmatrix}
=
\begin{bmatrix}
2 & 0 & 0 \\
0 & 6 & 0 \\
0 & 0 & -3 
\end{bmatrix} .
\end{equation*}

For this last reason, they are easy to invert, you simply invert
each diagonal element:
\begin{equation*}
\begin{bmatrix}
d_1 & 0 & 0 \\
0 & d_2 & 0 \\
0 & 0 & d_3 
\end{bmatrix}^{-1}
=
\begin{bmatrix}
d_1^{-1} & 0 & 0 \\
0 & d_2^{-1} & 0 \\
0 & 0 & d_3^{-1} 
\end{bmatrix} .
\end{equation*}
Let us check an example
\begin{equation*}
\underbrace{
\begin{bmatrix}
2 & 0 & 0 \\
0 & 3 & 0 \\
0 & 0 & 4 
\end{bmatrix}^{-1}
}_{A^{-1}}
\underbrace{
\begin{bmatrix}
2 & 0 & 0 \\
0 & 3 & 0 \\
0 & 0 & 4 
\end{bmatrix}
}_{A}
=
\underbrace{
\begin{bmatrix}
\frac{1}{2} & 0 & 0 \\
0 & \frac{1}{3} & 0 \\
0 & 0 & \frac{1}{4} 
\end{bmatrix}
}_{A^{-1}}
\underbrace{
\begin{bmatrix}
2 & 0 & 0 \\
0 & 3 & 0 \\
0 & 0 & 4 
\end{bmatrix}
}_{A}
=
\underbrace{
\begin{bmatrix}
1 & 0 & 0 \\
0 & 1 & 0 \\
0 & 0 & 1 
\end{bmatrix}
}_{I} .
\end{equation*}
It is no wonder that the way we solve many problems in linear algebra
(and in differential equations) is to try to reduce the problem to the
case of diagonal matrices.

Another type of matrix that has similarly nice properties are \emph{\myindex{triangular}} matrices. A matrix is
\emph{upper triangular} if all of the entries below the diagonal are zero. For a $3\times 3$ matrix, an upper triangular matrix 
looks like
\[ \begin{bmatrix} * & * & * \\ 0 & * & * \\ 0 & 0 & * \end{bmatrix} \] where the $*$ can be any number. Similarly, a \emph{lower triangular} matrix is one where all of the entries above the diagonal are zero, or, for a $3 \times 3$ matrix, something that looks like  
\[ \begin{bmatrix} * & 0 & 0 \\ * & * & 0 \\ * & * & * \end{bmatrix}. \] A matrix that is both upper and lower triangular is diagonal, because only the entries on the diagonal can be non-zero. 

\subsection{Transpose}

Vectors do not always have to be column vectors,
that is just a convention.
Swapping rows and columns is from time to time needed.
The operation that swaps rows and columns is the so-called
\emph{\myindex{transpose}}.
The transpose of $A$ is denoted by $A^T$.  Example:
\begin{equation*}
\begin{bmatrix}
1 & 2 & 3 \\
4 & 5 & 6
\end{bmatrix}^T =
\begin{bmatrix}
1 & 4 \\
2 & 5 \\
3 & 6 
\end{bmatrix} .
\end{equation*}
So transpose takes an $m \times n$ matrix to an $n \times m$ matrix.

A key fact about the transpose is that if the product $AB$ makes sense
then $B^TA^T$ also makes sense, at least from the point of view of sizes.
In fact, we get precisely the transpose of $AB$.  That is:
\begin{equation*}
{(AB)}^T = B^TA^T .
\end{equation*}
For example,
\begin{equation*}
{\left(
\begin{bmatrix}
1 & 2 & 3 \\
4 & 5 & 6
\end{bmatrix}
\begin{bmatrix}
0 & 1 \\
1 & 0 \\
2 & -2
\end{bmatrix}
\right)}^T =
\begin{bmatrix}
0 & 1 & 2 \\
1 & 0 & -2
\end{bmatrix}
\begin{bmatrix}
1 & 4 \\
2 & 5 \\
3 & 6 
\end{bmatrix} .
\end{equation*}
It is left to the reader to verify that computing the matrix product on the
left and then transposing is the same as computing the matrix product on the
right.

If we have a column vector $\vec{x}$ to which we apply a matrix $A$
and we transpose the result,
then the row vector $\vec{x}^T$ applies to $A^T$ from the left:
\begin{equation*}
{(A\vec{x})}^T = \vec{x}^TA^T .
\end{equation*}

Another place where transpose is useful is when we wish to apply the dot
product\footnote{As a side note, mathematicians
write $\vec{y}^T\vec{x}$ and physicists
write $\vec{x}^T\vec{y}$.  Shhh\ldots don't tell anyone, but the physicists
are probably right on this.}
to two column vectors:
\begin{equation*}
\vec{x} \cdot \vec{y} = \vec{y}^T \vec{x} .
\end{equation*}
That is the way that one often writes the dot product in software.

We say a matrix $A$ is \emph{symmetric}\index{symmetric matrix}
if $A = A^T$.  For example,
\begin{equation*}
\begin{bmatrix}
1 & 2 & 3 \\
2 & 4 & 5 \\
3 & 5 & 6
\end{bmatrix}
\end{equation*}
is a symmetric matrix.  Notice that a symmetric matrix is always
square, that is, $n \times n$.  Symmetric matrices 
have many nice properties\footnote{Although so far we have not learned
enough about matrices to really appreciate them.},
and come up quite often in applications.

To end the section, we notice how $A \vec{x}$ can be written more succintly.
Suppose
\begin{equation*}
A = 
\begin{bmatrix}
a_{11} & a_{12} & a_{13} \\
a_{21} & a_{22} & a_{23}
\end{bmatrix}
\qquad \text{and} \qquad
\vec{x} = 
\begin{bmatrix}
x_1 \\ x_2 \\ x_3 
\end{bmatrix} .
\end{equation*}
Then
\begin{equation*}
A \vec{x} = 
\begin{bmatrix}
a_{11} & a_{12} & a_{13} \\
a_{21} & a_{22} & a_{23}
\end{bmatrix}
\begin{bmatrix}
x_1 \\ x_2 \\ x_3 
\end{bmatrix} 
=
\begin{bmatrix}
a_{11} x_1 + a_{12} x_2 + a_{13} x_3 \\
a_{21} x_1 + a_{22} x_2 + a_{23} x_3
\end{bmatrix}  .
\end{equation*}
For example,
\begin{equation*}
\begin{bmatrix}
1 & 2 \\ 3 & 4
\end{bmatrix}
\begin{bmatrix}
2 \\ -1
\end{bmatrix} 
=
\begin{bmatrix}
1 \cdot 2 + 2 \cdot (-1) \\
3 \cdot 2 + 4 \cdot (-1)
\end{bmatrix}
=
\begin{bmatrix}
0 \\ 2
\end{bmatrix}  .
\end{equation*}

In other words, you take a row of the matrix, you multiply them by the
entries in your vector, you add things up, and that's the corresponding
entry in the resulting vector.

\subsection{Exercises}

\begin{exercise}
Add the following matrices
\begin{tasks}(2)
\task
$\begin{bmatrix}
-1 & 2 & 2 \\
5 & 8 & -1
\end{bmatrix}
+
\begin{bmatrix}
3 & 2 & 3 \\
8 & 3 & 5
\end{bmatrix}$
\task
$\begin{bmatrix}
1 & 2 & 4 \\
2 & 3 & 1 \\
0 & 5 & 1
\end{bmatrix}
+
\begin{bmatrix}
2 & -8 & -3 \\
3 & 1 & 0 \\
6 & -4 & 1
\end{bmatrix}$
\end{tasks}
\end{exercise}

\begin{exercise}\ansMark%
Add the following matrices
\begin{tasks}(2)
\task
$\begin{bmatrix}
2 & 1 & 0 \\
1 & 1 & -1
\end{bmatrix}
+
\begin{bmatrix}
5 & 3 & 4 \\
1 & 2 & 5
\end{bmatrix}$
\task
$\begin{bmatrix}
6 & -2 & 3 \\
7 & 3 & 3 \\
8 & -1 & 2
\end{bmatrix}
+
\begin{bmatrix}
-1 & -1 & -3 \\
6 & 7 & 3 \\
-9 & 4 & -1
\end{bmatrix}$
\end{tasks}
\end{exercise}
\exsol{%
a)~$\begin{bmatrix}
7 & 4 & 4 \\
2 & 3 & 4
\end{bmatrix}$
\quad
b)~$\begin{bmatrix}
5 & -3 & 0 \\
13 & 10 & 6 \\
-1 & 3 & 1
\end{bmatrix}$
}

\begin{exercise}
Compute
\begin{tasks}(2)
\task
$3\begin{bmatrix}
0 & 3 \\
-2 & 2
\end{bmatrix}
+
6
\begin{bmatrix}
1 & 5 \\
-1 & 5
\end{bmatrix}$
\task
$2\begin{bmatrix}
-3 & 1 \\
2 & 2
\end{bmatrix}
-
3
\begin{bmatrix}
2 & -1 \\
3 & 2
\end{bmatrix}$
\end{tasks}
\end{exercise}


\begin{exercise}\ansMark%
Compute
\begin{tasks}(2)
\task
$2\begin{bmatrix}
1 & 2 \\
3 & 4
\end{bmatrix}
+
3
\begin{bmatrix}
-1 & 3 \\
1 & 2
\end{bmatrix}$
\task
$3\begin{bmatrix}
2 & -1 \\
1 & 3
\end{bmatrix}
-
2
\begin{bmatrix}
2 & 1 \\
-1 & 2
\end{bmatrix}$
\end{tasks}
\end{exercise}
\exsol{%
a)~$\begin{bmatrix}
-1 & 13 \\
9 & 14
\end{bmatrix}$
\quad
b)~$\begin{bmatrix}
2 & -5 \\
5 & 5
\end{bmatrix}$
}

\begin{exercise}
Multiply the following matrices
\begin{tasks}(2)
\task
$\begin{bmatrix}
-1 & 2 \\
3 & 1 \\
5 & 8
\end{bmatrix}
\begin{bmatrix}
3 & -1 & 3 & 1 \\
8 & 3 & 2 & -3
\end{bmatrix}$
\task
$\begin{bmatrix}
1 & 2 & 3 \\
3 & 1 & 1 \\
1 & 0 & 3
\end{bmatrix}
\begin{bmatrix}
2 & 3 & 1 & 7 \\
1 & 2 & 3 & -1 \\
1 & -1 & 3 & 0
\end{bmatrix}$
\task
$\begin{bmatrix}
4 & 1 & 6 & 3 \\
5 & 6 & 5 & 0 \\
4 & 6 & 6 & 0
\end{bmatrix}
\begin{bmatrix}
2 & 5 \\
1 & 2 \\
3 & 5 \\
5 & 6
\end{bmatrix}$
\task
$\begin{bmatrix}
1 & 1 & 4 \\
0 & 5 & 1
\end{bmatrix}
\begin{bmatrix}
2 & 2 \\
1 & 0 \\
6 & 4
\end{bmatrix}$
\end{tasks}
\end{exercise}

\begin{exercise}\ansMark%
\pagebreak[2]
Multiply the following matrices
\begin{tasks}(2)
\task
$\begin{bmatrix}
2 & 1 & 4 \\
3 & 4 & 4
\end{bmatrix}
\begin{bmatrix}
2 & 4 \\
6 & 3 \\
3 & 5
\end{bmatrix}$
\task
$\begin{bmatrix}
0 & 3 & 3 \\
2 & -2 & 1 \\
3 & 5 & -2
\end{bmatrix}
\begin{bmatrix}
6 & 6 & 2 \\
4 & 6 & 0 \\
2 & 0 & 4
\end{bmatrix}$
\task
$\begin{bmatrix}
3 & 4 & 1 \\
2 & -1 & 0 \\
4 & -1 & 5
\end{bmatrix}
\begin{bmatrix}
0 & 2 & 5 & 0 \\
2 & 0 & 5 & 2 \\
3 & 6 & 1 & 6
\end{bmatrix}$
\task
$\begin{bmatrix}
-2 & -2 \\
5 & 3 \\
2 & 1
\end{bmatrix}
\begin{bmatrix}
0 & 3 \\
1 & 3
\end{bmatrix}$
\end{tasks}
\end{exercise}
\exsol{%
a)~$\begin{bmatrix}
22 & 31 \\
42 & 44
\end{bmatrix}$
\quad
b)~$\begin{bmatrix}
18 & 18 & 12 \\
6 & 0 & 8 \\
34 & 48 & -2
\end{bmatrix}$
\quad
c)~$\begin{bmatrix}
11 & 12 & 36 & 14 \\
-2 & 4 & 5 & -2 \\
13 & 38 & 20 & 28
\end{bmatrix}$
\quad
d)~$\begin{bmatrix}
-2 & -12 \\
3 & 24 \\
1 & 9
\end{bmatrix}$
}

\pagebreak

\begin{exercise}
\ 
\begin{tasks}
\task How must the dimensions of two matrices line up in order to multiply them together? If they can be multiplied, what is the dimension of the product?
\task If $A$ is a $3 \times 2$ matrix and the product $AB$ is a $3 \times 4$ matrix, then what are the dimensions of $B$?
\task If $A$ is a $5 \times 3$ matrix, is it possible to find a matrix $B$ so that the product $AB$ is a $4 \times 3$ matrix? What about a matrix $C$ so that the product $CA$ is a $4 \times 3$ matrix?
\end{tasks}
\end{exercise}

\begin{exercise} \label{ex:MatDims}
Assume that $A$ is a $3\times 4$ matrix.
\begin{tasks}
\task What must the dimensions of $B$ be in order for the product $AB$ to be defined?
\task What must the dimensions of $B$ be in order for the product $BA$ to be defined?
\task What about if we want to compute $ABA$ or $BAB$?
\end{tasks} 
\end{exercise}

\begin{exercise}
Complete \exerciseref{ex:MatDims} but with $A$ being a $2 \times 2$ matrix.
\end{exercise}

\begin{exercise}
Compute the inverse of the given matrices
\begin{tasks}(4)
\task
$\begin{bmatrix}
-3
\end{bmatrix}$
\task
$\begin{bmatrix}
0 & -1 \\
1 & 0
\end{bmatrix}$
\task
$\begin{bmatrix}
1 & 4 \\
1 & 3
\end{bmatrix}$
\task
$\begin{bmatrix}
2 & 2 \\
1 & 4
\end{bmatrix}$
\end{tasks}
\end{exercise}

\begin{exercise}\ansMark%
Compute the inverse of the given matrices
\begin{tasks}(4)
\task
$\begin{bmatrix}
2
\end{bmatrix}$
\task
$\begin{bmatrix}
0 & 1 \\
1 & 0
\end{bmatrix}$
\task
$\begin{bmatrix}
1 & 2 \\
3 & 5
\end{bmatrix}$
\task
$\begin{bmatrix}
4 & 2 \\
4 & 4
\end{bmatrix}$
\end{tasks}
\end{exercise}
\exsol{%
a)
$\begin{bmatrix}
\nicefrac{1}{2}
\end{bmatrix}$
\quad b)
$\begin{bmatrix}
0 & 1 \\
1 & 0
\end{bmatrix}$
\quad c)
$\begin{bmatrix}
-5 & 2 \\
3 & -1
\end{bmatrix}$
\quad d)
$\begin{bmatrix}
\nicefrac{1}{2} & \nicefrac{-1}{4} \\
\nicefrac{-1}{2} & \nicefrac{1}{2}
\end{bmatrix}$
}

\begin{exercise}
Compute the inverse of the given matrices
\begin{tasks}(3)
\task
$\begin{bmatrix}
-2 & 0 \\
0 & 1 
\end{bmatrix}$
\task
$\begin{bmatrix}
3 & 0 & 0 \\
0 & -2 & 0 \\ 
0 & 0 & 1
\end{bmatrix}$
\task
$\begin{bmatrix}
1 & 0 & 0 & 0 \\
0 & -1 & 0 & 0 \\ 
0 & 0 & 0.01 & 0 \\
0 & 0 & 0 & -5
\end{bmatrix}$
\end{tasks}
\end{exercise}

\begin{exercise}\ansMark%
Compute the inverse of the given matrices
\begin{tasks}(3)
\task
$\begin{bmatrix}
2 & 0 \\
0 & 3 
\end{bmatrix}$
\task
$\begin{bmatrix}
4 & 0 & 0 \\
0 & 5 & 0 \\ 
0 & 0 & -1
\end{bmatrix}$
\task
$\begin{bmatrix}
-1 & 0 & 0 & 0 \\
0 & 2 & 0 & 0 \\ 
0 & 0 & 3 & 0 \\
0 & 0 & 0 & 0.1
\end{bmatrix}$
\end{tasks}
\end{exercise}
\exsol{%
a)~$\begin{bmatrix}
\nicefrac{1}{2} & 0 \\
0 & \nicefrac{1}{3} 
\end{bmatrix}$
\quad
b)~$\begin{bmatrix}
\nicefrac{1}{4} & 0 & 0 \\
0 & \nicefrac{1}{5} & 0 \\ 
0 & 0 & -1
\end{bmatrix}$
\quad
c)~$\begin{bmatrix}
-1 & 0 & 0 & 0 \\
0 & \nicefrac{1}{2} & 0 & 0 \\ 
0 & 0 & \nicefrac{1}{3} & 0 \\
0 & 0 & 0 & 10
\end{bmatrix}$
}

\begin{exercise}
Consider the matrices
\begin{equation*}
A = \begin{bmatrix} 1 & 2 \\ 3 & 6 \end{bmatrix} \quad B = \begin{bmatrix} 1 & 1 \\ 0 & 2 \end{bmatrix} \quad C = \begin{bmatrix} 1 & 3 \\ 0 & 1 \end{bmatrix}.
\end{equation*}
\begin{tasks}
\task Compute the products $AB$ and $AC$.
\task Verify that for these matrices $AB = AC$, but $B \neq C$.
\end{tasks}
\end{exercise}

\begin{exercise}
Consider the matrices
\begin{equation*}
A = \begin{bmatrix} 1 & -1 \\ -1 & 1 \end{bmatrix} \quad B = \begin{bmatrix} 1 & 1 \\ 1 & 1 \end{bmatrix}.
\end{equation*}
Verify that $AB$ and $BA$ are both equal to zero, but neither of the matrices $A$ and $B$ are zero.
\end{exercise}

\begin{exercise}
\
\begin{tasks}
\task Let $A$ be a $3 \times 4$ matrix. What dimension does the vector $\vec{v}$ need to be in order for the product $A\vec{v}$ to be defined? If this product is defined, what is the dimension of the product $A\vec{v}$?
\task Let $B$ be a $3 \times 3$ matrix. What dimension does the vector $\vec{v}$ need to be in order for the product $B\vec{v}$ to be defined? If this product is defined, what is the dimension of the product $B\vec{v}$?
\end{tasks}
\end{exercise}

\setcounter{exercise}{100}

