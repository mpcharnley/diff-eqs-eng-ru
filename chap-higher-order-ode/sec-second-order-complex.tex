\section{Complex Roots and Euler's Formula}
\label{complexroots:section}

\LAtt{2.2}

\LO{
\item Understand the basics of complex numbers,
\item Use complex numbers to find complex solutions to second order constant coefficient equations, and
\item Use Euler's formula to find real-valued general solutions to these second order equations.
}

% \sectionnotes{1 lecture, \BDref{ \S3.3}}

The next case to consider for constant coefficient second order equations is the one where $b^2 - 4ac < 0$. This results in two roots $r_1$ and $r_2$, but they are complex roots. In order to solve differential equations with $b^2 - 4ac < 0$, we need to be able to manipulate and use some properties of complex numbers\index{complex number}. Complex numbers may seem a strange concept, especially because of the
terminology.  There is nothing imaginary or really complicated about complex
numbers. For more bachground information on complex numbers, see \Appendixref{sec:complexNums}.

To start with, we define $i = \sqrt{-1}$. Since this is the square root of a negative number, this $i$ is not a real number. A complex number is written in the form $z = x+iy$ where $x$ and $y$ are real numbers. For a complex number $x+iy$ we call
$x$ the \emph{\myindex{real part}} and $y$ the \emph{\myindex{imaginary part}} of the number.
Often the following notation is used,
\begin{equation*}
\operatorname{Re}(x+iy) = x
\qquad \text{and} \qquad
\operatorname{Im}(x+iy) = y.
\end{equation*}
The real numbers are contained in the complex numbers as those complex numbers with the imaginary part being zero. 

When trying to do arithmetic with complex numbers, we treat $i$ as though it is a variable, and do computations just as we would
with polynomials.
The important fact that we will use to simplify is the fact that since $i = \sqrt{-1}$, we have that $i^2 = -1$.
So whenever we see $i^2$, we replace it by $-1$.
For example,
\begin{equation*}
(2+3i)(4i) - 5i = 
(2\times 4)i + (3 \times 4) i^2 - 5i
=
8i + 12 (-1) - 5i
=
-12 + 3i .
\end{equation*}

The numbers
$i$ and $-i$ are the two roots of $r^2 + 1 = 0$.
Engineers often use the letter $j$ instead of $i$ for the square
root of $-1$.  We use the mathematicians' convention and use $i$.

\begin{exercise}
Make sure you understand (that you can justify)
the following identities:
\begin{tasks}(2)
\task $i^2 = -1$, $i^3 = -i$, $i^4 = 1$,
\task $\dfrac{1}{i} = -i$,
\task $(3-7i)(-2-9i) = \cdots = -69-13i$,
\task $(3-2i)(3+2i) = 3^2 - {(2i)}^2 = 3^2 + 2^2 = 13$,
\task $\frac{1}{3-2i} = \frac{1}{3-2i} \frac{3+2i}{3+2i} = \frac{3+2i}{13}
= \frac{3}{13}+\frac{2}{13}i$.
\end{tasks}
\end{exercise}

\pagebreak[2]
In order to solve differential equations where the characteristic equation has complex roots, we need to deal with the exponential $e^{a+bi}$ of complex numbers. 
We do this by writing down the Taylor series and plugging in the complex
number.  Because most properties of the exponential can be proved by looking
at the Taylor series, these
properties still hold for the complex
exponential.  For example the very important property: $e^{x+y} = e^x e^y$.  This means that
$e^{a+ib} = e^a e^{ib}$.  Hence if we can compute $e^{ib}$, we can
compute $e^{a+ib}$.  For $e^{ib}$, we use the so-called
\emph{\myindex{Euler's formula}}.

\begin{theorem1}[eulersformula]{Euler's formula} 
\begin{equation*}
%\mybxbg{~~
e^{i \theta} = \cos \theta + i \sin \theta
\qquad \text{ and } \qquad
e^{- i \theta} = \cos \theta - i \sin \theta .
%~~}
\end{equation*}
\end{theorem1}

In other words, $e^{a+ib} = e^a \bigl( \cos(b) + i \sin(b) \bigr) = e^a \cos(b) + i e^a \sin(b)$.

\begin{exercise}
Using Euler's formula, check the identities:
\begin{equation*}
\cos \theta = \frac{e^{i \theta} + e^{-i \theta}}{2}
\qquad \text{and} \qquad
\sin \theta = \frac{e^{i \theta} - e^{-i \theta}}{2i}.
\end{equation*}
\end{exercise}

\begin{exercise}
Double angle identities:
Start with $e^{i(2\theta)} = {\bigl(e^{i \theta} \bigr)}^2$.  Use Euler on
each side and deduce:
\begin{equation*}
\cos (2\theta) = \cos^2 \theta - \sin^2 \theta
\qquad \text{and} \qquad
\sin (2\theta) = 2 \sin \theta \cos \theta .
\end{equation*}
\end{exercise}

\subsection{Complex roots}

Suppose the equation $ay'' + by' + cy = 0$ has the 
characteristic equation
$a r^2 + b r + c = 0$ that has \myindex{complex roots}.
By the quadratic
formula, the roots are
$\frac{-b \pm \sqrt{b^2 - 4ac}}{2a}$.
These roots are complex if $b^2 - 4ac < 0$.  In this case the
roots are
\begin{equation*}
r_1, r_2 = \frac{-b}{2a} \pm i\frac{\sqrt{4ac - b^2}}{2a} .
\end{equation*}
As you can see, we always get a pair of roots of the form $\alpha \pm i
\beta$.  In this case we can still write the solution as
\begin{equation*}
y = C_1 e^{(\alpha+i\beta)x} + C_2 e^{(\alpha-i\beta)x} .
\end{equation*}
However, the exponential is now complex-valued, and so (real) linear combinations of these solutions will be complex valued. If we are using these equations to model physical problems, the answer should be real-valued, as the position of a mass-on-a-spring can not be a complex number. To do this, we need to determine two real-valued, linearly independent solutions to this differential equation.

To do this, we use the following result.

\begin{theorem1}[thm:realimagparts]{}
Consider the differential equation 
\begin{equation*}
y'' + p(x)y' + q(x)y = 0
\end{equation*}
where $p(t)$ and $q(t)$ are \emph{real-valued} continuous functions on some interval $I$. If $y$ is a complex-valued solution to this differential equation and we can split $y(x) = u(x) + i v(x)$ into its real and imaginary parts $u$ and $v$, then $u$ and $v$ are both solutions to $y'' + p(x)y' + q(x)y = 0$.
\end{theorem1} 

\begin{proof}
This is based on the fact that the differential equation is linear. We can compute derivatives of $y$
\begin{equation*}
\begin{split}
y(x)&= u(x) + iv(x) \\
y'(x) &= u'(x) + iv'(x) \\
y''(x) &= u''(x) + iv''(x)
\end{split}.
\end{equation*}
Then, we can plug this into the differential equation
\begin{equation*}
\begin{split}
0 &= y'' + p(x)y' + q(x) y \\
&= u''(x) + iv''(x) + p(x)(u'(x) + iv'(x)) + q(x) (u(x) + iv(x)) \\
0&= u''(x) + p(x)u'(x) + q(x)u(x) + i(v''(x) + p(x)v'(x) + q(x)v(x)) 
\end{split}.
\end{equation*}
Since the equation at the end of this chain is equal to zero, it must be zero as a complex number, which means that both the real and imaginary parts must be zero. This means that
\begin{equation*}
\begin{split}
u''(x) + p(x)u'(x) + q(x)u(x) &= 0 \\
v''(x) + p(x)v'(x) + q(x)v(x) &= 0
\end{split},
\end{equation*}
so that both $u$ and $v$ solve the original differential equation.
\end{proof}

To use this to solve the problem at hand, we have our solution
\begin{equation*}
y_1(x) = e^{\alpha + i\beta x}
\end{equation*}
and we need to split this into its real and imaginary parts. Since
\begin{equation*}
y_1  = e^{\alpha x} \cos (\beta x) + i e^{\alpha x} \sin (\beta x),
\end{equation*}
the real and imaginary parts of this function are
\begin{equation*}
\begin{split}
u(x) &= e^{\alpha x} \cos (\beta x) \\
v(x) &= e^{\alpha x} \sin (\beta x)
\end{split}
\end{equation*}
which, by the previous theorem, we know are also solutions. These are two solutions to our original differential equation that are also real-valued!

On the other hand, assume that we take the other complex solution, which will be  
\begin{equation*}
y_2(x) = e^{\alpha - i\beta x}.
\end{equation*}
If we split this into real and imaginary parts, we will get 
\begin{equation*}
y_2  = e^{\alpha x} \cos (\beta x) - i e^{\alpha x} \sin (\beta x),
\end{equation*}
so that the real and imaginary parts of this solution are
\begin{equation*}
\begin{split}
u_2(x) &= e^{\alpha x} \cos (\beta x) \\
v_2(x) &= -e^{\alpha x} \sin (\beta x).
\end{split}
\end{equation*}
These are exactly the same as the previous real and imaginary parts, up to the minus sign on $v_2$. Since we are going to incorporate these with constants $C_1$ and $C_2$ eventually, they will give rise to the same general solution. So, we only need one of these two complex solutions to generate our two linearly independent real-valued solutions, and either of the two complex solutions give the same pair of real-valued solutions.

\begin{exercise}
For $\beta \neq 0$, check that $e^{\alpha x} \cos (\beta x)$ and $e^{\alpha x} \sin (\beta x)$ are linearly independent.
\end{exercise}

With that fact, we have the following theorem. 

% We need to allow
%$C_1$ and $C_2$ to be complex numbers to obtain a real-valued solution (which
%is what we are after).  While there is nothing particularly wrong with this
%approach,
%it can make calculations harder and it is generally preferred
%to find two real-valued
%solutions.
%
%Here we can use \hyperref[eulersformula]{Euler's formula}.  Let
%\begin{equation*}
%y_1 = e^{(\alpha+i\beta)x} \qquad \text{and} \qquad y_2 = e^{(\alpha-i\beta)x} .
%\end{equation*}
%Then 
%\begin{align*}
%y_1 & = e^{\alpha x} \cos (\beta x) + i e^{\alpha x} \sin (\beta x) , \\
%y_2 & = e^{\alpha x} \cos (\beta x) - i e^{\alpha x} \sin (\beta x) .
%\end{align*}
%
%Linear combinations of solutions are also solutions.  Hence,
%\begin{align*}
%y_3 & = \frac{y_1 + y_2}{2} = e^{\alpha x} \cos (\beta x) , \\ 
%y_4 & = \frac{y_1 - y_2}{2i} = e^{\alpha x} \sin (\beta x) ,
%\end{align*}
%are also solutions.  Furthermore, they are real-valued.  It is not hard to
%see that they are linearly independent (not multiples of each other).
%Therefore, we have the following theorem.

\begin{theorem1}{}
Take the equation
\begin{equation*}
ay'' + by' + cy = 0 .
\end{equation*}
If the characteristic equation has the roots $\alpha \pm i \beta$
(when $b^2 - 4ac < 0$),
then the general solution is
\begin{equation*}
y = C_1 e^{\alpha x} \cos (\beta x) + C_2 e^{\alpha x} \sin (\beta x) .
\end{equation*}
\end{theorem1}

\begin{example} \label{example:sincossecondorder}
Find the general solution of $y'' + k^2 y = 0$, for a constant
$k > 0$.
\end{example}

\begin{exampleSol}
The characteristic equation is $r^2 + k^2 = 0$.  Therefore,
the roots are $r = \pm ik$, and by the theorem, we have the general solution
\begin{equation*}
y = C_1 \cos (kx) + C_2 \sin (kx) .
\end{equation*}
\end{exampleSol}

\begin{example}
Find the solution of $y'' - 6 y' + 13 y = 0$, $y(0) = 0$, $y'(0) =
10$.
\end{example}

\begin{exampleSol}
The characteristic equation is $r^2 - 6 r + 13 = 0$.  By completing the
square we get ${(r-3)}^2 + 2^2 = 0$ and hence the roots are
$r = 3 \pm 2i$.
By the theorem we have the general solution
\begin{equation*}
y = C_1 e^{3x} \cos (2x) + C_2 e^{3x} \sin (2x) .
\end{equation*}
To find the solution satisfying the initial conditions, we first plug in zero
to get
\begin{equation*}
0 = y(0) = C_1 e^{0} \cos 0 + C_2 e^{0} \sin 0  = C_1 .
\end{equation*}
Hence, $C_1 = 0$ and $y = C_2 e^{3x} \sin (2x)$.  We differentiate,
\begin{equation*}
y' = 3C_2 e^{3x} \sin (2x) + 2C_2 e^{3x} \cos (2x) .
\end{equation*}
We again plug in the initial condition and obtain $10 = y'(0) = 2C_2$, or
$C_2 = 5$.  The solution we are seeking is
\begin{equation*}
y = 5 e^{3x} \sin (2x) .
\end{equation*}
\end{exampleSol}

In this previous example, we can get a fairly good idea of how to sketch out the graph of this function. Since $\sin(2x)$ oscillates between $-1$ and $1$, the graph of $y = 5e^{3x}\sin(2x)$ will oscillate between the graphs of $5e^{3x}$ and $-5e^{3x}$. These curves that surround the graph of the solution are called \emph{envelope curves} for the solution. In \figureref{envelope:fig}, this phenomenon is illustrated for the function $y = 2e^x\sin(5x)$.

\begin{myfig}
\capstart
\myincludegraphics{width=3in}{width=4.5in}{envelopeCurves.png}
\caption{Plot of the function $y=2e^x \sin(5x)$ with envelope curves. \label{envelope:fig}}
\end{myfig}

This is simple when there is only one term in the function we want to draw. When both sine and cosine terms appear, this can get more tricky, but we can still work it out. In the more general case, the solution will look something like
\[ y = Ae^{\alpha x}\cos(\beta x) + B e^{\alpha x} \sin(\beta x). \] We can first factor out an $e^{\alpha x}$, and then we want to write $A \cos(\beta x) + B \sin(\beta x)$ as a single trigonometric function.  The identity we want to use here is the trigonometric identity
\[ \cos(\beta x - \delta) = \cos(\delta) \cos(\beta x) + \sin(\delta) \sin(\beta x). \] If there is an angle $\delta$ so that $A = \cos(\delta)$ and $B = \sin(\delta)$, then we could write
\[  A \cos(\beta x) + B \sin(\beta x) = \cos(\beta x - \delta) \] and we would be done. However, this does not always happen; the main issue being that $\cos^2(\delta) + \sin^2(\delta) = 1$ for all $\delta$, but it is not necessarily the case that $A^2 + B^2 = 1$. But we can force this last condition. If we define $R = \sqrt{A^2 + B^2}$, then we can rewrite this expression as
\[
\begin{split}
A \cos(\beta x) + B \sin(\beta x) &= R \left( \frac{A}{\sqrt{A^2 + B^2}} \cos(\beta x) + \frac{B}{\sqrt{A^2 + B^2}}\sin(\beta x)\right) \\
&= R \left( \cos(\delta) \cos(\beta x) + \sin(\delta) \sin(\beta x) \right) \\
&= R \cos(\beta x - \delta)
\end{split}
\]
where $\delta$ is the angle so that
\[ \cos(\delta) = \frac{A}{R} \qquad \sin(\delta) = \frac{B}{R} \] and such an angle will always exist. Therefore, we can represent the original solution 
\[ y = Ae^{\alpha x}\cos(\beta x) + B e^{\alpha x} \sin(\beta x) \] as 
\[ y = Re^{\alpha x}\cos(\beta x - \delta) \] where
\[ R = \sqrt{A^2 + B^2} \qquad \cos(\delta) = \frac{A}{R} \qquad \sin(\delta) = \frac{B}{R}. \] Therefore, the envelope curves for this solution will be 
\[ y = \pm Re^{\alpha x}. \]

Note that in order to determine these envelope curves, you do not need to determine the $\delta$ value in the representation of the solution. All you need is the value of $R$, which can be computed as $\sqrt{A^2 + B^2}$ where $A$ and $B$ are the coefficients of the sine and cosine terms in the solution. 

\begin{example}
Find the solution to the initial value problem
\[ y'' + 2y' + 5y = 0 \qquad y(0) = 1,\ y'(0) = 5. \]
Determine a value $T$ where the solution $y(x)$ satisfies $|y(x)| < 0.1$ for all $x > T$. 
\end{example}

\begin{exampleSol}
We solve the initial value problem by normal techniques from this section. The characteristic equation is $r^2 + 2r + 5 = 0$, which has roots $r = -1 \pm 2i$. Therefore, the general solution of the differential equation is
\[ y = C_1e^{-x}\cos(2x) + C_2 e^{-x}\sin(2x). \]
Plugging in $0$ gives that $y(0) = 1 = C_1$, and the derivative of this general solution is
\[ y' = -C_1e^{-x}\cos(2x) -2C_1e^{-x}\sin(2x) - C_2e^{-x}\sin(2x) + 2C_2e^{-x}\cos(2x). \] Plugging in $0$ here gives
\[ y'(0) = -C_1 + 2C_2. \] Since $C_1 = 1$, this gives that $C_2 = 3$. So, our solution is
\[ y(x) = e^{-x}\cos(2x) + 3e^{-x}\sin(2x). \]

Through the work above, we can find $R = \sqrt{1 + 9} = \sqrt{10}$. Therefore, the envelope curves for the solution are
\[ \pm \sqrt{10}e^{-x}. \] In order to find this threshold $T$ where the solution will stay within $0.1$ of zero, we need to figure out when this envelope curves get to the $0.1$ threshold. Once the envelope curves get to that level, we know that the full solution must be trapped there as well. We can solve
\[ 0.1 = \sqrt{10}e^{-T} \qquad T = - \ln\left(\frac{0.1}{\sqrt{10}}\right) \approx 3.454. \]

So, for all values of $x$ larger than 3.454, the solution will be within $0.1$ of zero. This is illustrated in \figureref{envBounds:fig}. Note that we did not find the \emph{best} value $T$ here, as it probably could be made smaller using the actual solution. The issue here is that because the solution is oscillating, it may end up staying inside the $0.1$ cutoff before that value of time, but this is the lowest value of $T$ that we can prove and validate using envelope curves.
\end{exampleSol}

\begin{myfig}
\capstart
\myincludegraphics{width=3in}{width=4.5in}{envBounds.png}
\caption{Plot of the function $e^{-x}\cos(2x) + 3e^{-x}\sin(2x)$ with envelope curves illustrating the bounds on the function for large values of $x$. \label{envBounds:fig}}
\end{myfig}


\subsection{Exercises}

\begin{exercise}\ansMark%
Write $3 \cos(2x) + 3\sin(2x)$ in the form $R \cos(\beta x - \delta)$.
\end{exercise}
\exsol{%
$3\sqrt{2}\cos\left(2x - \frac{\pi}{4}\right)$
}%

\begin{exercise}
Write $2 \cos(3x) + \sin(3x)$ in the form $R \cos(\beta x - \delta)$.
\end{exercise}

\begin{exercise}
Write $3 \cos(x) - 4\sin(x)$ in the form $R \cos(\beta x - \delta)$.
\end{exercise}

\begin{exercise}
Show that $e^{2x}\cos(x)$ and $e^{2x}\sin(x)$ are linearly independent.
\end{exercise}

\begin{exercise}
Find the general solution of $2y'' + 50y = 0$.
\end{exercise}

\begin{exercise}
Find the general solution of $y'' - 6 y' + 13 y = 0$.
\end{exercise}

\begin{exercise}
Find the solution to $y'' - 2y' + 5y = 0$ with $y(0) = 3$ and $y'(0) = 2$. 
\end{exercise} 

\begin{exercise}
Find the general solution of $y'' + 2y' - 3y = 0$.
\end{exercise}

\begin{exercise}\ansMark%
Find the solution to
$2y''+y'+y=0$, $y(0) = 1$, $y'(0)=-2$.
\end{exercise}
\exsol{%
$y =
e^{-x/4} \cos\bigl((\nicefrac{\sqrt{7}}{4})x\bigr)
-
\sqrt{7}
e^{-x/4} \sin\bigl((\nicefrac{\sqrt{7}}{4})x\bigr)$
}

\begin{exercise}\ansMark%
Find the solution to
$z''(t) = -2z'(t)-2z(t)$, $z(0) = 2$, $z'(0)= -2$.
\end{exercise}
\exsol{%
$z(t) =
2e^{-t} \cos(t)$
}


\begin{exercise}
Let us revisit the Cauchy--Euler equations\index{Cauchy--Euler equation} of
\exercisevref{sol:eulerex}.  Suppose now
that ${(b-a)}^2-4ac < 0$.  Find a formula for the general solution
of $a x^2 y'' + b x y' + c y = 0$.  Hint: Note that $x^r = e^{r \ln x}$.
\end{exercise}

\begin{exercise}
Construct an equation such that $y = C_1 e^{-2x} \cos(3x) + C_2 e^{-2x}
\sin(3x)$ is the general
solution.
\end{exercise}

\begin{exercise}\ansMark%
Find a second order, constant coefficient differential equation with general solution given by $y(t) = C_1e^{x} \cos(2x) + C_2e^{2x}\sin(x)$ or explain why there is no such thing.
\end{exercise}
\exsol{%
There is no such equation. The two roots will always be complex conjugates, which means the exponential parts will match, and the trigonometric functions will have the same argument.
}%

\begin{exercise}
Find a second order, constant coefficient differential equation with general solution given by $y(t) = C_1e^{x} \cos(2x) + C_2e^{x}\sin(2x)$ or explain why there is no such thing.
\end{exercise}

\begin{exercise}
Find the solution to the initial value problem 
\[ y'' + 4y' + 5y = 0 \qquad y(0) = 3,\ y'(0) = -1.\]
Determine a value $T$ so that $|y(x)| < 0.02$ for all $x > T$. 
\end{exercise}

\begin{exercise}
Find the solution to the initial value problem 
\[ y'' + 6y' + 13y = 0 \qquad y(0) = 4,\ y'(0) = 7.\]
Determine a value $T$ so that $|y(x)| < 0.01$ for all $x > T$. 
\end{exercise}

\setcounter{exercise}{100}






